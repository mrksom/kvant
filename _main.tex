% Options for packages loaded elsewhere
\PassOptionsToPackage{unicode}{hyperref}
\PassOptionsToPackage{hyphens}{url}
%
\documentclass[
]{book}
\usepackage{amsmath,amssymb}
\usepackage{lmodern}
\usepackage{ifxetex,ifluatex}
\ifnum 0\ifxetex 1\fi\ifluatex 1\fi=0 % if pdftex
  \usepackage[T1]{fontenc}
  \usepackage[utf8]{inputenc}
  \usepackage{textcomp} % provide euro and other symbols
\else % if luatex or xetex
  \usepackage{unicode-math}
  \defaultfontfeatures{Scale=MatchLowercase}
  \defaultfontfeatures[\rmfamily]{Ligatures=TeX,Scale=1}
\fi
% Use upquote if available, for straight quotes in verbatim environments
\IfFileExists{upquote.sty}{\usepackage{upquote}}{}
\IfFileExists{microtype.sty}{% use microtype if available
  \usepackage[]{microtype}
  \UseMicrotypeSet[protrusion]{basicmath} % disable protrusion for tt fonts
}{}
\makeatletter
\@ifundefined{KOMAClassName}{% if non-KOMA class
  \IfFileExists{parskip.sty}{%
    \usepackage{parskip}
  }{% else
    \setlength{\parindent}{0pt}
    \setlength{\parskip}{6pt plus 2pt minus 1pt}}
}{% if KOMA class
  \KOMAoptions{parskip=half}}
\makeatother
\usepackage{xcolor}
\IfFileExists{xurl.sty}{\usepackage{xurl}}{} % add URL line breaks if available
\IfFileExists{bookmark.sty}{\usepackage{bookmark}}{\usepackage{hyperref}}
\hypersetup{
  pdftitle={Kvantitatiivne andmeanalüüs},
  pdfauthor={Marko Sõmer},
  hidelinks,
  pdfcreator={LaTeX via pandoc}}
\urlstyle{same} % disable monospaced font for URLs
\usepackage{color}
\usepackage{fancyvrb}
\newcommand{\VerbBar}{|}
\newcommand{\VERB}{\Verb[commandchars=\\\{\}]}
\DefineVerbatimEnvironment{Highlighting}{Verbatim}{commandchars=\\\{\}}
% Add ',fontsize=\small' for more characters per line
\usepackage{framed}
\definecolor{shadecolor}{RGB}{248,248,248}
\newenvironment{Shaded}{\begin{snugshade}}{\end{snugshade}}
\newcommand{\AlertTok}[1]{\textcolor[rgb]{0.94,0.16,0.16}{#1}}
\newcommand{\AnnotationTok}[1]{\textcolor[rgb]{0.56,0.35,0.01}{\textbf{\textit{#1}}}}
\newcommand{\AttributeTok}[1]{\textcolor[rgb]{0.77,0.63,0.00}{#1}}
\newcommand{\BaseNTok}[1]{\textcolor[rgb]{0.00,0.00,0.81}{#1}}
\newcommand{\BuiltInTok}[1]{#1}
\newcommand{\CharTok}[1]{\textcolor[rgb]{0.31,0.60,0.02}{#1}}
\newcommand{\CommentTok}[1]{\textcolor[rgb]{0.56,0.35,0.01}{\textit{#1}}}
\newcommand{\CommentVarTok}[1]{\textcolor[rgb]{0.56,0.35,0.01}{\textbf{\textit{#1}}}}
\newcommand{\ConstantTok}[1]{\textcolor[rgb]{0.00,0.00,0.00}{#1}}
\newcommand{\ControlFlowTok}[1]{\textcolor[rgb]{0.13,0.29,0.53}{\textbf{#1}}}
\newcommand{\DataTypeTok}[1]{\textcolor[rgb]{0.13,0.29,0.53}{#1}}
\newcommand{\DecValTok}[1]{\textcolor[rgb]{0.00,0.00,0.81}{#1}}
\newcommand{\DocumentationTok}[1]{\textcolor[rgb]{0.56,0.35,0.01}{\textbf{\textit{#1}}}}
\newcommand{\ErrorTok}[1]{\textcolor[rgb]{0.64,0.00,0.00}{\textbf{#1}}}
\newcommand{\ExtensionTok}[1]{#1}
\newcommand{\FloatTok}[1]{\textcolor[rgb]{0.00,0.00,0.81}{#1}}
\newcommand{\FunctionTok}[1]{\textcolor[rgb]{0.00,0.00,0.00}{#1}}
\newcommand{\ImportTok}[1]{#1}
\newcommand{\InformationTok}[1]{\textcolor[rgb]{0.56,0.35,0.01}{\textbf{\textit{#1}}}}
\newcommand{\KeywordTok}[1]{\textcolor[rgb]{0.13,0.29,0.53}{\textbf{#1}}}
\newcommand{\NormalTok}[1]{#1}
\newcommand{\OperatorTok}[1]{\textcolor[rgb]{0.81,0.36,0.00}{\textbf{#1}}}
\newcommand{\OtherTok}[1]{\textcolor[rgb]{0.56,0.35,0.01}{#1}}
\newcommand{\PreprocessorTok}[1]{\textcolor[rgb]{0.56,0.35,0.01}{\textit{#1}}}
\newcommand{\RegionMarkerTok}[1]{#1}
\newcommand{\SpecialCharTok}[1]{\textcolor[rgb]{0.00,0.00,0.00}{#1}}
\newcommand{\SpecialStringTok}[1]{\textcolor[rgb]{0.31,0.60,0.02}{#1}}
\newcommand{\StringTok}[1]{\textcolor[rgb]{0.31,0.60,0.02}{#1}}
\newcommand{\VariableTok}[1]{\textcolor[rgb]{0.00,0.00,0.00}{#1}}
\newcommand{\VerbatimStringTok}[1]{\textcolor[rgb]{0.31,0.60,0.02}{#1}}
\newcommand{\WarningTok}[1]{\textcolor[rgb]{0.56,0.35,0.01}{\textbf{\textit{#1}}}}
\usepackage{longtable,booktabs,array}
\usepackage{calc} % for calculating minipage widths
% Correct order of tables after \paragraph or \subparagraph
\usepackage{etoolbox}
\makeatletter
\patchcmd\longtable{\par}{\if@noskipsec\mbox{}\fi\par}{}{}
\makeatother
% Allow footnotes in longtable head/foot
\IfFileExists{footnotehyper.sty}{\usepackage{footnotehyper}}{\usepackage{footnote}}
\makesavenoteenv{longtable}
\usepackage{graphicx}
\makeatletter
\def\maxwidth{\ifdim\Gin@nat@width>\linewidth\linewidth\else\Gin@nat@width\fi}
\def\maxheight{\ifdim\Gin@nat@height>\textheight\textheight\else\Gin@nat@height\fi}
\makeatother
% Scale images if necessary, so that they will not overflow the page
% margins by default, and it is still possible to overwrite the defaults
% using explicit options in \includegraphics[width, height, ...]{}
\setkeys{Gin}{width=\maxwidth,height=\maxheight,keepaspectratio}
% Set default figure placement to htbp
\makeatletter
\def\fps@figure{htbp}
\makeatother
\setlength{\emergencystretch}{3em} % prevent overfull lines
\providecommand{\tightlist}{%
  \setlength{\itemsep}{0pt}\setlength{\parskip}{0pt}}
\setcounter{secnumdepth}{5}
\usepackage{booktabs}

\renewcommand{\figurename}{Joonis}
\renewcommand\tablename{Tabel}
\renewcommand{\chaptername}{Peatükk}
\renewcommand{\contentsname}{Sisukord}
\renewcommand{\partname}{Osa}
\usepackage{fontspec}
\usepackage{multirow}
\usepackage{multicol}
\usepackage{colortbl}
\usepackage{hhline}
\usepackage{longtable}
\usepackage{array}
\usepackage{hyperref}
\ifluatex
  \usepackage{selnolig}  % disable illegal ligatures
\fi
\usepackage[]{natbib}
\bibliographystyle{plainnat}

\title{Kvantitatiivne andmeanalüüs}
\author{Marko Sõmer}
\date{2022-03-30}

\begin{document}
\maketitle

{
\setcounter{tocdepth}{1}
\tableofcontents
}
\hypertarget{teadmiseks}{%
\chapter{Teadmiseks}\label{teadmiseks}}

Siia on koondatud õppematerjalid kursusele ``Kvantitatiivne andmeanalüüs II''.

\hypertarget{part-meeldetuletus}{%
\part{Meeldetuletus}\label{part-meeldetuletus}}

\hypertarget{sissejuhatus-r-i}{%
\chapter{Sissejuhatus R-i}\label{sissejuhatus-r-i}}

\hypertarget{puxe4ris-algus}{%
\section{Päris algus}\label{puxe4ris-algus}}

\textbf{Mis see R on?}

\begin{itemize}
\tightlist
\item
  Analüüsikeskkond
\item
  Programmeerimiskeel
\item
  Vabavara
\item
  Põhineb kasutajate kirjutatud pakkettidel, ehk siis pidevalt arenev
\item
  Andmeanalüüsi seisukohalt praktiliselt piiramatute võimalustega
\item
  Järjest enam kasutatav
\item
  Publikatsioonitaseme andmegraafika
\item
  Suhteliselt järsu õppimiskõveraga, aga kui see ületatud, siis läheb lihtsamaks :)
\end{itemize}

\textbf{Miks R?}

\begin{itemize}
\tightlist
\item
  Kui on vajadus natukenegi tõsisemalt andmeanalüüsi või andmegraafikaga tegeleda, siis tuleb mingi hetk nii ehk naa R ära õppida (või siis näiteks Python).
\item
  Kui vahest harva on vaja mõni analüüs teha, siis ei ole ju mingit mõtet selle jaoks kallist (ja ebaefektiivset) kommertstarkvara soetada. R see-eest on tasuta.
\item
  Seega küsimus võiks pigem kõlada, et miks SPSS, Stata, SAS või Mplus?
\end{itemize}

\textbf{Mida R'i kasutamiseks vaja on?}

Baasprogramm\\
\url{https://cran.r-project.org/}

RStudio kasutajaliides\\
\url{https://www.rstudio.com/}

Saab tegelikult hakkama ka ainult baasprogrammiga. Kuid RStudio teeb töö päris palju mugavamaks ja efektiivsemaks ning lisab võimalusi (näiteks R Markdown analüüside kommunikeerimiseks, projektihaldus jne).\\
On ka teisi kasutajaliideseid ja scriptide kirjutamise abivahendeid, kuid RStudio on neist kahtlemata kõige populaarsem, funktsionaalsem ja mugavam.

\textbf{Kuidas R'i ja RStudiot kasutada?}

R'is ei ole rippmenüüsid, OK nuppe ega avanevaid aknaid. Kõik käsud tuleb käsureale sisse trükkida. Ühest küljest nõuab see kasutajalt mõnevõrra põhjalikumat arusaamist oma tegevusest, teisalt võimaldab tegevusi kombineerida, neist head ülevaadet saada ning tehtud analüüse lihtsalt korrata (\emph{reproducible research})

\begin{itemize}
\item
  Projektide haldamine\\
  RStudio võimaldab erinevaid töid hallata projektidena. Igal projektil on projektikaust (nn juurkataloog), kuhu saab salvestada kõik antud projektiga seonduvad scriptid, andmed jne. Uue projekti loomiseks klikkige \emph{File -\textgreater{} New Project}, misjärel saab valida kataloogi, kuhu projekt (ja seega kõik projektiga seotud matejalid) salvestub.
\item
  Skriptide loomine\\
  Kõik, mida me andmetega teeme, tuleks salvestada skripti (tegelikult on tegemist kõige tavalisema tekstifailiga). Uue skripti saab teha valides RStudios \emph{File -\textgreater{} New File -\textgreater{} R Script}\\
  Skriptide kaudu on meil ülevaade kõigest, mida me andmetega teinud oleme (andmeteisendused, analüüsikäik jne) ning samas saame iga hetk oma analüüsi korrata.\\
  Skriptis olevat koodi saame jooksutada kui märgime vajaliku koodi ära ja vajutame \emph{Cntr+Enter} või kasutame scriptiakna üleval paremas servas asuvat nuppu \emph{Run}.
\end{itemize}

\textbf{R kui kalkulaator}

\begin{Shaded}
\begin{Highlighting}[]
\DecValTok{2} \SpecialCharTok{+} \DecValTok{3}
\end{Highlighting}
\end{Shaded}

\begin{verbatim}
## [1] 5
\end{verbatim}

\begin{Shaded}
\begin{Highlighting}[]
\NormalTok{(}\DecValTok{4} \SpecialCharTok{{-}} \DecValTok{2}\NormalTok{) }\SpecialCharTok{/} \DecValTok{2} \CommentTok{\# sulud toimivad nii nagu nad toimima peavad}
\end{Highlighting}
\end{Shaded}

\begin{verbatim}
## [1] 1
\end{verbatim}

\begin{Shaded}
\begin{Highlighting}[]
\DecValTok{10} \SpecialCharTok{*} \DecValTok{10} \CommentTok{\# korrutamine}
\end{Highlighting}
\end{Shaded}

\begin{verbatim}
## [1] 100
\end{verbatim}

\begin{Shaded}
\begin{Highlighting}[]
\DecValTok{10} \SpecialCharTok{/} \DecValTok{5} \CommentTok{\# jagamine}
\end{Highlighting}
\end{Shaded}

\begin{verbatim}
## [1] 2
\end{verbatim}

\begin{Shaded}
\begin{Highlighting}[]
\DecValTok{3}\SpecialCharTok{\^{}}\DecValTok{2} \CommentTok{\#astendamine}
\end{Highlighting}
\end{Shaded}

\begin{verbatim}
## [1] 9
\end{verbatim}

\begin{Shaded}
\begin{Highlighting}[]
\DecValTok{3}\SpecialCharTok{**}\DecValTok{2} \CommentTok{\# ka nii saab astendada}
\end{Highlighting}
\end{Shaded}

\begin{verbatim}
## [1] 9
\end{verbatim}

\textbf{Loogilised tehted}

\begin{Shaded}
\begin{Highlighting}[]
\DecValTok{2} \SpecialCharTok{==} \DecValTok{2} \CommentTok{\# võrdub}
\end{Highlighting}
\end{Shaded}

\begin{verbatim}
## [1] TRUE
\end{verbatim}

\begin{Shaded}
\begin{Highlighting}[]
\DecValTok{1} \SpecialCharTok{!=} \DecValTok{2} \CommentTok{\# ei võrdu}
\end{Highlighting}
\end{Shaded}

\begin{verbatim}
## [1] TRUE
\end{verbatim}

\begin{Shaded}
\begin{Highlighting}[]
\DecValTok{2} \SpecialCharTok{\textgreater{}=} \DecValTok{2} \CommentTok{\# suurem kui või võrdne}
\end{Highlighting}
\end{Shaded}

\begin{verbatim}
## [1] TRUE
\end{verbatim}

\begin{Shaded}
\begin{Highlighting}[]
\DecValTok{1} \SpecialCharTok{\textless{}} \DecValTok{2}  \CommentTok{\# väiksem kui}
\end{Highlighting}
\end{Shaded}

\begin{verbatim}
## [1] TRUE
\end{verbatim}

\begin{Shaded}
\begin{Highlighting}[]
\NormalTok{(}\DecValTok{3}\SpecialCharTok{\textless{}}\DecValTok{6}\NormalTok{) }\SpecialCharTok{|}\NormalTok{ (}\DecValTok{6}\SpecialCharTok{\textless{}}\DecValTok{3}\NormalTok{) }\CommentTok{\# loogiline või}
\end{Highlighting}
\end{Shaded}

\begin{verbatim}
## [1] TRUE
\end{verbatim}

\begin{Shaded}
\begin{Highlighting}[]
\NormalTok{(}\DecValTok{3}\SpecialCharTok{\textless{}}\DecValTok{6}\NormalTok{) }\SpecialCharTok{\&}\NormalTok{ (}\DecValTok{6}\SpecialCharTok{\textless{}}\DecValTok{3}\NormalTok{) }\CommentTok{\# loogiline ja}
\end{Highlighting}
\end{Shaded}

\begin{verbatim}
## [1] FALSE
\end{verbatim}

\textbf{Andmeobjektid}

R töötab andmeobjektidega ehk andmetega, mis on salvestatud mingisse objekti. Andmeobjekt võib sisaldada üksikut numbrilist väärtust (või ka näiteks sõna), aga ka mitut üksiväärtus koondavat vektorit (seega tunnus) või hoopis mitut andmevektorit koondavat andmestikku.\\
Andmeobjektile väärtuse omistamine toimub ``\textless-'' märgiga.

\begin{Shaded}
\begin{Highlighting}[]
\NormalTok{x }\OtherTok{\textless{}{-}} \DecValTok{1}
\end{Highlighting}
\end{Shaded}

Omistasime objektile \emph{x} väärtuse 1. Nüüd käsitleb R \emph{x}'i kui 1'te ja me saame sellega näiteks tehteid teha.

\begin{Shaded}
\begin{Highlighting}[]
\NormalTok{x }\SpecialCharTok{+} \DecValTok{1}
\end{Highlighting}
\end{Shaded}

\begin{verbatim}
## [1] 2
\end{verbatim}

\begin{Shaded}
\begin{Highlighting}[]
\NormalTok{a }\OtherTok{\textless{}{-}} \DecValTok{10}
\NormalTok{a}
\end{Highlighting}
\end{Shaded}

\begin{verbatim}
## [1] 10
\end{verbatim}

\begin{Shaded}
\begin{Highlighting}[]
\NormalTok{a }\OtherTok{\textless{}{-}}\NormalTok{ a }\SpecialCharTok{+} \DecValTok{10} \CommentTok{\# kirjutame algse a üle ja omistame talle uue väärtuse}
\NormalTok{a}
\end{Highlighting}
\end{Shaded}

\begin{verbatim}
## [1] 20
\end{verbatim}

Andmeobjektide nimed ei tohi sisaldada tehtemärke või tühikut ega alata numbriga. Sõnu võib vajadusel eristada näiteks punkti või alakriipsuga: a.1, a\_1 . Ei ole ka soovitatav kasutada ka täpitähti (kuigi üldjuhul R neid tunnistab).\\
R eristab suuri ja väikeseid tähti. R ei võrdu r'iga.

Andmeobjektid võivad sisaldada ka sõnu, lauseid või terveid lõike.

\begin{Shaded}
\begin{Highlighting}[]
\NormalTok{linn }\OtherTok{\textless{}{-}} \StringTok{"Tallinn"}
\NormalTok{linn}
\end{Highlighting}
\end{Shaded}

\begin{verbatim}
## [1] "Tallinn"
\end{verbatim}

\begin{Shaded}
\begin{Highlighting}[]
\NormalTok{kool }\OtherTok{\textless{}{-}} \StringTok{"Tallinna Ülikool"}
\NormalTok{kool}
\end{Highlighting}
\end{Shaded}

\begin{verbatim}
## [1] "Tallinna Ülikool"
\end{verbatim}

Jutumärgid annavad R'ile teada, et tegemist on tekstiga ja mitte teise andmeobjekti või funktsiooniga.

\textbf{Loogilised andmed (TRUE või FALSE)}

\begin{Shaded}
\begin{Highlighting}[]
\NormalTok{a }\OtherTok{\textless{}{-}} \ConstantTok{TRUE}
\NormalTok{a}
\end{Highlighting}
\end{Shaded}

\begin{verbatim}
## [1] TRUE
\end{verbatim}

\begin{Shaded}
\begin{Highlighting}[]
\NormalTok{b }\OtherTok{\textless{}{-}} \ConstantTok{FALSE}
\NormalTok{b}
\end{Highlighting}
\end{Shaded}

\begin{verbatim}
## [1] FALSE
\end{verbatim}

Loogilisi väärtused on tulemuseks loogilistele tehetele.

\begin{Shaded}
\begin{Highlighting}[]
\NormalTok{c }\OtherTok{\textless{}{-}} \DecValTok{3} \SpecialCharTok{\textgreater{}} \DecValTok{2}
\NormalTok{c}
\end{Highlighting}
\end{Shaded}

\begin{verbatim}
## [1] TRUE
\end{verbatim}

R käsitleb loogilisi väärtusi sisemiselt 1 ja 0'ina, seega saame ka nendega tehteid teha:

\begin{Shaded}
\begin{Highlighting}[]
\NormalTok{a}
\end{Highlighting}
\end{Shaded}

\begin{verbatim}
## [1] TRUE
\end{verbatim}

\begin{Shaded}
\begin{Highlighting}[]
\NormalTok{b}
\end{Highlighting}
\end{Shaded}

\begin{verbatim}
## [1] FALSE
\end{verbatim}

\begin{Shaded}
\begin{Highlighting}[]
\NormalTok{a }\SpecialCharTok{+}\NormalTok{ b}
\end{Highlighting}
\end{Shaded}

\begin{verbatim}
## [1] 1
\end{verbatim}

\textbf{Puuduvad väärtused}

Puuduvate väärtuste jaoks on tähis NA.

\begin{Shaded}
\begin{Highlighting}[]
\NormalTok{c }\OtherTok{\textless{}{-}} \ConstantTok{NA}
\NormalTok{c}
\end{Highlighting}
\end{Shaded}

\begin{verbatim}
## [1] NA
\end{verbatim}

\begin{Shaded}
\begin{Highlighting}[]
\NormalTok{a}
\end{Highlighting}
\end{Shaded}

\begin{verbatim}
## [1] TRUE
\end{verbatim}

\begin{Shaded}
\begin{Highlighting}[]
\NormalTok{c }\SpecialCharTok{+}\NormalTok{ a}
\end{Highlighting}
\end{Shaded}

\begin{verbatim}
## [1] NA
\end{verbatim}

Miks on tulemuseks \emph{NA}? Kui me liidame mingi arvu millelegi, mida me ei tea, siis me ju ei tea ka vastust.

Ülesanne!

\begin{enumerate}
\def\labelenumi{\arabic{enumi}.}
\tightlist
\item
  tehke andmeobjekt \emph{synniaasta}, mille väärtuseks on Teie sünniaasta
\item
  tehke andmeobjekt \emph{aasta}, mille väärtuseks on 2017
\item
  arvutage kui vana te olete?
\item
  kontrollige looglise tehtega, kas Teie sünniaasta ikka on väiksem kui 2019
\item
  tehke andmeobjekt \emph{nimi}, mille väärtuseks on Teie nimi
\end{enumerate}

\textbf{Funktsioonid}

Enamik toimingutest toimub R'is funktsioonide abil.

\begin{Shaded}
\begin{Highlighting}[]
\FunctionTok{sqrt}\NormalTok{(}\DecValTok{4}\NormalTok{) }\CommentTok{\# ruutjuure funktsioon}
\end{Highlighting}
\end{Shaded}

\begin{verbatim}
## [1] 2
\end{verbatim}

Funktsioonile järgnevad alati sulud, milles tuleb määrata funktsiooni argument (antud juhul 4, ehk number millest tahame ruutjuurt võtta). Argumente võib olla ka rohkem kui üks (ja üldjuhul ongi). Sellisel juhul on nad eraldatud komaga. Funktsioonil \texttt{log()} on kaks argumenti: \texttt{x}, ehk arv millest me tahame logaritmi võtta ja \texttt{base} ehk logaritmi alus.

\begin{Shaded}
\begin{Highlighting}[]
\FunctionTok{log}\NormalTok{(}\AttributeTok{x =} \DecValTok{100}\NormalTok{, }\AttributeTok{base =} \DecValTok{10}\NormalTok{)}
\end{Highlighting}
\end{Shaded}

\begin{verbatim}
## [1] 2
\end{verbatim}

Kui me teame argumentide järjekorda, siis me ei pea nende tähiseid eksplitsiitselt välja kirjutama.

\begin{Shaded}
\begin{Highlighting}[]
\FunctionTok{log}\NormalTok{(}\DecValTok{100}\NormalTok{, }\DecValTok{10}\NormalTok{)}
\end{Highlighting}
\end{Shaded}

\begin{verbatim}
## [1] 2
\end{verbatim}

Osadel argumentidel on vaikeväärtused ( näit \texttt{log()} funktsiooni puhul \texttt{base} argumendi vaikeväärtuseks \emph{e} ehk 2.7) Kui me vaikeväärtusega argumenti välja ei kirjuta, kasutatakse vaikeväärtust.

\begin{Shaded}
\begin{Highlighting}[]
\FunctionTok{log}\NormalTok{(}\AttributeTok{x=}\DecValTok{100}\NormalTok{)}
\end{Highlighting}
\end{Shaded}

\begin{verbatim}
## [1] 4.60517
\end{verbatim}

See kehtib muidugi ainult sellisel juhul kui teisel argumendil on vaikeväärtus. Kuidas me seda aga teadma peaksime? Kõige lihtsam on vaadata funktsiooni abilehte, kus on kõik selle argumendid loetletud.

\textbf{Kuidas abi saada?}

Iga funktsiooni kohta on R'is abileht millele pääseb ligi kirjutade funktsiooni nime ette \texttt{?}.

\begin{Shaded}
\begin{Highlighting}[]
\NormalTok{?log}
\FunctionTok{help}\NormalTok{(log) }\CommentTok{\# saab ka nii}
\end{Highlighting}
\end{Shaded}

Kui on spetsiifilisemad probleemid, saab alati googeldada. Erinevaid materjale, tutoriale, foorumeid jne on väga palju.

\textbf{Vektorid}

Üldiselt ei tööta me üksikväärtustega (skalaaridega) vaid andmejadade ehk vektoritega. Vektori loomine ehk mitmest väärtusest andmeobjekti loomine käib \texttt{c()} funktsiooniga.

\begin{Shaded}
\begin{Highlighting}[]
\NormalTok{see }\OtherTok{\textless{}{-}} \FunctionTok{c}\NormalTok{(}\DecValTok{3}\NormalTok{,}\DecValTok{6}\NormalTok{,}\DecValTok{1}\NormalTok{,}\DecValTok{4}\NormalTok{,}\DecValTok{10}\NormalTok{)}
\NormalTok{see}
\end{Highlighting}
\end{Shaded}

\begin{verbatim}
## [1]  3  6  1  4 10
\end{verbatim}

\begin{Shaded}
\begin{Highlighting}[]
\NormalTok{too }\OtherTok{\textless{}{-}} \FunctionTok{c}\NormalTok{(}\StringTok{"a"}\NormalTok{, }\StringTok{"b"}\NormalTok{, }\StringTok{"c"}\NormalTok{) }\CommentTok{\# vektor võib sisaldada ka teksti}
\NormalTok{too}
\end{Highlighting}
\end{Shaded}

\begin{verbatim}
## [1] "a" "b" "c"
\end{verbatim}

Sarnaselt üksikväärtustega saab ka vektoritega tehteid teha.

\begin{Shaded}
\begin{Highlighting}[]
\NormalTok{see}
\end{Highlighting}
\end{Shaded}

\begin{verbatim}
## [1]  3  6  1  4 10
\end{verbatim}

\begin{Shaded}
\begin{Highlighting}[]
\NormalTok{too }\OtherTok{\textless{}{-}}\NormalTok{ see }\SpecialCharTok{*} \DecValTok{2}
\NormalTok{too}
\end{Highlighting}
\end{Shaded}

\begin{verbatim}
## [1]  6 12  2  8 20
\end{verbatim}

\begin{Shaded}
\begin{Highlighting}[]
\NormalTok{see }\SpecialCharTok{+}\NormalTok{ too}
\end{Highlighting}
\end{Shaded}

\begin{verbatim}
## [1]  9 18  3 12 30
\end{verbatim}

Kui üks vektor on teisest lühem, siis R taaskasutab lühema vektori väärtusi.

\begin{Shaded}
\begin{Highlighting}[]
\NormalTok{pikk }\OtherTok{\textless{}{-}} \FunctionTok{c}\NormalTok{(}\DecValTok{1}\NormalTok{,}\DecValTok{1}\NormalTok{,}\DecValTok{2}\NormalTok{,}\DecValTok{2}\NormalTok{,}\DecValTok{3}\NormalTok{,}\DecValTok{3}\NormalTok{)}
\NormalTok{lyhike }\OtherTok{\textless{}{-}} \FunctionTok{c}\NormalTok{(}\DecValTok{10}\NormalTok{, }\DecValTok{100}\NormalTok{)}
\NormalTok{pikk }\SpecialCharTok{*}\NormalTok{ lyhike}
\end{Highlighting}
\end{Shaded}

\begin{verbatim}
## [1]  10 100  20 200  30 300
\end{verbatim}

See juhtub ka siis, kui lühem vektor ei ole pikema täisarvuline jagatis:

\begin{Shaded}
\begin{Highlighting}[]
\NormalTok{pikk }\OtherTok{\textless{}{-}} \FunctionTok{c}\NormalTok{(}\DecValTok{1}\NormalTok{,}\DecValTok{1}\NormalTok{,}\DecValTok{2}\NormalTok{,}\DecValTok{2}\NormalTok{,}\DecValTok{3}\NormalTok{,}\DecValTok{3}\NormalTok{, }\DecValTok{4}\NormalTok{)}
\NormalTok{lyhike }\OtherTok{\textless{}{-}} \FunctionTok{c}\NormalTok{(}\DecValTok{10}\NormalTok{, }\DecValTok{100}\NormalTok{)}
\NormalTok{pikk }\SpecialCharTok{*}\NormalTok{ lyhike}
\end{Highlighting}
\end{Shaded}

\begin{verbatim}
## [1]  10 100  20 200  30 300  40
\end{verbatim}

Üldiselt me sellist asja teha ei taha. Õnneks antakse taolisest olukorrast meile ka hoiatusteatega märku. Aga miks taoline vektoriseeritus üldse vajalik peaks olema? Asi on selles, et R käsitleb kõiki objekte vektoritena. Ka üksik number on vektor, mille pikkus on 1. Ehk siis just vektoriseeritus võimaldab meil teha nii:

\begin{Shaded}
\begin{Highlighting}[]
\FunctionTok{c}\NormalTok{(}\DecValTok{1}\NormalTok{,}\DecValTok{2}\NormalTok{,}\DecValTok{3}\NormalTok{,}\DecValTok{4}\NormalTok{) }\SpecialCharTok{*} \DecValTok{2}
\end{Highlighting}
\end{Shaded}

\begin{verbatim}
## [1] 2 4 6 8
\end{verbatim}

Ja midagi taolist tahame me teha päris tihti.

Vektori elementidele saab ka nimesi anda (ja üksikväärtustele muidugi ka)

\begin{Shaded}
\begin{Highlighting}[]
\NormalTok{see2 }\OtherTok{\textless{}{-}} \FunctionTok{c}\NormalTok{(}\AttributeTok{a=}\DecValTok{1}\NormalTok{, }\AttributeTok{b=}\DecValTok{2}\NormalTok{, }\AttributeTok{c=}\DecValTok{3}\NormalTok{)}
\NormalTok{see2}
\end{Highlighting}
\end{Shaded}

\begin{verbatim}
## a b c 
## 1 2 3
\end{verbatim}

\textbf{Indekseerimine}

Kuidas üksikuid väärtusi vektorist kätte saada? Neile saab ligi kasutades {[} {]} funktsiooni koos soovitava väärtuse indeksiga (positsiooninumbriga).

\begin{Shaded}
\begin{Highlighting}[]
\NormalTok{see }\OtherTok{\textless{}{-}} \FunctionTok{c}\NormalTok{(}\DecValTok{3}\NormalTok{,}\DecValTok{6}\NormalTok{,}\DecValTok{1}\NormalTok{,}\DecValTok{4}\NormalTok{,}\DecValTok{10}\NormalTok{)}
\NormalTok{see[}\DecValTok{2}\NormalTok{] }\CommentTok{\# tahame teada vektori teist väärtust}
\end{Highlighting}
\end{Shaded}

\begin{verbatim}
## [1] 6
\end{verbatim}

\begin{Shaded}
\begin{Highlighting}[]
\NormalTok{a }\OtherTok{\textless{}{-}}\NormalTok{ see[}\DecValTok{2}\NormalTok{] }\CommentTok{\# tahame selle kirjutada uude andmeobjekti}
\NormalTok{a}
\end{Highlighting}
\end{Shaded}

\begin{verbatim}
## [1] 6
\end{verbatim}

\begin{Shaded}
\begin{Highlighting}[]
\NormalTok{see[}\DecValTok{3}\SpecialCharTok{:}\DecValTok{4}\NormalTok{] }\CommentTok{\# saame välja võtta mitu väärtust. ":" on "kuni" märk (teine kuni kolmas positsioon)}
\end{Highlighting}
\end{Shaded}

\begin{verbatim}
## [1] 1 4
\end{verbatim}

Andmeobjekti elemnte saab ka välja jätta.

\begin{Shaded}
\begin{Highlighting}[]
\NormalTok{see[}\SpecialCharTok{{-}}\DecValTok{2}\NormalTok{] }\CommentTok{\# kõik elemendid välja arvatud teine}
\end{Highlighting}
\end{Shaded}

\begin{verbatim}
## [1]  3  1  4 10
\end{verbatim}

Indekseerida saab ka nimega.

\begin{Shaded}
\begin{Highlighting}[]
\NormalTok{see2 }\OtherTok{\textless{}{-}} \FunctionTok{c}\NormalTok{(}\AttributeTok{a=}\DecValTok{1}\NormalTok{, }\AttributeTok{b=}\DecValTok{2}\NormalTok{, }\AttributeTok{c=}\DecValTok{3}\NormalTok{) }\CommentTok{\# teeme nimedega vektori}
\NormalTok{see2[}\StringTok{"a"}\NormalTok{] }\CommentTok{\# jällegi peame kasutama jutumärke, kuna ei viita mitte andmeobjektile, }
\end{Highlighting}
\end{Shaded}

\begin{verbatim}
## a 
## 1
\end{verbatim}

\begin{Shaded}
\begin{Highlighting}[]
          \CommentTok{\# vaid selle väärtusele}
\end{Highlighting}
\end{Shaded}

Kui tahame nimega indekseerida mitut väärtust, peame kasutama indeksite vektorit, mille teeme \texttt{c()} funktsiooniga. Viitame indeksite vektoriga andmevektorile.

\begin{Shaded}
\begin{Highlighting}[]
\NormalTok{see2}
\end{Highlighting}
\end{Shaded}

\begin{verbatim}
## a b c 
## 1 2 3
\end{verbatim}

\begin{Shaded}
\begin{Highlighting}[]
\NormalTok{see2[}\FunctionTok{c}\NormalTok{(}\StringTok{"a"}\NormalTok{,}\StringTok{"b"}\NormalTok{)]}
\end{Highlighting}
\end{Shaded}

\begin{verbatim}
## a b 
## 1 2
\end{verbatim}

Viidata saab ka loogiliste tehete või loogiliste vektoritega. Saame loogilise tehtega luua loogilise vektori, mida siis saab kasutada väärtuste väljavõtmiseks.

\begin{Shaded}
\begin{Highlighting}[]
\NormalTok{see }\OtherTok{\textless{}{-}} \FunctionTok{c}\NormalTok{(}\DecValTok{3}\NormalTok{,}\DecValTok{6}\NormalTok{,}\DecValTok{1}\NormalTok{,}\DecValTok{4}\NormalTok{,}\DecValTok{10}\NormalTok{)}
\NormalTok{see }\SpecialCharTok{\textgreater{}} \DecValTok{5} \CommentTok{\# loogiline vektor}
\end{Highlighting}
\end{Shaded}

\begin{verbatim}
## [1] FALSE  TRUE FALSE FALSE  TRUE
\end{verbatim}

\begin{Shaded}
\begin{Highlighting}[]
\NormalTok{see[see }\SpecialCharTok{\textgreater{}} \DecValTok{5}\NormalTok{] }\CommentTok{\# kasutame loogilist vektorit indekseerimiseks}
\end{Highlighting}
\end{Shaded}

\begin{verbatim}
## [1]  6 10
\end{verbatim}

\begin{Shaded}
\begin{Highlighting}[]
\CommentTok{\# see[c(F,T,F,F,T)] \# kui me kirjutaks loogilise vektori välja}
\end{Highlighting}
\end{Shaded}

Ülesanne!

\begin{enumerate}
\def\labelenumi{\arabic{enumi}.}
\tightlist
\item
  looge vektor \emph{a} milles sisalduvad numbrid 2 8 3 6 7
\item
  looge vektor \emph{b} milles sisalduvad numbrid 3 4 5 7 2
\item
  looge vektor \emph{c}, mis on kahe eelmise summa
\item
  looge andmeobjekt \emph{d}, mis sisaldab \emph{c} esimest väärtust
\item
  looge vektor \emph{e}, kus on \emph{c} väärtuseid, mis on suuremad kui 10
\end{enumerate}

\textbf{Maatriksid}

Andmevektoreid saab omakorda ühendada.

\begin{Shaded}
\begin{Highlighting}[]
\NormalTok{see }\OtherTok{\textless{}{-}} \FunctionTok{c}\NormalTok{(}\DecValTok{12}\NormalTok{,}\DecValTok{5}\NormalTok{)}
\NormalTok{too }\OtherTok{\textless{}{-}} \FunctionTok{c}\NormalTok{(}\DecValTok{6}\NormalTok{,}\DecValTok{9}\NormalTok{)}
\NormalTok{loo }\OtherTok{\textless{}{-}} \FunctionTok{cbind}\NormalTok{(see, too) }\CommentTok{\# cbind ühendab vektorid veergude kaupa, }
                       \CommentTok{\# ridade kaupa ühendamiseks on funktsioon rbind()}
\NormalTok{loo}
\end{Highlighting}
\end{Shaded}

\begin{verbatim}
##      see too
## [1,]  12   6
## [2,]   5   9
\end{verbatim}

Tulemuseks on uus andmeobjekt, mis kuulub klassi \emph{matrix}.

Saame ka \texttt{matrix()} funktsiooniga maatrikseid teha:

\begin{Shaded}
\begin{Highlighting}[]
\NormalTok{loo }\OtherTok{\textless{}{-}} \FunctionTok{matrix}\NormalTok{(}\FunctionTok{c}\NormalTok{(}\DecValTok{12}\NormalTok{,}\DecValTok{5}\NormalTok{,}\DecValTok{6}\NormalTok{,}\DecValTok{9}\NormalTok{), }\AttributeTok{nrow =} \DecValTok{2}\NormalTok{, }\AttributeTok{ncol =} \DecValTok{2}\NormalTok{, }\AttributeTok{byrow =}\NormalTok{ T)}
\NormalTok{loo}
\end{Highlighting}
\end{Shaded}

\begin{verbatim}
##      [,1] [,2]
## [1,]   12    5
## [2,]    6    9
\end{verbatim}

Maatriksi veergudele ja ridadele saame nimesid anda:

\begin{Shaded}
\begin{Highlighting}[]
\FunctionTok{colnames}\NormalTok{(loo) }\OtherTok{\textless{}{-}} \FunctionTok{c}\NormalTok{(}\StringTok{"esimene\_veerg"}\NormalTok{, }\StringTok{"teine\_veerg"}\NormalTok{)}
\FunctionTok{rownames}\NormalTok{(loo) }\OtherTok{\textless{}{-}} \FunctionTok{c}\NormalTok{(}\StringTok{"esimene\_rida"}\NormalTok{, }\StringTok{"teine\_rida"}\NormalTok{)}
\NormalTok{loo}
\end{Highlighting}
\end{Shaded}

\begin{verbatim}
##              esimene_veerg teine_veerg
## esimene_rida            12           5
## teine_rida               6           9
\end{verbatim}

\textbf{Andmeobjektide klassid}

Erinevatel andmeobjektidel on erinevad klassid. Klassid tulenevad sellest, millist tüüpi andmed selles andmeobjektis on (numbrilised, tekstilised, loogilised jne).

\begin{Shaded}
\begin{Highlighting}[]
\FunctionTok{class}\NormalTok{(see)}
\end{Highlighting}
\end{Shaded}

\begin{verbatim}
## [1] "numeric"
\end{verbatim}

\begin{Shaded}
\begin{Highlighting}[]
\FunctionTok{class}\NormalTok{(loo)}
\end{Highlighting}
\end{Shaded}

\begin{verbatim}
## [1] "matrix" "array"
\end{verbatim}

Enamikes andmeobjektides saab olla vaid ühte tüüpi elemente. Kui numbrilises vektoris on näiteks üks tekstiline väärtus, siis arvestab R seda vektorit kui tekstilist (kuna numrit on võimalik tekstiliseks teha, kuid teksti numbriks mitte). Näiteks maatriksis võivad olla vaid numbrilised väärtused.

Üks andmeobjekt on siinkohal erandlik. Selleks on ``list'', kus võib korraga olla erinevat tüüpi andmeid.

\begin{Shaded}
\begin{Highlighting}[]
\NormalTok{x }\OtherTok{\textless{}{-}} \FunctionTok{list}\NormalTok{(}\DecValTok{1}\NormalTok{, }\FunctionTok{c}\NormalTok{(}\StringTok{"b"}\NormalTok{, }\StringTok{"d"}\NormalTok{))}
\NormalTok{x}
\end{Highlighting}
\end{Shaded}

\begin{verbatim}
## [[1]]
## [1] 1
## 
## [[2]]
## [1] "b" "d"
\end{verbatim}

Ülesanne!

\begin{enumerate}
\def\labelenumi{\arabic{enumi}.}
\tightlist
\item
  teil on andmevektorid \emph{a, b, c}. Ühendage need veergupidi andmestikuks \emph{koos}
\item
  mis on selle andmeobjekti klass?
\end{enumerate}

\textbf{Data frame (andmestik)}

Kui meil on mingi andmestik, siis üldjuhul on seal erinevat liiki tunnuseid, nii arvtunnuseid kui kategoriaalseid ehk faktortunnuseid jne. Sellise andmebaasi jaoks on R'is eraldi andmeobjekti formaat - \emph{data.frame}.\\
\emph{data.frame} on iseenesest list, aga omapärane selles mõttes, et tema read peavad olema ühepikkused ja veerud peavad olema ühepikkused. Põhimõtteliselt on data.frame siis selline andmeobjekt, kus ridadeks on vaatlused ja veergudeks tunnused.

\begin{Shaded}
\begin{Highlighting}[]
\NormalTok{nimi }\OtherTok{\textless{}{-}} \FunctionTok{c}\NormalTok{(}\StringTok{"Jaan"}\NormalTok{, }\StringTok{"Mari"}\NormalTok{, }\StringTok{"Kadri"}\NormalTok{, }\StringTok{"Mati"}\NormalTok{)}
\NormalTok{vanus }\OtherTok{\textless{}{-}} \FunctionTok{c}\NormalTok{(}\DecValTok{29}\NormalTok{, }\DecValTok{42}\NormalTok{, }\DecValTok{35}\NormalTok{, }\DecValTok{52}\NormalTok{)}
\NormalTok{hinnang }\OtherTok{\textless{}{-}} \FunctionTok{c}\NormalTok{(}\FloatTok{1.438}\NormalTok{, }\FloatTok{2.763}\NormalTok{, }\FloatTok{1.548}\NormalTok{, }\DecValTok{2}\NormalTok{)}
\NormalTok{see }\OtherTok{\textless{}{-}} \FunctionTok{data.frame}\NormalTok{(nimi, vanus, hinnang) }\CommentTok{\# ühendame tunnused andmestikuks}
\NormalTok{see}
\end{Highlighting}
\end{Shaded}

\begin{verbatim}
##    nimi vanus hinnang
## 1  Jaan    29   1.438
## 2  Mari    42   2.763
## 3 Kadri    35   1.548
## 4  Mati    52   2.000
\end{verbatim}

\begin{Shaded}
\begin{Highlighting}[]
\CommentTok{\# saab ka nii (siin peame kasutama võrdusmärki)}
\NormalTok{see }\OtherTok{\textless{}{-}} \FunctionTok{data.frame}\NormalTok{(}\AttributeTok{nimi =} \FunctionTok{c}\NormalTok{(}\StringTok{"Jaan"}\NormalTok{, }\StringTok{"Mari"}\NormalTok{, }\StringTok{"Kadri"}\NormalTok{, }\StringTok{"Mati"}\NormalTok{),}
                  \AttributeTok{vanus =} \FunctionTok{c}\NormalTok{(}\DecValTok{29}\NormalTok{, }\DecValTok{42}\NormalTok{, }\DecValTok{35}\NormalTok{, }\DecValTok{52}\NormalTok{),}
                  \AttributeTok{hinnang =} \FunctionTok{c}\NormalTok{(}\FloatTok{1.438}\NormalTok{, }\FloatTok{2.763}\NormalTok{, }\FloatTok{1.548}\NormalTok{, }\DecValTok{2}\NormalTok{))}
\end{Highlighting}
\end{Shaded}

Kui meil juba on mingi andmetabel, näiteks maatriks, saame selle muuta data.frameiks funktsiooniga \texttt{as.data.frame()}

\begin{Shaded}
\begin{Highlighting}[]
\NormalTok{loo }\OtherTok{\textless{}{-}} \FunctionTok{cbind}\NormalTok{(}\FunctionTok{c}\NormalTok{(}\DecValTok{12}\NormalTok{,}\DecValTok{5}\NormalTok{), }\FunctionTok{c}\NormalTok{(}\DecValTok{6}\NormalTok{,}\DecValTok{9}\NormalTok{))}

\FunctionTok{as.data.frame}\NormalTok{(loo)}
\end{Highlighting}
\end{Shaded}

Sarnaselt vektoritele saame indekseerida ka data.frame'i (maatrikseid samuti). Kuna aga data.frame on kahedimensionaalne, peame kasutama kahte indeksit. Esiteks rea ja teiseks veeru indeks.
Tahame teada Kadri vanust, seega 3 rida ja 2 veerg:

\begin{Shaded}
\begin{Highlighting}[]
\NormalTok{see[}\DecValTok{2}\NormalTok{,}\DecValTok{3}\NormalTok{]}
\end{Highlighting}
\end{Shaded}

\begin{verbatim}
## [1] 2.763
\end{verbatim}

Kui jätame veeru koha tühjaks, valitakse kõik veerud.

\begin{Shaded}
\begin{Highlighting}[]
\NormalTok{see[}\DecValTok{2}\NormalTok{,]}
\end{Highlighting}
\end{Shaded}

\begin{verbatim}
##   nimi vanus hinnang
## 2 Mari    42   2.763
\end{verbatim}

Kui jätame rea koha tühjaks, valitakse kõik read

\begin{Shaded}
\begin{Highlighting}[]
\NormalTok{see[,}\DecValTok{3}\NormalTok{]}
\end{Highlighting}
\end{Shaded}

\begin{verbatim}
## [1] 1.438 2.763 1.548 2.000
\end{verbatim}

Kui tahame valida mingit tunnust (veergu) siis võime kasutada selle numbrilist või nimelist indeksit

\begin{Shaded}
\begin{Highlighting}[]
\NormalTok{see[,}\StringTok{"nimi"}\NormalTok{] }\CommentTok{\# tahame nime veeru kõiki ridu, seega jätame indekseerimisel rea koha tühjaks }
\end{Highlighting}
\end{Shaded}

\begin{verbatim}
## [1] "Jaan"  "Mari"  "Kadri" "Mati"
\end{verbatim}

Võime valida ka mitu veergu või rida korraga.

\begin{Shaded}
\begin{Highlighting}[]
\NormalTok{see[}\DecValTok{1}\SpecialCharTok{:}\DecValTok{2}\NormalTok{,}\DecValTok{1}\SpecialCharTok{:}\DecValTok{2}\NormalTok{]}
\end{Highlighting}
\end{Shaded}

\begin{verbatim}
##   nimi vanus
## 1 Jaan    29
## 2 Mari    42
\end{verbatim}

Seega saame valida ainult mingi, meile vajaliku osa datasetist

Teine viis veeru ehk tunnuse valimiseks on \$ märk

\begin{Shaded}
\begin{Highlighting}[]
\NormalTok{see}\SpecialCharTok{$}\NormalTok{nimi}
\end{Highlighting}
\end{Shaded}

\begin{verbatim}
## [1] "Jaan"  "Mari"  "Kadri" "Mati"
\end{verbatim}

Ülesanne!

\begin{enumerate}
\def\labelenumi{\arabic{enumi}.}
\tightlist
\item
  Teil on andmestik \emph{koos}. Tehke see data.frameiks
\item
  Võtke sealt välja esimene rida
\item
  Võtke sealt välja veerg \emph{c} ja salvestage see eraldi andmeobjektina
\item
  Kasutades alternatiivset viisi, võtke välja veerg \emph{b}
\end{enumerate}

\textbf{Data frame'i modifitseerimine}

Data.frame'i väärtuste muutmisel saame jälle indekseid kasutada:

\begin{Shaded}
\begin{Highlighting}[]
\NormalTok{see}\SpecialCharTok{$}\NormalTok{hinnang[}\DecValTok{1}\NormalTok{] }\OtherTok{\textless{}{-}} \DecValTok{1} \CommentTok{\# muudame hinnangu tunnuse esimese väärtuse 1{-}ks}
\NormalTok{see[}\DecValTok{1}\NormalTok{, }\DecValTok{3}\NormalTok{] }\OtherTok{\textless{}{-}} \DecValTok{1} \CommentTok{\# sama mis eelmine}
\NormalTok{see}
\end{Highlighting}
\end{Shaded}

\begin{verbatim}
##    nimi vanus hinnang
## 1  Jaan    29   1.000
## 2  Mari    42   2.763
## 3 Kadri    35   1.548
## 4  Mati    52   2.000
\end{verbatim}

Data.frame'i uute tunnuste lisamine:

\begin{Shaded}
\begin{Highlighting}[]
\NormalTok{see}\SpecialCharTok{$}\NormalTok{rahulolu }\OtherTok{\textless{}{-}} \FunctionTok{c}\NormalTok{(}\DecValTok{2}\NormalTok{, }\DecValTok{4}\NormalTok{, }\DecValTok{3}\NormalTok{, }\DecValTok{5}\NormalTok{)}
\NormalTok{see}\SpecialCharTok{$}\NormalTok{sugu }\OtherTok{\textless{}{-}} \FunctionTok{c}\NormalTok{(}\StringTok{"m"}\NormalTok{, }\StringTok{"n"}\NormalTok{, }\StringTok{"n"}\NormalTok{, }\StringTok{"m"}\NormalTok{)}
\NormalTok{see}
\end{Highlighting}
\end{Shaded}

\begin{verbatim}
##    nimi vanus hinnang rahulolu sugu
## 1  Jaan    29   1.000        2    m
## 2  Mari    42   2.763        4    n
## 3 Kadri    35   1.548        3    n
## 4  Mati    52   2.000        5    m
\end{verbatim}

Data.frame'i tunnuste kustutamine:

\begin{Shaded}
\begin{Highlighting}[]
\NormalTok{see}\SpecialCharTok{$}\NormalTok{rahulolu }\OtherTok{\textless{}{-}} \ConstantTok{NULL}
\NormalTok{see}
\end{Highlighting}
\end{Shaded}

\begin{verbatim}
##    nimi vanus hinnang sugu
## 1  Jaan    29   1.000    m
## 2  Mari    42   2.763    n
## 3 Kadri    35   1.548    n
## 4  Mati    52   2.000    m
\end{verbatim}

Saame ridade valimiseks (indekseerimiseks) kasutada loogilisi tehteid ja seega välja võtta just need vaatlused mida vajame.

\begin{Shaded}
\begin{Highlighting}[]
\NormalTok{see[see}\SpecialCharTok{$}\NormalTok{vanus }\SpecialCharTok{\textless{}} \DecValTok{40}\NormalTok{, ]}
\end{Highlighting}
\end{Shaded}

\begin{verbatim}
##    nimi vanus hinnang sugu
## 1  Jaan    29   1.000    m
## 3 Kadri    35   1.548    n
\end{verbatim}

Samal ajal saame saame võtta ka ainult vajalikud veerud:

\begin{Shaded}
\begin{Highlighting}[]
\NormalTok{see[see}\SpecialCharTok{$}\NormalTok{vanus }\SpecialCharTok{\textless{}} \DecValTok{40}\NormalTok{, }\FunctionTok{c}\NormalTok{(}\StringTok{"nimi"}\NormalTok{, }\StringTok{"vanus"}\NormalTok{)]}
\end{Highlighting}
\end{Shaded}

\begin{verbatim}
##    nimi vanus
## 1  Jaan    29
## 3 Kadri    35
\end{verbatim}

Nii saame teha andmestikust alamandmestiku, is vastab konkreetsetele tingimustele (subseti loomine):

\begin{Shaded}
\begin{Highlighting}[]
\NormalTok{uus }\OtherTok{\textless{}{-}}\NormalTok{ see[see}\SpecialCharTok{$}\NormalTok{sugu }\SpecialCharTok{==} \StringTok{"n"}\NormalTok{, }\FunctionTok{c}\NormalTok{(}\StringTok{"nimi"}\NormalTok{, }\StringTok{"hinnang"}\NormalTok{)]}
\NormalTok{uus}
\end{Highlighting}
\end{Shaded}

\begin{verbatim}
##    nimi hinnang
## 2  Mari   2.763
## 3 Kadri   1.548
\end{verbatim}

Selleks saab kasutada ka \texttt{subset()} funktsiooni:

\begin{Shaded}
\begin{Highlighting}[]
\FunctionTok{subset}\NormalTok{(see, sugu }\SpecialCharTok{==} \StringTok{"n"}\NormalTok{, }\AttributeTok{select =} \FunctionTok{c}\NormalTok{(}\StringTok{"nimi"}\NormalTok{, }\StringTok{"hinnang"}\NormalTok{))}
\end{Highlighting}
\end{Shaded}

\begin{verbatim}
##    nimi hinnang
## 2  Mari   2.763
## 3 Kadri   1.548
\end{verbatim}

Ülesanne!

\begin{enumerate}
\def\labelenumi{\arabic{enumi}.}
\tightlist
\item
  Lisage oma andmeobjektile \emph{koos} uus tunnus \emph{f}, milles sisalduvad tähed a i a i a
\item
  Looge uus data.frame, mis sisaldab tunnuseid \emph{a, b} ja ainult ridu, mille väärtus tunnuses \emph{f} on a
\end{enumerate}

\textbf{Faktorid}

Faktorid on R'i kategoriaalsed tunnused. Mõned meetodid vajavad sisendiks faktoreid. Võime tekstilise tunnuse (või ka numbrilise) muuta faktoriks funktsiooniga \texttt{as.factor()}.

\begin{Shaded}
\begin{Highlighting}[]
\NormalTok{x1 }\OtherTok{\textless{}{-}} \FunctionTok{as.factor}\NormalTok{(}\FunctionTok{c}\NormalTok{(}\StringTok{"punane"}\NormalTok{, }\StringTok{"roheline"}\NormalTok{, }\StringTok{"sinine"}\NormalTok{, }\StringTok{"sinine"}\NormalTok{))}
\NormalTok{x1}
\end{Highlighting}
\end{Shaded}

\begin{verbatim}
## [1] punane   roheline sinine   sinine  
## Levels: punane roheline sinine
\end{verbatim}

Aga faktortasemed on järjestatud tähestiku järgi. Üldjuhul on meil ikkagi mingi oma järjekord. Peaksime selle määrama nii:

\begin{Shaded}
\begin{Highlighting}[]
\NormalTok{x1 }\OtherTok{\textless{}{-}} \FunctionTok{c}\NormalTok{(}\StringTok{"punane"}\NormalTok{, }\StringTok{"roheline"}\NormalTok{, }\StringTok{"sinine"}\NormalTok{, }\StringTok{"sinine"}\NormalTok{)}
\NormalTok{x1 }\OtherTok{\textless{}{-}} \FunctionTok{factor}\NormalTok{(x1, }\AttributeTok{levels =} \FunctionTok{c}\NormalTok{(}\StringTok{"sinine"}\NormalTok{, }\StringTok{"roheline"}\NormalTok{, }\StringTok{"punane"}\NormalTok{))}
\NormalTok{x1}
\end{Highlighting}
\end{Shaded}

\begin{verbatim}
## [1] punane   roheline sinine   sinine  
## Levels: sinine roheline punane
\end{verbatim}

Mis aga juhtub kui me ühe taseme kogemata ära unustame:

\begin{Shaded}
\begin{Highlighting}[]
\NormalTok{x1 }\OtherTok{\textless{}{-}} \FunctionTok{c}\NormalTok{(}\StringTok{"punane"}\NormalTok{, }\StringTok{"roheline"}\NormalTok{, }\StringTok{"sinine"}\NormalTok{, }\StringTok{"sinine"}\NormalTok{)}
\NormalTok{x1 }\OtherTok{\textless{}{-}} \FunctionTok{factor}\NormalTok{(x1, }\AttributeTok{levels =} \FunctionTok{c}\NormalTok{(}\StringTok{"sinine"}\NormalTok{, }\StringTok{"roheline"}\NormalTok{))}
\NormalTok{x1}
\end{Highlighting}
\end{Shaded}

\begin{verbatim}
## [1] <NA>     roheline sinine   sinine  
## Levels: sinine roheline
\end{verbatim}

Seega faktorid võivad teatud kohtades natukene ohtlikud olla ning nende kasutamisel peab tähelepanelik olema.

Ülesanne!

\begin{enumerate}
\def\labelenumi{\arabic{enumi}.}
\tightlist
\item
  looge oma andmestikku \emph{koos} juurde faktortunnus \emph{g}, milles sisalduvad tähed r t r t r
\end{enumerate}

\textbf{Andmeobjektide kustutamine}

R jätab kõik konkreetse sessiooni ajal loodud või imporditud andmeobjektid mällu.
Andmeobjektide kustutamine käib funktsiooniga \texttt{rm()}.

\begin{Shaded}
\begin{Highlighting}[]
\FunctionTok{rm}\NormalTok{(x)}
\end{Highlighting}
\end{Shaded}

Kui tahame kustutada kõik mälus olevad andmeobjektid, siis \ldots{}

\begin{Shaded}
\begin{Highlighting}[]
\FunctionTok{rm}\NormalTok{(}\AttributeTok{list=}\FunctionTok{ls}\NormalTok{())}
\end{Highlighting}
\end{Shaded}

\textbf{R'i paketid}

Paljud funktsioonid on kaasas ``baas''R'iga. Lisaks neile on aga suur hulk funktsioone, mida on võimalik pakettidena juurde installida.\\
Paketid on kasutajate eneste poolt kirjutatud. Mõned neist on väga spetsiifilised, teised jällegi väga laialdaselt kasutatavad. Hetkel on ligi 13 700 paketti (kaks aastat tagasi oli neid veel 10 000).\\
Et paketti kasutada, tuleb see esmalt installida.

\begin{Shaded}
\begin{Highlighting}[]
\FunctionTok{install.packages}\NormalTok{(}\StringTok{"ggplot2"}\NormalTok{) }\CommentTok{\# jutumärgid on vajalikud}
\end{Highlighting}
\end{Shaded}

Kui pakett on installitud, tuleb see R'i mällu laadida (igaks sessiooniks uuesti).

\begin{Shaded}
\begin{Highlighting}[]
\FunctionTok{library}\NormalTok{(ggplot2) }\CommentTok{\# jutumärgid ei ole vajalikud}
\end{Highlighting}
\end{Shaded}

Miks peab enne igat sessiooni paketi uuesti laadima?\\
Kuna pakette on väga palju ja neis funktsioone veelgi rohkem, siis hakkavad funktsioonide nimed korduma. Et seda vältida, ongi mõistlik laadida vaid need paketid, mida konkreetse sessiooni ajal otseselt vaja on.\\
Funktsioonide nimed võivad kattuda isegi väheste laaditud pakettide korral. Sellisel juhul kasutab R viimati laetud paketi funktsiooni.

\hypertarget{andmetega-tuxf6uxf6tamine}{%
\section{Andmetega töötamine}\label{andmetega-tuxf6uxf6tamine}}

\textbf{Andmete sisselugemine}

Andmete sisselugemiseks on mitmeid erinevaid funktsioone, mille valik sõltub sellest mis formaadis meie andmed on.\\
Kõige mõistlikum viis andmeid hoida on .csv fail (\emph{comma separated value}). Näiteks Excelis saab andmetabeli csv'ks salvestada (save as). Samuti Statas, SPSS'is jne. Olenevalt sellest mida me numbri komakohana kasutame (``.'' või ``,''), saab csv faili laadida funktsiooniga \texttt{read.csv()} või \texttt{read.csv2}.

\begin{Shaded}
\begin{Highlighting}[]
\NormalTok{andmed }\OtherTok{\textless{}{-}} \FunctionTok{read.csv}\NormalTok{(}\StringTok{"C:/Users/Mina/Kvant analüüsi meetodid II (2019)/Andmed/andmed.csv"}\NormalTok{)}
\end{Highlighting}
\end{Shaded}

Kindlasti tuleb andmed kuhugi andmeobjekti (data.frame'i) sisse lugeda, muidu kuvatakse nad lihtsalt konsooli.\\
\emph{faili path} peab olema jutumärkides. Kaldkriipsud on teistpidi kui folderi käsujoonel.

Aegajalt juhtub, et loete täiesti korralikud andmed sisse, kuid kui neid R-is vaatate, siis on ü-de, ä-de ö-de või õ-de asemel mingid imelikud krõnksud. Sellisel puhul on üldjuhul tegemist \emph{encoding}'u probleemiga, st R ei saa aru kuidas arvutikeelt (see kuidas kõik tekstid ja andmed jne arvutisiseselt salvestatud on) inimkeelde tõlkida. Maailmas kasutatakse selleks erinevaid kodeeringuid, mis lähtuvad erinevatest tähestikest ja sümbolitest. Üks univesaalseid kodeeringuid, mis tuleb toime praktiliselt kõikvõimalike sümolite ja tähestikega on \emph{UTF-8}. Kui nüüd juhtub, et teie sissetõmmatud andmete puhul esineb taoline \emph{encoding}'u probleem, siis \texttt{read.csv()} funktsioonil on vastav parameeter, millega saate sobiliku \emph{encoding}'u määrata.

\begin{Shaded}
\begin{Highlighting}[]
\NormalTok{andmed }\OtherTok{\textless{}{-}} \FunctionTok{read.csv}\NormalTok{(}\StringTok{"C:/Users/Mina/Kvant analüüsi meetodid II (2019)/Andmed/andmed.csv"}\NormalTok{,}
                   \AttributeTok{fileEncoding =} \StringTok{"UTF{-}8"}\NormalTok{)}
\end{Highlighting}
\end{Shaded}

On ka üks mugav pakett \emph{csv} failide sissetõmbamiseks, kus see \emph{encoding} on juba automaatselt määratud - \emph{readr} ja selle funktsioon \texttt{read\_csv()} (või \texttt{read\_csv2()}).

\begin{Shaded}
\begin{Highlighting}[]
\FunctionTok{install.packages}\NormalTok{(}\StringTok{"readr"}\NormalTok{)}
\NormalTok{andmed }\OtherTok{\textless{}{-}} \FunctionTok{read\_csv}\NormalTok{(}\StringTok{"C:/Users/Mina/Kvant analüüsi meetodid II (2019)/Andmed/andmed.csv"}
\end{Highlighting}
\end{Shaded}

Kui andmed on näiteks SPSS faili kujul (ja teil ei ole SPSS'i, et neid ümber salvestada) aitab pakett \emph{haven} ja funktsioon \texttt{read\_spss()} (Stata faili puhul \texttt{read\_dta()}). Exceli faile saab sisse tõmmata paketi \emph{readxl} funktsiooniga \texttt{read\_excel()}.

\begin{Shaded}
\begin{Highlighting}[]
\FunctionTok{install.packages}\NormalTok{(}\StringTok{"haven"}\NormalTok{)}
\FunctionTok{install.packages}\NormalTok{(}\StringTok{"readxl"}\NormalTok{)}
\FunctionTok{library}\NormalTok{(haven)}
\NormalTok{andmed }\OtherTok{\textless{}{-}} \FunctionTok{read\_spss}\NormalTok{(}\StringTok{"andmed.sav"}\NormalTok{)}
\end{Highlighting}
\end{Shaded}

Viimase funktsiooniga ei kasutanud ma faili pathi. Kui me oleme määranud \emph{working directory}'ks ehk Ri konkreetse sessiooni töökataloogiks selle kataloogi, kus andmed parajasti on, siis ei ole seda vaja teha. \emph{working directory} saab määrata ka funktsiooniga \texttt{setwd()}.

\begin{Shaded}
\begin{Highlighting}[]
\FunctionTok{setwd}\NormalTok{(}\StringTok{"C:/Users/Mina/Kvant analüüsi meetodid II (2019)/Andmed/andmed.csv"}\NormalTok{)}
\end{Highlighting}
\end{Shaded}

\emph{Working directory}'sse salvestuvad ka kõik asjad mida me R'is salvestame (graafikud, andmed jne). Kui Kasutame R'i projekti, siis on wd automaatselt projektikataloog.

\textbf{Andmete salvestamine}

Andmete salvestamine .csv formaati käib \texttt{write.csv()} funktsiooniga (ja üldiselt me tahame neid sellesse formaati salvestada).

\begin{Shaded}
\begin{Highlighting}[]
\FunctionTok{write.csv}\NormalTok{(andmed, }\AttributeTok{file=}\StringTok{"C:/Users/Mina/Kvant analüüsi meetodid II/Andmed/andmed.csv"}\NormalTok{)}
\end{Highlighting}
\end{Shaded}

Teine (ja tegelikult eelistatum) salvestamisviis on \emph{readr} funktsioon \texttt{write\_csv()}.

R'il on ka oma salvestusformaat. Kui on aga plaanis andmeid pikemalt salvestada, teise arvutiga kasutada või kellegagi jagada, siis ei ole mõistlik Rdata salvestusvisi kasutada, kuna RData fail on konkreetse Ri konfiguratsiooni spetsiifiline.

\begin{Shaded}
\begin{Highlighting}[]
\FunctionTok{save}\NormalTok{(see, }\AttributeTok{file =} \StringTok{"see.RData"}\NormalTok{) }\CommentTok{\#salvestamine}
\FunctionTok{rm}\NormalTok{(see)}
\FunctionTok{load}\NormalTok{(}\StringTok{"see.RData"}\NormalTok{) }\CommentTok{\# sisse laadimine}
\end{Highlighting}
\end{Shaded}

\textbf{Andestikust ülevaate saamine}

R'is on mitmeid näidisandmestikke. Võtame neist ühe ja salvestame eraldi andmeobjekti.

\begin{Shaded}
\begin{Highlighting}[]
\NormalTok{dat }\OtherTok{\textless{}{-}}\NormalTok{ iris}
\end{Highlighting}
\end{Shaded}

Vaatame andmestiku struktuuri

\begin{Shaded}
\begin{Highlighting}[]
\FunctionTok{str}\NormalTok{(dat)}
\end{Highlighting}
\end{Shaded}

\begin{verbatim}
## 'data.frame':    150 obs. of  5 variables:
##  $ Sepal.Length: num  5.1 4.9 4.7 4.6 5 5.4 4.6 5 4.4 4.9 ...
##  $ Sepal.Width : num  3.5 3 3.2 3.1 3.6 3.9 3.4 3.4 2.9 3.1 ...
##  $ Petal.Length: num  1.4 1.4 1.3 1.5 1.4 1.7 1.4 1.5 1.4 1.5 ...
##  $ Petal.Width : num  0.2 0.2 0.2 0.2 0.2 0.4 0.3 0.2 0.2 0.1 ...
##  $ Species     : Factor w/ 3 levels "setosa","versicolor",..: 1 1 1 1 1 1 1 1 1 1 ...
\end{verbatim}

Mitu rida ja mitu veergu andmestikus on (ehk siis dimensioonid)?

\begin{Shaded}
\begin{Highlighting}[]
\FunctionTok{dim}\NormalTok{(dat)}
\end{Highlighting}
\end{Shaded}

\begin{verbatim}
## [1] 150   5
\end{verbatim}

Tunnuste nimed

\begin{Shaded}
\begin{Highlighting}[]
\FunctionTok{names}\NormalTok{(dat)}
\end{Highlighting}
\end{Shaded}

\begin{verbatim}
## [1] "Sepal.Length" "Sepal.Width"  "Petal.Length" "Petal.Width"  "Species"
\end{verbatim}

Andmestiku esimesed read

\begin{Shaded}
\begin{Highlighting}[]
\FunctionTok{head}\NormalTok{(dat)}
\end{Highlighting}
\end{Shaded}

\begin{verbatim}
##   Sepal.Length Sepal.Width Petal.Length Petal.Width Species
## 1          5.1         3.5          1.4         0.2  setosa
## 2          4.9         3.0          1.4         0.2  setosa
## 3          4.7         3.2          1.3         0.2  setosa
## 4          4.6         3.1          1.5         0.2  setosa
## 5          5.0         3.6          1.4         0.2  setosa
## 6          5.4         3.9          1.7         0.4  setosa
\end{verbatim}

Kuna R eristab suuri ja väikesi tähti, siis oleks lihtsam kui kõik tunnuste nimed olekid väikestes tähtedes. Kasutame funktsiooni \texttt{tolower()}:

\begin{Shaded}
\begin{Highlighting}[]
\FunctionTok{names}\NormalTok{(dat) }\OtherTok{\textless{}{-}} \FunctionTok{tolower}\NormalTok{(}\FunctionTok{names}\NormalTok{(dat))}
\end{Highlighting}
\end{Shaded}

Vaatame, mis klassist on tunnus \emph{species}:

\begin{Shaded}
\begin{Highlighting}[]
\FunctionTok{class}\NormalTok{(dat}\SpecialCharTok{$}\NormalTok{species)}
\end{Highlighting}
\end{Shaded}

\begin{verbatim}
## [1] "factor"
\end{verbatim}

Mhh. Faktor. Mis faktorlevelid on?

\begin{Shaded}
\begin{Highlighting}[]
\FunctionTok{levels}\NormalTok{(dat}\SpecialCharTok{$}\NormalTok{species)}
\end{Highlighting}
\end{Shaded}

\begin{verbatim}
## [1] "setosa"     "versicolor" "virginica"
\end{verbatim}

Ülevaade kõikidest tunnustest (kui on suur andmebaas, siis oleks mõistlik valida ainult mõned tunnused (indeksitega siis)):

\begin{Shaded}
\begin{Highlighting}[]
\FunctionTok{summary}\NormalTok{(dat)}
\end{Highlighting}
\end{Shaded}

\begin{verbatim}
##   sepal.length    sepal.width     petal.length    petal.width   
##  Min.   :4.300   Min.   :2.000   Min.   :1.000   Min.   :0.100  
##  1st Qu.:5.100   1st Qu.:2.800   1st Qu.:1.600   1st Qu.:0.300  
##  Median :5.800   Median :3.000   Median :4.350   Median :1.300  
##  Mean   :5.843   Mean   :3.057   Mean   :3.758   Mean   :1.199  
##  3rd Qu.:6.400   3rd Qu.:3.300   3rd Qu.:5.100   3rd Qu.:1.800  
##  Max.   :7.900   Max.   :4.400   Max.   :6.900   Max.   :2.500  
##        species  
##  setosa    :50  
##  versicolor:50  
##  virginica :50  
##                 
##                 
## 
\end{verbatim}

\textbf{Kirjeldav statistika}

Mõned olulisemad funktsioonid

\begin{Shaded}
\begin{Highlighting}[]
\FunctionTok{mean}\NormalTok{(dat}\SpecialCharTok{$}\NormalTok{sepal.length) }\CommentTok{\#aritmeetiline keskmine}
\end{Highlighting}
\end{Shaded}

\begin{verbatim}
## [1] 5.843333
\end{verbatim}

\begin{Shaded}
\begin{Highlighting}[]
\FunctionTok{median}\NormalTok{(dat}\SpecialCharTok{$}\NormalTok{sepal.length) }\CommentTok{\# mediaan}
\end{Highlighting}
\end{Shaded}

\begin{verbatim}
## [1] 5.8
\end{verbatim}

\begin{Shaded}
\begin{Highlighting}[]
\FunctionTok{sd}\NormalTok{(dat}\SpecialCharTok{$}\NormalTok{sepal.length) }\CommentTok{\# standardhälve}
\end{Highlighting}
\end{Shaded}

\begin{verbatim}
## [1] 0.8280661
\end{verbatim}

\begin{Shaded}
\begin{Highlighting}[]
\FunctionTok{var}\NormalTok{(dat}\SpecialCharTok{$}\NormalTok{sepal.length) }\CommentTok{\#dispersioon}
\end{Highlighting}
\end{Shaded}

\begin{verbatim}
## [1] 0.6856935
\end{verbatim}

\begin{Shaded}
\begin{Highlighting}[]
\FunctionTok{max}\NormalTok{(dat}\SpecialCharTok{$}\NormalTok{sepal.length) }\CommentTok{\#maksimaalne väärtus}
\end{Highlighting}
\end{Shaded}

\begin{verbatim}
## [1] 7.9
\end{verbatim}

\begin{Shaded}
\begin{Highlighting}[]
\FunctionTok{min}\NormalTok{(dat}\SpecialCharTok{$}\NormalTok{sepal.length) }\CommentTok{\#minimaalne väärtus}
\end{Highlighting}
\end{Shaded}

\begin{verbatim}
## [1] 4.3
\end{verbatim}

\begin{Shaded}
\begin{Highlighting}[]
\FunctionTok{length}\NormalTok{(dat}\SpecialCharTok{$}\NormalTok{sepal.length) }\CommentTok{\#vaatluste arv, tunnuse pikkus}
\end{Highlighting}
\end{Shaded}

\begin{verbatim}
## [1] 150
\end{verbatim}

\begin{Shaded}
\begin{Highlighting}[]
\FunctionTok{sum}\NormalTok{(dat}\SpecialCharTok{$}\NormalTok{sepal.length) }\CommentTok{\#summa}
\end{Highlighting}
\end{Shaded}

\begin{verbatim}
## [1] 876.5
\end{verbatim}

\begin{Shaded}
\begin{Highlighting}[]
\FunctionTok{cor}\NormalTok{(dat}\SpecialCharTok{$}\NormalTok{sepal.length, dat}\SpecialCharTok{$}\NormalTok{sepal.width) }\CommentTok{\#korrelatsioon}
\end{Highlighting}
\end{Shaded}

\begin{verbatim}
## [1] -0.1175698
\end{verbatim}

Kui tunnuses on puuduvad väärtused, siis paljud funktsioonid ei tööta

\begin{Shaded}
\begin{Highlighting}[]
\NormalTok{x }\OtherTok{\textless{}{-}}\NormalTok{ dat}\SpecialCharTok{$}\NormalTok{sepal.length}
\NormalTok{x[}\DecValTok{3}\NormalTok{] }\OtherTok{\textless{}{-}} \ConstantTok{NA} \CommentTok{\# muudame ühe väärtuse puuduolevaks}
\FunctionTok{mean}\NormalTok{(x)}
\end{Highlighting}
\end{Shaded}

\begin{verbatim}
## [1] NA
\end{verbatim}

Et puuduvaid väärtusi mitte arvestada, kasutame \texttt{na.rm} argumenti (mõnedel funktsioonidel on teistsugused missingute argumendid, vaadake help'i)

\begin{Shaded}
\begin{Highlighting}[]
\FunctionTok{mean}\NormalTok{(x, }\AttributeTok{na.rm=}\ConstantTok{TRUE}\NormalTok{)}
\end{Highlighting}
\end{Shaded}

\begin{verbatim}
## [1] 5.851007
\end{verbatim}

Ülesanne!

\begin{enumerate}
\def\labelenumi{\arabic{enumi}.}
\tightlist
\item
  Leidke tunnuse \emph{sepal.width} keskmine ilma funktsiooni \texttt{mean()} kasutamata
\item
  Tegime tunnusest \emph{sepal.length} uue andmeobjekti \emph{x} (kus on üks puuduv väärtus). Leidke \emph{x}'i korrelatsioon \emph{sepal.width}'iga (vajadusel kasutage helpi)
\end{enumerate}

\textbf{Tabelid}

Kategoriaalsetele tunnustele tabelid

\begin{Shaded}
\begin{Highlighting}[]
\FunctionTok{table}\NormalTok{(dat}\SpecialCharTok{$}\NormalTok{species)}
\end{Highlighting}
\end{Shaded}

\begin{verbatim}
## 
##     setosa versicolor  virginica 
##         50         50         50
\end{verbatim}

Risttabeli jaoks on meil ka teist faktortunnust (või tekstilist tunnust) vaja. Teeme ise ühe

\begin{Shaded}
\begin{Highlighting}[]
\NormalTok{dat}\SpecialCharTok{$}\NormalTok{kat }\OtherTok{\textless{}{-}} \FunctionTok{cut}\NormalTok{(dat}\SpecialCharTok{$}\NormalTok{sepal.length, }\AttributeTok{breaks =} \DecValTok{3}\NormalTok{, }\AttributeTok{labels =} \FunctionTok{c}\NormalTok{(}\StringTok{"L"}\NormalTok{, }\StringTok{"K"}\NormalTok{, }\StringTok{"P"}\NormalTok{)) }\CommentTok{\# funktsioon cut() lõikab arvtunnuse kategooriateks, breaks argumendiga saab määrata mitmeks kategooriaks (võib ka cut{-}pointid ette anda, nt breaks=c(2,3,4))}
\end{Highlighting}
\end{Shaded}

Risttabel

\begin{Shaded}
\begin{Highlighting}[]
\FunctionTok{table}\NormalTok{(dat}\SpecialCharTok{$}\NormalTok{species, dat}\SpecialCharTok{$}\NormalTok{kat)}
\end{Highlighting}
\end{Shaded}

\begin{verbatim}
##             
##               L  K  P
##   setosa     47  3  0
##   versicolor 11 36  3
##   virginica   1 32 17
\end{verbatim}

Saame juurde panna rea ja veeru summad (selleks peab tabel enne olemas olema)

\begin{Shaded}
\begin{Highlighting}[]
\NormalTok{x }\OtherTok{\textless{}{-}} \FunctionTok{table}\NormalTok{(dat}\SpecialCharTok{$}\NormalTok{species, dat}\SpecialCharTok{$}\NormalTok{kat)}
\FunctionTok{addmargins}\NormalTok{(x)}
\end{Highlighting}
\end{Shaded}

\begin{verbatim}
##             
##                L   K   P Sum
##   setosa      47   3   0  50
##   versicolor  11  36   3  50
##   virginica    1  32  17  50
##   Sum         59  71  20 150
\end{verbatim}

\emph{Margin}'eid võib lisada ka ainult veerule või ainult reale, samuti võivad need midagi muud kui summa olla (vaata funktsiooni helpi)

Proportsioonide tabel

\begin{Shaded}
\begin{Highlighting}[]
\FunctionTok{prop.table}\NormalTok{(x)}
\end{Highlighting}
\end{Shaded}

\begin{verbatim}
##             
##                        L           K           P
##   setosa     0.313333333 0.020000000 0.000000000
##   versicolor 0.073333333 0.240000000 0.020000000
##   virginica  0.006666667 0.213333333 0.113333333
\end{verbatim}

\emph{Default} on proportsioon kogusummast. Aga võime ka argumendiga 1 määrata rea proportsiooni või argumendiga 2 veeru proportsiooni.

\begin{Shaded}
\begin{Highlighting}[]
\FunctionTok{prop.table}\NormalTok{(x, }\DecValTok{1}\NormalTok{)}
\FunctionTok{prop.table}\NormalTok{(x, }\DecValTok{2}\NormalTok{)}
\end{Highlighting}
\end{Shaded}

\hypertarget{dplyr}{%
\subsection{dplyr}\label{dplyr}}

\emph{Dplyr} on väga funktsionaalne pakett data.frame'is olevate andmete töötlemiseks, kirjeldamiseks ja transformeerimiseks.\\
Praktiliselt kogu andmetöötluse saab \emph{dplyr}'i abil ära teha.\\
\emph{Dplyr} põhineb viiel peamisel funktsioonil:\\
- \texttt{filter()} - vaatluste filtreerimine mingite kriteeriumite alusel\\
- \texttt{select()} - tunnuste valimine\\
- \texttt{arrange()} - andmete järjestamine mingi tunnuse põhjal\\
- \texttt{mutate()} - uue tunnuse tegemine\\
- \texttt{summarise()} - tunnuste summeerimine

Lisaks veel funktsioon \texttt{group\_by()}, millega saab andmestiku mingi tunnuse alusel gruppideks jaotada ja siis igale grupile näiteks summarise() funktsiooni rakendada.

Kõigepealt installime ja laadime dplyr'i ja ühe näidisandmestiku

\begin{Shaded}
\begin{Highlighting}[]
\FunctionTok{install.packages}\NormalTok{(}\StringTok{"dplyr"}\NormalTok{)}
\FunctionTok{install.packages}\NormalTok{(}\StringTok{"nycflights13"}\NormalTok{)}
\end{Highlighting}
\end{Shaded}

\begin{Shaded}
\begin{Highlighting}[]
\FunctionTok{library}\NormalTok{(dplyr)}
\end{Highlighting}
\end{Shaded}

\begin{verbatim}
## 
## Attaching package: 'dplyr'
\end{verbatim}

\begin{verbatim}
## The following objects are masked from 'package:stats':
## 
##     filter, lag
\end{verbatim}

\begin{verbatim}
## The following objects are masked from 'package:base':
## 
##     intersect, setdiff, setequal, union
\end{verbatim}

\begin{Shaded}
\begin{Highlighting}[]
\FunctionTok{library}\NormalTok{(nycflights13)}
\end{Highlighting}
\end{Shaded}

Salvestame paketist \emph{nycflights13} andmestiku \emph{flights} käepärasema nimega :

\begin{Shaded}
\begin{Highlighting}[]
\NormalTok{dat }\OtherTok{\textless{}{-}}\NormalTok{ flights}
\end{Highlighting}
\end{Shaded}

Vaatame, mis andmestikuga tegu on

\begin{Shaded}
\begin{Highlighting}[]
\FunctionTok{head}\NormalTok{(dat)}
\end{Highlighting}
\end{Shaded}

\begin{verbatim}
## # A tibble: 6 x 19
##    year month   day dep_time sched_dep_time dep_delay arr_time sched_arr_time
##   <int> <int> <int>    <int>          <int>     <dbl>    <int>          <int>
## 1  2013     1     1      517            515         2      830            819
## 2  2013     1     1      533            529         4      850            830
## 3  2013     1     1      542            540         2      923            850
## 4  2013     1     1      544            545        -1     1004           1022
## 5  2013     1     1      554            600        -6      812            837
## 6  2013     1     1      554            558        -4      740            728
## # ... with 11 more variables: arr_delay <dbl>, carrier <chr>, flight <int>,
## #   tailnum <chr>, origin <chr>, dest <chr>, air_time <dbl>, distance <dbl>,
## #   hour <dbl>, minute <dbl>, time_hour <dttm>
\end{verbatim}

\begin{Shaded}
\begin{Highlighting}[]
\FunctionTok{names}\NormalTok{(dat)}
\end{Highlighting}
\end{Shaded}

\begin{verbatim}
##  [1] "year"           "month"          "day"            "dep_time"      
##  [5] "sched_dep_time" "dep_delay"      "arr_time"       "sched_arr_time"
##  [9] "arr_delay"      "carrier"        "flight"         "tailnum"       
## [13] "origin"         "dest"           "air_time"       "distance"      
## [17] "hour"           "minute"         "time_hour"
\end{verbatim}

\textbf{dplyr: filter}

Valime ainult need vaatlused, kus lennufirmaks on AA (tunnus \emph{carrier}) ja mis toimusid jaanuari- või veebruarikuus (\emph{dplyr}'i funktsioonides on andmestik alati esimeseks argumeniks)

\begin{Shaded}
\begin{Highlighting}[]
\NormalTok{dat.aa }\OtherTok{\textless{}{-}} \FunctionTok{filter}\NormalTok{(dat, carrier}\SpecialCharTok{==}\StringTok{"AA"} \SpecialCharTok{\&}\NormalTok{ (month}\SpecialCharTok{==}\DecValTok{1} \SpecialCharTok{|}\NormalTok{ month}\SpecialCharTok{==}\DecValTok{2}\NormalTok{))}
\FunctionTok{table}\NormalTok{(dat.aa}\SpecialCharTok{$}\NormalTok{month)}
\end{Highlighting}
\end{Shaded}

\begin{verbatim}
## 
##    1    2 
## 2794 2517
\end{verbatim}

\begin{Shaded}
\begin{Highlighting}[]
\FunctionTok{table}\NormalTok{(dat.aa}\SpecialCharTok{$}\NormalTok{carrier)}
\end{Highlighting}
\end{Shaded}

\begin{verbatim}
## 
##   AA 
## 5311
\end{verbatim}

\textbf{dplyr: select}

Valime \emph{dat.aa} andmestikust tunnused \emph{month}, \emph{arr\_delay} ja \emph{tailnum}

\begin{Shaded}
\begin{Highlighting}[]
\NormalTok{dat.aa1 }\OtherTok{\textless{}{-}} \FunctionTok{select}\NormalTok{(dat.aa, month, arr\_delay, tailnum)}
\FunctionTok{head}\NormalTok{(dat.aa1)}
\end{Highlighting}
\end{Shaded}

\begin{verbatim}
## # A tibble: 6 x 3
##   month arr_delay tailnum
##   <int>     <dbl> <chr>  
## 1     1        33 N619AA 
## 2     1         8 N3ALAA 
## 3     1        31 N3DUAA 
## 4     1       -12 N633AA 
## 5     1         5 N3EMAA 
## 6     1        -3 N3BAAA
\end{verbatim}

Küllaltki kasulikud on \texttt{select()}'i lisavõimalused, mis lasevad valida tunnuseid vastavalt sellele, mis tähekombinatsiooniga tunnusenimi algab, lõpeb või mida sisaldab (täpsemalt vaata \texttt{select()} helpi).

\begin{Shaded}
\begin{Highlighting}[]
\NormalTok{dat.aa2 }\OtherTok{\textless{}{-}} \FunctionTok{select}\NormalTok{(dat.aa, }\FunctionTok{contains}\NormalTok{(}\StringTok{"arr"}\NormalTok{))}
\end{Highlighting}
\end{Shaded}

\textbf{dplyr: arrange}

Järjestame \emph{dat.aa1} andmedtiku \emph{arr\_delay} tunnuse järgi

\begin{Shaded}
\begin{Highlighting}[]
\NormalTok{dat.aa1 }\OtherTok{\textless{}{-}} \FunctionTok{arrange}\NormalTok{(dat.aa1, arr\_delay)}
\FunctionTok{head}\NormalTok{(dat.aa1)}
\end{Highlighting}
\end{Shaded}

\begin{verbatim}
## # A tibble: 6 x 3
##   month arr_delay tailnum
##   <int>     <dbl> <chr>  
## 1     2       -69 N3EAAA 
## 2     2       -65 N320AA 
## 3     2       -60 N3FAAA 
## 4     1       -54 N335AA 
## 5     2       -54 N4UBAA 
## 6     2       -54 N350AA
\end{verbatim}

Suuremast väiksemaks järjestamieks tuleb kasutada \texttt{desc()} lisavõimalust

\begin{Shaded}
\begin{Highlighting}[]
\NormalTok{dat.aa1 }\OtherTok{\textless{}{-}} \FunctionTok{arrange}\NormalTok{(dat.aa1, }\FunctionTok{desc}\NormalTok{(arr\_delay))}
\end{Highlighting}
\end{Shaded}

\textbf{dplyr: mutate}

Teeme uue tunnuse, kus hilinemise aeg oleks tundides.

\begin{Shaded}
\begin{Highlighting}[]
\NormalTok{dat.aa1 }\OtherTok{\textless{}{-}} \FunctionTok{mutate}\NormalTok{(dat.aa1, }\AttributeTok{tunnid=}\NormalTok{arr\_delay}\SpecialCharTok{/}\DecValTok{60}\NormalTok{)}
\FunctionTok{head}\NormalTok{(dat.aa1)}
\end{Highlighting}
\end{Shaded}

\begin{verbatim}
## # A tibble: 6 x 4
##   month arr_delay tailnum tunnid
##   <int>     <dbl> <chr>    <dbl>
## 1     2       -69 N3EAAA   -1.15
## 2     2       -65 N320AA   -1.08
## 3     2       -60 N3FAAA   -1   
## 4     1       -54 N335AA   -0.9 
## 5     2       -54 N4UBAA   -0.9 
## 6     2       -54 N350AA   -0.9
\end{verbatim}

Saab luua ka funktsioonide alusel uusi tunnuseid, näiteks kui tahame mingil põhjusel tunnust, kus oleks kõikide hilinemiste keskmine.

\begin{Shaded}
\begin{Highlighting}[]
\FunctionTok{mutate}\NormalTok{(dat.aa1, }\AttributeTok{keskmine=}\FunctionTok{mean}\NormalTok{(arr\_delay, }\AttributeTok{na.rm=}\NormalTok{T))}
\end{Highlighting}
\end{Shaded}

\textbf{dplyr: summarise}

Tahame summeerida hilinemised keskmise, standardhälbe, maksimumi ja miinimumi alusel.

\begin{Shaded}
\begin{Highlighting}[]
\FunctionTok{summarise}\NormalTok{(dat.aa1, }
          \AttributeTok{keskmine=}\FunctionTok{mean}\NormalTok{(arr\_delay, }\AttributeTok{na.rm=}\NormalTok{T),}
          \AttributeTok{sdh=}\FunctionTok{sd}\NormalTok{(arr\_delay, }\AttributeTok{na.rm=}\NormalTok{T),}
          \AttributeTok{maks=}\FunctionTok{max}\NormalTok{(arr\_delay, }\AttributeTok{na.rm=}\NormalTok{T),}
          \AttributeTok{min=}\FunctionTok{min}\NormalTok{(arr\_delay, }\AttributeTok{na.rm=}\NormalTok{T))}
\end{Highlighting}
\end{Shaded}

\begin{verbatim}
## # A tibble: 1 x 4
##   keskmine   sdh  maks   min
##      <dbl> <dbl> <dbl> <dbl>
## 1     1.03  34.1   368   -69
\end{verbatim}

\textbf{dplyr: group\_by}

Tahame teada keskmist hilinemist kuude lõikes.

\begin{Shaded}
\begin{Highlighting}[]
\FunctionTok{summarise}\NormalTok{(}\FunctionTok{group\_by}\NormalTok{(dat.aa1, month), }\AttributeTok{keskmine=}\FunctionTok{mean}\NormalTok{(tunnid, }\AttributeTok{na.rm=}\NormalTok{T))}
\end{Highlighting}
\end{Shaded}

\begin{verbatim}
## # A tibble: 2 x 2
##   month keskmine
##   <int>    <dbl>
## 1     1   0.0164
## 2     2   0.0182
\end{verbatim}

\textbf{dplyr: piping}

dplyr toetab nn piping'ut. Kombinatsiooniga \%\textgreater\% saab saab ühe funktsiooni tulemuse võtta sisendiks järgmisele funktsioonile. Seega saame kõik vajalikud toimingud ühes reas ära teha. Kõikide dplyr'i funktsioonide puhul on esimene argument \emph{data}. Kui kasutame \%\textgreater\%, ei pea me enam \emph{data}'t defineerima ja dplyr saab ise aru, et andmeteks on eelmise funktsiooni tulemused.\\
Nii saame kõik eelnevad käsud panna ühte jadasse:

\begin{Shaded}
\begin{Highlighting}[]
\NormalTok{dat }\SpecialCharTok{\%\textgreater{}\%} \CommentTok{\#algsed andmed}
  \FunctionTok{filter}\NormalTok{(carrier}\SpecialCharTok{==}\StringTok{"AA"}\NormalTok{, month}\SpecialCharTok{==}\DecValTok{1} \SpecialCharTok{|}\NormalTok{ month}\SpecialCharTok{==}\DecValTok{2}\NormalTok{) }\SpecialCharTok{\%\textgreater{}\%}
  \FunctionTok{select}\NormalTok{(month, arr\_delay, tailnum) }\SpecialCharTok{\%\textgreater{}\%}
  \FunctionTok{arrange}\NormalTok{(arr\_delay) }\SpecialCharTok{\%\textgreater{}\%}
  \FunctionTok{mutate}\NormalTok{(}\AttributeTok{tunnid=}\NormalTok{arr\_delay}\SpecialCharTok{/}\DecValTok{60}\NormalTok{) }\SpecialCharTok{\%\textgreater{}\%}
  \FunctionTok{group\_by}\NormalTok{(month) }\SpecialCharTok{\%\textgreater{}\%}
  \FunctionTok{summarise}\NormalTok{(}\AttributeTok{keskmine=}\FunctionTok{mean}\NormalTok{(tunnid, }\AttributeTok{na.rm=}\NormalTok{T))}
\end{Highlighting}
\end{Shaded}

\begin{verbatim}
## # A tibble: 2 x 2
##   month keskmine
##   <int>    <dbl>
## 1     1   0.0164
## 2     2   0.0182
\end{verbatim}

Kui me tahame tulemusi kuhugi salvestada, peame uue andmeobjekti alguses määrama.

\begin{Shaded}
\begin{Highlighting}[]
\NormalTok{dat1 }\OtherTok{\textless{}{-}}\NormalTok{ dat }\SpecialCharTok{\%\textgreater{}\%}
  \FunctionTok{select}\NormalTok{(month, arr\_delay, tailnum)}
\end{Highlighting}
\end{Shaded}

Tänu piping'ule saab küllaltki keerulisi andmeteisendusi teha väga lihtsalt ja elegantselt (puhta ja arusaadava koodiga). Näiteks tunnus, milles on kõikide lennufirmade keskmine hilinemine kõikide kuude lõikes:

\begin{Shaded}
\begin{Highlighting}[]
\NormalTok{dat1 }\OtherTok{\textless{}{-}}\NormalTok{ dat}\SpecialCharTok{\%\textgreater{}\%}
  \FunctionTok{group\_by}\NormalTok{(carrier, month) }\SpecialCharTok{\%\textgreater{}\%} \CommentTok{\# saame grupeerida ka mitme tunnuse lõikes}
  \FunctionTok{mutate}\NormalTok{(}\AttributeTok{keskmine =} \FunctionTok{mean}\NormalTok{(arr\_delay, }\AttributeTok{na.rm=}\NormalTok{T))}
\end{Highlighting}
\end{Shaded}

\textbf{Veel mõned kasulikud funktsioonid}

Kui on vaja välja jätta dubleerivad vaatlused:

\begin{Shaded}
\begin{Highlighting}[]
\NormalTok{flights }\SpecialCharTok{\%\textgreater{}\%} 
  \FunctionTok{distinct}\NormalTok{(carrier, flight)}
\end{Highlighting}
\end{Shaded}

\begin{verbatim}
## # A tibble: 5,725 x 2
##    carrier flight
##    <chr>    <int>
##  1 UA        1545
##  2 UA        1714
##  3 AA        1141
##  4 B6         725
##  5 DL         461
##  6 UA        1696
##  7 B6         507
##  8 EV        5708
##  9 B6          79
## 10 AA         301
## # ... with 5,715 more rows
\end{verbatim}

Kui on vaja vaatluste arvu:

\begin{Shaded}
\begin{Highlighting}[]
\NormalTok{flights }\SpecialCharTok{\%\textgreater{}\%} 
  \FunctionTok{summarise}\NormalTok{(}\FunctionTok{n}\NormalTok{())}
\end{Highlighting}
\end{Shaded}

\begin{verbatim}
## # A tibble: 1 x 1
##    `n()`
##    <int>
## 1 336776
\end{verbatim}

\begin{Shaded}
\begin{Highlighting}[]
\CommentTok{\# Või vaatluste arv gruppide lõikes}
\NormalTok{flights }\SpecialCharTok{\%\textgreater{}\%} 
  \FunctionTok{group\_by}\NormalTok{(carrier) }\SpecialCharTok{\%\textgreater{}\%} 
  \FunctionTok{summarise}\NormalTok{(}\AttributeTok{kokku =} \FunctionTok{n}\NormalTok{())}
\end{Highlighting}
\end{Shaded}

\begin{verbatim}
## # A tibble: 16 x 2
##    carrier kokku
##    <chr>   <int>
##  1 9E      18460
##  2 AA      32729
##  3 AS        714
##  4 B6      54635
##  5 DL      48110
##  6 EV      54173
##  7 F9        685
##  8 FL       3260
##  9 HA        342
## 10 MQ      26397
## 11 OO         32
## 12 UA      58665
## 13 US      20536
## 14 VX       5162
## 15 WN      12275
## 16 YV        601
\end{verbatim}

Kui tahame välja võtta juhuvalimi:

\begin{Shaded}
\begin{Highlighting}[]
\CommentTok{\# Võtame välja 10 juhuslikku rida}
\NormalTok{flights }\SpecialCharTok{\%\textgreater{}\%} 
  \FunctionTok{sample\_n}\NormalTok{(}\DecValTok{10}\NormalTok{)}
\end{Highlighting}
\end{Shaded}

\begin{verbatim}
## # A tibble: 10 x 19
##     year month   day dep_time sched_dep_time dep_delay arr_time sched_arr_time
##    <int> <int> <int>    <int>          <int>     <dbl>    <int>          <int>
##  1  2013     1    23     1856           1900        -4     2125           2146
##  2  2013     7    30     1249           1255        -6     1511           1541
##  3  2013     7    15     1332           1339        -7     1448           1457
##  4  2013    11    17     1611           1610         1     1904           1918
##  5  2013     4     3      659            659         0     1002           1010
##  6  2013     9     5      622            630        -8      750            805
##  7  2013     1     3     2129           2125         4     2233           2233
##  8  2013     7    23     1502           1500         2     1809           1758
##  9  2013    10     6     1900           1900         0     2147           2154
## 10  2013    11     8     1745           1740         5     2120           2118
## # ... with 11 more variables: arr_delay <dbl>, carrier <chr>, flight <int>,
## #   tailnum <chr>, origin <chr>, dest <chr>, air_time <dbl>, distance <dbl>,
## #   hour <dbl>, minute <dbl>, time_hour <dttm>
\end{verbatim}

Kui tahame välja võtta konkreetsed read:

\begin{Shaded}
\begin{Highlighting}[]
\CommentTok{\# Võtame välja esimesed 5 rida}
\NormalTok{flights }\SpecialCharTok{\%\textgreater{}\%} 
  \FunctionTok{slice}\NormalTok{(}\DecValTok{1}\SpecialCharTok{:}\DecValTok{5}\NormalTok{)}
\end{Highlighting}
\end{Shaded}

\begin{verbatim}
## # A tibble: 5 x 19
##    year month   day dep_time sched_dep_time dep_delay arr_time sched_arr_time
##   <int> <int> <int>    <int>          <int>     <dbl>    <int>          <int>
## 1  2013     1     1      517            515         2      830            819
## 2  2013     1     1      533            529         4      850            830
## 3  2013     1     1      542            540         2      923            850
## 4  2013     1     1      544            545        -1     1004           1022
## 5  2013     1     1      554            600        -6      812            837
## # ... with 11 more variables: arr_delay <dbl>, carrier <chr>, flight <int>,
## #   tailnum <chr>, origin <chr>, dest <chr>, air_time <dbl>, distance <dbl>,
## #   hour <dbl>, minute <dbl>, time_hour <dttm>
\end{verbatim}

\hypertarget{andmete-uxfchendamine}{%
\subsection{Andmete ühendamine}\label{andmete-uxfchendamine}}

Andmestike ühendamisel võib olla kaks eesmärki: tahame lisada ridu või tahame lisada tunnuseid (veergusid).

Ridade lisamiseks on dplyr'is funktsioon \texttt{bind\_row()}:

\begin{Shaded}
\begin{Highlighting}[]
\CommentTok{\#Teeme kaks andmestikku}
\NormalTok{dt1 }\OtherTok{\textless{}{-}} \FunctionTok{data.frame}\NormalTok{(}\AttributeTok{a =} \FunctionTok{c}\NormalTok{(}\StringTok{"a"}\NormalTok{, }\StringTok{"b"}\NormalTok{, }\StringTok{"c"}\NormalTok{, }\StringTok{"d"}\NormalTok{, }\StringTok{"e"}\NormalTok{), }\AttributeTok{b =} \DecValTok{1}\SpecialCharTok{:}\DecValTok{5}\NormalTok{)}
\NormalTok{dt1}
\end{Highlighting}
\end{Shaded}

\begin{verbatim}
##   a b
## 1 a 1
## 2 b 2
## 3 c 3
## 4 d 4
## 5 e 5
\end{verbatim}

\begin{Shaded}
\begin{Highlighting}[]
\NormalTok{dt2 }\OtherTok{\textless{}{-}} \FunctionTok{data.frame}\NormalTok{(}\AttributeTok{a =} \FunctionTok{c}\NormalTok{(}\StringTok{"a"}\NormalTok{, }\StringTok{"b"}\NormalTok{, }\StringTok{"e"}\NormalTok{, }\StringTok{"f"}\NormalTok{), }\AttributeTok{c =} \DecValTok{6}\SpecialCharTok{:}\DecValTok{9}\NormalTok{)}
\NormalTok{dt2}
\end{Highlighting}
\end{Shaded}

\begin{verbatim}
##   a c
## 1 a 6
## 2 b 7
## 3 e 8
## 4 f 9
\end{verbatim}

\begin{Shaded}
\begin{Highlighting}[]
\CommentTok{\# Ühendame andmestikud ridadena}
\FunctionTok{bind\_rows}\NormalTok{(dt1, dt2)}
\end{Highlighting}
\end{Shaded}

\begin{verbatim}
##   a  b  c
## 1 a  1 NA
## 2 b  2 NA
## 3 c  3 NA
## 4 d  4 NA
## 5 e  5 NA
## 6 a NA  6
## 7 b NA  7
## 8 e NA  8
## 9 f NA  9
\end{verbatim}

Andmestike tunnuste kaupa ühendamiseks on meil vaja ID-tunnust või tunnuseid, mis identifitseeriks unikaalsed vaatlused. Antud juhul on meil selleks tunnus ``a''.

\begin{Shaded}
\begin{Highlighting}[]
\CommentTok{\# Ühendame teise andmestiku esimese külge }
\CommentTok{\# (ehk siis alles jäävad kõik esimese andmestiku vaatlused)}
\FunctionTok{left\_join}\NormalTok{(dt1, dt2, }\AttributeTok{by =} \StringTok{"a"}\NormalTok{)}
\end{Highlighting}
\end{Shaded}

\begin{verbatim}
##   a b  c
## 1 a 1  6
## 2 b 2  7
## 3 c 3 NA
## 4 d 4 NA
## 5 e 5  8
\end{verbatim}

\begin{Shaded}
\begin{Highlighting}[]
\CommentTok{\# Ühendame esimese andmestiku teise külge }
\CommentTok{\# (ehk siis alles jäävad kõik teise andmestiku vaatlused)}
\FunctionTok{right\_join}\NormalTok{(dt1, dt2, }\AttributeTok{by =} \StringTok{"a"}\NormalTok{)}
\end{Highlighting}
\end{Shaded}

\begin{verbatim}
##   a  b c
## 1 a  1 6
## 2 b  2 7
## 3 e  5 8
## 4 f NA 9
\end{verbatim}

\begin{Shaded}
\begin{Highlighting}[]
\CommentTok{\# Ühendame andmestikud nii, et alles jäävad kõik vaatlused mõlemast andmestikust}
\FunctionTok{full\_join}\NormalTok{(dt1, dt2)}
\end{Highlighting}
\end{Shaded}

\begin{verbatim}
## Joining, by = "a"
\end{verbatim}

\begin{verbatim}
##   a  b  c
## 1 a  1  6
## 2 b  2  7
## 3 c  3 NA
## 4 d  4 NA
## 5 e  5  8
## 6 f NA  9
\end{verbatim}

\begin{Shaded}
\begin{Highlighting}[]
\CommentTok{\# Ühendame andmestikud nii, et alles jäävad need vaatlused, mis mõlemas andmestikus olemas on}
\FunctionTok{inner\_join}\NormalTok{(dt1, dt2)}
\end{Highlighting}
\end{Shaded}

\begin{verbatim}
## Joining, by = "a"
\end{verbatim}

\begin{verbatim}
##   a b c
## 1 a 1 6
## 2 b 2 7
## 3 e 5 8
\end{verbatim}

\begin{Shaded}
\begin{Highlighting}[]
\CommentTok{\# Ühendame andmestikud nii, et alles jäävad need vaatlused, mida ei ole kummaski andmestikus}
\FunctionTok{anti\_join}\NormalTok{(dt1, dt2)}
\end{Highlighting}
\end{Shaded}

\begin{verbatim}
## Joining, by = "a"
\end{verbatim}

\begin{verbatim}
##   a b
## 1 c 3
## 2 d 4
\end{verbatim}

\hypertarget{andmestiku-kuju-muutmine}{%
\subsection{Andmestiku kuju muutmine}\label{andmestiku-kuju-muutmine}}

Andmestik võib olla nn ``pikal kujul'' või ``laial kujul''. Pikad andmed on sellised, mille puhul kõik muutujad on kirjeldatud tunnustena. Laial kujul andmed on sellised, mille puhul mõni muutuja on jaotatad erinevateks tunnusteks. \emph{tidyr} pakett võimaldab mugavalt andmestiku ühelt kujult teise tranformeerimist:

\begin{Shaded}
\begin{Highlighting}[]
\FunctionTok{library}\NormalTok{(tidyr)}

\CommentTok{\# Teeme "laia" näidisandmestiku}
\NormalTok{lai }\OtherTok{\textless{}{-}} \FunctionTok{data.frame}\NormalTok{(}\AttributeTok{nimi =} \FunctionTok{c}\NormalTok{(}\StringTok{"Jüri"}\NormalTok{, }\StringTok{"Mari"}\NormalTok{, }\StringTok{"Jaan"}\NormalTok{), }
                 \AttributeTok{test\_1 =} \FunctionTok{c}\NormalTok{(}\DecValTok{3}\NormalTok{,}\DecValTok{5}\NormalTok{,}\DecValTok{2}\NormalTok{), }
                 \AttributeTok{test\_2 =} \FunctionTok{c}\NormalTok{(}\DecValTok{8}\NormalTok{,}\DecValTok{4}\NormalTok{,}\DecValTok{5}\NormalTok{),}
                 \AttributeTok{test\_3 =} \FunctionTok{c}\NormalTok{(}\DecValTok{2}\NormalTok{,}\DecValTok{5}\NormalTok{,}\DecValTok{4}\NormalTok{))}

\NormalTok{lai}
\end{Highlighting}
\end{Shaded}

\begin{verbatim}
##   nimi test_1 test_2 test_3
## 1 Jüri      3      8      2
## 2 Mari      5      4      5
## 3 Jaan      2      5      4
\end{verbatim}

\begin{Shaded}
\begin{Highlighting}[]
\CommentTok{\# Antud andmestikus on erinevad testid eri tunnustena. }
\CommentTok{\# Aga kui me tahaksime, et test oleks tunnus. }
\CommentTok{\# Keerame andmestiku pikale kujule}

\NormalTok{pikk }\OtherTok{\textless{}{-}}\NormalTok{ lai }\SpecialCharTok{\%\textgreater{}\%}
  \FunctionTok{pivot\_longer}\NormalTok{(}\AttributeTok{cols =} \FunctionTok{starts\_with}\NormalTok{(}\StringTok{"test\_"}\NormalTok{), }
               \AttributeTok{names\_to =} \StringTok{"test"}\NormalTok{, }
               \AttributeTok{values\_to =} \StringTok{"tulemus"}\NormalTok{,}
               \AttributeTok{names\_prefix =} \StringTok{"test\_"}\NormalTok{)}
\NormalTok{pikk}
\end{Highlighting}
\end{Shaded}

\begin{verbatim}
## # A tibble: 9 x 3
##   nimi  test  tulemus
##   <chr> <chr>   <dbl>
## 1 Jüri  1           3
## 2 Jüri  2           8
## 3 Jüri  3           2
## 4 Mari  1           5
## 5 Mari  2           4
## 6 Mari  3           5
## 7 Jaan  1           2
## 8 Jaan  2           5
## 9 Jaan  3           4
\end{verbatim}

\begin{Shaded}
\begin{Highlighting}[]
\CommentTok{\# Keerame tagasi laiale kujule}
\NormalTok{lai }\OtherTok{\textless{}{-}}\NormalTok{ pikk }\SpecialCharTok{\%\textgreater{}\%} 
  \FunctionTok{pivot\_wider}\NormalTok{(}\AttributeTok{names\_from =} \StringTok{"test"}\NormalTok{, }\AttributeTok{values\_from =} \StringTok{"tulemus"}\NormalTok{, }\AttributeTok{names\_prefix =} \StringTok{"test\_"}\NormalTok{)}
\NormalTok{lai}
\end{Highlighting}
\end{Shaded}

\begin{verbatim}
## # A tibble: 3 x 4
##   nimi  test_1 test_2 test_3
##   <chr>  <dbl>  <dbl>  <dbl>
## 1 Jüri       3      8      2
## 2 Mari       5      4      5
## 3 Jaan       2      5      4
\end{verbatim}

\hypertarget{kuupuxe4evad}{%
\subsection{Kuupäevad}\label{kuupuxe4evad}}

Ris käsitletakse kuupevi ja kellaaegu eraldi ``Date'' klassina. See tagab, et kupäevad ja kellaajad on alati ühtses formaadis ning võimaldab nendega tehteid teha. Aja tunnustega tegelemiseks on mugav kasutada paketti \emph{lubridate}.

Numbriliste või tekstiliste tunnuste kuupäevadeks muutmine:

\begin{Shaded}
\begin{Highlighting}[]
\FunctionTok{library}\NormalTok{(lubridate)}
\end{Highlighting}
\end{Shaded}

\begin{verbatim}
## 
## Attaching package: 'lubridate'
\end{verbatim}

\begin{verbatim}
## The following objects are masked from 'package:base':
## 
##     date, intersect, setdiff, union
\end{verbatim}

\begin{Shaded}
\begin{Highlighting}[]
\CommentTok{\# Kui kuupäeva järjekord on kuupäev (d), kuu (m), aasta (y), siis:}
\FunctionTok{dmy}\NormalTok{(}\StringTok{\textquotesingle{}24.03.2017\textquotesingle{}}\NormalTok{)}
\end{Highlighting}
\end{Shaded}

\begin{verbatim}
## [1] "2017-03-24"
\end{verbatim}

\begin{Shaded}
\begin{Highlighting}[]
\CommentTok{\# või}
\FunctionTok{dmy}\NormalTok{(}\DecValTok{24032017}\NormalTok{)}
\end{Highlighting}
\end{Shaded}

\begin{verbatim}
## [1] "2017-03-24"
\end{verbatim}

\begin{Shaded}
\begin{Highlighting}[]
\CommentTok{\# või}
\FunctionTok{dmy}\NormalTok{(}\StringTok{\textquotesingle{}24{-}03{-}2017\textquotesingle{}}\NormalTok{)}
\end{Highlighting}
\end{Shaded}

\begin{verbatim}
## [1] "2017-03-24"
\end{verbatim}

\begin{Shaded}
\begin{Highlighting}[]
\CommentTok{\# või}
\FunctionTok{dmy}\NormalTok{(}\StringTok{\textquotesingle{}24/03/2017\textquotesingle{}}\NormalTok{)}
\end{Highlighting}
\end{Shaded}

\begin{verbatim}
## [1] "2017-03-24"
\end{verbatim}

\begin{Shaded}
\begin{Highlighting}[]
\CommentTok{\# Kui järjekord on teine, siis tuleb lihtsalt tähed funktsiooninimes vastavalt vahetada}
\FunctionTok{mdy}\NormalTok{(}\StringTok{\textquotesingle{}03{-}24{-}2017\textquotesingle{}}\NormalTok{)}
\end{Highlighting}
\end{Shaded}

\begin{verbatim}
## [1] "2017-03-24"
\end{verbatim}

\begin{Shaded}
\begin{Highlighting}[]
\FunctionTok{ymd}\NormalTok{(}\StringTok{\textquotesingle{}2017/03/24\textquotesingle{}}\NormalTok{)}
\end{Highlighting}
\end{Shaded}

\begin{verbatim}
## [1] "2017-03-24"
\end{verbatim}

\begin{Shaded}
\begin{Highlighting}[]
\CommentTok{\# jne}
\end{Highlighting}
\end{Shaded}

Kui tahame kuupevast aastat, kuud, päeva vms:

\begin{Shaded}
\begin{Highlighting}[]
\NormalTok{kp }\OtherTok{\textless{}{-}} \FunctionTok{dmy}\NormalTok{(}\DecValTok{24032017}\NormalTok{)}

\FunctionTok{year}\NormalTok{(kp)}
\end{Highlighting}
\end{Shaded}

\begin{verbatim}
## [1] 2017
\end{verbatim}

\begin{Shaded}
\begin{Highlighting}[]
\FunctionTok{month}\NormalTok{(kp)}
\end{Highlighting}
\end{Shaded}

\begin{verbatim}
## [1] 3
\end{verbatim}

\begin{Shaded}
\begin{Highlighting}[]
\FunctionTok{week}\NormalTok{(kp)}
\end{Highlighting}
\end{Shaded}

\begin{verbatim}
## [1] 12
\end{verbatim}

\begin{Shaded}
\begin{Highlighting}[]
\FunctionTok{day}\NormalTok{(kp)}
\end{Highlighting}
\end{Shaded}

\begin{verbatim}
## [1] 24
\end{verbatim}

\begin{Shaded}
\begin{Highlighting}[]
\FunctionTok{wday}\NormalTok{(kp)}
\end{Highlighting}
\end{Shaded}

\begin{verbatim}
## [1] 6
\end{verbatim}

\begin{Shaded}
\begin{Highlighting}[]
\CommentTok{\# või}
\FunctionTok{wday}\NormalTok{(kp, }\AttributeTok{label =}\NormalTok{ T)}
\end{Highlighting}
\end{Shaded}

\begin{verbatim}
## [1] Fri
## Levels: Sun < Mon < Tue < Wed < Thu < Fri < Sat
\end{verbatim}

Praegune aeg:

\begin{Shaded}
\begin{Highlighting}[]
\FunctionTok{today}\NormalTok{()}
\end{Highlighting}
\end{Shaded}

\begin{verbatim}
## [1] "2022-03-30"
\end{verbatim}

\begin{Shaded}
\begin{Highlighting}[]
\FunctionTok{now}\NormalTok{()}
\end{Highlighting}
\end{Shaded}

\begin{verbatim}
## [1] "2022-03-30 22:58:03 EEST"
\end{verbatim}

Kestus:

\begin{Shaded}
\begin{Highlighting}[]
\CommentTok{\# Mitu sekundit kestab päev}
\FunctionTok{duration}\NormalTok{(}\AttributeTok{day =} \DecValTok{1}\NormalTok{)}
\end{Highlighting}
\end{Shaded}

\begin{verbatim}
## [1] "86400s (~1 days)"
\end{verbatim}

\begin{Shaded}
\begin{Highlighting}[]
\CommentTok{\# mitu sekundit kestab nädal}
\FunctionTok{duration}\NormalTok{(}\AttributeTok{week =} \DecValTok{1}\NormalTok{)}
\end{Highlighting}
\end{Shaded}

\begin{verbatim}
## [1] "604800s (~1 weeks)"
\end{verbatim}

\begin{Shaded}
\begin{Highlighting}[]
\CommentTok{\# aasta}
\FunctionTok{duration}\NormalTok{(}\DecValTok{1}\NormalTok{, }\StringTok{"year"}\NormalTok{)}
\end{Highlighting}
\end{Shaded}

\begin{verbatim}
## [1] "31557600s (~1 years)"
\end{verbatim}

Interval:

\begin{Shaded}
\begin{Highlighting}[]
\NormalTok{kp1 }\OtherTok{\textless{}{-}} \FunctionTok{dmy}\NormalTok{(}\DecValTok{24032017}\NormalTok{)}
\NormalTok{kp2 }\OtherTok{\textless{}{-}} \FunctionTok{dmy}\NormalTok{(}\DecValTok{26062017}\NormalTok{)}

\FunctionTok{interval}\NormalTok{(kp1, kp2)}
\end{Highlighting}
\end{Shaded}

\begin{verbatim}
## [1] 2017-03-24 UTC--2017-06-26 UTC
\end{verbatim}

\begin{Shaded}
\begin{Highlighting}[]
\CommentTok{\# või}
\NormalTok{kp1 }\SpecialCharTok{\%{-}{-}\%}\NormalTok{ kp2}
\end{Highlighting}
\end{Shaded}

\begin{verbatim}
## [1] 2017-03-24 UTC--2017-06-26 UTC
\end{verbatim}

\begin{Shaded}
\begin{Highlighting}[]
\CommentTok{\# mitu päeva interval kestab}
\NormalTok{kp1 }\SpecialCharTok{\%{-}{-}\%}\NormalTok{ kp2 }\SpecialCharTok{\%/\%} \FunctionTok{days}\NormalTok{(}\DecValTok{1}\NormalTok{)}
\end{Highlighting}
\end{Shaded}

\begin{verbatim}
## [1] 94
\end{verbatim}

\begin{Shaded}
\begin{Highlighting}[]
\CommentTok{\# Kas mingi kuupäev jääb intervalli sisse}
\FunctionTok{dmy}\NormalTok{(}\DecValTok{23032017}\NormalTok{) }\SpecialCharTok{\%within\%} \FunctionTok{interval}\NormalTok{(kp1, kp2)}
\end{Highlighting}
\end{Shaded}

\begin{verbatim}
## [1] FALSE
\end{verbatim}

\begin{Shaded}
\begin{Highlighting}[]
\FunctionTok{dmy}\NormalTok{(}\DecValTok{25032017}\NormalTok{) }\SpecialCharTok{\%within\%} \FunctionTok{interval}\NormalTok{(kp1, kp2)}
\end{Highlighting}
\end{Shaded}

\begin{verbatim}
## [1] TRUE
\end{verbatim}

\hypertarget{tekstilised-tunnused}{%
\subsection{Tekstilised tunnused}\label{tekstilised-tunnused}}

Tekstiliste tunnuste jaoks on pakett \emph{stringr}

\begin{Shaded}
\begin{Highlighting}[]
\FunctionTok{library}\NormalTok{(stringr)}

\CommentTok{\# Teeme vektori tekstidega (stringidega)}
\NormalTok{tekst }\OtherTok{\textless{}{-}} \FunctionTok{c}\NormalTok{(}\StringTok{"Tekstiliste"}\NormalTok{, }\StringTok{"tunnuste"}\NormalTok{, }\StringTok{"jaoks"}\NormalTok{, }\StringTok{"on"}\NormalTok{, }\StringTok{"pakett"}\NormalTok{, }\StringTok{"stringr"}\NormalTok{)}

\CommentTok{\# Mitu tähemärki on igas sõnas}
\FunctionTok{str\_length}\NormalTok{(tekst) }
\end{Highlighting}
\end{Shaded}

\begin{verbatim}
## [1] 11  8  5  2  6  7
\end{verbatim}

\begin{Shaded}
\begin{Highlighting}[]
\CommentTok{\# Paneme erinevad sõnad kokku}
\FunctionTok{str\_c}\NormalTok{(tekst, }\AttributeTok{collapse =} \StringTok{" "}\NormalTok{)}
\end{Highlighting}
\end{Shaded}

\begin{verbatim}
## [1] "Tekstiliste tunnuste jaoks on pakett stringr"
\end{verbatim}

\begin{Shaded}
\begin{Highlighting}[]
\CommentTok{\# võtame välja iga sõna esimese ja teise tähemärgi}
\FunctionTok{str\_sub}\NormalTok{(tekst, }\AttributeTok{start =} \DecValTok{1}\NormalTok{, }\AttributeTok{end =} \DecValTok{2}\NormalTok{)}
\end{Highlighting}
\end{Shaded}

\begin{verbatim}
## [1] "Te" "tu" "ja" "on" "pa" "st"
\end{verbatim}

\begin{Shaded}
\begin{Highlighting}[]
\CommentTok{\# Võtame välja sõnad, mis sisaldavad "t" tähte}
\FunctionTok{str\_subset}\NormalTok{(tekst, }\StringTok{"t"}\NormalTok{)}
\end{Highlighting}
\end{Shaded}

\begin{verbatim}
## [1] "Tekstiliste" "tunnuste"    "pakett"      "stringr"
\end{verbatim}

\begin{Shaded}
\begin{Highlighting}[]
\CommentTok{\# Võtame välja sõnad, mis sisaldavad "a" või "o" tähte}
\FunctionTok{str\_subset}\NormalTok{(tekst, }\StringTok{"[ao]"}\NormalTok{)}
\end{Highlighting}
\end{Shaded}

\begin{verbatim}
## [1] "jaoks"  "on"     "pakett"
\end{verbatim}

\begin{Shaded}
\begin{Highlighting}[]
\CommentTok{\# Kas sõnas on "a" või "o" täht}
\FunctionTok{str\_detect}\NormalTok{(tekst, }\StringTok{"[ao]"}\NormalTok{)}
\end{Highlighting}
\end{Shaded}

\begin{verbatim}
## [1] FALSE FALSE  TRUE  TRUE  TRUE FALSE
\end{verbatim}

\begin{Shaded}
\begin{Highlighting}[]
\CommentTok{\# Võtame sõnadest välja "te" tähekombinatsioonid}
\FunctionTok{str\_extract}\NormalTok{(tekst, }\StringTok{"te"}\NormalTok{)}
\end{Highlighting}
\end{Shaded}

\begin{verbatim}
## [1] "te" "te" NA   NA   NA   NA
\end{verbatim}

\begin{Shaded}
\begin{Highlighting}[]
\CommentTok{\# Mitu "t" tähte igas sõnas on}
\FunctionTok{str\_count}\NormalTok{(tekst, }\StringTok{"t"}\NormalTok{)}
\end{Highlighting}
\end{Shaded}

\begin{verbatim}
## [1] 2 2 0 0 2 1
\end{verbatim}

\begin{Shaded}
\begin{Highlighting}[]
\CommentTok{\# Asendame kõik "t" tähed "T" tähega}
\FunctionTok{str\_replace}\NormalTok{(tekst, }\StringTok{"t"}\NormalTok{, }\StringTok{"T"}\NormalTok{)}
\end{Highlighting}
\end{Shaded}

\begin{verbatim}
## [1] "TeksTiliste" "Tunnuste"    "jaoks"       "on"          "pakeTt"     
## [6] "sTringr"
\end{verbatim}

\begin{Shaded}
\begin{Highlighting}[]
\CommentTok{\# Nagu näha, siis asendati ainult sõna esimene "t" täht}
\CommentTok{\# Kui tahame kõik "t" tähed asendada, siis:}
\FunctionTok{str\_replace\_all}\NormalTok{(tekst, }\StringTok{"t"}\NormalTok{, }\StringTok{"T"}\NormalTok{)}
\end{Highlighting}
\end{Shaded}

\begin{verbatim}
## [1] "TeksTilisTe" "TunnusTe"    "jaoks"       "on"          "pakeTT"     
## [6] "sTringr"
\end{verbatim}

\hypertarget{r-markdown}{%
\section{R markdown}\label{r-markdown}}

R markdown teeb tulemuste esitamise (koos koodiga) või raporti tegemise väga lihtsaks. Kõigepealt on vaja installida pakett ``rmarkdown'' (RStudio'ga tuleb see defaultis kaasa). Seejärel saame scriptifaili asemel valida \emph{markdown}'i dokumendi: File \textgreater{} New File \textgreater{} R Markdown. Saab valida formaadi, mida väljundina tahame saada (html, pdf, word). Avaneb markdown'i dokument, millesse saab kirjutada nii tavalist teksti kui ka R'i koodi, ning mille väljundis sisalduvad (kui me seda muidugi tahame) ka analüüsitulemused. Väljundi loomiseks tuleb vajutada Knit nuppu.

Lisainfo jaoks võite vaadata:\\
Help \textgreater{} Markdown Quick Reference\\
\url{http://rmarkdown.rstudio.com/}

\hypertarget{andmegraafika}{%
\section{Andmegraafika}\label{andmegraafika}}

Edward Tufte, üks andmegraafika legende, kirjeldab oma raamatutes \emph{Beautiful Evidence} ja \emph{The Visual Display of Quantitative Information} peamisi andmegraafika põhialuseid:

\begin{itemize}
\tightlist
\item
  Graafikul esitatud tunnuste representatsioonid peavad olema proportsionaalsed mõõdetud tunnustega reaalses maailmas
\item
  Graafikul esitatule peab olema selge, detailne ja läbiv tähistus ning selgitus
\item
  Esita andmete varieerumist, mitte graafiku disainist tulenevat varieerumist
\item
  Informatsiooni edastavaid dimensioone ei tohiks esitada rohkem kui andmed seda võimaldavad (3D tulpdiagrammid on saatanast)
\item
  Graafik peab edastama ainult andmetest lähtuvat informatsiooni (mida saaks graafikul kustudada, ilma et selle infoedastusvõime kannataks?)
\item
  Võrdlusmoment

  \begin{itemize}
  \tightlist
  \item
    Mingi kvantiteet (keskmine, sagedus, hajuvusnäitaja jne) omab mõtet vaid suhestudes mingi teise kvantiteedi või referentsiga
  \item
    Sisukas hüpotees võrdluses nullhüpoteesiga
  \end{itemize}
\item
  Mitmemõõtmelisus

  \begin{itemize}
  \tightlist
  \item
    Maailma on alati mitmemõõtmeline
  \item
    Näita võimalikult palju andmeid (aga mitte rohkem kui võimalik)
  \end{itemize}
\item
  Esitatud andmed peavad olema olulised (mõttekad)
\end{itemize}

Millekes üldse graafikud?

\begin{itemize}
\tightlist
\item
  Andmete mõistmine
\item
  Mustrite leidmine
\item
  Vigade leidmine
\item
  \textbf{Tulemuste kommunikeerimine}
\end{itemize}

\hypertarget{ri-baasgraafikud}{%
\subsection{R'i baasgraafikud}\label{ri-baasgraafikud}}

Baas-R'is on väga võimekas graafikamootor, millega on võimalik väga ilusaid ja sisukaid graafikuid teha. Tänapäeval kasutab aga enamik andmeanalüütikuid baas-R'i asemel paketti ``ggplot2'', kus on jooniste tegemine muudetud mõnevõrra lihtsamaks, loogilisemaks ja võimalusterohkemaks. Kuid, et oleksite vähemalt tuttav ka baas-R'i võimalustega, vaatame kiirelt üle ka selles leiduvad võimalused.

\begin{Shaded}
\begin{Highlighting}[]
\NormalTok{dt }\OtherTok{\textless{}{-}}\NormalTok{ iris }\CommentTok{\#Kasutame Iris\textquotesingle{}e näidisandmestikku}
\FunctionTok{names}\NormalTok{(dt) }\OtherTok{\textless{}{-}} \FunctionTok{tolower}\NormalTok{(}\FunctionTok{names}\NormalTok{(dt))}
\end{Highlighting}
\end{Shaded}

\textbf{Scatterplot}

\begin{Shaded}
\begin{Highlighting}[]
\FunctionTok{plot}\NormalTok{(}\AttributeTok{x =}\NormalTok{ dt}\SpecialCharTok{$}\NormalTok{sepal.length, }\AttributeTok{y =}\NormalTok{ dt}\SpecialCharTok{$}\NormalTok{sepal.width)}
\end{Highlighting}
\end{Shaded}

\begin{center}\includegraphics{00-sissejuhatus-ri_files/figure-latex/unnamed-chunk-110-1} \end{center}

Saab kasutada ka ainult ühte argumenti.

\begin{Shaded}
\begin{Highlighting}[]
\FunctionTok{plot}\NormalTok{(}\AttributeTok{x =}\NormalTok{ dt}\SpecialCharTok{$}\NormalTok{sepal.length)}
\end{Highlighting}
\end{Shaded}

\begin{center}\includegraphics{00-sissejuhatus-ri_files/figure-latex/unnamed-chunk-111-1} \end{center}

Argumendiga \emph{type=} saab määrata graafiku tüübi. Näiteks ``l'' joongraafik, ``b'' jooned ja sümolid koos jne (vaata ?plot).

\begin{Shaded}
\begin{Highlighting}[]
\FunctionTok{plot}\NormalTok{(dt}\SpecialCharTok{$}\NormalTok{sepal.length, }\AttributeTok{type=} \StringTok{"b"}\NormalTok{)}
\end{Highlighting}
\end{Shaded}

\begin{center}\includegraphics{00-sissejuhatus-ri_files/figure-latex/unnamed-chunk-112-1} \end{center}

\textbf{Histogram}

\begin{Shaded}
\begin{Highlighting}[]
\FunctionTok{hist}\NormalTok{(}\AttributeTok{x =}\NormalTok{ dt}\SpecialCharTok{$}\NormalTok{sepal.length)}
\end{Highlighting}
\end{Shaded}

\begin{center}\includegraphics{00-sissejuhatus-ri_files/figure-latex/unnamed-chunk-113-1} \end{center}

\begin{Shaded}
\begin{Highlighting}[]
\FunctionTok{hist}\NormalTok{(}\AttributeTok{x =}\NormalTok{ dt}\SpecialCharTok{$}\NormalTok{sepal.length, }\AttributeTok{breaks =} \DecValTok{20}\NormalTok{)}
\end{Highlighting}
\end{Shaded}

\begin{center}\includegraphics{00-sissejuhatus-ri_files/figure-latex/unnamed-chunk-114-1} \end{center}

\textbf{Boxplot}

Ühele grupile

\begin{Shaded}
\begin{Highlighting}[]
\FunctionTok{boxplot}\NormalTok{(dt}\SpecialCharTok{$}\NormalTok{sepal.length)}
\end{Highlighting}
\end{Shaded}

\begin{center}\includegraphics{00-sissejuhatus-ri_files/figure-latex/unnamed-chunk-115-1} \end{center}

Mitme grupi lõikes peab kasutama \emph{formula} märki (\textasciitilde)

\begin{Shaded}
\begin{Highlighting}[]
\FunctionTok{boxplot}\NormalTok{(dt}\SpecialCharTok{$}\NormalTok{sepal.length }\SpecialCharTok{\textasciitilde{}}\NormalTok{ dt}\SpecialCharTok{$}\NormalTok{species)}
\end{Highlighting}
\end{Shaded}

\begin{center}\includegraphics{00-sissejuhatus-ri_files/figure-latex/unnamed-chunk-116-1} \end{center}

\textbf{Barplot}

Barplot'i jaoks on sisendiks vaja tabelit

\begin{Shaded}
\begin{Highlighting}[]
\FunctionTok{library}\NormalTok{(dplyr)}
\NormalTok{d\_bar }\OtherTok{\textless{}{-}}\NormalTok{ dt}\SpecialCharTok{\%\textgreater{}\%}
  \FunctionTok{filter}\NormalTok{(sepal.length}\SpecialCharTok{\textgreater{}}\FloatTok{5.5}\NormalTok{)}\SpecialCharTok{\%\textgreater{}\%}
  \FunctionTok{select}\NormalTok{(species)}\SpecialCharTok{\%\textgreater{}\%}
  \FunctionTok{table}\NormalTok{()}
\NormalTok{d\_bar}
\end{Highlighting}
\end{Shaded}

\begin{verbatim}
## .
##     setosa versicolor  virginica 
##          3         39         49
\end{verbatim}

\begin{Shaded}
\begin{Highlighting}[]
\FunctionTok{barplot}\NormalTok{(d\_bar)}
\end{Highlighting}
\end{Shaded}

\begin{center}\includegraphics{00-sissejuhatus-ri_files/figure-latex/unnamed-chunk-118-1} \end{center}

\textbf{Baasgraafikute parameetrid}

\begin{itemize}
\tightlist
\item
  pch: graafikul esitatv sümbol (vaikimisi ring)
\item
  lty: joone tüüp (vaikimisi tavaline)
\item
  ldw: joone laius (numbriline vaikimisi 1)
\item
  col: värv (colors() funktsiooniga näeb võimalikke värve)
\item
  xlab ja ylab: telgede nimed (tekstiline väärtus)
\item
  xlim ja ylim: telgede limiidid (kui on vaja neid suurenda või vähendada)
\end{itemize}

\begin{Shaded}
\begin{Highlighting}[]
\FunctionTok{plot}\NormalTok{(}\AttributeTok{x =}\NormalTok{ dt}\SpecialCharTok{$}\NormalTok{sepal.length, }\AttributeTok{y =}\NormalTok{ dt}\SpecialCharTok{$}\NormalTok{sepal.width,}
     \AttributeTok{pch=}\DecValTok{2}\NormalTok{,}
     \AttributeTok{col=}\StringTok{"red"}\NormalTok{,}
     \AttributeTok{xlab=}\StringTok{"Sepal length"}\NormalTok{,}
     \AttributeTok{ylab=}\StringTok{"Sepal width"}\NormalTok{)}
\end{Highlighting}
\end{Shaded}

\begin{center}\includegraphics{00-sissejuhatus-ri_files/figure-latex/unnamed-chunk-119-1} \end{center}

\texttt{par()} funktsiooniga saab seada \emph{globaalseid} parameetrieid. Näiteks saab panna mitu graafikut üksteise kõrvale:

\begin{Shaded}
\begin{Highlighting}[]
\FunctionTok{par}\NormalTok{(}\AttributeTok{mfrow=}\FunctionTok{c}\NormalTok{(}\DecValTok{1}\NormalTok{,}\DecValTok{2}\NormalTok{))}
\end{Highlighting}
\end{Shaded}

Mis värviparameeter on vaikimisi \emph{globaalselt} määratletud?

\begin{Shaded}
\begin{Highlighting}[]
\FunctionTok{par}\NormalTok{(}\StringTok{"col"}\NormalTok{)}
\end{Highlighting}
\end{Shaded}

\begin{verbatim}
## [1] "black"
\end{verbatim}

\textbf{Baasgraafikute ehitamine}

Graafikutele saab lisada erinevaid komponente või ka teisi graafikuid. nii on võimalik vajalik graafik kokku ehitada. Mõned võimalused:

\begin{itemize}
\tightlist
\item
  lines() joonte lisamine
\item
  points() punktide lisamine
\item
  text() teksti lisamine
\item
  title() palkirja lisamine
\item
  legend() legendi lisamine
\end{itemize}

\begin{Shaded}
\begin{Highlighting}[]
\FunctionTok{plot}\NormalTok{(}\AttributeTok{x =}\NormalTok{ dt}\SpecialCharTok{$}\NormalTok{sepal.length, }\AttributeTok{y =}\NormalTok{ dt}\SpecialCharTok{$}\NormalTok{sepal.width,}
     \AttributeTok{xlab=}\StringTok{"Sepal length"}\NormalTok{,}
     \AttributeTok{ylab=}\StringTok{"Sepal width"}\NormalTok{)}
\NormalTok{dt2 }\OtherTok{\textless{}{-}}\NormalTok{ dt }\SpecialCharTok{\%\textgreater{}\%}
  \FunctionTok{filter}\NormalTok{(species}\SpecialCharTok{==}\StringTok{"setosa"}\NormalTok{)}
\FunctionTok{points}\NormalTok{(dt2}\SpecialCharTok{$}\NormalTok{sepal.length, dt2}\SpecialCharTok{$}\NormalTok{sepal.width, }\AttributeTok{col=}\StringTok{"red"}\NormalTok{)}
\FunctionTok{legend}\NormalTok{(}\StringTok{"topright"}\NormalTok{, }\AttributeTok{pch=}\DecValTok{1}\NormalTok{, }\AttributeTok{col=}\FunctionTok{c}\NormalTok{(}\StringTok{"black"}\NormalTok{, }\StringTok{"red"}\NormalTok{),}\AttributeTok{legend =} \FunctionTok{c}\NormalTok{(}\StringTok{"muu"}\NormalTok{, }\StringTok{"seotsa"}\NormalTok{))}
\FunctionTok{title}\NormalTok{(}\AttributeTok{main =} \StringTok{"Pealkiri"}\NormalTok{)}
\end{Highlighting}
\end{Shaded}

\begin{center}\includegraphics{00-sissejuhatus-ri_files/figure-latex/unnamed-chunk-122-1} \end{center}

\hypertarget{ggplot}{%
\subsection{ggplot}\label{ggplot}}

ggplot'i lähtekohaks on Leland Wilkinsoni \emph{graafika grammatika}, mis lähtub põhimõttest, et graafiku võib lahutada eraldiseisvateks komponentideks ja neist komponentidest saab saab uusi tervikuid ehitada.
\textgreater{} ``\ldots{} the grammar tells us that a statistical graphic is a \textbf{mapping from data} to
\textgreater{} \textbf{aesthetic attributes} (colour, shape, size) of \textbf{geometric objects} (points,
\textgreater{} lines, bars). The plot may also contain \textbf{statistical transformations} of the data
\textgreater{} and is drawn on a specific \textbf{coordinate system}. \textbf{Facetting} can be used to generate
\textgreater{} the same plot for diferent subsets of the dataset. It is the combination of these
\textgreater{} independent components that make up a graphic.''
\textgreater{} (Hadley Wickham, ``ggplot2: Elegant Graphics for Data Analysis'')

\textbf{ggploti elemendid}

\begin{itemize}
\tightlist
\item
  \texttt{data} : andmed. Üldiselt peaks olema dataframe kujul
\item
  \texttt{geom} : geomeetriline objekt, mille läbi me oma anmdeid esitame (punktid, jooned, tulbad jne)
\item
  \texttt{aes} : \emph{aesthetic} ehk siis kuidas ja mille läbi me oma andmeid geomeetriliste objektidega suhestame (mis on x ja y telg, värv, kuju, suurus). Need on joonise objektide visuaalsed omadused
\item
  \texttt{facet} : tahud ehk kuidas joonist alamosadeks (tahkudeks) jagada
\item
  \texttt{stat} : milliseid statistilisi transformatsioone on vaja kasutada
\item
  \texttt{scales} : kuidas andmete reaalsed väärtused joonise väärtusteks tõlgendatakse
\item
  \texttt{coord} : mis koordinaatsüsteemi kasutada. Üldiselt \texttt{cartesian}
\item
  \texttt{positsion} : andmeobjektide positsioonide nihutamine
\item
  \texttt{guides} : teljed, legendid jne
\item
  \texttt{theme} : joonise üldine kujundus (kus asub legend, mis värvi on tagapõhi jne)
\end{itemize}

Installime ggplot'i (kui me seda jua teinud ei ole) ja laadime selleks sessiooniks. Üritame teha regressioonijoontega \emph{scatterploti}.

\begin{Shaded}
\begin{Highlighting}[]
\CommentTok{\#install.packages("ggplot2")}
\FunctionTok{library}\NormalTok{(ggplot2)}
\NormalTok{dt }\OtherTok{\textless{}{-}}\NormalTok{ iris}
\FunctionTok{names}\NormalTok{(dt) }\OtherTok{\textless{}{-}} \FunctionTok{tolower}\NormalTok{(}\FunctionTok{names}\NormalTok{(dt))}
\end{Highlighting}
\end{Shaded}

\textbf{Scatterplot}

Kõigepealt \texttt{ggplot}i peafunktsioon, kus märgime andmestiku (tegelikult võime seda teha ka \emph{geom}'i sees). Seejärel lisame \emph{geom}'i kihi. Liidame selle peafunktsioonile otsa (kasutatdes \texttt{+} märki). Tahame punktdiagrammi, seega \texttt{geom\_point} (et saada aimu erinevatest võimalikest \emph{geom}'idest, võib uurida ggplot'i kodulehte \url{https://ggplot2.tidyverse.org/reference/} või \emph{cheatsheet}'i (Help \textgreater{} Cheatsheets \textgreater{} Data visualization with ggplot2)). Defineerime \texttt{aes}\emph{thetic}'u ehk siis \emph{mapime} tunnused x ja y teljele.

\begin{Shaded}
\begin{Highlighting}[]
\FunctionTok{ggplot}\NormalTok{(}\AttributeTok{data=}\NormalTok{dt)}\SpecialCharTok{+}
  \FunctionTok{geom\_point}\NormalTok{(}\AttributeTok{mapping =} \FunctionTok{aes}\NormalTok{(}\AttributeTok{x=}\NormalTok{sepal.width, }\AttributeTok{y=}\NormalTok{sepal.length))}
\end{Highlighting}
\end{Shaded}

\begin{center}\includegraphics{00-sissejuhatus-ri_files/figure-latex/unnamed-chunk-124-1} \end{center}

Tegelikult ei pea argumente välja kirjutama, vaid järjekord on tähtis. Saab ka nii:

\begin{Shaded}
\begin{Highlighting}[]
\FunctionTok{ggplot}\NormalTok{(dt)}\SpecialCharTok{+}
  \FunctionTok{geom\_point}\NormalTok{(}\FunctionTok{aes}\NormalTok{(sepal.width, sepal.length))}
\end{Highlighting}
\end{Shaded}

Tahame erinevad iirise liigid erinevate värvidega grupeerida. Kuna me tahame määrata seda, kuidas andmeid esitatakse (tunnuseid graafikule \emph{mapitakse}), peame seda tegema \texttt{aes}'i argumendi sees.

\begin{Shaded}
\begin{Highlighting}[]
\FunctionTok{ggplot}\NormalTok{(}\AttributeTok{data=}\NormalTok{dt)}\SpecialCharTok{+}
  \FunctionTok{geom\_point}\NormalTok{(}\AttributeTok{mapping =} \FunctionTok{aes}\NormalTok{(}\AttributeTok{x=}\NormalTok{sepal.width, }\AttributeTok{y=}\NormalTok{sepal.length, }\AttributeTok{color=}\NormalTok{species))}
\end{Highlighting}
\end{Shaded}

\begin{center}\includegraphics{00-sissejuhatus-ri_files/figure-latex/unnamed-chunk-126-1} \end{center}

Saaksime neid eristada ka näiteks kuju \texttt{shape=} või suuruse \texttt{size=} või ka läbipaistvuse \texttt{alpha=} järgi.

\begin{Shaded}
\begin{Highlighting}[]
\FunctionTok{ggplot}\NormalTok{(}\AttributeTok{data=}\NormalTok{dt)}\SpecialCharTok{+}
  \FunctionTok{geom\_point}\NormalTok{(}\AttributeTok{mapping =} \FunctionTok{aes}\NormalTok{(}\AttributeTok{x=}\NormalTok{sepal.width, }\AttributeTok{y=}\NormalTok{sepal.length, }\AttributeTok{shape=}\NormalTok{species))}
\end{Highlighting}
\end{Shaded}

\begin{center}\includegraphics{00-sissejuhatus-ri_files/figure-latex/unnamed-chunk-127-1} \end{center}

Kui me tahame lihtsalt punktide värvi muuta (ja mitte lähtuda mingist grupeerivast tunnusest), saame seda teha väljaspool \texttt{aes()} argumenti.

\begin{Shaded}
\begin{Highlighting}[]
\FunctionTok{ggplot}\NormalTok{(}\AttributeTok{data=}\NormalTok{dt)}\SpecialCharTok{+}
  \FunctionTok{geom\_point}\NormalTok{(}\AttributeTok{mapping =} \FunctionTok{aes}\NormalTok{(}\AttributeTok{x=}\NormalTok{sepal.width, }\AttributeTok{y=}\NormalTok{sepal.length), }\AttributeTok{color=}\StringTok{"green"}\NormalTok{)}
\end{Highlighting}
\end{Shaded}

\begin{center}\includegraphics{00-sissejuhatus-ri_files/figure-latex/unnamed-chunk-128-1} \end{center}

Oleks vaja joonisele ka regressioonijooned saada. Selleks lisame lihtsalt järgmise kihi (kasutades \texttt{+} märki).\\
Regressioonijoone joonistamiseks võime kasutada \texttt{geom\_abline()}'i, aga sellisel juhul peame regressioonikoefitsiendid enne välja arvutama (\texttt{geam\_abline()} vajab sisendiks \emph{intercept}'i ning \emph{slope}'i). Lihtsam on kasutada \texttt{geom\_smooth()}'i.

\begin{Shaded}
\begin{Highlighting}[]
\FunctionTok{ggplot}\NormalTok{(}\AttributeTok{data=}\NormalTok{dt)}\SpecialCharTok{+}
  \FunctionTok{geom\_point}\NormalTok{(}\AttributeTok{mapping =} \FunctionTok{aes}\NormalTok{(}\AttributeTok{x=}\NormalTok{sepal.width, }\AttributeTok{y=}\NormalTok{sepal.length, }\AttributeTok{color=}\NormalTok{species))}\SpecialCharTok{+}
  \FunctionTok{geom\_smooth}\NormalTok{(}\FunctionTok{aes}\NormalTok{(}\AttributeTok{x=}\NormalTok{sepal.width, }\AttributeTok{y=}\NormalTok{sepal.length))}
\end{Highlighting}
\end{Shaded}

\begin{verbatim}
## `geom_smooth()` using method = 'loess' and formula 'y ~ x'
\end{verbatim}

\begin{center}\includegraphics{00-sissejuhatus-ri_files/figure-latex/unnamed-chunk-129-1} \end{center}

Mhh, joon ei meenuta regressioonijoont. Asi on selles, et tegemist on küll regressioonijoonega, kuid mitte harjumuspärase lineaarse regressioonijoonega. \texttt{geom\_smooth} kasutab vaikimisi nn \emph{Local Polynomial Regression Fitting}'ut, mis üritab \emph{predictida} y väärtuse sõltuvust x'i väärtusest võimalikult täpselt ja lähtudes x'i lähiümbrusest. Aga saame tellida ka tavalise lineaarse regressioonijoone, kasutades argumenti \texttt{method="lm"}.

\begin{Shaded}
\begin{Highlighting}[]
\FunctionTok{ggplot}\NormalTok{(}\AttributeTok{data=}\NormalTok{dt)}\SpecialCharTok{+}
  \FunctionTok{geom\_point}\NormalTok{(}\AttributeTok{mapping =} \FunctionTok{aes}\NormalTok{(}\AttributeTok{x=}\NormalTok{sepal.width, }\AttributeTok{y=}\NormalTok{sepal.length, }\AttributeTok{color=}\NormalTok{species))}\SpecialCharTok{+}
  \FunctionTok{geom\_smooth}\NormalTok{(}\FunctionTok{aes}\NormalTok{(}\AttributeTok{x=}\NormalTok{sepal.width, }\AttributeTok{y=}\NormalTok{sepal.length), }\AttributeTok{method=}\StringTok{"lm"}\NormalTok{)}
\end{Highlighting}
\end{Shaded}

\begin{verbatim}
## `geom_smooth()` using formula 'y ~ x'
\end{verbatim}

\begin{center}\includegraphics{00-sissejuhatus-ri_files/figure-latex/unnamed-chunk-130-1} \end{center}

Kõikide iirise liikide kohta eraldi joonte saamiseks tuleb jällegi määrata grupeerimine \texttt{geom\_smooth()}'i \texttt{aes()}'i sees (kuna see on eraldi kiht ja eelmise kihi määrangud siin enam ei kehti). Kui me usaldusintervalle mingil põhjusel ei taha, võime need tühistada argumendiga \texttt{se=F}.

\begin{Shaded}
\begin{Highlighting}[]
\FunctionTok{ggplot}\NormalTok{(}\AttributeTok{data=}\NormalTok{dt)}\SpecialCharTok{+}
  \FunctionTok{geom\_point}\NormalTok{(}\AttributeTok{mapping =} \FunctionTok{aes}\NormalTok{(}\AttributeTok{x=}\NormalTok{sepal.width, }\AttributeTok{y=}\NormalTok{sepal.length, }\AttributeTok{color=}\NormalTok{species))}\SpecialCharTok{+}
  \FunctionTok{geom\_smooth}\NormalTok{(}\FunctionTok{aes}\NormalTok{(}\AttributeTok{x=}\NormalTok{sepal.width, }\AttributeTok{y=}\NormalTok{sepal.length, }\AttributeTok{color=}\NormalTok{species), }\AttributeTok{method=}\StringTok{"lm"}\NormalTok{, }\AttributeTok{se=}\NormalTok{F)}
\end{Highlighting}
\end{Shaded}

\begin{verbatim}
## `geom_smooth()` using formula 'y ~ x'
\end{verbatim}

\begin{center}\includegraphics{00-sissejuhatus-ri_files/figure-latex/unnamed-chunk-131-1} \end{center}

Pidime \texttt{aes()} argumendi määrangud kaks korda järjest välja kirjutama, kuigi nad on identsed. Ei tundu väga mõistlik. Õnneks saab ka lihtsamalt. Võime need määrata ka \texttt{ggplot()} funktsiooni enda sees. Sellisel juhul kehtivad nad ka kõikide järgenvate kihtide kohta (välja arvatud juhul kui järgenvates kihtides on teisiti määratud).

\begin{Shaded}
\begin{Highlighting}[]
\FunctionTok{ggplot}\NormalTok{(dt, }\FunctionTok{aes}\NormalTok{(sepal.width, sepal.length, }\AttributeTok{color=}\NormalTok{species))}\SpecialCharTok{+}
  \FunctionTok{geom\_point}\NormalTok{()}\SpecialCharTok{+}
  \FunctionTok{geom\_smooth}\NormalTok{(}\AttributeTok{method=}\StringTok{"lm"}\NormalTok{, }\AttributeTok{se=}\NormalTok{F)}
\end{Highlighting}
\end{Shaded}

\textbf{Facetid}

Gruppe saab eristada ka neid erinevatele tahkudele pannes, kasutades selleks \texttt{facet\_wrap()}'i või \texttt{facet\_grid}'i. \texttt{facet\_wrap()} eristab ühe tunnuse lõikes, \texttt{facet\_grid()} mitme tunnuse lõikes. Mõlema puhul tuleb kasutada \emph{formula} määrangut, st. tuleb kasutada \textasciitilde{} märki (tegelikult ggplot'i viimase versiooni puhul saame kasutada ka argumente \texttt{rows=} ja \texttt{cols=}).

\begin{Shaded}
\begin{Highlighting}[]
\FunctionTok{ggplot}\NormalTok{(dt, }\FunctionTok{aes}\NormalTok{(sepal.width, sepal.length))}\SpecialCharTok{+}
  \FunctionTok{geom\_point}\NormalTok{()}\SpecialCharTok{+}
  \FunctionTok{geom\_smooth}\NormalTok{(}\AttributeTok{method=}\StringTok{"lm"}\NormalTok{, }\AttributeTok{se=}\NormalTok{F)}\SpecialCharTok{+}
  \FunctionTok{facet\_wrap}\NormalTok{(}\SpecialCharTok{\textasciitilde{}}\NormalTok{species)}
\end{Highlighting}
\end{Shaded}

\begin{verbatim}
## `geom_smooth()` using formula 'y ~ x'
\end{verbatim}

\begin{center}\includegraphics{00-sissejuhatus-ri_files/figure-latex/unnamed-chunk-133-1} \end{center}

Kui tahame tahkusid näiteks ainult kahes tulbas, saame kasutada argumenti \texttt{nrow=} või \texttt{ncol=}.

\begin{Shaded}
\begin{Highlighting}[]
\FunctionTok{ggplot}\NormalTok{(dt, }\FunctionTok{aes}\NormalTok{(sepal.width, sepal.length))}\SpecialCharTok{+}
  \FunctionTok{geom\_point}\NormalTok{()}\SpecialCharTok{+}
  \FunctionTok{geom\_smooth}\NormalTok{(}\AttributeTok{method=}\StringTok{"lm"}\NormalTok{, }\AttributeTok{se=}\NormalTok{F)}\SpecialCharTok{+}
  \FunctionTok{facet\_wrap}\NormalTok{(}\SpecialCharTok{\textasciitilde{}}\NormalTok{species, }\AttributeTok{ncol=}\DecValTok{2}\NormalTok{)}
\end{Highlighting}
\end{Shaded}

\begin{verbatim}
## `geom_smooth()` using formula 'y ~ x'
\end{verbatim}

\begin{center}\includegraphics{00-sissejuhatus-ri_files/figure-latex/unnamed-chunk-134-1} \end{center}

\texttt{facet\_grid()}'i ja kahe tunnuse lõikes tahkude illustreerimiseks meil \emph{iris}'e andmestikus piisavalt kategoriaalseid tunnuseid ei ole. Aga ggplotiga tuleb kaasa \texttt{diamonds} andmebaas. Vaatame seda:

\begin{Shaded}
\begin{Highlighting}[]
\NormalTok{dt1 }\OtherTok{\textless{}{-}}\NormalTok{ diamonds}
\FunctionTok{str}\NormalTok{(dt1)}
\end{Highlighting}
\end{Shaded}

\begin{verbatim}
## tibble [53,940 x 10] (S3: tbl_df/tbl/data.frame)
##  $ carat  : num [1:53940] 0.23 0.21 0.23 0.29 0.31 0.24 0.24 0.26 0.22 0.23 ...
##  $ cut    : Ord.factor w/ 5 levels "Fair"<"Good"<..: 5 4 2 4 2 3 3 3 1 3 ...
##  $ color  : Ord.factor w/ 7 levels "D"<"E"<"F"<"G"<..: 2 2 2 6 7 7 6 5 2 5 ...
##  $ clarity: Ord.factor w/ 8 levels "I1"<"SI2"<"SI1"<..: 2 3 5 4 2 6 7 3 4 5 ...
##  $ depth  : num [1:53940] 61.5 59.8 56.9 62.4 63.3 62.8 62.3 61.9 65.1 59.4 ...
##  $ table  : num [1:53940] 55 61 65 58 58 57 57 55 61 61 ...
##  $ price  : int [1:53940] 326 326 327 334 335 336 336 337 337 338 ...
##  $ x      : num [1:53940] 3.95 3.89 4.05 4.2 4.34 3.94 3.95 4.07 3.87 4 ...
##  $ y      : num [1:53940] 3.98 3.84 4.07 4.23 4.35 3.96 3.98 4.11 3.78 4.05 ...
##  $ z      : num [1:53940] 2.43 2.31 2.31 2.63 2.75 2.48 2.47 2.53 2.49 2.39 ...
\end{verbatim}

Kuidas on seotud teemandite karaadid (\emph{carat}) ja nende hind (\emph{price})?

\begin{Shaded}
\begin{Highlighting}[]
\FunctionTok{ggplot}\NormalTok{(dt1)}\SpecialCharTok{+}
  \FunctionTok{geom\_point}\NormalTok{(}\AttributeTok{mapping =} \FunctionTok{aes}\NormalTok{(}\AttributeTok{x=}\NormalTok{carat, }\AttributeTok{y=}\NormalTok{price))}
\end{Highlighting}
\end{Shaded}

\begin{center}\includegraphics{00-sissejuhatus-ri_files/figure-latex/unnamed-chunk-136-1} \end{center}

Kuidas siia suhestub teemandite selgus (\emph{clarity})?

\begin{Shaded}
\begin{Highlighting}[]
\FunctionTok{ggplot}\NormalTok{(dt1)}\SpecialCharTok{+}
  \FunctionTok{geom\_point}\NormalTok{(}\AttributeTok{mapping =} \FunctionTok{aes}\NormalTok{(}\AttributeTok{x=}\NormalTok{carat, }\AttributeTok{y=}\NormalTok{price, }\AttributeTok{color=}\NormalTok{clarity))}
\end{Highlighting}
\end{Shaded}

\begin{center}\includegraphics{00-sissejuhatus-ri_files/figure-latex/unnamed-chunk-137-1} \end{center}

Aga nende lõige \emph{cut}? kasutame selleks \texttt{facet\_grid()}'i. Kui tahame \texttt{facet\_grid()}'iga ainult ühe tunnuse lõikes tahke tekitada, tuleb teise tunnuse asemel kasuatada punkti. Seda, kas tahud on tulbas või reas, saab määrata sellega, kuhupoole \textasciitilde{} märki punkt panna.

\begin{Shaded}
\begin{Highlighting}[]
\FunctionTok{ggplot}\NormalTok{(dt1)}\SpecialCharTok{+}
  \FunctionTok{geom\_point}\NormalTok{(}\AttributeTok{mapping =} \FunctionTok{aes}\NormalTok{(}\AttributeTok{x=}\NormalTok{carat, }\AttributeTok{y=}\NormalTok{price, }\AttributeTok{color=}\NormalTok{clarity))}\SpecialCharTok{+}
  \FunctionTok{facet\_grid}\NormalTok{(cut }\SpecialCharTok{\textasciitilde{}}\NormalTok{ .)}
\end{Highlighting}
\end{Shaded}

\begin{center}\includegraphics{00-sissejuhatus-ri_files/figure-latex/unnamed-chunk-138-1} \end{center}

Lisame veel phe tunnuse, mille lõikes teemantide erisusi vaadata, värvi (\emph{color}).

\begin{Shaded}
\begin{Highlighting}[]
\FunctionTok{ggplot}\NormalTok{(dt1)}\SpecialCharTok{+}
  \FunctionTok{geom\_point}\NormalTok{(}\AttributeTok{mapping =} \FunctionTok{aes}\NormalTok{(}\AttributeTok{x=}\NormalTok{carat, }\AttributeTok{y=}\NormalTok{price, }\AttributeTok{color=}\NormalTok{clarity))}\SpecialCharTok{+}
  \FunctionTok{facet\_grid}\NormalTok{(cut }\SpecialCharTok{\textasciitilde{}}\NormalTok{ color)}
\end{Highlighting}
\end{Shaded}

\begin{center}\includegraphics{00-sissejuhatus-ri_files/figure-latex/unnamed-chunk-139-1} \end{center}

\textbf{Geomid ja aestetikud}

Erinevaid \texttt{geom}'e on päris palju. Kõik nad on üles loetletud ggplot'i kodulehel (koos suure hulga muu infoga): \url{https://ggplot2.tidyverse.org/reference/}. Aga mõned olulisemad:

\begin{itemize}
\tightlist
\item
  \texttt{geom\_bar()}
\item
  \texttt{geom\_histogram()} ja \texttt{geom\_freqpoly()}
\item
  \texttt{geom\_boxplot()} ja \texttt{geom\_violin()}
\item
  \texttt{geom\_line()} ja \texttt{geom\_path()}
\item
  \texttt{geom\_density()}
\item
  \texttt{geom\_abline()}, \texttt{geom\_hline()} ja \texttt{geom\_vline()}
\item
  \texttt{geom\_text()}
\end{itemize}

\textbf{Barplot}

Tavaline \emph{barplot}

\begin{Shaded}
\begin{Highlighting}[]
\FunctionTok{ggplot}\NormalTok{(dt1)}\SpecialCharTok{+}
  \FunctionTok{geom\_bar}\NormalTok{(}\FunctionTok{aes}\NormalTok{(clarity))}
\end{Highlighting}
\end{Shaded}

\begin{center}\includegraphics{00-sissejuhatus-ri_files/figure-latex/unnamed-chunk-140-1} \end{center}

Kahe tunnuse lõikes tulpdiagrammi jaoks peame kasutama \texttt{aes()} sees argumenti \texttt{fill=}.

\begin{Shaded}
\begin{Highlighting}[]
\FunctionTok{ggplot}\NormalTok{(dt1)}\SpecialCharTok{+}
  \FunctionTok{geom\_bar}\NormalTok{(}\FunctionTok{aes}\NormalTok{(clarity, }\AttributeTok{fill=}\NormalTok{cut))}
\end{Highlighting}
\end{Shaded}

\begin{center}\includegraphics{00-sissejuhatus-ri_files/figure-latex/unnamed-chunk-141-1} \end{center}

Mhh, see vist ei ole päris see mida me silmas pidasime. Pigem tahaksime, et gruppide tulbad asuksid kõrvuti. Selleks peame määratlema geomi positsiooni:

\begin{Shaded}
\begin{Highlighting}[]
\FunctionTok{ggplot}\NormalTok{(dt1)}\SpecialCharTok{+}
  \FunctionTok{geom\_bar}\NormalTok{(}\FunctionTok{aes}\NormalTok{(clarity, }\AttributeTok{fill=}\NormalTok{cut), }\AttributeTok{position=}\StringTok{"dodge"}\NormalTok{)}
\end{Highlighting}
\end{Shaded}

\begin{center}\includegraphics{00-sissejuhatus-ri_files/figure-latex/unnamed-chunk-142-1} \end{center}

Või siis kui tahame 100\% barplot

\begin{Shaded}
\begin{Highlighting}[]
\FunctionTok{ggplot}\NormalTok{(dt1)}\SpecialCharTok{+}
  \FunctionTok{geom\_bar}\NormalTok{(}\FunctionTok{aes}\NormalTok{(clarity, }\AttributeTok{fill=}\NormalTok{cut), }\AttributeTok{position =} \StringTok{"fill"}\NormalTok{)}
\end{Highlighting}
\end{Shaded}

\begin{center}\includegraphics{00-sissejuhatus-ri_files/figure-latex/unnamed-chunk-143-1} \end{center}

\textbf{Boxplot ja violin plot}

\begin{Shaded}
\begin{Highlighting}[]
\FunctionTok{ggplot}\NormalTok{(dt1)}\SpecialCharTok{+}
  \FunctionTok{geom\_boxplot}\NormalTok{(}\FunctionTok{aes}\NormalTok{(}\AttributeTok{x=}\NormalTok{color, }\AttributeTok{y=}\NormalTok{carat))}
\end{Highlighting}
\end{Shaded}

\begin{center}\includegraphics{00-sissejuhatus-ri_files/figure-latex/unnamed-chunk-144-1} \end{center}

\begin{Shaded}
\begin{Highlighting}[]
\FunctionTok{ggplot}\NormalTok{(dt1)}\SpecialCharTok{+}
  \FunctionTok{geom\_violin}\NormalTok{(}\FunctionTok{aes}\NormalTok{(}\AttributeTok{x=}\NormalTok{color, }\AttributeTok{y=}\NormalTok{carat))}
\end{Highlighting}
\end{Shaded}

\begin{center}\includegraphics{00-sissejuhatus-ri_files/figure-latex/unnamed-chunk-145-1} \end{center}

\textbf{Histogram ja frequency polygon}

\begin{Shaded}
\begin{Highlighting}[]
\FunctionTok{ggplot}\NormalTok{(dt1)}\SpecialCharTok{+}
  \FunctionTok{geom\_histogram}\NormalTok{(}\FunctionTok{aes}\NormalTok{(}\AttributeTok{x=}\NormalTok{carat), }\AttributeTok{bins =} \DecValTok{50}\NormalTok{, }\AttributeTok{color=}\StringTok{"white"}\NormalTok{)}
\end{Highlighting}
\end{Shaded}

\begin{center}\includegraphics{00-sissejuhatus-ri_files/figure-latex/unnamed-chunk-146-1} \end{center}

\begin{Shaded}
\begin{Highlighting}[]
\FunctionTok{ggplot}\NormalTok{(dt1)}\SpecialCharTok{+}
  \FunctionTok{geom\_freqpoly}\NormalTok{(}\FunctionTok{aes}\NormalTok{(}\AttributeTok{x=}\NormalTok{carat))}
\end{Highlighting}
\end{Shaded}

\begin{verbatim}
## `stat_bin()` using `bins = 30`. Pick better value with `binwidth`.
\end{verbatim}

\begin{center}\includegraphics{00-sissejuhatus-ri_files/figure-latex/unnamed-chunk-147-1} \end{center}

\textbf{Keskmised ja usalduspiirid}

Väga tihti on meil vaja esitada keskmiste või proportsioonide punkthinnanguid mingite gruppide lõikes koos usaldusintervallidega. Selleks on meil kõigepealt vaja keskmisi ja usaldusintervalle. ggplot neid ise ei arvuta. Aga \emph{dplyr}'i abil saab need võrdlemisi lihtsalt kätte. t-jaotuse kvartiilid on leitavad \texttt{qt(p,\ df)} funktsiooniga (p on siis tõenäosus).\\
Leiame teemantite keskmised hinnad koos usalduspiiridega teemadite lõiketi (\emph{cut}):

\begin{Shaded}
\begin{Highlighting}[]
\FunctionTok{library}\NormalTok{(dplyr)}
\NormalTok{keskmised }\OtherTok{\textless{}{-}}\NormalTok{ dt1 }\SpecialCharTok{\%\textgreater{}\%}
  \FunctionTok{group\_by}\NormalTok{(cut)}\SpecialCharTok{\%\textgreater{}\%}
  \FunctionTok{summarise}\NormalTok{(}\AttributeTok{keskmine=}\FunctionTok{mean}\NormalTok{(price),}
            \AttributeTok{se=}\FunctionTok{sd}\NormalTok{(price)}\SpecialCharTok{/}\FunctionTok{sqrt}\NormalTok{(}\FunctionTok{length}\NormalTok{(price)),}
            \AttributeTok{l.ci=}\NormalTok{keskmine}\SpecialCharTok{{-}}\FunctionTok{qt}\NormalTok{(}\FloatTok{0.975}\NormalTok{, }\FunctionTok{length}\NormalTok{(price)}\SpecialCharTok{{-}}\DecValTok{1}\NormalTok{)}\SpecialCharTok{*}\NormalTok{se,}
            \AttributeTok{u.ci=}\NormalTok{keskmine}\SpecialCharTok{+}\FunctionTok{qt}\NormalTok{(}\FloatTok{0.975}\NormalTok{, }\FunctionTok{length}\NormalTok{(price)}\SpecialCharTok{{-}}\DecValTok{1}\NormalTok{)}\SpecialCharTok{*}\NormalTok{se)}
\NormalTok{keskmised}
\end{Highlighting}
\end{Shaded}

\begin{verbatim}
## # A tibble: 5 x 5
##   cut       keskmine    se  l.ci  u.ci
##   <ord>        <dbl> <dbl> <dbl> <dbl>
## 1 Fair         4359.  88.7 4185. 4533.
## 2 Good         3929.  52.6 3826. 4032.
## 3 Very Good    3982.  35.8 3912. 4052.
## 4 Premium      4584.  37.0 4512. 4657.
## 5 Ideal        3458.  25.9 3407. 3508.
\end{verbatim}

Kasutame \texttt{geom\_point()}'i keskmiste visualiseerimiseks ja \texttt{geom\_linerange()}'i usalduspiiride märkimiseks

\begin{Shaded}
\begin{Highlighting}[]
\FunctionTok{ggplot}\NormalTok{(keskmised, }\FunctionTok{aes}\NormalTok{(cut, keskmine))}\SpecialCharTok{+}
  \FunctionTok{geom\_point}\NormalTok{()}\SpecialCharTok{+}
  \FunctionTok{geom\_linerange}\NormalTok{(}\FunctionTok{aes}\NormalTok{(}\AttributeTok{ymin=}\NormalTok{l.ci, }\AttributeTok{ymax=}\NormalTok{u.ci))}\SpecialCharTok{+}
  \FunctionTok{coord\_flip}\NormalTok{() }\CommentTok{\# saame joonise teljed ära vahetada}
\end{Highlighting}
\end{Shaded}

\begin{center}\includegraphics{00-sissejuhatus-ri_files/figure-latex/unnamed-chunk-149-1} \end{center}

\textbf{Joonise disain}

ggplot võimaldab kontrollida praktiliselt kogu joonise väljanägemist. Vaatame mõnda olulisemat võimalust: telgede nimed ja joonise pealkiri

\begin{Shaded}
\begin{Highlighting}[]
\FunctionTok{ggplot}\NormalTok{(keskmised, }\FunctionTok{aes}\NormalTok{(cut, keskmine))}\SpecialCharTok{+}
  \FunctionTok{geom\_point}\NormalTok{()}\SpecialCharTok{+}
  \FunctionTok{geom\_linerange}\NormalTok{(}\FunctionTok{aes}\NormalTok{(}\AttributeTok{ymin=}\NormalTok{l.ci, }\AttributeTok{ymax=}\NormalTok{u.ci))}\SpecialCharTok{+}
  \FunctionTok{coord\_flip}\NormalTok{()}\SpecialCharTok{+}
  \FunctionTok{ylab}\NormalTok{(}\StringTok{"Hind"}\NormalTok{)}\SpecialCharTok{+}
  \FunctionTok{xlab}\NormalTok{(}\StringTok{"Lõige"}\NormalTok{)}\SpecialCharTok{+}
  \FunctionTok{ggtitle}\NormalTok{(}\StringTok{"Teemantite hind"}\NormalTok{)}
\end{Highlighting}
\end{Shaded}

\begin{center}\includegraphics{00-sissejuhatus-ri_files/figure-latex/unnamed-chunk-150-1} \end{center}

Kui me tahame telgede nimedest lahti saada:

\begin{Shaded}
\begin{Highlighting}[]
\FunctionTok{ggplot}\NormalTok{(keskmised, }\FunctionTok{aes}\NormalTok{(cut, keskmine))}\SpecialCharTok{+}
  \FunctionTok{geom\_point}\NormalTok{()}\SpecialCharTok{+}
  \FunctionTok{geom\_linerange}\NormalTok{(}\FunctionTok{aes}\NormalTok{(}\AttributeTok{ymin=}\NormalTok{l.ci, }\AttributeTok{ymax=}\NormalTok{u.ci))}\SpecialCharTok{+}
  \FunctionTok{coord\_flip}\NormalTok{()}\SpecialCharTok{+}
  \FunctionTok{ylab}\NormalTok{(}\ConstantTok{NULL}\NormalTok{)}\SpecialCharTok{+}
  \FunctionTok{xlab}\NormalTok{(}\ConstantTok{NULL}\NormalTok{)}
\end{Highlighting}
\end{Shaded}

\begin{center}\includegraphics{00-sissejuhatus-ri_files/figure-latex/unnamed-chunk-151-1} \end{center}

\begin{Shaded}
\begin{Highlighting}[]
\CommentTok{\# või ka nii:}
\CommentTok{\# ggplot(keskmised, aes(cut, keskmine))+}
\CommentTok{\#   geom\_point()+}
\CommentTok{\#   geom\_linerange(aes(ymin=l.ci, ymax=u.ci))+}
\CommentTok{\#   coord\_flip()+}
\CommentTok{\#   labs(x = NULL, y = NULL)}
\end{Highlighting}
\end{Shaded}

\texttt{xlab} ja \texttt{ylab} on tegelikult mugavdatud variandid \texttt{scale} funktsioonidest. \texttt{scale} funktsioonid kontrollivad seda kuidas andmed mappitakse \texttt{aes()}'i. Võetakse andmed ja teakse neist midagi joonisel nähtavat. Igal \emph{aestheticul} on oma \texttt{scale}:\\
- Kui x telg on pidev: \texttt{scale\_x\_continuous}
- Kui y telg on kategoriaalne: \texttt{scale\_y\_discrete}
- Kui kasutame \texttt{fill} \texttt{aes}'i: \texttt{scale\_fill\_discrete}

\begin{Shaded}
\begin{Highlighting}[]
\FunctionTok{ggplot}\NormalTok{(keskmised, }\FunctionTok{aes}\NormalTok{(cut, keskmine))}\SpecialCharTok{+}
  \FunctionTok{geom\_point}\NormalTok{()}\SpecialCharTok{+}
  \FunctionTok{geom\_linerange}\NormalTok{(}\FunctionTok{aes}\NormalTok{(}\AttributeTok{ymin=}\NormalTok{l.ci, }\AttributeTok{ymax=}\NormalTok{u.ci))}\SpecialCharTok{+}
  \FunctionTok{coord\_flip}\NormalTok{()}\SpecialCharTok{+}
  \FunctionTok{scale\_x\_discrete}\NormalTok{(}\AttributeTok{name=}\StringTok{"Lõige"}\NormalTok{)}\SpecialCharTok{+}
  \FunctionTok{scale\_y\_continuous}\NormalTok{(}\AttributeTok{name=}\StringTok{"Hind"}\NormalTok{)}
\end{Highlighting}
\end{Shaded}

\begin{center}\includegraphics{00-sissejuhatus-ri_files/figure-latex/unnamed-chunk-152-1} \end{center}

Saame kontrollida ka \emph{tick mark}'e ja \emph{label}'eid

\begin{Shaded}
\begin{Highlighting}[]
\FunctionTok{ggplot}\NormalTok{(keskmised, }\FunctionTok{aes}\NormalTok{(cut, keskmine))}\SpecialCharTok{+}
  \FunctionTok{geom\_point}\NormalTok{()}\SpecialCharTok{+}
  \FunctionTok{geom\_linerange}\NormalTok{(}\FunctionTok{aes}\NormalTok{(}\AttributeTok{ymin=}\NormalTok{l.ci, }\AttributeTok{ymax=}\NormalTok{u.ci))}\SpecialCharTok{+}
  \FunctionTok{coord\_flip}\NormalTok{()}\SpecialCharTok{+}
  \FunctionTok{scale\_x\_discrete}\NormalTok{(}\AttributeTok{name=}\StringTok{"Lõige"}\NormalTok{, }\AttributeTok{labels=}\FunctionTok{c}\NormalTok{(}\DecValTok{1}\SpecialCharTok{:}\DecValTok{5}\NormalTok{))}\SpecialCharTok{+}
  \FunctionTok{scale\_y\_continuous}\NormalTok{(}\AttributeTok{name=}\StringTok{"Hind"}\NormalTok{, }\AttributeTok{breaks =} \FunctionTok{seq}\NormalTok{(}\DecValTok{3000}\NormalTok{,}\DecValTok{5000}\NormalTok{, }\AttributeTok{by=}\DecValTok{100}\NormalTok{))}
\end{Highlighting}
\end{Shaded}

\begin{center}\includegraphics{00-sissejuhatus-ri_files/figure-latex/unnamed-chunk-153-1} \end{center}

Kõige võimsam joonise visuaali tööriist, millega saab kontrollida pea kõike, on \texttt{theme()}. Täpsemalt saab selle kohta lugeda \url{https://ggplot2.tidyverse.org/reference/theme.html}. Vaatame näiteks, kuidas \texttt{theme()} abil muula legendi asukohta ning kustutada x-telje skaala:

\begin{Shaded}
\begin{Highlighting}[]
\NormalTok{dt }\OtherTok{\textless{}{-}}\NormalTok{ iris}
\FunctionTok{names}\NormalTok{(dt) }\OtherTok{\textless{}{-}} \FunctionTok{tolower}\NormalTok{(}\FunctionTok{names}\NormalTok{(dt))}
\FunctionTok{ggplot}\NormalTok{(dt, }\FunctionTok{aes}\NormalTok{(sepal.width, sepal.length, }\AttributeTok{color=}\NormalTok{species))}\SpecialCharTok{+}
  \FunctionTok{geom\_point}\NormalTok{()}\SpecialCharTok{+}
  \FunctionTok{theme}\NormalTok{(}\AttributeTok{legend.position=}\StringTok{"bottom"}\NormalTok{,}
        \AttributeTok{axis.text.x =} \FunctionTok{element\_blank}\NormalTok{())}
\end{Highlighting}
\end{Shaded}

\begin{center}\includegraphics{00-sissejuhatus-ri_files/figure-latex/unnamed-chunk-154-1} \end{center}

Saame üksikasjalikult muuta praktiliselt kogu joonise väljanägemist. Saame kasutada ka juba mõningaid valmistehtud \texttt{theme()}'ide \emph{template}'e. Näiteks \texttt{theme\_bw()}.

\begin{Shaded}
\begin{Highlighting}[]
\NormalTok{dt }\OtherTok{\textless{}{-}}\NormalTok{ iris}
\FunctionTok{names}\NormalTok{(dt) }\OtherTok{\textless{}{-}} \FunctionTok{tolower}\NormalTok{(}\FunctionTok{names}\NormalTok{(dt))}
\FunctionTok{ggplot}\NormalTok{(dt, }\FunctionTok{aes}\NormalTok{(sepal.width, sepal.length, }\AttributeTok{color=}\NormalTok{species))}\SpecialCharTok{+}
  \FunctionTok{geom\_point}\NormalTok{()}\SpecialCharTok{+}
  \FunctionTok{theme\_bw}\NormalTok{()}
\end{Highlighting}
\end{Shaded}

\begin{center}\includegraphics{00-sissejuhatus-ri_files/figure-latex/unnamed-chunk-155-1} \end{center}

\hypertarget{edasiseks-lugemiseks}{%
\section{Edasiseks lugemiseks}\label{edasiseks-lugemiseks}}

\begin{itemize}
\tightlist
\item
  R'i baasteadmised

  \begin{itemize}
  \tightlist
  \item
    Daniel Navarro, ``Learning statistics with R: A tutorial for psychology students and other beginners'', peatükid 3, 4, 5, 7\\
  \end{itemize}
\item
  dplyr

  \begin{itemize}
  \tightlist
  \item
    \url{https://cran.r-project.org/web/packages/dplyr/vignettes/dplyr.html}
  \end{itemize}
\item
  ggplot

  \begin{itemize}
  \tightlist
  \item
    Grolemund, G., Wickham, H., ``R for Data Science''., peatükk 3 ``Data visualisation'' \url{http://r4ds.had.co.nz/}
  \end{itemize}
\end{itemize}

\hypertarget{lineaarne-regressioon}{%
\chapter{Lineaarne regressioon}\label{lineaarne-regressioon}}

\hypertarget{lihtne-lineaarne-regressioon}{%
\section{Lihtne lineaarne regressioon}\label{lihtne-lineaarne-regressioon}}

Lihtne lineaarne regressioon (\emph{simple linear regression}) on statistiline meetod mis võimaldab hinnata ja kvantifitseerida kahe arvtunnuse vahelist suhet. Regressioonsuhte puhul eeldatakse, et üks tunnustest oleks nn sõltuv tunnus ja teine sõltumatu\footnote{Inglisekeelses terminoloogias kasutatakse sõltuva tunnuse puhul peale \emph{dependent variable} ka nimetusi \emph{response} või \emph{outcome variable} ja sõltumatu tunnuse puhul peale \emph{independent variable} ka \emph{predictor} või \emph{explanatory variable}. Prediktor on ka eesti keeles kasutusel.}, kus sõltuva tunnuse väärtus on mõjutatud (sõltub) sõltumatu tunnuse väärtusest. Kui sõltumatuid tunnuseid on rohkem kui üks, on tegemist mitmese regressiooniga (sellest hiljem), ühe sõltumatu tunnuse korral nn ``lihtsa'' regressiooniga (\emph{simple linear regression}). Keskendume esialgu ``lihtsale'' variandile.

Kasutame näitena Piaaci andmestikku. Tõmbame andmestiku sisse ja uurime graafiliselt sissetuleku (\emph{sissetulek}) ning matemaatilise kirjaokuse (\emph{numeracy}) vahelist seost.

\begin{Shaded}
\begin{Highlighting}[]
\CommentTok{\# Loeme kõigepealt sisse vajalikud paketid}
\FunctionTok{library}\NormalTok{(dplyr)}
\FunctionTok{library}\NormalTok{(ggplot2)}
\FunctionTok{library}\NormalTok{(readr)}

\CommentTok{\# Tõmbame  sisse andmestiku}
\NormalTok{piaac }\OtherTok{\textless{}{-}} \FunctionTok{read\_csv}\NormalTok{(}\StringTok{"https://github.com/mrksom/kvant/raw/master/data/piaac.csv"}\NormalTok{)}

\NormalTok{piaac }\SpecialCharTok{\%\textgreater{}\%} 
  \FunctionTok{ggplot}\NormalTok{(}\FunctionTok{aes}\NormalTok{(}\AttributeTok{x =}\NormalTok{ numeracy, }\AttributeTok{y =}\NormalTok{ sissetulek))}\SpecialCharTok{+}
  \FunctionTok{geom\_point}\NormalTok{(}\AttributeTok{size =} \FloatTok{0.5}\NormalTok{, }\AttributeTok{alpha =} \FloatTok{0.3}\NormalTok{)}\SpecialCharTok{+}
  \FunctionTok{theme\_minimal}\NormalTok{()}
\end{Highlighting}
\end{Shaded}

\includegraphics{01-regressioon_files/figure-latex/reg-plot-1-1.pdf}

Tundub, et nende kahe tunnuse vahel on seos olemas. Mida kõrgem on matemaatilise kirjaoskuse skoor, seda kõrgem on sissetulek. Me saame selle suhte kokku võtta regressioonisirge abil. ggplotis on olemas vastav funktsioon \texttt{geom\_smooth()}, mis selle joone meile graafikule paneb. Kuna me tahame saada lineaarse regressiooni sirget, siis peame \texttt{geom\_smooth}'is kasutama argumenti \texttt{method\ =\ "lm"}\footnote{\emph{Defaultis} annab \texttt{geom\_smooth} meile mittelineaarse regressioonijoone (vastavalt sellele palju vaatlusi on, kas \emph{gam} või \emph{loess}), mis üritab tunnustevahelist suhet andmete kõikides punktides võimalikult täpselt kirjeldada.}

\begin{Shaded}
\begin{Highlighting}[]
\NormalTok{piaac }\SpecialCharTok{\%\textgreater{}\%} 
  \FunctionTok{ggplot}\NormalTok{(}\FunctionTok{aes}\NormalTok{(}\AttributeTok{x =}\NormalTok{ numeracy, }\AttributeTok{y =}\NormalTok{ sissetulek))}\SpecialCharTok{+}
  \FunctionTok{geom\_point}\NormalTok{(}\AttributeTok{size =} \FloatTok{0.3}\NormalTok{, }\AttributeTok{alpha =} \FloatTok{0.2}\NormalTok{)}\SpecialCharTok{+}
  \FunctionTok{geom\_smooth}\NormalTok{(}\AttributeTok{method =} \StringTok{"lm"}\NormalTok{, }\AttributeTok{se =}\NormalTok{ F, }\AttributeTok{color =} \StringTok{"\#972D15"}\NormalTok{)}\SpecialCharTok{+}
  \FunctionTok{theme\_minimal}\NormalTok{()}
\end{Highlighting}
\end{Shaded}

\includegraphics{01-regressioon_files/figure-latex/reg-plot-2-1.pdf}

Regressioonisirge on väljendatav tavalise joone võrrandiga:

\begin{equation}
Y=a+bX
\end{equation}

kus \(a\) on vabaliige (\emph{intercept}) ja \(b\) on sirge tõus (\emph{slope}). Regressiooni kontekstis kutsutakse seda sirge tõusu regressioonikoefitsiendiks või regressioonikordajaks. Vabaliige tähistab \(Y\) väärtust juhul kui \(X\) on \(0\) (sirge lõikumine y-teljega) ja sirge tõus ühikulist muutust \(Y\) väärtuses kui \(X\) väärtus muutub ühe ühiku võrra. Eelneva näite puhul oleks vabaliige võrdne sissetulekuga (\(Y\)) juhul kui matemaatilise kirjaoskuse tase (\(X\)) oleks \(0\) ja sirge tõus võrdne keskmise sissetuleku muutusega, mis lisandub iga matemaatilise kirjaoskuse punktiga. Kui sirge tõus on positiivne, siis \(X\)'i väärtuse kasvades \(Y\) väärtus suureneb, kui negatiivne, siis kahaneb. Kui sirge tõus on aga \(0\), siis seos kahe tunnuse vahel puudub (iga \(X\) väärtuse korral on keskmine \(Y\) sama).

Linaarse regressioonanalüüsi eesmärgiks ongi leida parim võimalik sirge (st leida vabaliige ja regressioonikoefitsient, mis seda sirget määratlevad) tunnustevahelise lineaarse suhte kirjeldamiseks. Parim võimalik tähendab siinjuures seda, et see sirge läheb punktiparvest läbi võimalikult keskelt, st kirjeldab kõiki punkte võimalikult hästi.

\hypertarget{regressioon-r-is}{%
\section{Regressioon R-is}\label{regressioon-r-is}}

R-is käib lihtsa regressioonimudeli tegemine \texttt{lm()} (\emph{linear model}) funktsiooniga. Loomulikult on ka teisi funktsioone, mis regressiooni jooksutamisega hakkama saavad ja hea tahtmise korral võib vastava funktsiooni ka mõningase vaevaga ise valmis kirjutada. Kuid jätame teised variandid hetkel kõrvale.

\texttt{lm()} funktsioonis tuleb defineerida regressioonivõrrand. Selleks peame määratlema sõltuva tunnuse, seejärel kasutama tildet (\texttt{\textasciitilde{}}) ning seejärel määratlema sõltumatu(d) tunnuse(d): \texttt{sõltuv\_tunnus\ \textasciitilde{}\ sõltumatu\_tunnus}\footnote{Hiljem, kui meil on mitu sõltumatut tunnust, eristame tunnused plussiga: \texttt{sõltuv\_tunnus\ \textasciitilde{}\ sõltumatu\_tunnus\_1\ +\ sõltumatu\_tunnus\_2\ +\ ...}}. Võtame eelpool toodud näite sissetuleku ja matemaatilise kirjaoskuse seosest ning defineerime regressioonimudeli, millega hindame matemaatilise kirjaoskuse mõju sissetulekule\footnote{Tegelikult ei ole selline mudel korrektne. Sissetuleku jaotus ei vasta hästi regressiooni nõuetele. Miks ei vasta ja kuidas see vastama panna, sellest natuke hiljem. Kuid hetkel kasutame seda puhtalt didaktilistest kaalutustest lähtuvalt.}:

\begin{Shaded}
\begin{Highlighting}[]
\FunctionTok{lm}\NormalTok{(sissetulek }\SpecialCharTok{\textasciitilde{}}\NormalTok{ numeracy, }\AttributeTok{data =}\NormalTok{ piaac)}
\end{Highlighting}
\end{Shaded}

\begin{verbatim}
## 
## Call:
## lm(formula = sissetulek ~ numeracy, data = piaac)
## 
## Coefficients:
## (Intercept)     numeracy  
##    -140.887        3.606
\end{verbatim}

Lihtsalt \texttt{lm()} funktsiooni jookustades saame kaks numbrit - vabaliikme (\emph{intercept}), mis antud näite puhul on \(-140\), ja regressioonikoefitsiendi (\emph{regression coefficient}), mis antud näite puhul on \(3.6\). Mida need meile ütlevad? Nagu eelnevalt juttu oli, siis vabaliige on \(Y\) väärtus kui \(X\) on \(0\), ehk siis inimesel, kelle matemaatilise kirjaoskuse skoor on \(0\), peaks meie mudeli kohaselt sissetulek olema \(-140\). Regressioonikoefitsient aga annab meile teada kui palju \(Y\) muutub, kui \(X\) muutub ühe ühiku võrra, ehk siis kui matemaatilise kirjaoskuse skoor tõuseb ühe punkti võrra, tõuseb sissetulek keskmiselt \(3.6\) euro võrra.
Nüüd, kui teame mudeli parameetreid, saame nende abil regressioonijoone graafikule kanda ka ilma \texttt{geom\_smooth}'ita:

\begin{Shaded}
\begin{Highlighting}[]
\NormalTok{piaac }\SpecialCharTok{\%\textgreater{}\%} 
  \FunctionTok{ggplot}\NormalTok{(}\FunctionTok{aes}\NormalTok{(}\AttributeTok{x =}\NormalTok{ numeracy, }\AttributeTok{y =}\NormalTok{ sissetulek))}\SpecialCharTok{+}
  \FunctionTok{geom\_point}\NormalTok{(}\AttributeTok{size =} \FloatTok{0.3}\NormalTok{, }\AttributeTok{alpha =} \FloatTok{0.2}\NormalTok{)}\SpecialCharTok{+}
  \FunctionTok{geom\_abline}\NormalTok{(}\AttributeTok{slope =} \FloatTok{3.6}\NormalTok{, }\AttributeTok{intercept =} \SpecialCharTok{{-}}\DecValTok{140}\NormalTok{, }\AttributeTok{color =} \StringTok{"\#972D15"}\NormalTok{)}\SpecialCharTok{+}
  \FunctionTok{coord\_cartesian}\NormalTok{(}\AttributeTok{xlim =} \FunctionTok{c}\NormalTok{(}\DecValTok{0}\NormalTok{,}\DecValTok{450}\NormalTok{), }\AttributeTok{ylim =} \FunctionTok{c}\NormalTok{(}\DecValTok{0}\NormalTok{,}\DecValTok{3500}\NormalTok{))}\SpecialCharTok{+}
  \FunctionTok{theme\_minimal}\NormalTok{()}
\end{Highlighting}
\end{Shaded}

\includegraphics{01-regressioon_files/figure-latex/reg-plot-3-1.pdf}

Kui me teame regressioonisirge tõusu ehk regressioonikoefitsienti ja vabaliiget, siis lähtuvalt sõltumatu tunnuse väärtustest saame prognoosida sõltuva tunnuse väärtuse:

\begin{equation}
  \hat{y}_i=b_0+b_1x_i
\end{equation}

\(\hat{y}_i\) antud võrrandis tähistab hinnatud või prognoositud \(y\) väärtust (sellest ka see müts \(y\) peal) vaatlusele \(i\). Kui meil on regressioonivõrrand \(\hat{y}_i=-140+3.6x_i\) ja meil on mingi vaatlus \(i\), kelle \(x\) väärtus on näiteks \(200\), siis saame sellele vaatlusele prognoosida \(y\) väärtuseks \(-140+3.6\times200=580\). Ehk siis inimesel, kelle matemaatilise kirjaoskuse skoor on 200, peaks meie mudeli järgi sissetulek olema \emph{ca} 580 eurot. Inimesel, kelle matemaatilise kirjaoskuse skoor on 400, peaks sissetulek olema keskmiselt \(-140+3.6\times400=1300\) eurot

\begin{figure}
\centering
\includegraphics{01-regressioon_files/figure-latex/reg-plot-4-1.pdf}
\caption{\label{fig:reg-plot-4}Prognoosime y väärtust kui x on 200 ja kui x = 400}
\end{figure}

\hypertarget{regressiooni-juxe4uxe4gid}{%
\section{Regressiooni jäägid}\label{regressiooni-juxe4uxe4gid}}

Samas on muidugi võimatu ühe sirgega kõiki punkte ideaalselt kirjeldada. Iga punkti ja sirge vahele jääb alati mingi viga või teisisõnu, kõik punktid (või vähemalt enamus neist) hälbivad suuremal või vähemal määral regressioonisirgest.

Mida suuremad need hälbed on, seda vähem suudab on meie mudel (regressioonisirge) kirjeldada sõltuva tunnuse variatsiooni ja seda suurem on vea määr meie mudelis. Neid hälbeid kutusutakse \textbf{regressiooni jääkideks} (\emph{regression residuals}).

\begin{figure}
\centering
\includegraphics{01-regressioon_files/figure-latex/reg-plot-5-1.pdf}
\caption{\label{fig:reg-plot-5}Regressiooni jäägid}
\end{figure}

Ehk siis iga kord, kui prognoosime \(\hat{y}_i=\beta_0+\beta_1x_i\) abil \(y_i\) väärtust, teeme me mingi vea\footnote{Mida saab väljendada kui \(\epsilon_i=y_i-\hat{y}_i\)}. Seetõttu tuleb regressioonivõrrandile lisada vea komponent (\(\epsilon\)) ning võrrand ise muutub vastavalt:

\begin{equation}
  \hat{y_i}=\beta_0+\beta x_i+\epsilon
\end{equation}

Kõige parem regressioonisirge annab joon, mille puhul jäägid on minimaalsed, ehk siis joon, mille puhul kõikide vaatluste jääkide summa oleks võimalikult väike. Kuna me ei saa jääke kokku võtta neid lihtsalt kokku liites (\emph{ca} pooled jäägid on väiksemad kui regressioonijoon ja \emph{ca} pooled suuremad, seega nende summa oleks \(0\)), siis tuleb nad enne liitmist ruutu panna. Ja meie eesmärgiks on nüüd leida regressioonisirge, mis minimeeriks \textbf{ruutjääkide summa} (\emph{residual sum of squares} ehk \(RSS\)) ehk siis regressioonisirge, mille puhul \(RSS\) oleks võimalikult väike\footnote{\(RSS= = e_1^2 + e_2^2 + ... + e_n = \sum_{i=1}^{n}(y_i-\hat{y}_i)^2\)}.

Eelnevast lähtuvalt on ka küllaltki loogiline, et meetodit, millega \(RSS\) minimeeritakse ja regressioonisirge ning vastavad koefitsiendid leitakse, nimetatakse \textbf{vähimruutude meetodiks}.

Ülesanne!

\begin{itemize}
\tightlist
\item
  Kasutades ggplot'i ja tehke punktdiagramm \texttt{geom\_point()} matemaatilise kirjaoskuse (\emph{numeracy}) ja funktsionaalse lugemisoskuse (\emph{literacy}) vahelisest seosest. Pange \emph{numeracy} x-teljele ja \emph{literacy} y-teljele.\\
\item
  Kasutades \texttt{geom\_abline()}'i, lisage joonisele lineaarne regressioonijoon (seega peate eelnevalt \texttt{lm()} funktsiooniga leidma regressioonijoone vabaliikme ja regressioonikoefitsiendi)
\end{itemize}

\hypertarget{regressioonimudeli-sobitumine}{%
\section{Regressioonimudeli sobitumine}\label{regressioonimudeli-sobitumine}}

Olles leidnud joone, mis kirjeldab kahe tunnuse vahelist seost kõige paremini, võiks ju eeldada, et ülesanne on täidetud. Aga kas ikka on? Ükskõik, millisest punktiparvest võib regressioonijoone läbi panna. Kuid tulenevalt regressioonijääkide (vaatluste hälbed regressioonijoonest) suurusest saame selle joone kohta teha väga erinevaid järeldusi. Kui jäägid on väikesed, siis võime suhteliselt täpselt prognoosida sõltuva tunnuse väärtust või teha järeldusi seose kohta. Kuid mida suuremad on jäägid, seda ebatäpsem on ka meie prognoos/järeldus.

Üldjuhul kasutame regresioonanalüüsi, et teha valimi baasil järeldusi mingi üldkogumi kohta. Meid huvitab, kas see seos, mida näeme oma valimi andmete põhjal, kehtib ka üldkogumis. Saame küll eeldada, et valimipõhiselt leitud regressioonisirge on suhteliselt sarnane üldkogumi sirgele (sirge, mille me saaksime, kui kaasaksime analüüsi kõik üldkogumi liikmed), aga kui sarnane, seda me ei tea. Kui me võtaksime samast üldkogumist teise valimi, siis juhul, kui mõlemad valimid on võetud korrektselt\footnote{Korrektse valimi võtmise all peame siinkohal silmas eelkõige juhuvalikut. Kõikidel populatsiooni liikmetel/elementidel peab olema võrdne võimalus valimisse sattuda. Kui üldpopulatsiooniks on Eesti elanikkond, aga valimisse võtaksime ainult Tallinna elanikud, siis antud valimi põhjal tehtavad järeldused ei oleks kuidagi üldistatavad kõigile Eesti elanikele, vaid ikkagi ainult tallinnlastele. Lisaks juhuvalimile on veel terve rida spetsiifilisemaid valimidisaine (stratifitseeritud valim, klastervalim jne) mida me hetkel ei käsitle. Kuid tuleb meeles pidada, et keerulisemate valimidisainide puhul tuleb hilisemas analüüsis ja järelduste tegemise käigus valimi moodustamise loogikat arvesse võtta.} ja valimid on piisavalt suured, siis peaksid nende põhjal leitud regressioonisirged olema suhteliselt sarnased, aga identsed ei ole nad praktiliselt kunagi. Kõikide võimalike valimite puhul me mingil määral alahindame või ülehindame tegelikku, populatsiooni regressioonikoefitsienti (ja ka vabaliiget). Seega, et saada aimu valimipõhise hinnangu täpsusest (vastavusest tegelikule tegelikule üldkogumi parameetrile), peaksime kuidagi välja selgitama valimi kasutamisest tuleneva vea võimaliku suuruse.

Et hinnata mudeli sobivust andmetega ja sellega leitud hinnagute täpsust, vajame mudeli kohta täiendavat infot. Eelnevalt regressioonimudelit \texttt{lm()} funktsiooniga jooksutades oli väljund väga lakooniline. Saime teada ainult vabaliikme ja regressioonikoefitsinedi väärtused. Tegelikult on \texttt{lm()} tulem muidugi märksa põhjalikum. Muule mudeliga kaasnevale infole saame ligi kui salvestame mudeli esmalt mingisse andmeobjekti ja kasutame selle andmeobjekti peal \texttt{summary()} käsku\footnote{Ka \texttt{summary()} ei anna välja kogu mudeliobjektis sisalduvat infot. Et näha mida mudeliobjekt veel sisaldab, võib kasutada \texttt{str(mudeliobjekt)} käsku.}.

\begin{Shaded}
\begin{Highlighting}[]
\NormalTok{mudel1 }\OtherTok{\textless{}{-}} \FunctionTok{lm}\NormalTok{(sissetulek }\SpecialCharTok{\textasciitilde{}}\NormalTok{ numeracy, }\AttributeTok{data =}\NormalTok{ piaac)}
\FunctionTok{summary}\NormalTok{(mudel1)}
\end{Highlighting}
\end{Shaded}

\begin{verbatim}
## 
## Call:
## lm(formula = sissetulek ~ numeracy, data = piaac)
## 
## Residuals:
##     Min      1Q  Median      3Q     Max 
## -1016.7  -351.5  -129.1   179.4  2923.4 
## 
## Coefficients:
##             Estimate Std. Error t value Pr(>|t|)    
## (Intercept) -140.887     56.510  -2.493   0.0127 *  
## numeracy       3.606      0.202  17.849   <2e-16 ***
## ---
## Signif. codes:  0 '***' 0.001 '**' 0.01 '*' 0.05 '.' 0.1 ' ' 1
## 
## Residual standard error: 555.8 on 3982 degrees of freedom
##   (3648 observations deleted due to missingness)
## Multiple R-squared:  0.07408,    Adjusted R-squared:  0.07385 
## F-statistic: 318.6 on 1 and 3982 DF,  p-value: < 2.2e-16
\end{verbatim}

\begin{Shaded}
\begin{Highlighting}[]
\CommentTok{\# Kui me ei taha mudelit salvestada, siis saab ka nii:}
\FunctionTok{summary}\NormalTok{(}\FunctionTok{lm}\NormalTok{(numeracy }\SpecialCharTok{\textasciitilde{}}\NormalTok{ literacy, }\AttributeTok{data =}\NormalTok{ piaac))}
\end{Highlighting}
\end{Shaded}

Nüüd näeme juba märksa põhjalikumat väljundit. Vaatame mis seal kirjas on ja kuidas seda tõlgendada. Käime väljundi sektsioonide kaupa läbi (v.a. esimene rida, mis on vist niigi suht selge)

\hypertarget{juxe4uxe4kide-jaotus}{%
\subsection{Jääkide jaotus}\label{juxe4uxe4kide-jaotus}}

\begin{verbatim}
## Residuals:
##     Min      1Q  Median      3Q     Max 
## -1016.7  -351.5  -129.1   179.4  2923.4
\end{verbatim}

Väljundis on kirjeldatud regressiooni jääkide (\emph{residuals}) jaotus. Enne nägime, et regressiooni jäägid on regressioonijoone ja tegelike, vaadeldud väärtuste vahe. Mida väiksemad on jäägid, seda täpsemini kirjeldab regressioonijoon andmete vahelist seost. Nägime ka, et pooled jäägid peaksid ideaalis olema suuremad (positiivse märgiga) kui regressioonisirge ja pooled väiksemad (negatiivse märgiga). Seega peaks jääkide keskmine olema ligikaudu \(0\) ning jääkide jaotus normaaljaotuse sarnane, kus esimene ja kolmas kvartiil, aga ka maksimum ja miinimum, on keskväärtusest umbes sama kaugel. Hiljem vaatame jääkide jaotust ka graafiliselt, mis on märksa mõistlikum viis neid uurida, kuid esmase mulje saab ka siit kätte.

\hypertarget{regressioonikoefitsiendid-ja-nende-olulisus}{%
\subsection{Regressioonikoefitsiendid ja nende olulisus}\label{regressioonikoefitsiendid-ja-nende-olulisus}}

\begin{verbatim}
##             Estimate Std. Error t value Pr(>|t|)    
## (Intercept) -140.887     56.510  -2.493   0.0127 *  
## numeracy       3.606      0.202  17.849   <2e-16 ***
## ---
## Signif. codes:  0 '***' 0.001 '**' 0.01 '*' 0.05 '.' 0.1 ' ' 1
\end{verbatim}

Koefitsientide sektsioonis on esitatud mudeli oluliseim info. \textbf{\emph{Estimate}} on hinnang mudeliga leitud regressioonikoefitsientidele. Lihtsa regressiooni puhul on meil ainult vabaliige ja ühe sõltumatu tunnuse koefitsient. Hiljem, mitmese regressiooni kontekstis, on neid koefitsiente rohkem. Vabaliikmeid on aga mudeli kohta alati üks.

Tulbas \textbf{\emph{Std. Error}} on toodud koefitsientide standardvead. Standardviga kirjeldab meie mudeli hinnangus sisalduvat määramatust. Me kasutame regressioonikoefitsientide leidmiseks üldjuhul valimipõhiseid andmeid, kuigi tegelikult huvitavad meid ju üldkogums esinevad seosed. Valimipõhine hinnang peaks piisavalt suure valimi korral olema tõenäoliselt küllaltki sarnane üldkogumi vastavale parameetrile, kuid väikese valimi korral puhta juhuse läbi sellest arvestatavalt erineda. Standardviga näitabki kui kindlad me oma mudeli hinnangus olla saame. Mida väiksem on standardviga (võrreldes hinnangu endaga), seda kindlamad võime olla ka oma hinnangus. Standardvea suurs sõltub eelkõige jääkide hajuvusest ja valimi suurusest. Mida väiksemad on jäägid ja mida suurem on valim, seda väiksem on ka standardviga.

Standardvea abil saame \emph{t}-testi abil testida, kas regressioonikoefitsient erineb oluliselt nullist (kui koefitsient on null, siis seos tunnuste vahel puudub). \emph{t}-testi tulemust näitab veerg \textbf{\emph{t value}}. \emph{t}-väärtus ütleb meile kui mitme standardvea kaugusel meie regressioonikoefitsient 0-st on. Kui on piisavalt kaugel, siis saame järeldada, et leitud koefitsient on ka üldkogumis 0-st erinev. Kui kaugel on aga piisavalt kaugel? See sõltub sellest, kui suurt vea tõenäosust me oleme valmis tolereerima (mingi vea tõenäosus jääb seejuures alati). Üldjuhul valitakse selleks tõenäosuseks \(5\%\) (ütleme, et regressioonikoefitsient on statistiliselt oluline usaldusnivool \(95 \%\) või olulisusnivool \(p < 0.05\)), aga see võib olla ka \(1\%\) või \(10\%\). Siin tegelikult ei ole mingit väga konkreetset piirmäära, millest juhinduda. Kui me aga lepime kokku, et võimaliku vea tõenäosusena aktsepteerime \(5\)-te protsenti, siis peab \emph{t}-väärtus olema suurem kui \emph{ca} \(\pm2\) (täpne väärtus sõltub vaatluste arvust). Antud näite puhul on \emph{t}-väärtused \(-2.5\) ja \(17.8\), ehk siis mõnevõrra suuremad kui \(\pm2\) ja me võime järeldada, et nii vabaliige kui regressioonikoefitsient erinevad olulisusnivool \(95\%\) oluliselt nullist (kuigi jah, vabaliige on suhteliselt piiri peal).

Õnneks ei pea me seda täpset \emph{t}-väärtuse piirmäära ise välja nuputama. R arvutab meile automaatselt võimaliku vea tõenäosuse konkreetse \emph{t}-väärtuse kohta. See tõenäosus on ära toodud veerus \textbf{\emph{Pr(\textgreater\textbar t\textbar)}} ja seda nimetatakse \emph{p}-väärtuseks. \emph{p}-väärtuse tõlgendus on: kui tõenäoline on, et me saaksime niivõrd suure või suurema \emph{t}-väärtuse nagu me saime, kui regressioonikoefitsient oleks üldkogumis tegelikult \(0\). Seega kui \emph{p}-väärtus on näiteks \(0.04\), siis oleks tõenäosus, et me saaksime sellise regressioonikoefitsiendi, juhul kui üldkogumis oleks regressioonikoefitsient tegelikult \(0\) (ehk tunnuste vahe seost ei oleks), \(0.04\) ehk \(4\)\% või väiksem. Üldjuhul tahaksime näha \emph{p}-väärtust, mis on väiksem kui \(0.05\). Sellisel juhul oleks koefitsient statistiliselt oluline usaldusnivool \(95\%\). Antud näites on meil regressioonikoefitsiendi puhul tegemist väga väikeste \emph{p} väärtustega (\textless2e-16 tähendab väiksem kui \(2\times10^{-16}\)) ja me võime olla päris kindlad, et koefitsient erineb nullist. Vabaliikme \emph{p}-väärtus on aga \(0.012\), ehk kui me kasutaksime usaldusnivood \(99\%\) (mille puhul \emph{p}-väärtus peaks olema väiksem kui \(0.01\)), siis me ei saaks järeldada, et see on statistiliselt oluliselt erinev nullist. Lisaks kuvab R iga \emph{p}-väärtuse taha ka tärnid, mis indikeerivad selle väärtuse suurust lähtuvalt allolevast legendist.

Miks meil on üldse vaja teada kas koefitsiendid erinevad oluliselt nullist? Aga sellepärast, et kui regressioonisirge oleks \(0\), siis meie tunnuste vahel ei oleks seost (kui \(X\) muutub \(1\) ühiku võrra, siis \(Y\) muutub \(0\) ühiku võrra, ehk siis \(Y\) väärtus ei sõltu \(X\)'i väärtusest). Aga kuidas on lood vabaliikmega? Kas ka see peab erinema nullist, et meie mudelist mingit tolku oleks? Tegelikult ju ei pea. Võib täitsa vabalt juhtuda, et regressioonisirge lähebki läbi \(X\) ja \(Y\) telgede ristumiskoha (\(Y\) on \(0\) kui \(X\) on \(0\)). Sellisel juhul oleks vabaliikme \emph{t}-väärtus väiksem kui \(2\) ja \emph{p}-väärtus suurem kui 0.05, kuid mudeli tõlgendust see ei mõjutaks. Ehk siis tavaliselt meid vabaliikme \emph{p} ja \emph{t} väärtused väga ei huvita. Küll aga peaks jälgima, et standardviga väga suur (võrreldes vabaliikme endaga) ei oleks.

\hypertarget{juxe4uxe4kide-standardviga}{%
\subsection{Jääkide standardviga}\label{juxe4uxe4kide-standardviga}}

\begin{verbatim}
## Residual standard error: 555.8 on 3982 degrees of freedom
##   (3648 observations deleted due to missingness)
\end{verbatim}

Kuidas hinnata regressiooniprognoosi täpsust, ehk siis seda kui hästi regressioonimudel sobitub andmetega (\emph{model fit})? Üheks võimaluseks on lähtuda samast loogikast mida kasutame tunnuse keskväärtuse täpsuse hindamisel. Ehk kui palju vaatlused keskmiselt erinevad keskväärtusest. Regressioonijoone puhul ei ole meil ühte keskväärtust, mille suhtes vaatluste hälbimist määrata. Kuid iga vaatluse sõltumatu tunnuse väärtuse \(x\) kohta on meil ``hinnatud'' sõltuva tunnuse väärtus \(\hat{y}\). Seega tuleb meil lihtsalt vaadata kui palju vaatluste \(y\) ja \(\hat{y}\) väärtused keskmiselt erinevad, ehk kui suur on keskmine viga meie mudelis. Regressioonanalüüsi kontekstis kutsutakse seda vaatluste varieeruvuse näitajat keskmiseks ruutveaks (\emph{mean squared error}) ehk lühidalt \(MSE\)\footnote{\(MSE=\frac{\sum_{i=1}^{n}(y_i-\hat{y}_i)^2}{n-2}\)}. Kuna aga \(MSE\) väärtus on ruudus, siis on seda keeruline interpreteerida (samamoodi nagu ka dispersiooni). Kui me võtame ruutjuure \(MSE\)'st, \(\sqrt{MSE}\), saame regressiooni jääkide standardhälbe, mida nimetatakse \textbf{jääkide standardveaks} (\emph{residual standard error} ehk RSE). Mida väiksem on mudeli RSE, seda paremini mudel andmetega sobitub (seda vähem hälbivad vaatlused regressioonijoonest ehk seda väiksemad on regresiooni jäägid). See, kui väike peaks RSE väärtus hea mudeli korral olema, sõltub eelkõige kontekstist ja sõltuva tunnuse skaalast (samamoodi nagu keskväärtuse standardhälve). Mingeid konkreetseid piirväärtusi siinkohal tuua ei ole võimalik.\\
Lisaks on siin ära toodud ka \emph{degrees of freedom} ehk vabadusastmete arv jääkide standardvea arvutamisel. Sisuliselt on siin kirjas analüüsi kaasatud vaatluste arv (miinus regressioonikordajate arv, siinses mudelis 2). Ära on toodud ka analüüsist välja jäetud vaatluste arv. Need on need, kellel puudus väärtus vähemalt ühe analüüsitava tunnuse jaoks.

\hypertarget{r-ruut}{%
\subsection{R ruut}\label{r-ruut}}

\begin{verbatim}
## Multiple R-squared:  0.07408,    Adjusted R-squared:  0.07385
\end{verbatim}

Vast oluliseimaks mudeli headuse näitajaks on \(R^2\). Regressioonanalüüsi eesmärk on seletada mingit osa sõltuva tunnuse variatiivsusest sõltumatu tunnuse abil. Seega saame regressioonimudeli puhul hinnata ja mudeli kvaliteedi iseloomustusena kasutada sõltumatu tunnuse poolt seletatud variatiivsuse osakaalu sõltuva tunnuse koguvariatiivsusest. Sõltuva tunnuse variatiivsuse (seda nimetatakse \(TSS\) ehk \emph{total sums of squares}) saab jagada komponentideks: variatiivsus, mis on seletatud regressioonijoone poolt (\(ESS\) ehk \emph{explained sums of squares}) ja variatiivsus, mis ei ole regressioonijoone poolt seletatud ehk siis mudeli seisukohast viga (\(RSS\) ehk \emph{residual sums of squares}):

\[TSS=RSS+ESS\]

\begin{equation}
  ESS=\sum_{i=1}^{n}(\hat{y}_i-\bar{y})^2
\end{equation}

\begin{equation}
  RSS=\sum_{i=1}^{n}(y_i-\hat{y}_i)^2
\end{equation}

\begin{equation}
  TSS=\sum_{i=1}^{n}(y_i-\bar{y})^2
\end{equation}

Teades erinevaid variatiivsuse komponente, saame määrata kui suur osa (mitu protsenti) sõltuva tunnuse koguvariatsioonist on seletatav regressioonijoone poolt (ehk siis sõltumatu tunnuse poolt). Seda suurust nimetatakse \textbf{determinatsioonikordajaks} ehk lühidalt \(R^2\)-ks.

\begin{equation}
  R^2=\frac{TSS-RSS}{TSS}=1-\frac{RSS}{TSS}
\end{equation}

\begin{figure}
\centering
\includegraphics{01-regressioon_files/figure-latex/ss-1.pdf}
\caption{\label{fig:ss}Variatsiivsuse jagunemine}
\end{figure}

\(R^2\) jääb vahemikku \(0-1\). See mõõdab seose tugevust, st mida lähemal \(R^2\) on \(1\)'le, seda tugevam lineaarne seos tunnuste vahel on ja seda enam sõltumatu tunnus sõltuva tunnuse variatsiooni seletab, seega seda efektiivsem on regressioonifunktsiooni kasutamine selle asemel, et lihtsalt sõltuva tunnuse keskmist hinnata (kui \(R^2\) on \(0\), siis regressioonijoon langeb kokku sõltuva tunnuse keskmist tähistava joonega, st et \(ESS=0\) ja \(TSS=RSS\)).

R annab meile lisaks tavalisele \(R^2\) väärtusele (\emph{Multiple R-squared}) ka nn korrigeeritud \(R^2\) väärtuse (\emph{Adjusted R-squared}). Korrigeeritud \(R^2\) puhul võetakse arvesse ka sõltumatute tunnuste arvu. Iga lisanduva sõltumatu tunnusega läheb ``tavaline'' \(R^2\) suuremaks. Kui lisanduv tunnus eriti midagi ei seleta, siis võib see tõus olla väga väike, kuid mingi tõus paratamatult on. Korrigeeritud \(R^2\), arvestades oma valemis ka sõltumatute tunnuste arvu, annab mitme sõltumatu tunnuse korral korrektsema tulemuse. Hetkel, lihtsa regressiooni kontekstis, kus meil on ainult üks sõltumatu tunnus, annavad mõlemad variandid (enam-vähem) sama tulemuse.

\hypertarget{f-vuxe4uxe4rtus-ja-f-test}{%
\subsection{F-väärtus ja F-test}\label{f-vuxe4uxe4rtus-ja-f-test}}

\begin{verbatim}
## F-statistic: 318.6 on 1 and 3982 DF,  p-value: < 2.2e-16
\end{verbatim}

\emph{F}-väärtus, sarnaselt \emph{t}-väärtusele, aitab meil hinnata kas meie mudel on statistiliselt oluline, ehk siis kas meie analüüsitavate tunnuste vahel on oluline lineaarne seos. \emph{F}-väärtuseks nimetatakse mudeli abil seletatud variatiivsuse ja seletamata variatiivsuse suhet\footnote{Natuke täpsemalt väljendades \(F = \frac{(TSS-RSS)/p}{RSS/(n-p-1)}\), kus \(n\) on valimi suurus ja \(p\) on regressioonikoefitsientide (sõltumatute muutujate) arv.}:

\begin{equation}
\text{F-suhe} = \frac{\text{regressioonimudeli poolt seletatud variatiivsus}}{\text{regressioonimudeli poolt seletamata variatiivus}}
\end{equation}

Kui mudeli regressioonisirge on \(0\), siis peaks see suhe olema \(1\). See tähendab, et regressioonisirge ei seleta üldse sõltuva tunnuse varieeruvust. Kui regressioonisirge on suurem kui \(0\) siis peaks regressioonisirge poolt seletatud varieeruvus (koos juhusliku varieeruvusega) olema suurem kui ainult juhuslik dispersioon. Saame jällegi kasutada \emph{F}-väärtusega kaasnevat \emph{p} väärtust, et hinnata kas see \emph{F}-väärtus on piisavalt suur, et saaksime mudelist lähtuvalt mingeid sisukaid järeldusi teha.

Võite märgata, et need kaks testi regressioonimudeli kohta annavad sama \emph{p} väärtuse. Ja tegelikult annavad nad ka sama teststatistiku. \emph{t}-statistik on lihtsalt ruutjuur \emph{F} statistikust\footnote{\((t^{*}_{(n-2)})^2=F^{*}_{(1,n-2)}\)}. Võib tekkida küsimus, et miks me siis kahte testi peame kasutama. Ühe sõltumatu tunnusega regressioonimudelis otseselt ei peagi. Samas kui meil on mitu sõltumatut tunnust (nagu meil hiljem on), siis \emph{F} ja \emph{t} väärtused muutuvad. \emph{F}-testiga saab sel juhul testida terve mudeli headust, st kas meie sõltumatud tunnused koos suudavad seletada piisavalt sõltuva tunnuse variatiivsust (tegelikult testib \emph{F}-test seda, et kas vähemalt üks koefitsientidest erineb nullist). \emph{t}-statistikud aga arvutatakse igale regressioonikoefitsiendile eraldi ning nendega saame kontrollida iga üksiku koefitsiendi erinevust nullist.

Ülesanne!

\begin{itemize}
\tightlist
\item
  Looge regressioonimudel, millega hindate \emph{numeracy} mõju \emph{literacy}'le.\\
\item
  Salvestage see mudel ja uurige \texttt{summary()} funktsiooniga.
\item
  Kas \emph{numeracy} mõju \emph{literacy}'le on statistiliselt oluline?\\
\item
  Mitu protsenti \emph{literacy} variatsioonist on selgitatav läbi \emph{numeracy}?
\end{itemize}

\hypertarget{kategoriaalsed-tunnused-regressioonis}{%
\section{Kategoriaalsed tunnused regressioonis}\label{kategoriaalsed-tunnused-regressioonis}}

\hypertarget{uxfcks-binaarne-suxf5ltumatu-tunnus}{%
\subsection{Üks binaarne sõltumatu tunnus}\label{uxfcks-binaarne-suxf5ltumatu-tunnus}}

Siiani oleme käsitlenud ainult mudeleid, kus sõltumatuteks tunnusteks on pidevad muutujad. Kuid me saame mudelisse lülitada ka kategoriaalseid tunnuseid. Vaatame esmalt mudelit, kus on üks kategoriaalne sõltumatu muutuja\footnote{Sellist mudelit nimetatakse ka ANOVA-ks või täpsemlat One-Way ANOVA-ks (kuna tegemist on ainult ühe kategoriaalse sõltumatu muutujaga)}. Teeme Piaaci andmete põhjal mudeli, millega hindame soo mõju sissetulekule

\begin{Shaded}
\begin{Highlighting}[]
\NormalTok{mudel2 }\OtherTok{\textless{}{-}} \FunctionTok{lm}\NormalTok{(sissetulek }\SpecialCharTok{\textasciitilde{}}\NormalTok{ sugu, }\AttributeTok{data =}\NormalTok{ piaac)}
\FunctionTok{summary}\NormalTok{(mudel2)}
\end{Highlighting}
\end{Shaded}

\begin{verbatim}
## 
## Call:
## lm(formula = sissetulek ~ sugu, data = piaac)
## 
## Residuals:
##    Min     1Q Median     3Q    Max 
## -974.0 -344.0 -122.7  198.1 2755.2 
## 
## Coefficients:
##             Estimate Std. Error t value Pr(>|t|)    
## (Intercept)  1077.90      13.35   80.73   <2e-16 ***
## suguNaine    -383.15      17.52  -21.86   <2e-16 ***
## ---
## Signif. codes:  0 '***' 0.001 '**' 0.01 '*' 0.05 '.' 0.1 ' ' 1
## 
## Residual standard error: 545.8 on 3982 degrees of freedom
##   (3648 observations deleted due to missingness)
## Multiple R-squared:  0.1072, Adjusted R-squared:  0.107 
## F-statistic:   478 on 1 and 3982 DF,  p-value: < 2.2e-16
\end{verbatim}

Kuidas seda tulemust tõlgendada?

Regressioonimudeliga hindame \(\hat{y_i}\) väärtust vaatlusele \(i\), kui sõltumatu tunnuse väärtus muutub ühe ühiku võrra. \(\hat{y_i}\)'i väärtus kujuneb siis lähtuvalt vabaliikme \(\beta_0\) ja regressioonikoefitsiendi \(\beta\) ning sõltumatu tunnuse \(x_i\)'i korrutise summast (pluss mingi viga): \(\hat{y_i}=\beta_0+\beta x_i+\epsilon\). Sealjuures vabaliige on \(y\) väärtus kui \(x\) on \(0\).

Meil on tunnus \(x\) (sugu) kahe kategooriaga. Mis juhtub, kui kodeerime selle ümber väärtusteks \(0\) ja \(1\) (vastavalt mees ja naine). R kusjuures teeb seda automaatselt.

\[ x_{i} =
  \begin{cases}
    1  & \quad \text{kui on naine}\\
    0  & \quad \text{kui on mees}
  \end{cases}
\]

Kui me selle tunnuse nüüd regressioonivõrrandisse paneme, siis mis on \(y_i\) väärtus kui \(x_i\) on \(1\) (ehk siis vaatluse sugu on naine) ja mis on \(y_i\) väärtus kui \(x_i\) on \(0\) (ehk siis vaatluse sugu on mees)?

\[ \hat{y_i}=\beta_0+\beta_1 x_i =
  \begin{cases}
    \beta_0+(\beta_1 \times 1) = \beta_0+\beta_1  & \quad \text{kui on naine}\\
    \beta_0+(\beta_1 \times 0) = \beta_0  & \quad \text{kui on mees}
  \end{cases}
\]

Ehk siis kui \(x_i\) väärtus on \(0\) (mehed), siis võrdub \(\hat{y_i}\) vabaliikmega \(\beta_0\) (sest \(\beta_1\) korrutatakse läbi nulliga) ja kui \(x_i\) väärtus on \(1\) (naised), siis vabaliikme ja regressioonikoefitsiendi summaga \(\beta_0+\beta\). Mida \(\hat{y}\) antud juhul üldse tähistab? Pidevmuutujaga regressioonis tähistas see keskmist \(y\)-i väärtust erinevate \(x\) väärtuste korral. Ja siin täpselt samamoodi. Aga nüüd on meil ainult kaks \(x\) väärtust ja \(\hat{y}\) on vastavate gruppide (meeste ja naiste) keskmine \(y\).

Seega saame regressioonivõrrandiga väljenda binaarse tunnuse mõju sõltuva tunnuse keskmisele. Lihtsalt käsitleme ühte kategooriat nn \textbf{referentskategooriana} ja kodeerime selle \(0\)'ks. Kui \(x\) on \(0\), siis \(y\) väärtus on võrdne vabaliikme väärtusega. Ja kui sõltumatu tunnuse väärtus muutub ühe ühiku võrra (ja rohkem ta ei saagi muutuda), siis on \(y\) väärtus võrdne vabaliikme väärtus pluss regressioonikoefitsiendi väärtus.

Kuidas me eelneva valguses oma näidet siis tõlgendama peaksime?

\begin{verbatim}
## Coefficients:
##             Estimate Std. Error t value Pr(>|t|)    
## (Intercept)  1077.90      13.35   80.73   <2e-16 ***
## suguNaine    -383.15      17.52  -21.86   <2e-16 ***
\end{verbatim}

\emph{Sugu} oli tekstiline tunnus. R saab aru, et tegemist on kategoriaalse tunnusega ja kodeerib selle sisemiselt ümber \(0\)-ks ja \(1\)-ks. Antud juhul määras ta kategooria \emph{Naine} \(1\)-ks ja kategooria \emph{Mees} \(0\)-ks. Kuna tegemist oli tekstilise tunnusega, siis lähtub R siin tähestikulisest järjekorrast. Ümberkodeeritud (dihhotomiseeritud) tunnuse nimeks on alati mitte-referentskategooria. Antud juhul siis \emph{suguNaine}. Regressioonivõrrand oli järgmine:

\[\hat{y_i}=\beta_0+\beta x_i+\epsilon\]

kus

\begin{align}
x_{i} =
  \begin{cases}
    1  & \quad \text{kui on naine}\\
    0  & \quad \text{kui on mees}
  \end{cases}
\end{align}

Paneme mudeli tulemused sellesse võrrandisse:

\begin{align}
\text{keskmine sissetulek}&=1077.90+(-383.15)\times \text{naine}\\
&=
  \begin{cases}
    1077.90-383.15\times 1 & \quad \text{kui on naine}\\
    1077.90-383.15\times 0 & \quad \text{kui on mees}
  \end{cases}\\
&=
  \begin{cases}
    1077.90-383.15 & \quad \text{kui on naine}\\
    1077.90-0 & \quad \text{kui on mees}
  \end{cases}\\
&=
  \begin{cases}
    694.75 & \quad \text{kui on naine}\\
    1077.90 & \quad \text{kui on mees}
  \end{cases}
\end{align}

Ehk siis naiste keskmine sissetulek on \(694.8\) eurot (vabaliige + regressioonikoefitsient) ja meeste oma \(1077.9\) eurot (vabaliige). Erinevus on statistiliselt oluline, kuna \emph{p}-väärtused nii koefitsiendi \emph{t}-testi kui ka mudeli \emph{F}-testi puhul olid olulisusnivool \(95\%\) olulised (väiksemad kui \(0.05\)).

Kui me paneme need keskmised joonisele ja ühendame nad joonega, näeme, et selle joone tõus (\emph{slope}) on võrdne regressioonikoefitsiendiga, täpselt samuti nagu pidevtunnusega regressioonis.

\includegraphics{01-regressioon_files/figure-latex/reg-plot-6-1.pdf}

\hypertarget{kolme-vuxf5i-enama-kategooriaga-suxf5ltumatu-tunnus}{%
\subsection{Kolme või enama kategooriaga sõltumatu tunnus}\label{kolme-vuxf5i-enama-kategooriaga-suxf5ltumatu-tunnus}}

Kusjuures me ei pea piirduma vaid binaarsete tunnustega. Aga kui kategooriaid on rohkem, tuleb meil nad binaarseks teha ehk dihhotomiseerida. Määratleme ühe kategooria referentskategooriana ja ülejäänud kategooriad kodeerime erinevates tunnustes \(1\)'ks. Seega, kui meil on näiteks hariduse tunnus kolme kategooriaga (põhiharidus, keskharidus, kõrgharidus), peame määratlema ühe referentskategooria (näiteks põhiharidus) ja tegema kaks uut tunnust (vastavalt keskhariduse ja kõrghariduse kategooriatele):

\[ kesk_{i} =
  \begin{cases}
    1  & \quad \text{kui inimene on keskharidusega}\\
    0  & \quad \text{kui inimene ei ole keskharidusega}
  \end{cases}
\]

\[ korg_{i} =
  \begin{cases}
    1  & \quad \text{kui inimene on kõrgharidusega}\\
    0  & \quad \text{kui inimene ei ole kõrgaridusega}
  \end{cases}
\]

Nüüd saame iga inimese hariduse määratleda kahe tunnuse kaudu. Ehk siis inimene, kelle puhul \(kesk = 1\) ja \(korg = 0\), on keskharidusega; inimene kelle puhul \(kesk = 0\) ja \(korg = 1\), on kõrgharidusega ja inimene kelle puhul \(keks = 0\) ja \(korg = 0\), on põhiharidusega. \(y\) väärtus kujuneb täpselt samamoodi nagu binaarse tunnuse puhul:

\[y_i=\beta_0+\beta_1 \times kesk_i+\beta_2 \times korg_i\]

\begin{align}
&=
  \begin{cases}
    \beta_0+\beta_1 \times 1+\beta_2 \times 0  & \quad \text{keskharidusega}\\
    \beta_0+\beta_1 \times 0+\beta_2 \times 1  & \quad \text{kõrgaridusega}\\
    \beta_0+\beta_1 \times 0+\beta_2 \times 0  & \quad \text{põhiharidusega}
  \end{cases}\\
&=
  \begin{cases}
    \beta_0+\beta_1  & \quad \text{keskharidusega}\\
    \beta_0+\beta_2  & \quad \text{kõrgaridusega}\\
    \beta_0  & \quad \text{põhiharidusega}
  \end{cases}
\end{align}

Vaatame kuidas see kõik R-is välja näeb. Hindame hariduse (tunnus \emph{haridustase}) mõju sissetulekule:

\begin{Shaded}
\begin{Highlighting}[]
\NormalTok{mudel3 }\OtherTok{\textless{}{-}} \FunctionTok{lm}\NormalTok{(sissetulek }\SpecialCharTok{\textasciitilde{}}\NormalTok{ haridustase, }\AttributeTok{data =}\NormalTok{ piaac)}
\FunctionTok{summary}\NormalTok{(mudel3)}
\end{Highlighting}
\end{Shaded}

\begin{verbatim}
## 
## Call:
## lm(formula = sissetulek ~ haridustase, data = piaac)
## 
## Residuals:
##    Min     1Q Median     3Q    Max 
## -868.3 -362.5 -145.2  187.1 2881.2 
## 
## Coefficients:
##                  Estimate Std. Error t value Pr(>|t|)    
## (Intercept)        763.35      13.33  57.279   <2e-16 ***
## haridustaseKõrge   217.23      19.06  11.400   <2e-16 ***
## haridustaseMadal   -22.20      30.17  -0.736    0.462    
## ---
## Signif. codes:  0 '***' 0.001 '**' 0.01 '*' 0.05 '.' 0.1 ' ' 1
## 
## Residual standard error: 567.1 on 3981 degrees of freedom
##   (3648 observations deleted due to missingness)
## Multiple R-squared:  0.03631,    Adjusted R-squared:  0.03583 
## F-statistic:    75 on 2 and 3981 DF,  p-value: < 2.2e-16
\end{verbatim}

R sai jällegi ise aru, et \emph{haridustase} on tekstiline tunnus ja dihotomiseeris selle automaatselt ära, tehes kaks uut tunnust: \emph{haridustaseKõrge} (kus kõik kõrgharitud on kodeeritud \(1\)-na ja kõik teised \(0\)-na) ja \emph{haridustaseMadal} (kus kõik madala haridustasemega on kodeeritud \(1\)-na ja kõik teised \(0\)-na). Referentskategooriaks võttis ta tähestiku järjekorras esimese kategooria \emph{Keskmine} (kõik vaatlused, mille puhul nii \emph{haridustaseKõrge} kui ka \emph{haridustaseMadal} on \(0\)-d, on keskmise haridustasemega).

Tulemuste interpreteerimine toimub samamoodi nagu binaarse tunnuse puhul. Vabaliige tähistab referentskategooria, ehk antud juhul keskmise haridustasemega inimeste, keskmistsissetulekut (\(763.35\)), \emph{haridustaseKõrge} regressioonikordaja tähistab kõrge haridustasemega inimeste skoori erinevust referentskategooria keskmisest (vabaliikmest) ja \emph{haridustaseMadal} madala haridustasemega inimeste skoori erinevust referentskategooria keskmisest (vabaliikmest).

Võrrandi kujul näeb tulem välja järgmine:
\[y_i=\beta_0+\beta_1 \times korge_i+\beta_2 \times madal_i\]

\begin{align}
&=
  \begin{cases}
    763.35+217.23 \times 1+(-22.20) \times 0 & \quad \text{kõrge haridustase}\\
    763.35+217.23 \times 0+(-22.20) \times 1 & \quad \text{madal haridustase}\\
    763.35+217.23 \times 0+(-22.20) \times 0 & \quad \text{keskmine haridustase}
  \end{cases}\\
&=
  \begin{cases}
    763.35+217.23+0  & \quad \text{kõrge haridustase}\\
    763.35+0-22.20  & \quad \text{madal haridustase}\\
    763.35+0+0  & \quad \text{keskmine haridustase}
  \end{cases}\\
&=
  \begin{cases}
    980.58  & \quad \text{kõrge haridustase}\\
    741.15  & \quad \text{madal haridustase}\\
    763.35  & \quad \text{keskmine haridustase}
  \end{cases}
\end{align}

Kui kategoriaalne sõltumatu tunnus on tekstiline (\emph{character}), siis valib R referentskategooriaks tähestikuliselt esimese kategooria. Kui tunnus on faktortunnus (\emph{factor}), siis valib R esimese faktortaseme. Faktortasemeid saame me aga muuta. Tihti tahame referentskategooria ise valida (näiteks kõige suurema grupi või grupi, mida on loogiline teistega võrrelda). Näiteks tahame haridustasemete puhul määrata referentskategooriaks põhihariduse. Selleks teeme tunnuse faktoriks ja määrame tasemete järjestuse nii, et madal haridustase oleks esimene:

\begin{Shaded}
\begin{Highlighting}[]
\CommentTok{\# vaatame kõigepealt mis kategooriad tunnuses on}
\FunctionTok{unique}\NormalTok{(piaac}\SpecialCharTok{$}\NormalTok{haridustase)}
\end{Highlighting}
\end{Shaded}

\begin{verbatim}
## [1] "Keskmine" "Madal"    "Kõrge"    NA
\end{verbatim}

\begin{Shaded}
\begin{Highlighting}[]
\CommentTok{\# Laeme forcats paketti, millega on mugav faktoritega toimetada}
\FunctionTok{library}\NormalTok{(forcats)}
\CommentTok{\# Kasutame funktsiooni fct\_relevel()}
\CommentTok{\# Meil on antud juhul vaja määrata ainult esimene tasand,}
\CommentTok{\#  ülejäänud tulevad tähestiku järjekorras.}
\NormalTok{piaac }\OtherTok{\textless{}{-}}\NormalTok{ piaac }\SpecialCharTok{\%\textgreater{}\%} 
  \FunctionTok{mutate}\NormalTok{(}\AttributeTok{haridustase\_f =} \FunctionTok{fct\_relevel}\NormalTok{(haridustase, }\StringTok{"Madal"}\NormalTok{))}

\CommentTok{\# Baas{-}R{-}is käiks faktori tegemine nii:}
\CommentTok{\#piaac$haridustase\_f \textless{}{-} factor(piaac$haridustase, }
\CommentTok{\#                              levels = c("Madal","Keskmine","Kõrge"))}

\CommentTok{\# ja kui me nüüd regressiooni jooksutame, on referentsiks madal tase}
\FunctionTok{summary}\NormalTok{(}\FunctionTok{lm}\NormalTok{(sissetulek }\SpecialCharTok{\textasciitilde{}}\NormalTok{ haridustase\_f, }\AttributeTok{data =}\NormalTok{ piaac))}
\end{Highlighting}
\end{Shaded}

\begin{verbatim}
## 
## Call:
## lm(formula = sissetulek ~ haridustase_f, data = piaac)
## 
## Residuals:
##    Min     1Q Median     3Q    Max 
## -868.3 -362.5 -145.2  187.1 2881.2 
## 
## Coefficients:
##                       Estimate Std. Error t value Pr(>|t|)    
## (Intercept)             741.15      27.07  27.381  < 2e-16 ***
## haridustase_fKeskmine    22.20      30.17   0.736    0.462    
## haridustase_fKõrge      239.43      30.30   7.902 3.54e-15 ***
## ---
## Signif. codes:  0 '***' 0.001 '**' 0.01 '*' 0.05 '.' 0.1 ' ' 1
## 
## Residual standard error: 567.1 on 3981 degrees of freedom
##   (3648 observations deleted due to missingness)
## Multiple R-squared:  0.03631,    Adjusted R-squared:  0.03583 
## F-statistic:    75 on 2 and 3981 DF,  p-value: < 2.2e-16
\end{verbatim}

Kui meil juba on faktortunnus, aga tahame selle tasemete järjekorda muuta, saame jälle kasutada käsku \texttt{fct\_relevel()}. Muudame haridustaseme faktortunnuses kõrgema hariduse esimeseks tasemeks:

\begin{Shaded}
\begin{Highlighting}[]
\NormalTok{piaac}\SpecialCharTok{$}\NormalTok{haridustase\_f }\OtherTok{\textless{}{-}} \FunctionTok{fct\_relevel}\NormalTok{(piaac}\SpecialCharTok{$}\NormalTok{haridustase\_f, }\StringTok{"Kõrge"}\NormalTok{)}

\CommentTok{\# Baas{-}R{-}is käiks see nii:}
\CommentTok{\#piaac$haridustase\_f \textless{}{-} relevel(piaac$haridustase\_f, ref = "Kõrge")}

\FunctionTok{summary}\NormalTok{(}\FunctionTok{lm}\NormalTok{(numeracy }\SpecialCharTok{\textasciitilde{}}\NormalTok{ haridustase\_f, }\AttributeTok{data =}\NormalTok{ piaac))}
\end{Highlighting}
\end{Shaded}

\begin{verbatim}
## 
## Call:
## lm(formula = numeracy ~ haridustase_f, data = piaac)
## 
## Residuals:
##      Min       1Q   Median       3Q      Max 
## -202.991  -26.446    2.328   28.800  150.020 
## 
## Coefficients:
##                       Estimate Std. Error t value Pr(>|t|)    
## (Intercept)           289.2173     0.8172  353.93   <2e-16 ***
## haridustase_fMadal    -40.8718     1.4011  -29.17   <2e-16 ***
## haridustase_fKeskmine -21.0568     1.0944  -19.24   <2e-16 ***
## ---
## Signif. codes:  0 '***' 0.001 '**' 0.01 '*' 0.05 '.' 0.1 ' ' 1
## 
## Residual standard error: 42.72 on 7583 degrees of freedom
##   (46 observations deleted due to missingness)
## Multiple R-squared:  0.1065, Adjusted R-squared:  0.1063 
## F-statistic: 451.9 on 2 and 7583 DF,  p-value: < 2.2e-16
\end{verbatim}

Ülesanne!

\begin{itemize}
\tightlist
\item
  Piaaci andmestikus on tunnus \emph{meeldib\_oppida}. Tehke see faktortunnuseks nii, et esimene kategooria oleks ``Mõningal määral'' (kategooriate nimed saate teada näiteks funnktsiooniga \texttt{unique(piaac\$meeldib\_oppida)})
\item
  Tehke regressioonimudel, kus hindate õppimishimu mõju sissetulekule
\end{itemize}

\hypertarget{mitmene-regressioon}{%
\section{Mitmene regressioon}\label{mitmene-regressioon}}

Siiani oleme käsitlenud lineaarset regressiooni, kus sõltumatuid tunnuseid oli üks. Aga on võimalik lülitada ühte mudelisse ka mitu sõltumatut tunnust. Miks see hea peaks olema?

Valdavalt üritame välja selgitada (või tegelikult mingi teooria põhjal testida) mingi tunnuse kausaalset mõju teisele tunnusele (sõltumatu tunnuse mõju sõltuvale tunnusele). Kausaalsusel on aga teatud eeldused:

\begin{enumerate}
\def\labelenumi{\arabic{enumi}.}
\tightlist
\item
  Tunnuste vaheline seos (seose olemasolu ei tähenda muidugi kohe põhjalikkust)\\
\item
  Ajaline järgnevus (vastupidi ei saaks ju kuidagi olla)\\
\item
  Alternatiivse seletuse/põhjuse kõrvaldamine (sõltuv tunnus võib olla sõltumatu tunnuse poolt mõjutatud läbi mõne muu tunnuse, st kaudselt)
\end{enumerate}

Mitmene regressioon võimaldabki meil testida sõltumatute tunnuste otsest mõju sõltuvale tunnusele, kontrollides samal ajal teiste mudelisse lülitatud sõltumatute tunnuste mõjude suhtes (hoides teisi tunnuseid konstantsetena). Regressioonivõrrand mitme sõltumatu tunnuse puhul on sarnane ühese regressiooni võrrandiga, välja arvatud siis sõltumatute tunnuste arv. Mudel \(y\) prognoosimiseks \(p\) sõltumatute tunnuste kaudu on väljendatav järgmiselt:

\begin{equation}  
y_{i}=\beta_{0}+\beta_{1}x_{i,1}+\beta_{2}x_{i,2}+\ldots+\beta_{p}x_{i,p}+\epsilon_{i}
\end{equation}

Kus:

\(\beta_0\) on vabaliige (ehk \(y\) väärtus kui kõik sõltumatud tunnused on \(0\)'id)

\(\beta_1\) regressioonikoeffitsient esimesele sõltumatule tunnusele \(x_1\)

\(\beta_2\) regressioonikoeffitsient teisele sõltumatule tunnusele \(x_2\)

\(\beta_{p}\) regressioonikoeffitsient tunnusele \(x_{p}\)

\(\epsilon\) on mudeli jääk igale vaatlusele

\(\beta\) coefitsinedid on leitud nii, et nendega kaalutud tunnuste väärtused minimeerivad \(\epsilon\)'i ehk mudeli viga (kogu mudeli mõistes minimeerivad ruuthälvete summat). \(\beta\) väärtus on tõlgendatav kui muutus \(y\) väärtuses, kui vastava sõltumatu tunnuse väärtus muutub ühe ühiku võrra, hoides samal ajal teisi sõltumatuid tunnuseid konstantsetena. See tähendab, et koefitsientides on teiste tunnuste mõju arvesse võetud ja meie tulemused peegeldavad nn ``puhast'' mõju.

Mudeli defineerimisel R-is saame sõltumatuid tunnuseid lisada \texttt{+} märgi abil:

\begin{Shaded}
\begin{Highlighting}[]
\NormalTok{mudel4 }\OtherTok{\textless{}{-}} \FunctionTok{lm}\NormalTok{(sissetulek }\SpecialCharTok{\textasciitilde{}}\NormalTok{ numeracy }\SpecialCharTok{+}\NormalTok{ sugu, }\AttributeTok{data =}\NormalTok{ piaac)}
\FunctionTok{summary}\NormalTok{(mudel4)}
\end{Highlighting}
\end{Shaded}

\begin{verbatim}
## 
## Call:
## lm(formula = sissetulek ~ numeracy + sugu, data = piaac)
## 
## Residuals:
##      Min       1Q   Median       3Q      Max 
## -1181.73  -323.46   -98.77   167.86  2813.48 
## 
## Coefficients:
##              Estimate Std. Error t value Pr(>|t|)    
## (Intercept)  140.7631    55.0470   2.557   0.0106 *  
## numeracy       3.3533     0.1915  17.509   <2e-16 ***
## suguNaine   -365.0623    16.9196 -21.576   <2e-16 ***
## ---
## Signif. codes:  0 '***' 0.001 '**' 0.01 '*' 0.05 '.' 0.1 ' ' 1
## 
## Residual standard error: 526 on 3981 degrees of freedom
##   (3648 observations deleted due to missingness)
## Multiple R-squared:  0.171,  Adjusted R-squared:  0.1706 
## F-statistic: 410.7 on 2 and 3981 DF,  p-value: < 2.2e-16
\end{verbatim}

Mitmese regressiooni tõlgendus on analoogne lihtsa regressiooni tõlgendusega. Võrrandi kujul on see väljendatav järgmiselt:

\[\hat{y}_{sissetulek}=\beta_0+\beta_1 \times numeracy + \beta_2 \times naine\]

\begin{align}
&=
  \begin{cases}
    \beta_0+\beta_1 \times numeracy + \beta_2 \times 1 =  & \quad \text{naine}\\
    \beta_0+\beta_1 \times numeracy + \beta_2 \times 0 =  & \quad \text{mees}
  \end{cases}\\
&=
  \begin{cases}
    (\beta_0+\beta_2)+\beta_1 \times numeracy  & \quad \text{naine}\\
    \beta_0+\beta_1 \times numeracy & \quad \text{mees}
  \end{cases}\\
\end{align}

Vabaliige näitab kategoriaalse tunnuse referentskategooria (antud juhul mees) keskmist sõltuva tunnuse väärtust. Aga kuna nüüd on meil mudelis ka sõltumatu pidevtunnus, siis on see referentskategooria keskmine juhul, kui sõltumatu pidevtunnus on \(0\). Ehk siis meie näite puhul tähistab vabaliige (\(140.8\)) meeste sissetulekut juhul kui nende matemaatilise kirjaoskuse skoor on \(0\). \emph{suguNaine} regressioonikordaja näitab naiste sissetuleku erinevust meestest. See võtab arvesse ka matemaatilise kirjaoskuse skoori. Ehk siis kõikide matemaatilise kirjaoskuse väärtuste puhul on on naiste sissetulek \(365\) eurot meestest madalam (st soo mõju on kontrollitud matemaatilise kirjaoskuse suhtes). \emph{numeracy} regressioonikordaja näitab jällegi sissetuleku muutust (\(3.35\)) kui matemaatiline kirjaoskus muutub ühe ühiku võrra. Kuna ka sugu on mudelis arvesse võetud, kehtib see muutus võrdselt nii naistele kui meestele (mõju on kontrollitud soo suhtes).

Vaatame, milliseks kujunevad mudeli järgi meeste ja naiste keskmised sissetulekud, kui nende \emph{numeracy} skoor on 300.

\[\hat{y}_{sissetulek}=140.8+3.35 \times numeracy + (-365) \times naine\]

\begin{align}
&=
  \begin{cases}
    140.8+3.35 \times 300 + (-365) \times 1  & \quad \text{naine}\\
    140.8+3.35 \times 300 + (-365) \times 0  & \quad \text{mees}
  \end{cases}\\
&=
  \begin{cases}
    (140.8-365)+3.35 \times 300  & \quad \text{naine}\\
    140.8+3.35 \times 300 & \quad \text{mees}
  \end{cases}\\
&=
  \begin{cases}
    -224.2+1005  & \quad \text{naine}\\
    140.8+1005 & \quad \text{mees}
  \end{cases}\\
&=
  \begin{cases}
    780.8  & \quad \text{naine}\\
    1145.8 & \quad \text{mees}
  \end{cases}
\end{align}

Mõnevõrra lihtsam on seda tulemust interpreteerida graafiliselt:

\begin{Shaded}
\begin{Highlighting}[]
\NormalTok{piaac }\SpecialCharTok{\%\textgreater{}\%} 
  \FunctionTok{ggplot}\NormalTok{(}\FunctionTok{aes}\NormalTok{(}\AttributeTok{x =}\NormalTok{ numeracy, }\AttributeTok{y =}\NormalTok{ sissetulek, }\AttributeTok{color =}\NormalTok{ sugu))}\SpecialCharTok{+}
  \FunctionTok{geom\_point}\NormalTok{(}\AttributeTok{alpha =} \FloatTok{0.1}\NormalTok{, }\AttributeTok{size =} \FloatTok{0.3}\NormalTok{)}\SpecialCharTok{+}
  \FunctionTok{geom\_abline}\NormalTok{(}\AttributeTok{intercept =} \FloatTok{140.8}\NormalTok{, }\AttributeTok{slope =} \FloatTok{3.35}\NormalTok{, }\AttributeTok{color =} \StringTok{"\#972D15"}\NormalTok{)}\SpecialCharTok{+}
  \FunctionTok{geom\_abline}\NormalTok{(}\AttributeTok{intercept =} \FloatTok{140.8}\DecValTok{{-}365}\NormalTok{, }\AttributeTok{slope =} \FloatTok{3.35}\NormalTok{, }\AttributeTok{color =} \StringTok{"\#02401B"}\NormalTok{)}\SpecialCharTok{+}
  \FunctionTok{scale\_colour\_manual}\NormalTok{(}\AttributeTok{values =} \FunctionTok{c}\NormalTok{(}\StringTok{"Mees"} \OtherTok{=} \StringTok{"\#972D15"}\NormalTok{, }\StringTok{"Naine"} \OtherTok{=} \StringTok{"\#02401B"}\NormalTok{))}\SpecialCharTok{+}
  \FunctionTok{theme\_minimal}\NormalTok{()}\SpecialCharTok{+}
  \FunctionTok{guides}\NormalTok{(}\AttributeTok{color =} \FunctionTok{guide\_legend}\NormalTok{(}\AttributeTok{override.aes =} \FunctionTok{list}\NormalTok{(}\AttributeTok{size =} \DecValTok{2}\NormalTok{, }\AttributeTok{alpha =} \DecValTok{1}\NormalTok{)))}
\end{Highlighting}
\end{Shaded}

\includegraphics{01-regressioon_files/figure-latex/reg-plot-7-1.pdf}

Lihtsam võimalus seoseid graafiliselt esitada on kasutada paketist \texttt{interactions} funktsiooni \texttt{interaction\_plot()}. See on küll mõeldud eelkõige koosmõjude plottimiseks, kuid toimib ka tavalisete seoste kujutamisel.

\begin{Shaded}
\begin{Highlighting}[]
\FunctionTok{library}\NormalTok{(interactions)}
\FunctionTok{interact\_plot}\NormalTok{(mudel4, }\AttributeTok{pred =}\NormalTok{ numeracy, }\AttributeTok{modx =}\NormalTok{ sugu, }\AttributeTok{colors =} \FunctionTok{c}\NormalTok{(}\StringTok{"\#972D15"}\NormalTok{, }\StringTok{"\#02401B"}\NormalTok{))}
\end{Highlighting}
\end{Shaded}

\includegraphics{01-regressioon_files/figure-latex/unnamed-chunk-15-1.pdf}

\hypertarget{kaks-pidevat-suxf5ltumatut-muutujat}{%
\subsection{Kaks pidevat sõltumatut muutujat}\label{kaks-pidevat-suxf5ltumatut-muutujat}}

Vaatame ka olukorda, kus meil on kaks pidevat sõltumatut tunnust - matemaatiline kirjaoskus ja vanus:

\begin{Shaded}
\begin{Highlighting}[]
\NormalTok{mudel5 }\OtherTok{\textless{}{-}} \FunctionTok{lm}\NormalTok{(sissetulek}\SpecialCharTok{\textasciitilde{}}\NormalTok{numeracy}\SpecialCharTok{+}\NormalTok{vanus, }\AttributeTok{data =}\NormalTok{ piaac)}
\FunctionTok{summary}\NormalTok{(mudel5)}
\end{Highlighting}
\end{Shaded}

\begin{verbatim}
## 
## Call:
## lm(formula = sissetulek ~ numeracy + vanus, data = piaac)
## 
## Residuals:
##     Min      1Q  Median      3Q     Max 
## -1036.6  -349.5  -128.9   178.7  2944.0 
## 
## Coefficients:
##             Estimate Std. Error t value Pr(>|t|)    
## (Intercept)  25.3372    67.3417   0.376    0.707    
## numeracy      3.4841     0.2033  17.137  < 2e-16 ***
## vanus        -3.2210     0.7138  -4.512 6.59e-06 ***
## ---
## Signif. codes:  0 '***' 0.001 '**' 0.01 '*' 0.05 '.' 0.1 ' ' 1
## 
## Residual standard error: 554.5 on 3981 degrees of freedom
##   (3648 observations deleted due to missingness)
## Multiple R-squared:  0.07879,    Adjusted R-squared:  0.07833 
## F-statistic: 170.3 on 2 and 3981 DF,  p-value: < 2.2e-16
\end{verbatim}

Tõlgendame seda järmiselt:

\begin{enumerate}
\def\labelenumi{\arabic{enumi}.}
\tightlist
\item
  Kui matemaatiline kirjaoskus tõuseb ühe punkti võrra, siis sissetulek tõuseb \(3.48\) euro võrra, hoides vanust konstantsena (st see seos kehtib kõikide vanuste jaoks).\\
\item
  Kui vanus tõuseb ühe aasta võrra, siis sissetulek langeb \(3.2\) euro võrra, hoides funktsionaalset lugemisoskust konstantsena (st see seos kehtib kogu funktsionaalse lugemisoskuse skaala ulatuses).\\
\item
  Juhul kui nii vanus oleks \(0\) aastat ja matemaatiline kirjaoskus oleks \(0\) punkti, oleks sissetulek \(25.3\) eurot (kuna selline olukord on suhteliselt võimatu, siis me sellistel puhkudel vabaliiget ei interpreteeri).
\end{enumerate}

Et taolisest mudelist paremini aru saada võime kasutada 3D punktdiagrammi

\begin{Shaded}
\begin{Highlighting}[]
\CommentTok{\#library(car)}
\CommentTok{\#scatter3d(piaac$numeracy,piaac$sissetulek, piaac$vanus)}
\end{Highlighting}
\end{Shaded}

Ülesanne!

\begin{itemize}
\tightlist
\item
  Looge regressioonimudel, millega hindate \emph{numeracy}, \emph{vanus}, \emph{sugu} ja \emph{haridustase} mõju sissetulekule.
\item
  Milliste tunnuste mõju sissetulekule on statistiliselt oluline?
\item
  Esitage vanuse ja soo mõju sissetulekule graafiliselt.
\end{itemize}

\hypertarget{koosmuxf5jud}{%
\section{Koosmõjud}\label{koosmuxf5jud}}

Eelnevas näites vaatasime sissetuleku sõltuvust matemaatilise kirjaoskuse tasemest soo lõikes, ja nägime, et kui lisame mudelisse soo tunnuse, siis saame klasside kohta eraldi regressioonijooned. Kuid need regressioonijooned olid paralleelsed, mis tähendab et nii meeste kui naiste hulgas oli funktsionaalse lugemisoskuse ja matemaatilise kirjaoskuse suhe mudeli järgi sama. Kuid kas see on alati väga realistlik eeldus? Võib ju vabalt olla, et see seos erineb soo lõikes.

Kui me arvame, et see võib nii olla, st sõltumatu tunnuse mõju sõltuvale tunnusele sõltub omakorda mingist muust tunnusest, saame mudelisse lisada nende kahe tunnuse koosmõju (interaktsiooni). Selleks peame moodustame uue tunnuse, mis tuleneb nende tunnuste, mille suhtes me koosmõju hinnata tahame, korrutisest. Kui me nüüd selle uue tunnuse mudelisse kaasame, siis hindame sellele ka regressioonikoefitsiendi. Regressioonivõrrand pidevtunnuse ja kategoriaalse tunnuse koosmõjuga näeks välja nii:

\[\hat{y}_{sissetulek}=\beta_0+\beta_1 \times numeracy + \beta_2 \times naine + \beta_3 \times naine \times numeracy \]

\begin{align}
&=
  \begin{cases}
    \beta_0+\beta_1 \times numeracy + \beta_2 \times 1 + \beta_3 \times 1 \times numeracy  & \quad \text{naised}\\
    \beta_0+\beta_1 \times numeracy + \beta_2 \times 0 + \beta_3 \times 0 \times numeracy & \quad \text{mehed}
  \end{cases}\\
&=
  \begin{cases}
    (\beta_0+\beta_2)+(\beta_1+\beta_3) \times numeracy  & \quad \text{naised}\\
    \beta_0+\beta_1 \times numeracy & \quad \text{mehed}
  \end{cases}
\end{align}

Mis siin nüüd siis toimub? Meeste keskmine sissetulek on, nagu varasemaltki, defineeritud vabaliikme ja \(\beta_1\) regressioonikoefitsiendiga (ülejäänud kaks koefitsienti lähevad meeste jaoks \(0\)-ks, kuna \emph{naine} tunnus on nende jaoks \(0\)). Naistel on aga lisaks veel kaks koefitsienti. \(\beta_2\), mis nagu varasemaltki kirjeldab naiste vabaliikme erinevust meeste vabaliikmest, ning siis veel \(\beta_3\), mis kirjeldab naiste regressioonisirge erinevust meeste regressioonisirgest. Koosmõjudega mudelis on naiste \emph{numeracy} koefitsient \(\beta_1+\beta_3\) (sest \(\beta_1 \times numeracy + \beta_3 \times numeracy = (\beta_1+\beta_3) \times numeracy\)).

R-is saame taolise mudeli defineerida järgmiselt:

\begin{Shaded}
\begin{Highlighting}[]
\NormalTok{mudel8 }\OtherTok{\textless{}{-}} \FunctionTok{lm}\NormalTok{(sissetulek }\SpecialCharTok{\textasciitilde{}}\NormalTok{ numeracy }\SpecialCharTok{*}\NormalTok{ sugu, }\AttributeTok{data =}\NormalTok{ piaac)}
\FunctionTok{summary}\NormalTok{(mudel8)}
\end{Highlighting}
\end{Shaded}

\begin{verbatim}
## 
## Call:
## lm(formula = sissetulek ~ numeracy * sugu, data = piaac)
## 
## Residuals:
##      Min       1Q   Median       3Q      Max 
## -1213.35  -322.79   -99.79   166.87  2802.32 
## 
## Coefficients:
##                     Estimate Std. Error t value Pr(>|t|)    
## (Intercept)          28.4609    79.7161   0.357   0.7211    
## numeracy              3.7552     0.2815  13.340   <2e-16 ***
## suguNaine          -157.9875   107.6870  -1.467   0.1424    
## numeracy:suguNaine   -0.7476     0.3840  -1.947   0.0516 .  
## ---
## Signif. codes:  0 '***' 0.001 '**' 0.01 '*' 0.05 '.' 0.1 ' ' 1
## 
## Residual standard error: 525.8 on 3980 degrees of freedom
##   (3648 observations deleted due to missingness)
## Multiple R-squared:  0.1718, Adjusted R-squared:  0.1712 
## F-statistic: 275.2 on 3 and 3980 DF,  p-value: < 2.2e-16
\end{verbatim}

\begin{Shaded}
\begin{Highlighting}[]
\CommentTok{\# Sama tulemuse saaksime, kui kirjutaksime:}
\CommentTok{\#lm(numeracy \textasciitilde{} literacy + sugu + literacy:sugu, data = piaac)}
\end{Highlighting}
\end{Shaded}

Koosmõjudega mudeli vabaliikmed ja regressioonikoefitsiendid kujunevad järgmiselt:

\[\hat{y}_{sissetulek}=\beta_0+\beta_1 \times numeracy + \beta_2 \times naine + \beta_3 \times naine \times numeracy\]

\begin{align}
&=
  \begin{cases}
    28.5+3.8 \times numeracy + (-157) \times 1 + (-0.7) \times 1 \times numeracy & \quad \text{naised}\\
    28.5+3.8 \times numeracy + (-157) \times 0 + (-0.7) \times 0 \times numeracy & \quad \text{mehed}
  \end{cases}\\
&=
  \begin{cases}
    (28.5-157)+(3.8-0.7) \times numeracy  & \quad \text{naised}\\
    28.5+3.8 \times numeracy & \quad \text{mehed}
  \end{cases}\\
&=
  \begin{cases}
    -128.5+3.1 \times numeracy  & \quad \text{naised}\\
    28.5+3.8 \times numeracy & \quad \text{mehed}
  \end{cases}
\end{align}

Seega selles mudelis erinevad kategoriaalse tunnuse lõikes nii vabaliikme väärtused kui ka regressioonisirge tõusud. Kui me nüüd selle mudeli tulemused graafikule paneme, siis näeme, et regressioonisirged ei ole enam paralleelsed. Mida suurem on matemaatilise kirjaoskuse tase, seda suurem on erinevus meeste ja naiste sissetulekutes.

\begin{Shaded}
\begin{Highlighting}[]
\FunctionTok{interact\_plot}\NormalTok{(mudel8, }\AttributeTok{pred =}\NormalTok{ numeracy, }\AttributeTok{modx =}\NormalTok{ sugu,  }\AttributeTok{colors =}  \FunctionTok{c}\NormalTok{(}\StringTok{"\#972D15"}\NormalTok{, }\StringTok{"\#02401B"}\NormalTok{))}
\end{Highlighting}
\end{Shaded}

\includegraphics{01-regressioon_files/figure-latex/reg-plot-8-1.pdf}

\hypertarget{koosmuxf5jud-kategoriaalsete-tunnuste-puhul}{%
\subsection{Koosmõjud kategoriaalsete tunnuste puhul}\label{koosmuxf5jud-kategoriaalsete-tunnuste-puhul}}

Enne oli juttu, et kahe kategoriaalse sõltumatu tunnusega mudel ilma koosmõjudeta pole väga mõistlik. Vaatame nüüd kuidas see koosmõjudega välja näeks:

\begin{Shaded}
\begin{Highlighting}[]
\NormalTok{mudel9 }\OtherTok{\textless{}{-}} \FunctionTok{lm}\NormalTok{(sissetulek }\SpecialCharTok{\textasciitilde{}}\NormalTok{ sugu }\SpecialCharTok{*}\NormalTok{ haridustase, }\AttributeTok{data =}\NormalTok{ piaac)}
\FunctionTok{summary}\NormalTok{(mudel9)}
\end{Highlighting}
\end{Shaded}

\begin{verbatim}
## 
## Call:
## lm(formula = sissetulek ~ sugu * haridustase, data = piaac)
## 
## Residuals:
##      Min       1Q   Median       3Q      Max 
## -1126.25  -311.40   -99.32   163.85  2639.98 
## 
## Coefficients:
##                            Estimate Std. Error t value Pr(>|t|)    
## (Intercept)                 1004.53      17.95  55.967   <2e-16 ***
## suguNaine                   -458.80      24.76 -18.533   <2e-16 ***
## haridustaseKõrge             252.60      28.59   8.836   <2e-16 ***
## haridustaseMadal             -73.55      37.63  -1.955   0.0507 .  
## suguNaine:haridustaseKõrge    50.69      36.65   1.383   0.1667    
## suguNaine:haridustaseMadal    10.76      56.52   0.190   0.8491    
## ---
## Signif. codes:  0 '***' 0.001 '**' 0.01 '*' 0.05 '.' 0.1 ' ' 1
## 
## Residual standard error: 526.1 on 3978 degrees of freedom
##   (3648 observations deleted due to missingness)
## Multiple R-squared:  0.1715, Adjusted R-squared:  0.1705 
## F-statistic: 164.7 on 5 and 3978 DF,  p-value: < 2.2e-16
\end{verbatim}

Sellisest mudelist saame välja lugeda kõikide gruppide ristlõigete (kõrge haridustasemega naised, kõrge haridustasemega mehed jne) keskmised. Arvutame näiteks välja kõrge haridustasemega (\emph{koh}) naiste ja madala haridustasemega (\emph{mh}) meeste keskmised matemaatilise lugemisoskuse skoorid:

\[\hat{y}_{sissetulek}=\beta_0+\beta_1 \times naine + \beta_2 \times koh + \beta_3 \times mh + \beta_4 \times naine \times koh + \beta_5 \times naine \times mh\]

\begin{align}
&=
  \begin{cases}
    1004.5+(-458.8) \times 1 + 252.60 \times 1 + (-73.55) \times 0 + 50.69 \times 1 + 10.76 \times 0 & \quad \text{kõrge haridustasemega naised}\\
    1004.5+(-458.8) \times 0 + 252.60 \times 0 + (-73.55) \times 1 + 50.69 \times 0 + 10.76 \times 0 & \quad \text{madala haridustasemega mehed}
  \end{cases}\\
&=
  \begin{cases}
    1004.5+(-458.8) + 252.60 + 50.69 & \quad \text{kõrge haridustasemega naised}\\
    1004.5 + (-73.55) & \quad \text{madala haridustasemega mehed}
  \end{cases}\\
&=
  \begin{cases}
    849 & \quad \text{kõrge haridustasemega naised}\\
    931 & \quad \text{madala haridustasemega mehed}
  \end{cases}
\end{align}

Vaatame seda mudelit ka graafiliselt (kasutame paketi \texttt{interactions} funktsiooni \texttt{cat\_plot()}):

\begin{Shaded}
\begin{Highlighting}[]
\FunctionTok{cat\_plot}\NormalTok{(mudel9, }\AttributeTok{pred =}\NormalTok{ haridustase, }\AttributeTok{modx =}\NormalTok{ sugu, }\AttributeTok{colors =}  \FunctionTok{c}\NormalTok{(}\StringTok{"\#972D15"}\NormalTok{, }\StringTok{"\#02401B"}\NormalTok{))}
\end{Highlighting}
\end{Shaded}

\includegraphics{01-regressioon_files/figure-latex/unnamed-chunk-20-1.pdf}

Ülesanne!

\begin{itemize}
\tightlist
\item
  Looge koosmõjuga regressioonimudel, millega hindate soo ja laste olemasolu mõju sissetulekule.
\item
  Esitage koosmõjud graafikul (\texttt{cat\_plot()} abiga)
\end{itemize}

\hypertarget{mudelite-vuxf5rdlemine}{%
\section{Mudelite võrdlemine}\label{mudelite-vuxf5rdlemine}}

Milline on hea mudel? See peaks muidugi seletama võimalikult palju sõltuva tunnuse varieeruvusest. Samas peaks see olema ka võimalikult ökonoomne, st see peaks sisaldama ainult tunnuseid, mis mudelit oluliselt paremaks teevad. Siin on rõhk sõnal ``oluliselt''. Iga lisanduv tunnus teeb mudeli mingil määral paremaks, kuid see paranemine võib olla mikroskoopiline. Kuidas siis hinnata, kas mudel \(n+1\) tunnusega on oluliselt parem kui \(n\) tunnusega mudel?

Me saame vaadata lisanduva tunnuse standardviga, \emph{t}-väärtust ja sellega seonduvat \emph{p}-väärtust. Kuid nagu enne jutuks oli, testib see ainult konkreetse koefitsiendi erinevust nullist. Meid aga huvitab kogu mudeli kvaliteet. Võimalus on ka võrrelda mudelite \(R^2\) väärtusi, kuid need on pigem kirjeldavad, ega anna meile indikatsiooni sellest kas üks väärtus on oluliselt parem kui teine.

Erinevate mudelite statistiliselt olulist erinevust saame testida hii-ruut testiga kasutades \texttt{anova()} funktsiooni. Seda saab teha ainult siis kui mudelid on omavehl seotud (\emph{nested}), st keerukam (rohkemate tunnustega) mudel peab sisdaldama kõiki lihtsama mudeli tunnuseid.

\begin{Shaded}
\begin{Highlighting}[]
\NormalTok{mudel\_test1 }\OtherTok{\textless{}{-}} \FunctionTok{lm}\NormalTok{(numeracy }\SpecialCharTok{\textasciitilde{}}\NormalTok{ literacy, }\AttributeTok{data =}\NormalTok{ piaac)}
\NormalTok{mudel\_test2 }\OtherTok{\textless{}{-}} \FunctionTok{lm}\NormalTok{(numeracy }\SpecialCharTok{\textasciitilde{}}\NormalTok{ literacy }\SpecialCharTok{+}\NormalTok{ sugu, }\AttributeTok{data =}\NormalTok{ piaac)}
\FunctionTok{anova}\NormalTok{(mudel\_test1, mudel\_test2, }\AttributeTok{test =} \StringTok{"Chisq"}\NormalTok{)}
\end{Highlighting}
\end{Shaded}

\begin{verbatim}
## Analysis of Variance Table
## 
## Model 1: numeracy ~ literacy
## Model 2: numeracy ~ literacy + sugu
##   Res.Df     RSS Df Sum of Sq  Pr(>Chi)    
## 1   7584 4848657                           
## 2   7583 4734885  1    113772 < 2.2e-16 ***
## ---
## Signif. codes:  0 '***' 0.001 '**' 0.01 '*' 0.05 '.' 0.1 ' ' 1
\end{verbatim}

Tõlgendame jällegi testi \emph{p}-väärtust. Kui see on väiksem kui \(0.05\) (usaldusnivool \(95\%\)), siis võime järeldada, et mudelid on oluliselt erinevad, mis tähendab omakorda, et lisatud tunnus tõstis mudeli selgitusvõimet olulisel määral.

\hypertarget{regressioonimudeli-eeldused}{%
\section{Regressioonimudeli eeldused}\label{regressioonimudeli-eeldused}}

Nagu iga meetodi puhul, on ka lineaarsel regressioonanalüüsil rida eeldusi, mis peavad olema täidetud, et analüüsist korrektseid järeldusi oleks võimalik teha.

\begin{enumerate}
\def\labelenumi{\arabic{enumi}.}
\tightlist
\item
  Esimene ja vahest ka kõige olulisem eeldus on \textbf{lineaarne suhe sõltuva ja sõltumatu(te) tunnuse vahel}. Kõrvaloleval joonisel on esitatud neli andmestikku, mille regressioonisirged on identsed (\(y=3+0.5x\)). Tegelikult on identsed ka kõik muud andmete statistilised omadused (\(x\)'i keskmine, \(y\)'i keskmine, \(x\)'i dispersioon, \(y\)'i dispersioon ja ka korrelatsioon). Ometi on visuaalselt näha, et kõik andmestikud on väga erinevad. Seega peaks regressioonanalüüsi (või tegelikult ükskõik mis analüüsi) puhul olema alati esimene samm neid graafiliselt uurida. Kui tunnuste vaheline seos ei ole lineaarne, piisab mõnel juhul tunnuste mittelineaarsest transformeerimisest (see peaks olema ka muidugi teoreetiliselt põhjendatud). Kui seos on eksponentsiaalne, siis võib kaaluda \emph{log}-transformatsiooni. Kui seos on paraboolne, siis võib kaaluda ruutu tõstetud tunnuse lisamist (\(y = \beta_0+\beta_1x+\beta_2x^2\)). Taoliste transformatsioonide juures peab meeles pidama, et koos nendega muutub ka mudeli tõlgendus.
\end{enumerate}

\begin{figure}
\centering
\includegraphics{01-regressioon_files/figure-latex/anscombe-1.pdf}
\caption{\label{fig:anscombe}Anscombe kvartett}
\end{figure}

\begin{enumerate}
\def\labelenumi{\arabic{enumi}.}
\setcounter{enumi}{1}
\item
  Lineaarse regressiooni puhul peaks tähelepanelik olema ka \textbf{erindite} (\emph{outliers})suhtes, st vaatluste suhtes, mis erinevad teistest väga olulisel määral (nagu ka kõrvalolevalt jooniselt näha). Mõnede andmete puhul on erindid paratamatud (näiteks sissetuleku puhul, kus suurem osa inimesi on koondunud keskmise sissetuleku ümber, kuid mõned üksikud teenivad sellest oluliselt enam). Sellisel juhul tasuks kaaluda jällegi tunnuse transformeerimist (sissetuleku puhul näiteks log-skaalale). Kui tegemist on mõne üksiku erindiga, võiks ju selle aluseks oleva vaatluse ka lihtsalt analüüsist välja jätta. Siin tuleks aga olla väga ettevaatlik. Andmete või sellest saadava informatsiooni tahtlik vähendamine (näiteks pidevtunnuste kategoriseerimine) ei ole üldiselt kunagi hea mõte. Seda enam ei ole hea mõte andmete vähendamine eesmärgiga mudelit paremaks teha. Kui aga erindite tekkimine on mingil moel teoreetiliselt seletatav või tulenenud näiteks veast andmekorjel, siis võib seda loomulikult teha.
\item
  \textbf{Jääkide dispersiooni homogeensus} (\emph{homoscedasticity}). Jäägid peaksid hinnatud väärtuste lõikes olema homogeense ja konstantse variatiivsusega, st ühtlaselt jaotunud kõikide \(\hat{y}\) väärtuste ümber. Selle eelduse rikkumine mõjutab eelkõige standardvigu (need ei kehti enam kõikidele \(\hat{y}\) väärtustele ühtlaselt) ja seeläbi loomulikult ka usaldusintervalle ning \emph{p}-väärtusi. Lahenduseks võivad olla nn robustsed standardvead (\emph{robust standard errors}), mis võtavad varieeruvuse erinevust arvesse.
\end{enumerate}

\begin{Shaded}
\begin{Highlighting}[]
\DocumentationTok{\#\# Robust standard errors}
\NormalTok{mudel6 }\OtherTok{\textless{}{-}} \FunctionTok{lm}\NormalTok{(numeracy }\SpecialCharTok{\textasciitilde{}}\NormalTok{ literacy }\SpecialCharTok{*}\NormalTok{ sugu, }
             \AttributeTok{data =}\NormalTok{ piaac)}

\CommentTok{\# Tavalised standardvead}
\FunctionTok{summary}\NormalTok{(mudel6)}
\end{Highlighting}
\end{Shaded}

\begin{verbatim}
## 
## Call:
## lm(formula = numeracy ~ literacy * sugu, data = piaac)
## 
## Residuals:
##      Min       1Q   Median       3Q      Max 
## -102.441  -16.551   -0.094   16.867   88.714 
## 
## Coefficients:
##                     Estimate Std. Error t value Pr(>|t|)    
## (Intercept)        36.129029   2.620050  13.789  < 2e-16 ***
## literacy            0.871523   0.009417  92.543  < 2e-16 ***
## suguNaine           4.596729   3.619797   1.270 0.204164    
## literacy:suguNaine -0.044937   0.012972  -3.464 0.000535 ***
## ---
## Signif. codes:  0 '***' 0.001 '**' 0.01 '*' 0.05 '.' 0.1 ' ' 1
## 
## Residual standard error: 24.97 on 7582 degrees of freedom
##   (46 observations deleted due to missingness)
## Multiple R-squared:  0.6948, Adjusted R-squared:  0.6946 
## F-statistic:  5753 on 3 and 7582 DF,  p-value: < 2.2e-16
\end{verbatim}

\begin{Shaded}
\begin{Highlighting}[]
\CommentTok{\# Robustsed standardvead}
\FunctionTok{library}\NormalTok{(sandwich)}
\FunctionTok{library}\NormalTok{(lmtest)}

\FunctionTok{coeftest}\NormalTok{(mudel6, }\AttributeTok{vcov. =} \FunctionTok{vcovHC}\NormalTok{(mudel6))}
\end{Highlighting}
\end{Shaded}

\begin{verbatim}
## 
## t test of coefficients:
## 
##                      Estimate Std. Error t value  Pr(>|t|)    
## (Intercept)        36.1290294  2.7887350 12.9553 < 2.2e-16 ***
## literacy            0.8715229  0.0098976 88.0538 < 2.2e-16 ***
## suguNaine           4.5967288  3.8313081  1.1998 0.2302621    
## literacy:suguNaine -0.0449372  0.0135552 -3.3151 0.0009203 ***
## ---
## Signif. codes:  0 '***' 0.001 '**' 0.01 '*' 0.05 '.' 0.1 ' ' 1
\end{verbatim}

\begin{Shaded}
\begin{Highlighting}[]
\CommentTok{\# Saab ka nii}
\FunctionTok{library}\NormalTok{(sandwich)}
\FunctionTok{vcovHC}\NormalTok{(mudel6) }\SpecialCharTok{\%\textgreater{}\%} 
  \FunctionTok{diag}\NormalTok{() }\SpecialCharTok{\%\textgreater{}\%} 
  \FunctionTok{sqrt}\NormalTok{()}
\end{Highlighting}
\end{Shaded}

\begin{verbatim}
##        (Intercept)           literacy          suguNaine literacy:suguNaine 
##        2.788735004        0.009897618        3.831308067        0.013555193
\end{verbatim}

\begin{enumerate}
\def\labelenumi{\arabic{enumi}.}
\setcounter{enumi}{3}
\item
  \textbf{Jääkide normaaljaotus}. Regressiooni jäägid peaksid olema normaaljaotusega \(e_i \sim N(0, \sigma^2)\), seega enamus jääke peaks jääma nulli ümber ning mida suuremad jäägid, seda vähem neid olema peaks. See eeldus on eelkõige oluline regressioonikoefitsientide \emph{t}-testi jaoks.
\item
  \textbf{Jääkide sõltumatus}. Ühe vaatluse jäägid ei tohiks olla korreleeritud teise vaatluse jääkidega. Selline olukord võib tekkida näiteks siis kui meil mudelist välja jäänud mingi oluline tunnus (ühe tunnuse regressiooni puhul on see muidugi vaid hüpoteetiline olukord), näiteks hindame õpilaste testiskoore lähtuvalt nende õppimisele kulunud ajast, kuid ei arvesta, et õpilased tulevad näiteks erinevatest koolidest, kus võib olla erinev tase. Seega õpilaste tulemused ei ole enam sõltumatud, vaid sõltuvad koolist. Regressioonikoefitsientide standardvigade arvutamisel lähtutakse eeldusest, et jäägid on sõltumatud. Kui jäägid on korreleeritud, siis võib juhtuda, et me alahindame standardvigade suurust ehk siis oleme oma tulemustes ülemäära kindlad (usaldusintervallid ning \emph{p}-väärtused tulevad liialt väikesed) ning võime näha seoseid seal kus neid tegelikult ei ole. Lahenduseks võiks olla puuduolevate tunnuste lisamine mudelisse (konkrteetse näite puhul kooli tunnus).
\item
  Kui kaks sõltumatut tunnust on teineteisega väga tugevalt seotud põhjusteab see nn \textbf{kollineaarsust}. See võib tekitada probleeme mudeli hindamisel ning ka tõlgendusel. Lisaks kipuvad standardvead liialt suureks minema, mis tähendab seda, et kaotame oma tulemuste täpsuses ja võime mitte näha seoseid seal, kus need tegelikult olemas on. Seega võiks tähele panna, et korrelatsioon sõltumatute muutujate vahel peaks alati olema väiksem kui korrelatsioon sõltuva ja sõltumatu muutuja vahel.
\end{enumerate}

\hypertarget{kuidas-eelduste-tuxe4idetust-hinnata}{%
\section{Kuidas eelduste täidetust hinnata?}\label{kuidas-eelduste-tuxe4idetust-hinnata}}

Eelduste hindamiseks on loomulikult mitmeid teste, kuid kõige lihtsam on seda mudeli diagnostiliste joonistega.

\begin{Shaded}
\begin{Highlighting}[]
\NormalTok{mod }\OtherTok{\textless{}{-}} \FunctionTok{lm}\NormalTok{(}\AttributeTok{formula =}\NormalTok{ sissetulek }\SpecialCharTok{\textasciitilde{}}\NormalTok{ numeracy }\SpecialCharTok{+}\NormalTok{ vanus }\SpecialCharTok{+}\NormalTok{ sugu }\SpecialCharTok{+}\NormalTok{ haridustase, }
    \AttributeTok{data =}\NormalTok{ piaac)}
\end{Highlighting}
\end{Shaded}

\begin{Shaded}
\begin{Highlighting}[]
\FunctionTok{plot}\NormalTok{(mod, }\DecValTok{1}\NormalTok{)}
\end{Highlighting}
\end{Shaded}

\includegraphics{01-regressioon_files/figure-latex/unnamed-chunk-24-1.pdf}

Kontrollime mittelineaarse seose olemasolu. Punktid peaksid olema ühtlaselt ümber keskmise joone jaotunud. Ei tohiks olla mingit ilmset mustrit.

\begin{Shaded}
\begin{Highlighting}[]
\FunctionTok{plot}\NormalTok{(mod, }\DecValTok{2}\NormalTok{)}
\end{Highlighting}
\end{Shaded}

\includegraphics{01-regressioon_files/figure-latex/unnamed-chunk-25-1.pdf}

Kas jäägid on normaalselt jaotunud? Punktid peaksid ühtima diagonaalse joonega.

\begin{Shaded}
\begin{Highlighting}[]
\FunctionTok{plot}\NormalTok{(mod, }\DecValTok{3}\NormalTok{)}
\end{Highlighting}
\end{Shaded}

\includegraphics{01-regressioon_files/figure-latex/unnamed-chunk-26-1.pdf}

Kas jääkide dispersioon on homogeenne? Punane joon peaks olema horisontaalne ja punktid peaksid olema ühtlaselt jaotunud ega tohiks mingit mustrit moodustada.

\begin{Shaded}
\begin{Highlighting}[]
\FunctionTok{plot}\NormalTok{(mod, }\DecValTok{5}\NormalTok{)}
\end{Highlighting}
\end{Shaded}

\includegraphics{01-regressioon_files/figure-latex/unnamed-chunk-27-1.pdf}

Kas mudelis on mudelit oluliselt mõjutavaid erindeid? Kui on, siis peaksid need olema paremal all või paremal üleval nurgas ning kaugemal kui punktiirjoon (antud juhul neid ei ole ja seega ei ole ka punktiirjoont näha).

\hypertarget{part-uxfcldistatud-lineaarsed-mudelid}{%
\part{Üldistatud lineaarsed mudelid}\label{part-uxfcldistatud-lineaarsed-mudelid}}

\hypertarget{logistiline-regressioon}{%
\chapter{Logistiline regressioon}\label{logistiline-regressioon}}

Logistilise regressiooniga (logit-mudeliga) saame hinnata sõltumatute tunnuste mõju binaarsele sõltuvale tunnusele (töötav/töötu, käis valimas/ei käinud valimas, surnud/ei ole surnud). Teisisõnu, hindame tõenäosust mingi sündmuse toimumiseks (\emph{success}/\emph{failure}). Sõltuva tunnuse \(y\) jaotus on määratletud kui sündmuse toimumise tõenäosus \(P(Y=1)=\pi\).\\
Tavalise regressiooni mudel oli väljendatav kui \(\bar{y}=\beta_0+\beta_p x_p\). Miks me ei võiks pidevtunnuselise \(y\) keskmist asendada \(\pi\)'ga: \(\bar{\pi}=\beta_0+\beta_k x_k\)? Aga sellepärast, et tõenäosus on piiritletud \(0\) ja \(1\)'ga, samas kui lineaarne funktsioon hõlmab kõiki reaalarvulisi väärtusi. Seega on ülimalt tõenäoline, et mingite \(x\)'i väärtuste puhul on prognoositav \(y\) väärtus suurem kui \(1\) või väiksem kui \(0\). Lisaks tekivad probleemid jääkide struktuuriga (tavaline regressioon eeldab normaaljaotust) ja jääkide dispersiooniga (tavaline regressioon eeldab konstantset hajuvust).

\begin{figure}
\centering
\includegraphics{02-logit_files/figure-latex/glm-1-1.pdf}
\caption{\label{fig:glm-1}Lineaarse regressiooni kasutamine binaarse sõltuva tunnusega}
\end{figure}

\hypertarget{ux161ansid}{%
\section{Šansid}\label{ux161ansid}}

Kuidas me saaksime tõenäosuse skaala (\(0 \dots1\)) teisendada pidevaks skaalaks (\(-\infty \dots \infty\))? Et saada lahti maksimaalsest väärtusest (\(1\)), on võimalik kasutada sündmuse toimumise tõenäosuse asemel sündmuse toimumise šanssi (\emph{odds}). Šanssideks nimetatakse sündmuse toimumise ja mittetoimumise suhet:

\[\text{šansid}=\frac{p}{(1-p)}\]

Näiteks kulli ja kirja viskamisel on kulli saamise šanss \(\frac{0.5}{(1-0.5)}=1\). Šanss võtta kaardipakist ruutu on \(\frac{0.25}{(1-0.25)}=\frac{1}{3}=0.33\).

\begin{figure}
\centering
\includegraphics{02-logit_files/figure-latex/odds-1.pdf}
\caption{\label{fig:odds}Šansside ja tõenäosuse suhe}
\end{figure}

Šansid saab omakorda teisenda tagasi tõenäosuseks:

\[p=\frac{\text{šansid}}{1+\text{šansid}}\]

\hypertarget{logit}{%
\section{Logit}\label{logit}}

Kuid tõenäosuse alumine piir jääb sel juhul ikkagi ette. Ka šansid on altpoolt piiratud (nad ei saa olla väiksemad kui \(0\)). Lahenduseks on võtta šansside logaritm. Saadud väärtust nimetatakse \textbf{logit}-iks (\emph{log odds}):

\[\text{logit}=\log \bigg(\frac{p}{(1-p)}\bigg)\]

\begin{figure}
\centering
\includegraphics{02-logit_files/figure-latex/glm-3-1.pdf}
\caption{\label{fig:glm-3}Logit-i ja tõenäosuse suhe}
\end{figure}

\hypertarget{logit-mudel}{%
\section{Logit mudel}\label{logit-mudel}}

Lõpuks saame mudeli võrrandi kokku panna:

\[\text{logit}(\pi_i)=\text{log} \left(\dfrac{\pi_i}{1-\pi_i}\right)=\beta_0+\beta_1 x_i\]

Või kui võtame mõlemast poolest eksponendi:

\[\frac{\pi_i}{1-\pi_i}=e^{({\beta_0}+\beta_1 x_i)}\]
Sama võrrandit saab esitada ka nii:

\[\pi_i=Pr(Y_i=1|X_i=x_i)=\dfrac{e^{(\beta_0+\beta_1 x_i)}}{1+e^{(\beta_0+\beta_1 x_i)}}\]\\
või hoopis nii:

\[\pi_i=Pr(Y_i=1|X_i=x_i)=\frac{1}{1+e^{-\beta_0-\beta_1 x_i}}\]

\begin{figure}
\centering
\includegraphics{02-logit_files/figure-latex/glm-2-1.pdf}
\caption{\label{fig:glm-2}Logistiline regressioon võrdluses lineaarse regressiooniga}
\end{figure}

\hypertarget{mudeli-tuxf5lgendus}{%
\section{Mudeli tõlgendus}\label{mudeli-tuxf5lgendus}}

Tavalise regresioonimudeliga saime prognoosida \(y\) väärtust mingite \(x\) väärtuste korral (ja \(y\) muutust, kui \(x\) muutub ühe ühiku võrra). Sama kehtib ka logistilise regressiooni korral. Kuid mida me siinjuures täpsemalt prognoosime? Tahaksime kindlasti prognoosida (uuritava sündmuse toimumise) tõenäosust. Kuid kuna me teisendasime tõenäosuse logititeks, siis tegelikult saame prognoosida hoopis logitit. Ja ka ühe ühikuline muutus \(x\)-is ei peegelda mitte \(y\) tõenäosuse muutust, vaid muutust logit(\(y\))-is. Logiteid ei oska me (vähemalt esialgu) kuidagi tõenäosuslikult tõlgendada. Mida siis teha? Lahenduseks on võtta \emph{logit}-i võrrandi mõlemast poolest eksponent \(exp(logit) = exp(\beta_0+\beta_1 x_i) \implies \frac{\pi_i}{1-\pi_i}=e^{({\beta_0}+\beta_1 x_i)}\). Sellisel juhul saab \(y\)-t tõlgendada kui šansse ja \(\beta\)-t kui muutust šanssides (mitu korda \(x\)-i ühe ühiku muutudes \(y\) šansid suurenevad või vähenevad). Seda šansside muutust väljendavat kordajat nimetatakse šansside suhteks.

\hypertarget{ux161ansside-suhe}{%
\subsection{Šansside suhe}\label{ux161ansside-suhe}}

Šansid saime leida valemiga:

\[\text{šansid}=\frac{p}{(1-p)}\]\\
Valemist võime välja lugeda järgmist:

\begin{enumerate}
\def\labelenumi{\arabic{enumi}.}
\tightlist
\item
  Šansid on alati positiivsed\\
\item
  Kui šansid on \(1\), siis on sündmuse toimumise ja mittetoimumise tõenäosus võrdsed (\(p=0.5\)).\\
\item
  Kui šansid on suuremad kui \(1\), siis on sündmuse toimumise tõenäosus suurem kui mittetoimumise tõenäosus (\(p>0.5\)) ja vastupidi.\\
  Näiteks kui sündmuse toimumise tõenäosus on \(0.8\), siis on šansid \(\frac{0.8}{1-0.8}=\frac{0.8}{0.2}=4\). Seega sündmuse toimumise tõenäosus on \(4\) korda suurem kui selle mittetoimumise tõenäosus. Kui sündmuse toimumise tõenäosus on \(0.2\), siis on šansid \(\frac{0.2}{1-0.2}=\frac{0.2}{0.8}=\frac{1}{4}=0.25\). Sündmuse toimumise tõenäosus on \(4\) korda väiksem kui selle mittetoimumise tõenäosus.
\end{enumerate}

Vaatame näidet, kus hindame hääletamise tõenäosust ning abielu mõju sellele:

\begin{longtable}[]{@{}lll@{}}
\toprule
& Hääletab & Ei hääleta \\
\midrule
\endhead
Abielus & 0.75 & 0.25 \\
Ei ole abielus & 0.54 & 0.46 \\
\bottomrule
\end{longtable}

Abielus inimeste puhul on šanss hääletamiseks \(\frac{0,75}{0,25} = \frac{3}{1}= 3\) (iga mittehääletaja kohta on kolm hääletajat).

Vallaliste puhul on šanss hääletamiseks \(\frac{0,54}{0,46} = 1,17\) (iga mittehääletaja kohta on \(1,17\) hääletajat).

Meid huvitab kuidas sõltumatu tunnuse muutus sündmuse toimumise šansse mõjutab, ehk kui palju muutuvad šansid kui sõltumatu tunnus muutub ühe ühiku võrra. Seda muutust väljendabki šansside suhe (\emph{odds ratio} ehk OR)

\[OR=\frac{y \text{ šanss juhul kui } x \text{ väärtus muutub ühe ühiku võrra}}{y \text{ šanss juhul kui } x \text{ väärtus jääb samaks}}\]

Kui palju on abielus olijate šansid hääletamiseks suuremad kui vallalistel? Šansside suhe on \(\frac{3}{1,17}=2,56\). Ehk siis abielus olijate šanss hääletada on kaks ja pool korda suurem. Abielu tunnuse ühe ühiku muutumisega muutuvad šansid \(2,56\) korda ehk suurenevad \(156\%\).

\hypertarget{logistiline-regressioon-r-is}{%
\section{Logistiline regressioon R-is}\label{logistiline-regressioon-r-is}}

Võtame R-i näidisandmestiku \texttt{Titanic}, mis kirjeldab Titanicul hukkunute ja ellujäänute sugu, vanust ja reisijaklassi. Üritame hinnata kuidas ja kas need tunnused mõjutasid ellujäämist.\\
GLMi mudeleid saab R'is defineerida \texttt{glm()} funktsiooniga. Selle loogika ja argumendid on sarnased \texttt{lm()} funktsiooni omadele. Peamiseks erinevuseks on see, et nüüd peame defineerima ka sõltuva tunnuse jaotuse ja linkfunktsiooni. See käib argumendiga \texttt{family}. Logistilise regressiooni jaoks peame defineerima \texttt{family\ =\ binomial(link\ =\ \textquotesingle{}logit\textquotesingle{})} (sõltuva tunnuse jaotus on binoomjaotus ja linkfunktsioon on logit).\\
Vaatame kõigepealt soo mõju:

\begin{Shaded}
\begin{Highlighting}[]
\CommentTok{\# Andmestik on algselt tabeli kujul.}

\CommentTok{\# Saaksime seda ka sellisel kujul analüüsida,}
\CommentTok{\# kuid mugavam ja selgem on, kui keerame ta}
\CommentTok{\# nn tavalisele kujule. Kasutame selleks }
\CommentTok{\# paketi tidyr funktsiooni uncount()}

\NormalTok{titanic }\OtherTok{\textless{}{-}}\NormalTok{ datasets}\SpecialCharTok{::}\NormalTok{Titanic }\SpecialCharTok{\%\textgreater{}\%} 
  \FunctionTok{as.data.frame}\NormalTok{() }\SpecialCharTok{\%\textgreater{}\%} 
\NormalTok{  tidyr}\SpecialCharTok{::}\FunctionTok{uncount}\NormalTok{(Freq)}

\CommentTok{\# vaatame andmestiku esimesi ridu}
\FunctionTok{head}\NormalTok{(titanic)}
\end{Highlighting}
\end{Shaded}

\begin{verbatim}
##   Class  Sex   Age Survived
## 1   3rd Male Child       No
## 2   3rd Male Child       No
## 3   3rd Male Child       No
## 4   3rd Male Child       No
## 5   3rd Male Child       No
## 6   3rd Male Child       No
\end{verbatim}

\begin{Shaded}
\begin{Highlighting}[]
\CommentTok{\# Defineerime mudeli}
\NormalTok{mudel7 }\OtherTok{\textless{}{-}} \FunctionTok{glm}\NormalTok{(}\FunctionTok{I}\NormalTok{(Survived }\SpecialCharTok{==} \StringTok{"Yes"}\NormalTok{)}\SpecialCharTok{\textasciitilde{}}\NormalTok{Sex, }\AttributeTok{data =}\NormalTok{ titanic, }\AttributeTok{family =} \FunctionTok{binomial}\NormalTok{())}
\FunctionTok{summary}\NormalTok{(mudel7)}
\end{Highlighting}
\end{Shaded}

\begin{verbatim}
## 
## Call:
## glm(formula = I(Survived == "Yes") ~ Sex, family = binomial(), 
##     data = titanic)
## 
## Deviance Residuals: 
##     Min       1Q   Median       3Q      Max  
## -1.6226  -0.6903  -0.6903   0.7901   1.7613  
## 
## Coefficients:
##             Estimate Std. Error z value Pr(>|z|)    
## (Intercept)  -1.3128     0.0588  -22.32   <2e-16 ***
## SexFemale     2.3172     0.1196   19.38   <2e-16 ***
## ---
## Signif. codes:  0 '***' 0.001 '**' 0.01 '*' 0.05 '.' 0.1 ' ' 1
## 
## (Dispersion parameter for binomial family taken to be 1)
## 
##     Null deviance: 2769.5  on 2200  degrees of freedom
## Residual deviance: 2335.0  on 2199  degrees of freedom
## AIC: 2339
## 
## Number of Fisher Scoring iterations: 4
\end{verbatim}

\begin{Shaded}
\begin{Highlighting}[]
\CommentTok{\# I(Survived == "Yes") notatsiooniga saame tekstilise tunnuse }
\CommentTok{\# teisendada loogilisek tunnuseks }
\CommentTok{\# saaksime seda teha ka näiteks nii:}
\CommentTok{\#   titanic$surv \textless{}{-} titanic$Survived == "Yes"}
\CommentTok{\#   glm(surv\textasciitilde{}Sex, data = titanic, family = binomial())}
\CommentTok{\# Tulemus on sama}
\end{Highlighting}
\end{Shaded}

Väljundist leiame kõigepealt regressioonikoefitsiendid, nende standardvead, z-väärtused ja z-testi \emph{p}-väärtuse\footnote{z-test puhul on tegemist t-testi analoogiga, mis ei lähtu mitte t-jaotusest, vaid normaaljaotusest. Tõlgendus on aga sama}. Kuid koefitsiendid on nüüd logititskaalal ja seepärast küllaltki raskesti tõlgendatavad. Saame siiski järeldada, et naiste tõenäosus ellu jääda oli suurem kui meestel (koefitsient on positiivne). Mõnevõrra lihtsam on tõlgendada šansside suhet. Selleks peame koefitsientidest eksponendi võtma:

\begin{Shaded}
\begin{Highlighting}[]
\FunctionTok{exp}\NormalTok{(}\FunctionTok{coef}\NormalTok{(mudel7))}
\end{Highlighting}
\end{Shaded}

\begin{verbatim}
## (Intercept)   SexFemale 
##   0.2690616  10.1469660
\end{verbatim}

Vabaliiget tõlgendame kui referentsgrupi (antud juhul meeste) šansse ellu jääda. Seega mehe šanss Titanicul ellu jääda oli 0.26, ehk siis iga hukkunud mehe kohta jäi ellu 0.26 meest, või vastupidi \(1 / 0.269 = 3.7\), iga ellujäänud mehe kohta hukkus 3.7 meest. Saame välja arvutada ka meeste ellujaamise tõenäosuse:

\[\pi=\frac{\text{šansid}}{1+\text{šansid}} = \frac{0.269}{1+0.269} = 0.21\]

Naiste puhul tõlgendame šansside suhet. Ehk kui palju muudab naiseksolemine võrreldes meestega ellujäämise šansse. Tuleb välja, et ligi 10 korda. Seega naiste šansid ellu jääda olid \(10.147 \times 0.269 = 2.73\). Iga hukkunud naise kohta jäi 2.7 naist ellu. Naiste ellujäämise tõenäosus oli:

\[\pi=\frac{\text{šansid}}{1+\text{šansid}} = \frac{2.73}{1+2.73} = 0.73\]

Saame selle tõenäosuse ka otse välja arvutada, kui paneme koefitsiendid regressioonivõrrandisse (eelnevalt toodud valemi järgi):

\[\pi=\dfrac{e^{(\beta_0+\beta_1 x_i)}}{1+e^{(\beta_0+\beta_1 x_i)}} = \dfrac{e^{(-1.313+2.317 \times 1)}}{1+e^{(-1.313+2.317 \times 1)}} = 0.73\]

Vaatme ka, kuidas muudab ellujäämise tõenäosust lisaks soole vanus (\emph{Age} on siin kategoriaalne tunnus kategooriatega \emph{Child} ja \emph{Adult}). Eeldame ka soo ja vanuse koosmõju:

\begin{Shaded}
\begin{Highlighting}[]
\NormalTok{mudel8 }\OtherTok{\textless{}{-}} \FunctionTok{glm}\NormalTok{(}\FunctionTok{I}\NormalTok{(Survived }\SpecialCharTok{==} \StringTok{"Yes"}\NormalTok{)}\SpecialCharTok{\textasciitilde{}}\NormalTok{Sex}\SpecialCharTok{*}\NormalTok{Age, }\AttributeTok{data =}\NormalTok{ titanic, }\AttributeTok{family =} \FunctionTok{binomial}\NormalTok{())}
\FunctionTok{summary}\NormalTok{(mudel8)}
\end{Highlighting}
\end{Shaded}

\begin{verbatim}
## 
## Call:
## glm(formula = I(Survived == "Yes") ~ Sex * Age, family = binomial(), 
##     data = titanic)
## 
## Deviance Residuals: 
##     Min       1Q   Median       3Q      Max  
## -1.6497  -0.6732  -0.6732   0.7699   1.7865  
## 
## Coefficients:
##                    Estimate Std. Error z value Pr(>|z|)    
## (Intercept)         -0.1881     0.2511  -0.749   0.4539    
## SexFemale            0.6870     0.3970   1.731   0.0835 .  
## AgeAdult            -1.1811     0.2584  -4.571 4.86e-06 ***
## SexFemale:AgeAdult   1.7465     0.4167   4.191 2.77e-05 ***
## ---
## Signif. codes:  0 '***' 0.001 '**' 0.01 '*' 0.05 '.' 0.1 ' ' 1
## 
## (Dispersion parameter for binomial family taken to be 1)
## 
##     Null deviance: 2769.5  on 2200  degrees of freedom
## Residual deviance: 2312.8  on 2197  degrees of freedom
## AIC: 2320.8
## 
## Number of Fisher Scoring iterations: 4
\end{verbatim}

Täiskasvanuks olemine mõnevõrra langetab ellujäämise tõenäosust, kuid seda ainult meeste puhul (soo ja vanuse interaktsioon on positiivne). Tulemuste tõlgendamiseks võtame jälle koefitsientidest eksponendi:

\begin{Shaded}
\begin{Highlighting}[]
\FunctionTok{exp}\NormalTok{(}\FunctionTok{coef}\NormalTok{(mudel8))}
\end{Highlighting}
\end{Shaded}

\begin{verbatim}
##        (Intercept)          SexFemale           AgeAdult SexFemale:AgeAdult 
##          0.8285714          1.9878296          0.3069458          5.7344228
\end{verbatim}

Vabaliige kirjeldab ellujäämise šansse juhul kui sõltumatud tunnused on nullid. Ehk siis antud juhul ellujäämise šansse referentsgruppide kombinatsiooni puhul (lastest mehed ehk poisid). Seega poiste ellujäämise tõenäosus oli:

\[\pi = \frac{0.83}{1+0.83} = 0.45\]
Tüdrukute (lastest naised) ellujäämise šhansid olid ca kaks korda (1.99) suuremad kui poistel (tõenäosus \(\frac{0.83\times1.99}{1+(0.83\times1.99)} = 0.62\)). Täiskasvanud meeste šansid olid \(0.3\times0.83 = 0.24\) ja seega tõenäosus \(\frac{0.24}{1+0.24} = 0.19\). Täiskasvanud naiste puhul peame appi võtma koosmõju koefitsiendi. Täiskavanud naiste šansid moodustuvad \(0.83\times1.99\times0.3\times5.7 = 2.8\). Tõenäosusena teeb see \(0.74\).

Näeme, et koosmõju on antud mudeli puhul vägagi sisukas. Meeste puhul täiskavanuks olemine langetas ellujäämise šansse, naiste puhul aga tõstis.

Ülesanne!

\begin{itemize}
\tightlist
\item
  Piaaci andmestikus on tunnus \emph{staatus3}. Võtke see aluseks ja tehke uus loogiline (TRUE/FALSE) tunnus \emph{hoiv}, mis kirjeldaks kas inimene on või ei ole hõivatud.\\
\item
  Hinneke logistilise regressiooniga, kas hõivatus on mõjutatud inimese haridusest ja vanusest.
\end{itemize}

\hypertarget{mudeli-kvaliteet}{%
\section{Mudeli kvaliteet}\label{mudeli-kvaliteet}}

Kuidas hinnata mudeli kvaliteeti? Meile ei anta ei jääkide standardviga ega determinatsioonikordajat. Küll on aga väljunis toodud \emph{Null deviance} ja \emph{Residual deviance}. \emph{Deviance} kirjeldab mudeli hälvet ehk seda kui hästi (või õigem oleks öelda kui halvasti) meie mudel andmetega sobitub. Mida väiksem on \emph{deviance}, seda paremini mudel andmetes leiduvaid seoseid peegeldab. \emph{Null deviance} kirjeldab hälbimust nullmudelis, ehk ainult vabaliikmega mudelis (ainult keskmisega mudelis), ning \emph{Residual deviance} hälbimust sõltumatute tunnustega mudelis.

\hypertarget{mudeli-sobivus}{%
\subsection{Mudeli sobivus}\label{mudeli-sobivus}}

Mudeli sobivust andmetega (\emph{goodness of fit}) saame hinnata jääkhälbimuse (\emph{Residual deviance}) näitaja abil. Jääkhälbimus näitab kui palju mudeliga hinnatud \(Y\) väärtused empiirilistest \(Y\) väärtustest erinevad (analoogne asi lineaarse regressiooni puhul oli \emph{residual sum of squares}). Jääkhälbimuse abil saame võrrelda kui palju meie sobitatud mudel erineb küllastunud (\emph{saturated}) mudelist, st mudelist mis sobituks täiel määral andmetega ehk kus jääkälbimus oleks \(0\). Jääkhälbimus näitabki sisuliselt meie mudeli ja küllastunud mudeli erinevust. Juhul kui mudel on andmetega ``sobiv'', siis peaks jääkhälbimus olema võimalikult väike. Seda, kas see on piisavalt väike, saame testida hi-ruut testiga (arvestades mudeli vabadusasteid (\emph{degrees of freedom}).

\begin{Shaded}
\begin{Highlighting}[]
\NormalTok{res\_dev }\OtherTok{\textless{}{-}} \FunctionTok{deviance}\NormalTok{(mudel8)}
\NormalTok{res\_df }\OtherTok{\textless{}{-}} \FunctionTok{df.residual}\NormalTok{(mudel8)}
\FunctionTok{pchisq}\NormalTok{(res\_dev, res\_df, }\AttributeTok{lower.tail =}\NormalTok{ F)}
\end{Highlighting}
\end{Shaded}

\begin{verbatim}
## [1] 0.04209986
\end{verbatim}

\texttt{pchisq()} funktsiooniga saame testitulemusele ka \emph{p}-väärtuse. Näeme, et see on väiksem kui \(0.05\), mis tähendab, et meie mudel ei sobitu andmetega väga hästi (siin tahame, et \emph{p}-väärtus oleks võimalikult suur). Reaaleluliste andmetega ongi tegelikult väga keeruline hästi sobituvat mudelit leida. Seega üldjuhul me lihtsalt lepime, et meie mudel ei ole täiuslik ja jätame selle testi tähelepanuta.

\hypertarget{mudeli-statistiline-olulisus}{%
\subsection{Mudeli statistiline olulisus}\label{mudeli-statistiline-olulisus}}

Näeme, et sisuka mudeli hälve on võrreldes nullmudeliga tunduvalt väiksem\footnote{Peame siin arvestama ka erinevust vabadusastmetes. Kuigi sisuka mudeli hälve on väiksem, on selles ka vähem vabadusastmeid}. See tähendab, et tänu sõltumatutele tunnustele suudame me sõltuva tunnuse variatsiooni seletada paremini kui ainult keskmise abil. Aga kas mudeli hälve läks väiksemaks piisavalt paju, et me saaksime selle kohta ka statistiliselt olulisi järeldusi teha? Ehk siis kas me saame järeldada, et sõltumatud tunnused seletavad statistiliselt olulisel määral sõltuva tunnuse variatsiooni ja meie mudel on parem kui lihtsalt sõltuva tunnuse keskmine? Saame seda testida \emph{likelihood ratio} testiga. Arvutame esmalt hälvete erinevuse:

\begin{Shaded}
\begin{Highlighting}[]
\NormalTok{dev\_vahe }\OtherTok{\textless{}{-}}\NormalTok{ mudel8}\SpecialCharTok{$}\NormalTok{null.deviance }\SpecialCharTok{{-}}\NormalTok{ mudel8}\SpecialCharTok{$}\NormalTok{deviance}
\NormalTok{dev\_vahe}
\end{Highlighting}
\end{Shaded}

\begin{verbatim}
## [1] 456.6809
\end{verbatim}

Ja ka vabadusasteme erinevuse:

\begin{Shaded}
\begin{Highlighting}[]
\NormalTok{df\_vahe }\OtherTok{\textless{}{-}}\NormalTok{ mudel8}\SpecialCharTok{$}\NormalTok{df.null}\SpecialCharTok{{-}}\NormalTok{mudel8}\SpecialCharTok{$}\NormalTok{df.residual}
\NormalTok{df\_vahe}
\end{Highlighting}
\end{Shaded}

\begin{verbatim}
## [1] 3
\end{verbatim}

Hälvete vahe on jaotunud hii-ruut jaotuse alusel, seega saame hii-ruut jaotuse alusel määrata selle olulisust. Arvutame hälvete vahele olulisustõenäosuse. Kasutame selleks hii-ruut jaotuse funktsiooni \texttt{pchisq()}, mis tahab sisendina teatstatisikut (hälvete vahe) ja vabadusastemeid (vabadusasteme vahe). Samuti peame ütlema, et meid huvitab jaotuse parempoolse saba alla jääv tõenäosus.

\begin{Shaded}
\begin{Highlighting}[]
\FunctionTok{pchisq}\NormalTok{(dev\_vahe, df\_vahe, }\AttributeTok{lower.tail =}\NormalTok{ F)}
\end{Highlighting}
\end{Shaded}

\begin{verbatim}
## [1] 1.163316e-98
\end{verbatim}

Võime kasutada ka \texttt{anova()} funktsiooni, kus võrdleme kahte mudelt:

\begin{Shaded}
\begin{Highlighting}[]
\CommentTok{\# kasutame update() funktsiooni, millega}
\CommentTok{\# uuendame oma mudelit nii, et selle prediktoriks}
\CommentTok{\# oleks ainult vabaliige (tähistatud \textasciitilde{}1)}
\FunctionTok{anova}\NormalTok{(mudel8,}
      \FunctionTok{update}\NormalTok{(mudel8, }\SpecialCharTok{\textasciitilde{}}\DecValTok{1}\NormalTok{),    }
      \AttributeTok{test=}\StringTok{"Chisq"}\NormalTok{)}
\end{Highlighting}
\end{Shaded}

\begin{verbatim}
## Analysis of Deviance Table
## 
## Model 1: I(Survived == "Yes") ~ Sex * Age
## Model 2: I(Survived == "Yes") ~ 1
##   Resid. Df Resid. Dev Df Deviance  Pr(>Chi)    
## 1      2197     2312.8                          
## 2      2200     2769.5 -3  -456.68 < 2.2e-16 ***
## ---
## Signif. codes:  0 '***' 0.001 '**' 0.01 '*' 0.05 '.' 0.1 ' ' 1
\end{verbatim}

Või kasutame \texttt{lmtest} paketi \texttt{lrtest()} funtsiooni:

\begin{Shaded}
\begin{Highlighting}[]
\FunctionTok{library}\NormalTok{(lmtest)}
\FunctionTok{lrtest}\NormalTok{(mudel8)}
\end{Highlighting}
\end{Shaded}

\begin{verbatim}
## Likelihood ratio test
## 
## Model 1: I(Survived == "Yes") ~ Sex * Age
## Model 2: I(Survived == "Yes") ~ 1
##   #Df  LogLik Df  Chisq Pr(>Chisq)    
## 1   4 -1156.4                         
## 2   1 -1384.7 -3 456.68  < 2.2e-16 ***
## ---
## Signif. codes:  0 '***' 0.001 '**' 0.01 '*' 0.05 '.' 0.1 ' ' 1
\end{verbatim}

Kõikide eelnevate testide puhul huvitab meid eelkõige \emph{p} väärtus. Kui see on piisavalt väike (näiteks väiksem kui \(0,05\)), siis saame järeldada, et meie testitavad mudelid on piisavalt erinevad ehk siis sõltumatute tunnuste lisamine vähendas \emph{deviance}'i olulisel määral.

Eelnevas näites on \emph{p}- väärtus on väga väike, seega meie mudel on võrreldes nullmudeliga oluliselt parem.

Sama loogikaga saame ka testida kas uue sõltumatu tunnuse lisamine teeb mudeli oluliselt paremaks.

Lisaks saame \texttt{anova()} funktsiooniga testida kui palju iga sõltumatu tunnus mudelit paremaks tegi ja kas see paranemine oli statistiliselt oluline:

\begin{Shaded}
\begin{Highlighting}[]
\FunctionTok{anova}\NormalTok{(mudel8, }\AttributeTok{test =} \StringTok{"Chisq"}\NormalTok{)}
\end{Highlighting}
\end{Shaded}

\begin{verbatim}
## Analysis of Deviance Table
## 
## Model: binomial, link: logit
## 
## Response: I(Survived == "Yes")
## 
## Terms added sequentially (first to last)
## 
## 
##         Df Deviance Resid. Df Resid. Dev  Pr(>Chi)    
## NULL                     2200     2769.5              
## Sex      1   434.47      2199     2335.0 < 2.2e-16 ***
## Age      1     5.89      2198     2329.1    0.0152 *  
## Sex:Age  1    16.32      2197     2312.8 5.352e-05 ***
## ---
## Signif. codes:  0 '***' 0.001 '**' 0.01 '*' 0.05 '.' 0.1 ' ' 1
\end{verbatim}

\hypertarget{pseudo-r2}{%
\subsection{\texorpdfstring{Pseudo-\(R^2\)}{Pseudo-R\^{}2}}\label{pseudo-r2}}

Kui tavalise regressiooni puhul hindasime mudeli sobivust andmetega determinatsioonikordaja (\(R^2\)) abil, siis GLM-ide puhul vastavat näitajat ei ole. Küll on aga nn pseudo-\(R^2\) statistikud, mida võib analoogsel viisil kasutada (need ei näita küll päris sama asja). Üheks selliseks on näiteks Mcfadden'i \(R^2\):

\begin{Shaded}
\begin{Highlighting}[]
\FunctionTok{library}\NormalTok{(pscl)}
\NormalTok{titanic}\SpecialCharTok{$}\NormalTok{surv }\OtherTok{\textless{}{-}}\NormalTok{ titanic}\SpecialCharTok{$}\NormalTok{Survived }\SpecialCharTok{==} \StringTok{"Yes"}
\NormalTok{mudel\_r2 }\OtherTok{\textless{}{-}} \FunctionTok{glm}\NormalTok{(surv}\SpecialCharTok{\textasciitilde{}}\NormalTok{Sex}\SpecialCharTok{*}\NormalTok{Age, }\AttributeTok{data =}\NormalTok{ titanic, }\AttributeTok{family =} \FunctionTok{binomial}\NormalTok{())}
\FunctionTok{pR2}\NormalTok{(mudel\_r2)}
\end{Highlighting}
\end{Shaded}

\begin{verbatim}
## fitting null model for pseudo-r2
\end{verbatim}

\begin{verbatim}
##           llh       llhNull            G2      McFadden          r2ML 
## -1156.3879160 -1384.7283644   456.6808969     0.1648991     0.1873769 
##          r2CU 
##     0.2617525
\end{verbatim}

\hypertarget{predict}{%
\section{Predict}\label{predict}}

Sageli tahame oma mudeli alusel prognoosida mingitele kindlatele sõltumatute tunnuste väärtustele sõltuva tunnuse hinnanguid. Saame loomulikult need sõltumatute tunnuste väärtused regressioonivõrrandisse sisse panna ja hinnangu käsitsi välja arvutada. Aga on ka mugavam variant. Nimelt \texttt{predict()} funktsioon\footnote{\texttt{predict()} funktsiooni saab kasutada ka tavalise regressiooni puhul}.

\texttt{predict()} vajab sisendiks mudelit ning referentsandmestikku vajalike sõltumatute tunnuste kategooriate kombinatsioonidega. Referentsandmestiku saame valmis teha käsitsi või kasutada näiteks \texttt{expand.grid()} funktsiooni.

Tahame teada \texttt{titanic} andmestiku põhjal täiskasvanud meeste tõenäosust ellu jääda:

\begin{Shaded}
\begin{Highlighting}[]
\CommentTok{\# Teeme referentsandmestiku}
\NormalTok{ref\_data }\OtherTok{\textless{}{-}} \FunctionTok{data.frame}\NormalTok{(}\AttributeTok{Sex =} \StringTok{"Male"}\NormalTok{, }\AttributeTok{Age =} \StringTok{"Adult"}\NormalTok{)}

\CommentTok{\# Kasutame predict() funktsiooni ja lisame referentsandmestikule}
\CommentTok{\# pred tunnuse, millesse kirjutame prognoosi}
\CommentTok{\# Kuna tegemist on logit mudeliga, siis defaultis}
\CommentTok{\# prognoosib predict() logiteid Kui tahame teada}
\CommentTok{\# tõenäosusi, siis peame määrama type = \textquotesingle{}response\textquotesingle{}}

\NormalTok{ref\_data}\SpecialCharTok{$}\NormalTok{pred }\OtherTok{\textless{}{-}} \FunctionTok{predict}\NormalTok{(mudel8, }\AttributeTok{newdata =}\NormalTok{ ref\_data, }\AttributeTok{type =} \StringTok{"response"}\NormalTok{)}
\NormalTok{ref\_data}
\end{Highlighting}
\end{Shaded}

\begin{verbatim}
##    Sex   Age      pred
## 1 Male Adult 0.2027594
\end{verbatim}

Kui tahame prognoosi rohkematele kategooriate kombinatsioonidele, saame kasutada \texttt{expand.grid()} funktsiooni:

\begin{Shaded}
\begin{Highlighting}[]
\CommentTok{\# Teeme kõigepealt uue andmestiku, kus on sees kõik }
\CommentTok{\# tunnuse ja väärtused, mille kohta predictioni tahame}
\NormalTok{ndata }\OtherTok{\textless{}{-}} \FunctionTok{expand.grid}\NormalTok{(}\AttributeTok{Sex =} \FunctionTok{c}\NormalTok{(}\StringTok{"Male"}\NormalTok{, }\StringTok{"Female"}\NormalTok{), }\AttributeTok{Age =} \FunctionTok{c}\NormalTok{(}\StringTok{"Adult"}\NormalTok{, }\StringTok{"Child"}\NormalTok{))}

\CommentTok{\# Lisame andmestikule predictioni}
\NormalTok{ndata}\SpecialCharTok{$}\NormalTok{pred }\OtherTok{\textless{}{-}} \FunctionTok{predict}\NormalTok{(mudel8, }\AttributeTok{newdata =}\NormalTok{ ndata, }\AttributeTok{type =} \StringTok{"response"}\NormalTok{)}
\NormalTok{ndata}
\end{Highlighting}
\end{Shaded}

\begin{verbatim}
##      Sex   Age      pred
## 1   Male Adult 0.2027594
## 2 Female Adult 0.7435294
## 3   Male Child 0.4531250
## 4 Female Child 0.6222222
\end{verbatim}

Nüüd saame oma tulemused näiteks joonisele panna:

\begin{Shaded}
\begin{Highlighting}[]
\FunctionTok{ggplot}\NormalTok{(ndata, }\FunctionTok{aes}\NormalTok{(}\AttributeTok{x =}\NormalTok{ Sex, }\AttributeTok{y =}\NormalTok{ pred, }\AttributeTok{color =}\NormalTok{ Age))}\SpecialCharTok{+}
  \FunctionTok{geom\_point}\NormalTok{(}\AttributeTok{position =} \FunctionTok{position\_dodge}\NormalTok{(}\AttributeTok{width =} \FloatTok{0.5}\NormalTok{), }\AttributeTok{size =} \DecValTok{3}\NormalTok{)}\SpecialCharTok{+}
  \FunctionTok{labs}\NormalTok{(}\AttributeTok{y =} \StringTok{"Survival probability"}\NormalTok{)}\SpecialCharTok{+}
  \FunctionTok{scale\_y\_continuous}\NormalTok{(}\AttributeTok{labels =}\NormalTok{ scales}\SpecialCharTok{::}\NormalTok{percent)}\SpecialCharTok{+}
  \FunctionTok{scale\_color\_manual}\NormalTok{(}\AttributeTok{values =} \FunctionTok{c}\NormalTok{(}\StringTok{"\#972D15"}\NormalTok{, }\StringTok{"\#02401B"}\NormalTok{))}\SpecialCharTok{+}
  \FunctionTok{theme\_minimal}\NormalTok{()}
\end{Highlighting}
\end{Shaded}

\includegraphics{02-logit_files/figure-latex/unnamed-chunk-15-1.pdf}

\hypertarget{broom}{%
\subsection{Broom}\label{broom}}

Prognoositud väärtused kõikidele meie andmetes olevatele vaatlustele saame mõnevõrra lihtsamalt kätte paketi \emph{broom} abil. \emph{broom}i funktsioon \texttt{augment()} loob mudeli objektist andmestiku, milles on lisaks algsetele tunnusetele ka kõikidele vaatlustele prognoositud väärtudsed (\emph{.fitted}), prognoositud väärtuste standardvead (\emph{.se.fit}), jäägid (\emph{.resid}) jne.

\begin{Shaded}
\begin{Highlighting}[]
\FunctionTok{library}\NormalTok{(broom)}
\CommentTok{\# Kasutame broomi funktsiooni augment }
\NormalTok{mudel\_fit }\OtherTok{\textless{}{-}} \FunctionTok{augment}\NormalTok{(mudel8, }\AttributeTok{type.predict =} \StringTok{"response"}\NormalTok{)}
\FunctionTok{head}\NormalTok{(mudel\_fit)}
\end{Highlighting}
\end{Shaded}

\begin{verbatim}
## # A tibble: 6 x 9
##   `I(Survived == "~` Sex   Age   .fitted .resid .std.resid   .hat .sigma .cooksd
##   <I<lgl>>           <fct> <fct>   <dbl>  <dbl>      <dbl>  <dbl>  <dbl>   <dbl>
## 1 FALSE              Male  Child   0.453  -1.10      -1.11 0.0156   1.03 0.00334
## 2 FALSE              Male  Child   0.453  -1.10      -1.11 0.0156   1.03 0.00334
## 3 FALSE              Male  Child   0.453  -1.10      -1.11 0.0156   1.03 0.00334
## 4 FALSE              Male  Child   0.453  -1.10      -1.11 0.0156   1.03 0.00334
## 5 FALSE              Male  Child   0.453  -1.10      -1.11 0.0156   1.03 0.00334
## 6 FALSE              Male  Child   0.453  -1.10      -1.11 0.0156   1.03 0.00334
\end{verbatim}

\hypertarget{marginaalsed-efektid}{%
\section{Marginaalsed efektid}\label{marginaalsed-efektid}}

Marginaalsed efeketid (\emph{marginal effects}) kirjeldavad sõltuva tunnuse muutust kui mingi sõltumatu tunnus muutub ühe ühiku võrra. Seega võimaldavad need logistilise regressiooni puhul kasutada lineaarse regressiooniga analoogset tõlgendamisloogikat. Marginaalsete efektide arvutamiseks on erinevaid viise. Üheks levinuimaks meetodiks on nn \emph{Keskmised marginaalsed efektid} (\emph{Average Marginal Effects} ehk AME).

Oletame, et tahame Titanicu andmestiku alusel hinnata kui palju muutub inimese ellujäämise tõenäosus sõltuvalt tema soost. Logistilise regressioonimudeli abil saame teada vastava šansside suhte. Meid aga huvitaks tõenäosus. Me saame ka tõenäosuse välja arvutada (näiteks \texttt{predict()} funktsiooniga), kuid selleks peame defineerima mingi konkreetse grupi, kellele me regressioonivõrrandi abil tõenäosust prognoosime (näiteks saame võrrelda esimese klassi kajutis elvate täiskasvanud meeste ellujäämise tõenäosust esimese klassi kajutis elavate täsikasvanud naiste ellujäämise tõenäosusega). Meid aga huvitaks lihtsalt keskmine tõenäosuse erinevus meeste ja naiste vahel. Kuidas seda saavutada?

Marginaalsete efektide (täpsemalt selle AME variandi) leidmiseks prognoositakse kõikidele andmestiku vaatlustele mudelipõhine hinnang kahel juhul - esimesel juhul nii, et kõikide vaatluste puhul määratakse nende sooks mees ja teisel juhul nii, et kõikide vaatluste puhul määratakse nende sooks naine. Kõik muud tunnused on mõlemal puhul nii nagu nad algselt olid. Keskmine marginaalne efekt ongi keskmine kahe prognoositud hinnangu vahe.

\begin{Shaded}
\begin{Highlighting}[]
\FunctionTok{library}\NormalTok{(margins)}
\FunctionTok{summary}\NormalTok{(}\FunctionTok{margins}\NormalTok{(mudel8))}
\end{Highlighting}
\end{Shaded}

\begin{verbatim}
##     factor     AME     SE       z      p   lower   upper
##   AgeAdult -0.1710 0.0521 -3.2827 0.0010 -0.2731 -0.0689
##  SexFemale  0.5224 0.0227 23.0123 0.0000  0.4779  0.5669
\end{verbatim}

Saame järeldada, et täiskasvanute tõenäosus ellu jääda oli \(17\%\) väiksem kui lastel ning naiste tõenäosus ellu jääda oli \(52\%\) kõrgem kui meestel.

\hypertarget{prognoosi-tuxe4psus}{%
\section{Prognoosi täpsus}\label{prognoosi-tuxe4psus}}

\emph{Confusion matrix}'i (segaduse maatriks?) abiga saame hinnata oma prognoosi täpsust. Võrdleme tegelikke ja hinnatuid väärtusi. Kasutame jälle \texttt{predict()} funktsiooni ning prognoosime seekord kõikidele titanic andmestiku vaatlustele mudelipõhised hinnangud. Seejärel võrdleme neid hinnanguid vaatluste tegelike väärtustega:

\begin{Shaded}
\begin{Highlighting}[]
\CommentTok{\# Anname table() funktsioonile ette kaks loogilist vektorit.}
\CommentTok{\# Kui me predict funktsioonile newdata argumeti ei anna,}
\CommentTok{\# siis võtab ta automaatselt mudeli objektist kogu andmestiku}
\CommentTok{\# ja prognoosib hinnangu igale vaatlusele. Kuna prognoos on }
\CommentTok{\# tõenäosusskaalal, siis teeme selle loogiliseks vektoriks nii,}
\CommentTok{\# et kõik üle 0.5 tõenäosused oleksd T ja väiksemad F}
\NormalTok{vaadeldud }\OtherTok{\textless{}{-}}\NormalTok{ titanic}\SpecialCharTok{$}\NormalTok{Survived }\SpecialCharTok{==} \StringTok{"Yes"}
\NormalTok{prognoos }\OtherTok{\textless{}{-}} \FunctionTok{predict}\NormalTok{(mudel8, }\AttributeTok{type =} \StringTok{"response"}\NormalTok{)}\SpecialCharTok{\textgreater{}} \FloatTok{0.5}
\FunctionTok{table}\NormalTok{(vaadeldud, prognoos)}
\end{Highlighting}
\end{Shaded}

\begin{verbatim}
##          prognoos
## vaadeldud FALSE TRUE
##     FALSE  1364  126
##     TRUE    367  344
\end{verbatim}

Saadud maatriksist näeme, et prognoosisime oma mudeliga õigesti \(1364 + 344 = 1708\) juhul ning valesti \(367+126 = 493\) juhul, ehk siis meie mudeli \textbf{täpsus} (\emph{accuracy}) on \(\frac{1364 + 344}{1364 + 344 + 367+126} = 0.776 = 78\%\).

Maatriksist saame välja lugeda ka prognoosi \textbf{tundlikkuse} (\emph{sensitivity}) ja \textbf{spetsiifilisuse} (\emph{specificity}).

\textbf{Tundlikkus} väljendab õigesti prognoositud positiivsete väärtuste osakaalu kõikidest positiivsetest väärtustest
\[\text{tundlikkus} = \frac{\text{õige positiivne}}{\text{õige positiivne} + \text{vale negatiivne}} = \frac{344}{(344+367)} = 0.48\]
\textbf{Spetsiifilisus} omakorda väljendab õigesti prognoositud negatiivsete väärtuste osakaalu kõikidest negatiivsetest väärtustest

\[\text{spetsiifilisus} = \frac{\text{õige negatiivne}}{\text{õige negatiivne} + \text{vale positiivne}} = \frac{1364}{(1364+126)} = 0.92\]

Saame need arvutused teha ka \emph{caret} paketi ja \texttt{confusionMatrix()} funktsiooniga.

\begin{Shaded}
\begin{Highlighting}[]
\FunctionTok{library}\NormalTok{(caret)}
\CommentTok{\# confusionMatrix vajab sisendiuna faktoreid, }
\CommentTok{\# positive = TRUE arguimendiga ütleme, et ellujäämine oli positiivne sündmus}
\FunctionTok{confusionMatrix}\NormalTok{(}\AttributeTok{data =} \FunctionTok{as.factor}\NormalTok{(prognoos), }
                \AttributeTok{reference =} \FunctionTok{as.factor}\NormalTok{(vaadeldud), }\AttributeTok{positive =} \StringTok{\textquotesingle{}TRUE\textquotesingle{}}\NormalTok{)}
\end{Highlighting}
\end{Shaded}

\begin{verbatim}
## Confusion Matrix and Statistics
## 
##           Reference
## Prediction FALSE TRUE
##      FALSE  1364  367
##      TRUE    126  344
##                                          
##                Accuracy : 0.776          
##                  95% CI : (0.758, 0.7933)
##     No Information Rate : 0.677          
##     P-Value [Acc > NIR] : < 2.2e-16      
##                                          
##                   Kappa : 0.4381         
##                                          
##  Mcnemar's Test P-Value : < 2.2e-16      
##                                          
##             Sensitivity : 0.4838         
##             Specificity : 0.9154         
##          Pos Pred Value : 0.7319         
##          Neg Pred Value : 0.7880         
##              Prevalence : 0.3230         
##          Detection Rate : 0.1563         
##    Detection Prevalence : 0.2135         
##       Balanced Accuracy : 0.6996         
##                                          
##        'Positive' Class : TRUE           
## 
\end{verbatim}

Nii mudeli täpsus, tundlikkus, kui ka spetsiifilisus lähtusid eeldusest, et me klassifitseerisime vaatlused positiivseteks või negatiivseteks lähtuvalt sellest kas nende prognoositud tõenäosus oli suurem või väiksem kui \(0.5\) (nn \emph{treshold} või \emph{cutoff value}). Mida suurem on see \emph{cutoff}, seda rohkem õigeid positiivseid väärtusi saame prognoosida. Kuid samas, seda vähem saame prognoosida õigeid negatiivseid väärtusi. Ehk siis tundlikkuse ja spetsiifilisuse vahel on pöördvõrdeline seos. Mida suurem on üks, seda väiksem peab teine olema ja vastupidi. Seda seost saame vaadelda ROCi (\emph{receiver operating characteristics}) graafiku abil.

\begin{Shaded}
\begin{Highlighting}[]
\FunctionTok{library}\NormalTok{(ROCit)}
\FunctionTok{library}\NormalTok{(broom)}
\CommentTok{\# Kasutame broomi funktsiooni augment }
\NormalTok{mudel\_fit }\OtherTok{\textless{}{-}} \FunctionTok{augment}\NormalTok{(mudel8, }\AttributeTok{type.predict =} \StringTok{"response"}\NormalTok{)}
\NormalTok{roc\_obj }\OtherTok{\textless{}{-}} \FunctionTok{rocit}\NormalTok{(}\AttributeTok{score =}\NormalTok{ mudel\_fit}\SpecialCharTok{$}\NormalTok{.fitted,}\AttributeTok{class=}\NormalTok{mudel\_fit}\SpecialCharTok{$}\StringTok{\textasciigrave{}}\AttributeTok{I(Survived == "Yes")}\StringTok{\textasciigrave{}}\NormalTok{)}
\FunctionTok{plot}\NormalTok{(roc\_obj)}
\end{Highlighting}
\end{Shaded}

\includegraphics{02-logit_files/figure-latex/unnamed-chunk-20-1.pdf}

Mida suurem on pind graafiku kurvi all, seda parema mudeliga meil tegemist on (seda täpsemini võimaldab mudel prognoosida). Seda kurvi alust pindala suurust kasutataksegi prognoosi täpsuse hindamiseks. Vastavat statistikut kutsustaksegi kurvialuseks pindalaks (AUC ehk \emph{area under the curve}). Mida lähemal AUC \(1\)'le on, seda parema prognoosivõimega mudeliga meil tegemist on.

\begin{Shaded}
\begin{Highlighting}[]
\FunctionTok{summary}\NormalTok{(roc\_obj)}
\end{Highlighting}
\end{Shaded}

\begin{verbatim}
##                             
##  Method used: empirical     
##  Number of positive(s): 711 
##  Number of negative(s): 1490
##  Area under curve: 0.7133
\end{verbatim}

\hypertarget{poissoni-regressioon}{%
\chapter{Poissoni regressioon}\label{poissoni-regressioon}}

Poissoni regressioon kuulub üldistatud lineaarsete mudelite (GLM) raamistikku ja sellega saame hinnata sõltumatute tunnuste mõju mingile loendavale (\emph{count}) sõltuvale tunnusele (mitu inimest on kursusel, mitu inimest on poejärjekorras, mitu last on peres jne)\footnote{Ka kategoriaalsetest tunnustest moodustatud risttabelite sagedused on loendilised väärtused. Taoliste risttabelite sageduste baasil moodustatud mudeleid kutsutakse log-lineaarseteks mudeliteks. Lisaks loenditele kasutatakse Poissoni regressiooni ka määrade (\emph{rates}) mudeldamiseks, kuna määrasid võib käsitleda kui standardiseeritud loendeid. Siinkohal me taolisi mudeleid lähemalt ei käsitle, aga olgu see lihtsalt ära mainitud.}.

Sarnaselt logistilise regressiooniga (ja isegi õigustatumalt) tekib küsimus, et miks me ei saa taoliste tunnuste korral kasutada tavalist lineaarset regressiooni? Loend on ju suhteliselt sarnane tavalisele arvtunnusele. Välja arvatud asjaolu, et nii nagu tõenäosus logistilise regressiooni puhul, ei saa ka loend olla negatiivne. Kui me modelleeriksime loendi tunnust tavalise lineaarse regressiooniga, siis võib vabalt juhtuda, et mingite sõltumatute tunnuste väärtuste korral oleks prognoositav sõltuv tunnus väiksem kui 0. See aga ei ole loendilise tunnuse puhul realistlik. Lisaks, nii nagu ka logistilise regressiooni ja binaarsete sõltuvate tunnuste puhul, tekivad probleemid jääkide struktuuriga (tavaline regressioon eeldab normaaljaotust) ja jääkide dispersiooniga (tavaline regressioon eeldab konstantset hajuvust)\footnote{Loendiline tunnus \(Y\) järgib Poissoni jaotust, mille puhul \(E(Y)=Var(Y)=\lambda\), kus \(\lambda\) on nn keskmine loend (keskmine kursuse suurus, järjekorra pikkus, laste arv). Seega Poissoni jaotusega tunnuse puhul peaks keskmine ja dispersioon võrdsed olema ning suurem keskmine tähendab ka suuremat dispersioon.}.

\begin{figure}
\centering
\includegraphics{03-poisson_files/figure-latex/pois-dist-1.pdf}
\caption{\label{fig:pois-dist}Poissoni jaotus erinevate keskmiste loendusväärtuste (Lambda) korral}
\end{figure}

Logistilise regressiooni puhul saime tõenäosuse ülemisest piirist lahti seeläbi, et teisendasime tõenäosused šanssideks ning alumise piiri seeläbi, et võtsime šanssidest logaritmid. Loendilise tunnuse puhul meil tunnuse ülemise piiriga probleemi ei ole. Loend võib potentsiaalselt olla lõputu. Alumise piiri puhul saame aga kasutada sedasama logit mudelitest tuttavat logaritmimise nippi. Kui \(Y\) on meie hinnatav loendiline sõltuv tunnus ja \(x\) seda selgitav sõltumatu tunnus, siis Poissoni regressioonimudel on väljendatav järgmiselt\footnote{\(log(Y_i)\) ei ole siin sama mis lineaarse regressiooni puhul log-transformeeritud tunnus (kui me kasutame analüüsis eelnevalt logaritmitud sõltuvat tunnust). Teine ja sellest aspektist võib-olla selgem notatsioon oleks \(log(E(Y))=\beta_0+\beta_1 x_i\), millest nähtub, et me peame logaritmima \(Y\) oodatavat väärtust (\emph{expected value}) ehk \(Y\) keskmist. Keskmine logaritmitud \(Y\) ja logaritm keskmisest \(Y\)-ist ei ole aga samad asjad (võite järgi proovida: \texttt{mean(log(c(1,2,3,4)))} vs \texttt{log(mean(c(1,2,3,4)))}).}:

\[log(Y_i)=\beta_0+\beta_1 x_i\]
kui me võtame mõlemast võrrandi pooles eksponendi, saame sama asja väljendada ka nii:

\[Y_i = e^{\beta_0+\beta_1 x_i}\]

Eeldame siin, et \(Y_i\) järgib Poissoni jaotust. GLM'i sõnavara kasutades ütleme, et mudeli juhuslik komponent (ehk siis sõltuv tunnus) on Poissoni jaotusega ning linkfunktsioonina (funktsiooon, mille abil sõltuva ja sõltumatute tunnuste vaheline mittelineaarne seos muudetakse lineaarseks seoseks) kasutame log-funktsiooni.

Poissoni jaotuse kuju sõltub tunnuse keskmisest (võrdluseks, normaaljaotuse kuju sõltub tunnuse keskmisest ja standardhälbest). Mida suurem on keskmine, seda enam sarnaneb Poissoni jaotus normaaljaotusele (vt joonis 1). Seega suurte loendite puhul saaksime põhimõtteliselt ka tavalist lineaarset regressiooni kasutada (kuigi negatiivsete väärtuste probleem jääb ka sel juhul). Üldiselt on mõistlik Poissoni regressiooni kasutada siis, kui loendite maksimaalsed väärtused ei ole väga suured ja tunnus on eripäraselt Poissoni jaotuse kujuga. Kui meil on tegemist suuremate väärtustega loendiga (näiteks ülikoolide tudengite arvud), siis saame tunnuse näiteks mingi arvuga läbi jagada.

Loendilise tunnuse puhul tuleb tihti ette olukordi, kus tunnuses on palju nulle. Näiteks tunnus, mis kirjeldab bakalaureusetudengite laste arvu. Põhimõtteliselt on muidugi tegemist loendilise tunnusega, aga kuna suuremal osal tudengitest veel ei ole lapsi, siis enamik vaatlusi on paratamatult nullid. Sellise tunnuse jaotus ei vasta väga hästi Poissoni jaotusele ja selle kasutamine Poissoni regressioonimudeliga ei anna tõenäoliselt väga head tulemust\footnote{Paljude nullidega tunnuse puhul oleks mõistlikum kasutada näiteks \emph{zero-inflated Poisson}'i regressiooni (\texttt{zeroinfl()} funktsioon \emph{pscl} paketis) või \emph{negative binomial} regressiooni (\texttt{glm.nb()} funktsioon \emph{MASS} paketis)}.

\hypertarget{mudeli-tuxf5lgendus-1}{%
\section{Mudeli tõlgendus}\label{mudeli-tuxf5lgendus-1}}

Kuidas me neid regressioonimudeli \(\beta_0\) ja \(\beta_1\) koefitsiente tõlgendama peaksime? Lineaarse regressiooni puhul oli asi lihtne: vabaliige (\(\beta_0\)) oli tõlgendatav \(Y\) väärtusena kui \(x\) on \(0\) ja regressioonikoefitsient (\(\beta_1\)) näitas \(Y\) muutust kui \(x\) muutub ühe ühiku võrra. Logistilise regressiooni puhul pidime aga esmalt koefitsientidest eksponendi võtma ja saime neid seejärel tõlgendada šansside ja šansside suhetena. Mis siis sekord?

Kuna meil on jälle tegemist logaritmidega, siis koefitsientide otsene (lineaarse regressiooni moodi) tõlgendamine on keeruline. Mõistlikum on koefitsientidest jällegi eksponent võtta, misjärel saame vabaliiget tõlgendada tavapärasel moel (\(Y\) väärtus kui \(x\) on \(0\)) ja regressioonikoefitsienti kui \(Y\)-i multiplikatiivset muutust kui \(x\) muutub ühe ühiku võrra. Ehk kui \(\beta_1\) väärtus on näiteks 0.25, siis tema eksponent on \(e^{0.25} = exp(0.25) = 1.28\) ja saame järeldada, et kui \(x\) kasvab ühe ühiku võrra, siis \(Y\) kasvab \(1.28\) korda. Teisisõnu, \(Y\) kasvab \(28\%\). Või kui \(\beta_1\) väärtus on \(-0.5\), siis tema eksponent on \(e^{-0.5} = exp(-0.5) = 0.6\) ja saame järeldada, et kui \(x\) kasvab ühe ühiku võrra, siis \(Y\) kasvab \(0.6\) korda (ehk siis tegelikult kahaneb). Kui \(\beta = 0\), siis \(e^{0} = 1\) ehk multiplikatiivne efekt on \(1\) (\(Y\times1\)) ja \(Y\) \(x\)-i kasvades või kahanedes ei muutu. Kui \(\beta < 0\), siis \(Y\) \(x\)-i kasvades väheneb, kui \(\beta > 0\), siis \(Y\) \(x\)-i kasvades kasvab.

\hypertarget{poissoni-regressiooni-eeldused}{%
\section{Poissoni regressiooni eeldused}\label{poissoni-regressiooni-eeldused}}

\begin{itemize}
\tightlist
\item
  Sõltuv tunnus \(Y\) peaks enam-vähem vastama Poissoni jaotusele (st olema loendiline tunnus).
\item
  Vaatlused peavad olema üksteisest sõltumatud (st kogu vaatluste vaheline seos peaks olema kirjeldatud mudeli sõltumatute tunnuste poolt).
\item
  Dispersioon (\emph{variance}) peaks olema võrdne keskmisega. Juhul kui see eeldus ei ole täidetud, ja tihti juhtub, et ei ole, on meil tegemist nn üledispersiooniga (\emph{overdispersion}). Sellisel juhul tuleks Poissoni mudeli asemel kasutada nn \emph{quasipoisson}'i mudelit.
\item
  Sõltuva tunnuse ja sõltumatute prediktorite seos peaks läbi linkfunktsiooni olema lineaarne.
\end{itemize}

\hypertarget{mudeli-hindamine-r-is}{%
\section{Mudeli hindamine R-is}\label{mudeli-hindamine-r-is}}

\hypertarget{andmete-kirjeldus-ja-ettevalmistus}{%
\subsection{Andmete kirjeldus ja ettevalmistus}\label{andmete-kirjeldus-ja-ettevalmistus}}

Kasutame näitena \emph{PhDPublications} andmestikku\footnote{Long, J.S. (1997). The Origin of Sex Differences in Science. Social Forces, 68, 1297--1315.} paketist \emph{AER}. Andmestikus on loendatud biokeemia doktorantide publikatsioonide arv (tunnus \emph{atricles}) kolme aasta jooksul. Sõltumatute tunnustena on kasutada:

\begin{itemize}
\tightlist
\item
  \emph{gender},
\item
  \emph{married} - kas doktorant oli abielus,
\item
  \emph{kids} - mitu last doktorandil oli,
\item
  \emph{prestige} - kooli maine skoor) ja
\item
  \emph{mentor} - juhendaja publikatsioonide arv.
\end{itemize}

\begin{Shaded}
\begin{Highlighting}[]
\CommentTok{\# Kui pakett ei ole installitud, }
\CommentTok{\# tuleb seda teha käsuga: install.packages("AER")}

\CommentTok{\# Loeme paketi sisse}
\FunctionTok{library}\NormalTok{(AER)}
\CommentTok{\# Võtame andmestiku}
\FunctionTok{data}\NormalTok{(PhDPublications)}
\CommentTok{\# Paneme andmestikule lihtsama nime}
\NormalTok{phd }\OtherTok{\textless{}{-}}\NormalTok{ PhDPublications}
\end{Highlighting}
\end{Shaded}

Vaatame artiklite tunnust lähemalt:

\begin{Shaded}
\begin{Highlighting}[]
\FunctionTok{ggplot}\NormalTok{(phd)}\SpecialCharTok{+}
  \FunctionTok{geom\_histogram}\NormalTok{(}\FunctionTok{aes}\NormalTok{(}\AttributeTok{x =}\NormalTok{ articles), }
                 \AttributeTok{binwidth =} \DecValTok{1}\NormalTok{, }
                 \AttributeTok{fill =} \StringTok{\textquotesingle{}grey\textquotesingle{}}\NormalTok{, }
                 \AttributeTok{color =} \StringTok{\textquotesingle{}black\textquotesingle{}}\NormalTok{)}\SpecialCharTok{+}
  \FunctionTok{theme\_minimal}\NormalTok{()}
\end{Highlighting}
\end{Shaded}

\includegraphics{03-poisson_files/figure-latex/unnamed-chunk-2-1.pdf}

Ilmselgelt on tegemist loendilise tunnusega. Samas päris Poissoni jaotusega tegemist vist siiski ei ole, kuna tundub, et nulle on selleks natukene liiga palju. ggploti abil saame joonisele panna ka tunnust iseloomustava teoreetilise Poissoni jaotuse (lähtuvalt tunnuse keskmisest ehk \(\lambda\) parameetrist). Vaatame kuidas see võrreldes reaalse jaotusega välja näeb:

\begin{Shaded}
\begin{Highlighting}[]
\CommentTok{\# Poissoni jaotuse parameetrina on meil vaja keskmist}
\NormalTok{keskmine }\OtherTok{\textless{}{-}} \FunctionTok{mean}\NormalTok{(phd}\SpecialCharTok{$}\NormalTok{articles)}
\NormalTok{phd }\SpecialCharTok{\%\textgreater{}\%} 
  \CommentTok{\# standardiseerime  artiklite arvu, }
  \CommentTok{\# et saaksime seda jaotusega võrrelda}
  \FunctionTok{group\_by}\NormalTok{(articles) }\SpecialCharTok{\%\textgreater{}\%} 
  \FunctionTok{summarise}\NormalTok{(}\AttributeTok{n =} \FunctionTok{n}\NormalTok{()) }\SpecialCharTok{\%\textgreater{}\%} 
  \FunctionTok{mutate}\NormalTok{(}\AttributeTok{n\_scaled =}\NormalTok{ n}\SpecialCharTok{/}\FunctionTok{sum}\NormalTok{(n)) }\SpecialCharTok{\%\textgreater{}\%} 
\FunctionTok{ggplot}\NormalTok{(}\FunctionTok{aes}\NormalTok{(}\AttributeTok{x =}\NormalTok{ articles, }\AttributeTok{y =}\NormalTok{ n\_scaled))}\SpecialCharTok{+}
  \CommentTok{\# kasutame stat = \textquotesingle{}identity\textquotesingle{}, }
  \CommentTok{\# st kasutame joonisel olemasolevaid väärtusi}
  \CommentTok{\# (mitte ei lase ggplot\textquotesingle{}il neid välja arvutada)}
  \FunctionTok{geom\_histogram}\NormalTok{(}\AttributeTok{stat =} \StringTok{\textquotesingle{}identity\textquotesingle{}}\NormalTok{, }
                 \AttributeTok{fill =} \StringTok{\textquotesingle{}grey\textquotesingle{}}\NormalTok{, }
                 \AttributeTok{color =} \StringTok{\textquotesingle{}black\textquotesingle{}}\NormalTok{)}\SpecialCharTok{+}
  \CommentTok{\# dpois funktsioon annab meile poissoni tihedusfunktsiooni}
  \FunctionTok{geom\_line}\NormalTok{(}\FunctionTok{aes}\NormalTok{(}\AttributeTok{x =}\NormalTok{ articles, }
                \AttributeTok{y =} \FunctionTok{dpois}\NormalTok{(articles, }\AttributeTok{lambda =}\NormalTok{ keskmine)), }
            \AttributeTok{color =} \StringTok{\textquotesingle{}red\textquotesingle{}}\NormalTok{)}\SpecialCharTok{+}
  \FunctionTok{theme\_minimal}\NormalTok{()}
\end{Highlighting}
\end{Shaded}

\includegraphics{03-poisson_files/figure-latex/pois-hist-dist-1.pdf}

Nagu näeme, siis tõesti, nulle on natuke liiga palju ja artiklite loendi jaotus ei vasta jaotuse alguses päris täpselt Poissoni jaotuse kujule. Aga jätame selle asjaolu hetkel tähelepanuta ja kasutame ikkagi Poissoni regressiooni.

Vaatame üle ka teised andmestiku tunnused:

\begin{Shaded}
\begin{Highlighting}[]
\NormalTok{phd }\SpecialCharTok{\%\textgreater{}\%} 
  \FunctionTok{group\_by}\NormalTok{(gender) }\SpecialCharTok{\%\textgreater{}\%} 
  \FunctionTok{summarize}\NormalTok{(}\FunctionTok{n}\NormalTok{())}
\end{Highlighting}
\end{Shaded}

\begin{verbatim}
## # A tibble: 2 x 2
##   gender `n()`
##   <fct>  <int>
## 1 male     494
## 2 female   421
\end{verbatim}

\begin{Shaded}
\begin{Highlighting}[]
\NormalTok{phd }\SpecialCharTok{\%\textgreater{}\%} 
  \FunctionTok{group\_by}\NormalTok{(married) }\SpecialCharTok{\%\textgreater{}\%} 
  \FunctionTok{summarize}\NormalTok{(}\FunctionTok{n}\NormalTok{())}
\end{Highlighting}
\end{Shaded}

\begin{verbatim}
## # A tibble: 2 x 2
##   married `n()`
##   <fct>   <int>
## 1 no        309
## 2 yes       606
\end{verbatim}

Soo ja abielu tunnus tunduvad korras olevat, kui välja arvata see, et kuidagi paljud doktorandid paistavad abielus olevat. Aga on nagu on.

\begin{Shaded}
\begin{Highlighting}[]
\NormalTok{phd }\SpecialCharTok{\%\textgreater{}\%} 
  \FunctionTok{group\_by}\NormalTok{(kids) }\SpecialCharTok{\%\textgreater{}\%} 
  \FunctionTok{summarize}\NormalTok{(}\FunctionTok{n}\NormalTok{())}
\end{Highlighting}
\end{Shaded}

\begin{verbatim}
## # A tibble: 4 x 2
##    kids `n()`
##   <int> <int>
## 1     0   599
## 2     1   195
## 3     2   105
## 4     3    16
\end{verbatim}

Laste tunnus on originaalis arvuline. Me kindlasti ei taha nelja väärtusega tunnust arvulisena käsitleda. Parem oleks see faktoriks teha ja seda mudelis kategoriaalsena käsitleda. Antud juhul, kuna lastetuid doktorante on niivõrd palju, oleks vast kõige mõistlik see tunnus üldse binaarseks teha, st kas on või ei ole lapsi (kuigi teoreetiliselt võiks ju eeldada, et laste arv võib mõjutada artiklite kirjutamiseks jäävat aega, siis siin, mulle tundub, on laste arvu variatiivsus selle kasutamiseks liiga väike).

\begin{Shaded}
\begin{Highlighting}[]
\CommentTok{\# kasutame laste arvu teisendamiseks ifelse() funktsiooni}
\CommentTok{\# vajadusel vaadake sellekohast abiinfot ?ifelse}
\NormalTok{phd }\OtherTok{\textless{}{-}}\NormalTok{ phd }\SpecialCharTok{\%\textgreater{}\%} 
  \FunctionTok{mutate}\NormalTok{(}\AttributeTok{kids2 =} \FunctionTok{ifelse}\NormalTok{(kids }\SpecialCharTok{==} \DecValTok{0}\NormalTok{, }\StringTok{\textquotesingle{}Ei\textquotesingle{}}\NormalTok{, }\StringTok{\textquotesingle{}Jah\textquotesingle{}}\NormalTok{))}

\CommentTok{\# kontrollime uut tunnust}
\NormalTok{phd }\SpecialCharTok{\%\textgreater{}\%} 
  \FunctionTok{group\_by}\NormalTok{(kids, kids2) }\SpecialCharTok{\%\textgreater{}\%} 
  \FunctionTok{summarise}\NormalTok{(}\FunctionTok{n}\NormalTok{())}
\end{Highlighting}
\end{Shaded}

\begin{verbatim}
## # A tibble: 4 x 3
## # Groups:   kids [4]
##    kids kids2 `n()`
##   <int> <chr> <int>
## 1     0 Ei      599
## 2     1 Jah     195
## 3     2 Jah     105
## 4     3 Jah      16
\end{verbatim}

Pidevast tunnusest ülevaate saamiseks on mugav kasutada histogrammi:

\begin{Shaded}
\begin{Highlighting}[]
\FunctionTok{hist}\NormalTok{(phd}\SpecialCharTok{$}\NormalTok{prestige)}
\end{Highlighting}
\end{Shaded}

\includegraphics{03-poisson_files/figure-latex/unnamed-chunk-7-1.pdf}

PhD programmi maine küsimus tundub suht OK.

\begin{Shaded}
\begin{Highlighting}[]
\FunctionTok{hist}\NormalTok{(phd}\SpecialCharTok{$}\NormalTok{mentor)}
\end{Highlighting}
\end{Shaded}

\includegraphics{03-poisson_files/figure-latex/unnamed-chunk-8-1.pdf}

Juhendaja artiklit arvu tunnus on iseenesest samuti loendav tunnus (kuigi Poissoni jaotust see samuti päris täpselt ei järgi).

\hypertarget{poissoni-mudeli-defineerimine}{%
\subsection{Poissoni mudeli defineerimine}\label{poissoni-mudeli-defineerimine}}

Vaatame esmalt kuidas sugu artiklite avaldamist mõjutab. Defineerime mudeli kasutades \texttt{glm()} funktsiooni ja selles \texttt{family\ =} argumendina \texttt{poisson()} funktsiooni (kõlbaks ka \texttt{family\ =\ "poisson"} või \texttt{family\ =\ poisson}). \texttt{poisson()} funktsiooni puhul on vaikimisi eeldatud linkfunktsioonina \texttt{log}-linki, kuid vajadusel saaksime ka mingit muud linkfunktsiooni kasutada või selle eksplitsiitselt välja tuua: \texttt{poisson(link\ =\ \textquotesingle{}log\textquotesingle{})}. Salvestame mudeli kõigepealt andmeobjektiks ning seejärel uurime seda \texttt{summary()} funktsiooniga.

\begin{Shaded}
\begin{Highlighting}[]
\NormalTok{m1 }\OtherTok{\textless{}{-}} \FunctionTok{glm}\NormalTok{(articles}\SpecialCharTok{\textasciitilde{}}\NormalTok{gender, }\AttributeTok{family =} \FunctionTok{poisson}\NormalTok{(), }\AttributeTok{data =}\NormalTok{ phd)}
\FunctionTok{summary}\NormalTok{(m1)}
\end{Highlighting}
\end{Shaded}

\begin{verbatim}
## 
## Call:
## glm(formula = articles ~ gender, family = poisson(), data = phd)
## 
## Deviance Residuals: 
##     Min       1Q   Median       3Q      Max  
## -1.9404  -1.7148  -0.4119   0.4139   7.3221  
## 
## Coefficients:
##              Estimate Std. Error z value Pr(>|z|)    
## (Intercept)   0.63265    0.03279  19.293  < 2e-16 ***
## genderfemale -0.24718    0.05187  -4.765 1.89e-06 ***
## ---
## Signif. codes:  0 '***' 0.001 '**' 0.01 '*' 0.05 '.' 0.1 ' ' 1
## 
## (Dispersion parameter for poisson family taken to be 1)
## 
##     Null deviance: 1817.4  on 914  degrees of freedom
## Residual deviance: 1794.4  on 913  degrees of freedom
## AIC: 3466.1
## 
## Number of Fisher Scoring iterations: 5
\end{verbatim}

Nagu näeme, siis Poissoni regressiooni väljund on praktiliselt identne logistilise regressiooni väljundiga (mis on ka loogiline, kuna mõlemad on loodud sama funktsiooniga).

Tõlgendame kõigepealt regressioonikoefitsiente. Nagu eelnevalt mainitud, siis tuleks neist eelnevalt eksponent võtta. Aga ka ilma selleta saame öelda, et naised on meestest vähem artikleid avaldanud. Mehed on antud juhul referentsgrupp ning \emph{genderfemale} koefitsient kirjeldab naiste erinevust meestest. Kuna koefitsient on negatiivne, siis saame järeldada, et keskmine artiklite arv on naiste hulgas väiksem kui meeste hulgas. Aga et teada saada kui palju väiksem, peame koefitsientidest eksponendi võtma:

\begin{Shaded}
\begin{Highlighting}[]
\CommentTok{\# koefitsiendid saame mudeli objektist kätte coef() funktsiooniga}
\FunctionTok{exp}\NormalTok{(}\FunctionTok{coef}\NormalTok{(m1))}
\end{Highlighting}
\end{Shaded}

\begin{verbatim}
##  (Intercept) genderfemale 
##    1.8825911    0.7810027
\end{verbatim}

Vabaliige on \(Y\) väärtus kui \(x\) on \(0\). Kuna mehed on referentsgrupp (ehk siis \(0\)), siis kirjeldab vabaliige siinkohal meeste keskmist artiklite arvu. Võime tulemuse \texttt{dplyr}'iga verifitseerida:

\begin{Shaded}
\begin{Highlighting}[]
\NormalTok{phd }\SpecialCharTok{\%\textgreater{}\%} 
  \FunctionTok{filter}\NormalTok{(gender }\SpecialCharTok{==} \StringTok{\textquotesingle{}male\textquotesingle{}}\NormalTok{) }\SpecialCharTok{\%\textgreater{}\%} 
  \FunctionTok{summarize}\NormalTok{(}\FunctionTok{mean}\NormalTok{(articles))}
\end{Highlighting}
\end{Shaded}

\begin{verbatim}
##   mean(articles)
## 1       1.882591
\end{verbatim}

Tundub tõesti nii olevat.

Eksponenti võetud \emph{genderfemale} koefitsient näitab naiste multiplikatiivset erinevust meestest. Ehk siis naiste keskmine artiklite arv peaks olema \(1.88 \times 0.78 = 1.47\). Teiste sõnadega naiste keskmine artiklite arv on \(1-0.78 = 22\%\) väiksem kui meestel. Kontrollime üle:

\begin{Shaded}
\begin{Highlighting}[]
\NormalTok{phd }\SpecialCharTok{\%\textgreater{}\%} 
  \FunctionTok{filter}\NormalTok{(gender }\SpecialCharTok{==} \StringTok{\textquotesingle{}female\textquotesingle{}}\NormalTok{) }\SpecialCharTok{\%\textgreater{}\%} 
  \FunctionTok{summarize}\NormalTok{(}\FunctionTok{mean}\NormalTok{(articles))}
\end{Highlighting}
\end{Shaded}

\begin{verbatim}
##   mean(articles)
## 1       1.470309
\end{verbatim}

Jälle sama tulemus.

Mida me veel tulemist näeme ja näha tahame? Eelkõige huvitab meid muidugi see, kas meie tulemused on statistiliselt usaldusväärsed. Regressioonikordajate statistilise olulisuse hindamiseks on meil analoogselt logistilise regressiooni väljundiga kasutada z-test, mis testib kas regressioonikordaja on oluliselt erinev nullist. Näeme nii teststatistikut (z-väärtust), kui ka z-testi iseloomustavat \emph{p}-väärtust. \emph{p} väärtused (tulp \texttt{Pr(\textgreater{}\textbar{}z\textbar{})}) on nii vabaliikme kui regressioonikoefitsiendi puhul oluliselt väiksemad kui \(0.05\), misläbi saame järeldada, et usaldusnivool \(95\%\) on meie mudeli koefitsiendid statistiliselt oluliselt erinevad nullist\footnote{\emph{p}-väärtus ütleb meile, et juhul kui nullhüpotees oleks tõsi (antud juhul on nullhüpoteesiks, et tunnuste vahel ei ole seost, st regressioonikoefitsient on \(0\)), siis saaksime sellise tulemuse nagu me saime (täpsemalt öeldes sellise z-väärtuse) tõenäosusega mis on võrdne \emph{p}-väärtusega. Kui \emph{p}-väärtus on näiteks \(0.05\), siis uuringut korrates saaksime sellise tulemuse \(5\%\) kordadest. Antud juhul on \emph{p}-väärtus oluliselt väiksem kui \(0.5\), st tõenäosus, et meie regressioonikordaja on tegelikult \(0\), on hästi väike.}, ehk siis sugu mõjutab oluliselt avaldatud artiklite arvu. Kui koefitsient ei oleks statistiliselt oluline (\(p > 0.05\)), siis peaksime järeldama, et meie koefitsient ei erine oluliselt nullist ning seost sõltumatu ja sõltuva tunnuse vahel ei ole.

Sama järelduse saaksime tegelikult teha ka z-väärtuste ja standardvigade (tulp \texttt{Std.\ Error}) põhjal. Mida väiksem on z-väärtus, seda väiksem on \emph{p}. Ja kui z-väärtus on (absoluutarvuna) suurem kui \(1.96\), siis on \emph{p} on väiksem kui \(0.05\). Ning z-väärtus (ja z-test) omakorda tuleneb standardveast: \(\frac{-0.24718}{0.05187} = -4.765\).

Aga vaatame oma mudelit edasi. Lisame ka teised sõltumatud tunnused:

\begin{Shaded}
\begin{Highlighting}[]
\NormalTok{m2 }\OtherTok{\textless{}{-}} \FunctionTok{glm}\NormalTok{(articles}\SpecialCharTok{\textasciitilde{}}\NormalTok{gender}\SpecialCharTok{+}\NormalTok{prestige}\SpecialCharTok{+}\NormalTok{married}\SpecialCharTok{+}\NormalTok{kids2}\SpecialCharTok{+}\NormalTok{mentor, }\AttributeTok{family =} \FunctionTok{poisson}\NormalTok{(), }\AttributeTok{data =}\NormalTok{ phd)}
\FunctionTok{summary}\NormalTok{(m2)}
\end{Highlighting}
\end{Shaded}

\begin{verbatim}
## 
## Call:
## glm(formula = articles ~ gender + prestige + married + kids2 + 
##     mentor, family = poisson(), data = phd)
## 
## Deviance Residuals: 
##     Min       1Q   Median       3Q      Max  
## -3.4477  -1.5669  -0.3587   0.5705   5.4715  
## 
## Coefficients:
##               Estimate Std. Error z value Pr(>|z|)    
## (Intercept)   0.306243   0.103093   2.971  0.00297 ** 
## genderfemale -0.217925   0.054717  -3.983 6.81e-05 ***
## prestige      0.010275   0.026460   0.388  0.69776    
## marriedyes    0.151697   0.063028   2.407  0.01609 *  
## kids2Jah     -0.249563   0.063342  -3.940 8.15e-05 ***
## mentor        0.025817   0.002019  12.788  < 2e-16 ***
## ---
## Signif. codes:  0 '***' 0.001 '**' 0.01 '*' 0.05 '.' 0.1 ' ' 1
## 
## (Dispersion parameter for poisson family taken to be 1)
## 
##     Null deviance: 1817.4  on 914  degrees of freedom
## Residual deviance: 1640.9  on 909  degrees of freedom
## AIC: 3320.7
## 
## Number of Fisher Scoring iterations: 5
\end{verbatim}

Näeme, et ülikooli maine ei mõjuta statistiliselt oluliselt artiklite arvu. Ka abielustaatus on suhteliselt piiripealse mõjuga. Jätame maine tunnuse mudelist välja (tahame alati leida võimalikult lihtsa mudeli, seega tunnused, mis mudelisse ei ei panusta, jätame välja).

\begin{Shaded}
\begin{Highlighting}[]
\NormalTok{m3 }\OtherTok{\textless{}{-}} \FunctionTok{glm}\NormalTok{(articles}\SpecialCharTok{\textasciitilde{}}\NormalTok{gender}\SpecialCharTok{+}\NormalTok{married}\SpecialCharTok{+}\NormalTok{kids2}\SpecialCharTok{+}\NormalTok{mentor, }\AttributeTok{family =} \FunctionTok{poisson}\NormalTok{(), }\AttributeTok{data =}\NormalTok{ phd)}
\end{Highlighting}
\end{Shaded}

Kontrollime igaks juhuks ka \texttt{anova()}-ga, kas maine tunnuse väljajätmine ikka oli õigustatud:

\begin{Shaded}
\begin{Highlighting}[]
\FunctionTok{anova}\NormalTok{(m3, m2, }\AttributeTok{test =} \StringTok{\textquotesingle{}Chisq\textquotesingle{}}\NormalTok{)}
\end{Highlighting}
\end{Shaded}

\begin{verbatim}
## Analysis of Deviance Table
## 
## Model 1: articles ~ gender + married + kids2 + mentor
## Model 2: articles ~ gender + prestige + married + kids2 + mentor
##   Resid. Df Resid. Dev Df Deviance Pr(>Chi)
## 1       910     1641.1                     
## 2       909     1640.9  1  0.15093   0.6976
\end{verbatim}

Hii-ruut test ütleb meile, et keerulisema ja lihtsama mudeli vahel ei ole statistiliselt olulist erinevust (\(p > 0.05\)), seega võime tunnuse vabalt välja jätta.

Näeme, et nii sugu, abielustaatus, laste arv, kui ka juhendaja publikatsioonide arv mõjutavad artiklite arvu oluliselt. Kontrollime igaks juhuks ka soo ja laste olemasolu koosmõju:

\begin{Shaded}
\begin{Highlighting}[]
\NormalTok{m4 }\OtherTok{\textless{}{-}} \FunctionTok{glm}\NormalTok{(articles}\SpecialCharTok{\textasciitilde{}}\NormalTok{gender}\SpecialCharTok{*}\NormalTok{kids2}\SpecialCharTok{+}\NormalTok{married}\SpecialCharTok{+}\NormalTok{mentor, }\AttributeTok{family =} \FunctionTok{poisson}\NormalTok{(), }\AttributeTok{data =}\NormalTok{ phd)}
\FunctionTok{summary}\NormalTok{(m4)}
\end{Highlighting}
\end{Shaded}

\begin{verbatim}
## 
## Call:
## glm(formula = articles ~ gender * kids2 + married + mentor, family = poisson(), 
##     data = phd)
## 
## Deviance Residuals: 
##     Min       1Q   Median       3Q      Max  
## -3.4995  -1.5593  -0.3582   0.5639   5.4570  
## 
## Coefficients:
##                       Estimate Std. Error z value Pr(>|z|)    
## (Intercept)            0.34220    0.06249   5.476 4.35e-08 ***
## genderfemale          -0.22492    0.06299  -3.571 0.000356 ***
## kids2Jah              -0.25695    0.07148  -3.595 0.000325 ***
## marriedyes             0.14875    0.06291   2.365 0.018045 *  
## mentor                 0.02601    0.00196  13.268  < 2e-16 ***
## genderfemale:kids2Jah  0.02522    0.12539   0.201 0.840628    
## ---
## Signif. codes:  0 '***' 0.001 '**' 0.01 '*' 0.05 '.' 0.1 ' ' 1
## 
## (Dispersion parameter for poisson family taken to be 1)
## 
##     Null deviance: 1817.4  on 914  degrees of freedom
## Residual deviance: 1641.0  on 909  degrees of freedom
## AIC: 3320.8
## 
## Number of Fisher Scoring iterations: 5
\end{verbatim}

Koosmõju koefitsient ei ole statistiliselt oluline, seega jääme mudeli m3 juurde:

\begin{Shaded}
\begin{Highlighting}[]
\FunctionTok{summary}\NormalTok{(m3)}
\end{Highlighting}
\end{Shaded}

\begin{verbatim}
## 
## Call:
## glm(formula = articles ~ gender + married + kids2 + mentor, family = poisson(), 
##     data = phd)
## 
## Deviance Residuals: 
##     Min       1Q   Median       3Q      Max  
## -3.5080  -1.5615  -0.3626   0.5614   5.4494  
## 
## Coefficients:
##              Estimate Std. Error z value Pr(>|z|)    
## (Intercept)   0.33873    0.06011   5.635 1.75e-08 ***
## genderfemale -0.21864    0.05470  -3.997 6.42e-05 ***
## marriedyes    0.14953    0.06279   2.381   0.0172 *  
## kids2Jah     -0.25029    0.06332  -3.953 7.73e-05 ***
## mentor        0.02600    0.00196  13.268  < 2e-16 ***
## ---
## Signif. codes:  0 '***' 0.001 '**' 0.01 '*' 0.05 '.' 0.1 ' ' 1
## 
## (Dispersion parameter for poisson family taken to be 1)
## 
##     Null deviance: 1817.4  on 914  degrees of freedom
## Residual deviance: 1641.1  on 910  degrees of freedom
## AIC: 3318.8
## 
## Number of Fisher Scoring iterations: 5
\end{verbatim}

Järgmiseks võtame mudeli koefitsientide eksponendid, et neid natukene inimlikumal kujul kuvada ja tõlgendada:

\begin{Shaded}
\begin{Highlighting}[]
\FunctionTok{exp}\NormalTok{(}\FunctionTok{coef}\NormalTok{(m3))}
\end{Highlighting}
\end{Shaded}

\begin{verbatim}
##  (Intercept) genderfemale   marriedyes     kids2Jah       mentor 
##    1.4031677    0.8036087    1.1612850    0.7785778    1.0263406
\end{verbatim}

Mida need koefitsiendid meile ütlevad?

\begin{itemize}
\tightlist
\item
  Vabaliige on antud näite puhul artiklite arv juhul kui kõikide sõltumatute tunnuste väärtused on nullid (ehk siis kõik kategoorilised tunnused on referentsväärtusega ja pidevtunnused lihtsalt nullid). Konkreetsel juhul on see ka sisukas tulemus, kuna mudeli ainsa pidevtunnuse nullväärtus on täiesti realistilik (juhendaja artiklite arv võib vabalt \(0\) olla). Aga üldiselt see pigem nii ei ole (mida me hakkame peale \(Y\) väärtusega, mis kehtib olukorras kus vanus või kehakaal on \(0\)). Seega, meie mudeli järgi on vallaliste, ilma lasteta meeste, kelle juhendajad pole viimase kolme aasta jooksul ühtegi publikatsiooni avaldanud, artiklite arv keskmiselt 1.4.
\item
  Naiste keskmine publikatsioonide arv (hoides teisi tunnuseid konstantsetena) on keskmiselt 20\% väiksem (\(1-0.8 = 0.2 = 20\%\)) kui meestel.
\item
  Abielus doktorantide keskmine publikatsioonide arv on \(1.16\) ehk \(16\%\) suurem kui vallalistel doktorantidel.
\item
  Laste olemasolu pärsib artiklite avaldamist keskmiselt \(22\%\) võrra.
\item
  Iga juhendaja lisanduv artikkel tõstab juhendatava publitseerimisvõimekust \(2\%\) võrra.
\end{itemize}

\hypertarget{mudeli-eelduste-kontroll}{%
\section{Mudeli eelduste kontroll}\label{mudeli-eelduste-kontroll}}

\hypertarget{uxfcledispersioon}{%
\subsection{Üledispersioon}\label{uxfcledispersioon}}

Mudeli üledispersioon (\emph{overdispersion}) on olukord, kus mudeli dispersioon on suurem kui mudeli aluseks olev jaotusfunktsioon eeldaks. Kui see on nii, siis on mudeli standardvead tõenäoliselt liiga väikesed (ja seega mudeli alusel tehtavad järldused valed). Üledispersiooni olemasolu saame kontrollida võrreldes jääkhälbimust (\emph{Residual deviance}) ja selle vabadusastemid (\emph{degrees of freedom} ehk \emph{df}). Kui need on enam-vähem võrdsed, ehk \(\frac{\text{Residual deviance}}{\text{df}} \approx 1\), siis üledispersiooni ei ole. Aga kui see suhe on oluliselt suurem kui \(1\), siis on tegemist probleemiga. Juhul kui taoline olukord esineb, peaksime Poissoni mudeli asemel kasutama \emph{quasipoisson}'i mudelit, kus üledispersiooni on eraldi dispersiooni parameetrina mudelis arvesse võetud. Ka meie näite puhul on tegemist kerge üledispersiooniga (mitte küll väga suurega, aga siiski), seega võiksime kasutada quasipoissonit:

\begin{Shaded}
\begin{Highlighting}[]
\NormalTok{m5 }\OtherTok{\textless{}{-}} \FunctionTok{glm}\NormalTok{(articles}\SpecialCharTok{\textasciitilde{}}\NormalTok{gender}\SpecialCharTok{+}\NormalTok{married}\SpecialCharTok{+}\NormalTok{kids2}\SpecialCharTok{+}\NormalTok{mentor, }\AttributeTok{family =}\NormalTok{ quasipoisson, }\AttributeTok{data =}\NormalTok{ phd)}
\FunctionTok{summary}\NormalTok{(m5)}
\end{Highlighting}
\end{Shaded}

\begin{verbatim}
## 
## Call:
## glm(formula = articles ~ gender + married + kids2 + mentor, family = quasipoisson, 
##     data = phd)
## 
## Deviance Residuals: 
##     Min       1Q   Median       3Q      Max  
## -3.5080  -1.5615  -0.3626   0.5614   5.4494  
## 
## Coefficients:
##               Estimate Std. Error t value Pr(>|t|)    
## (Intercept)   0.338732   0.081482   4.157 3.53e-05 ***
## genderfemale -0.218643   0.074151  -2.949  0.00327 ** 
## marriedyes    0.149527   0.085110   1.757  0.07928 .  
## kids2Jah     -0.250286   0.085834  -2.916  0.00363 ** 
## mentor        0.026000   0.002656   9.788  < 2e-16 ***
## ---
## Signif. codes:  0 '***' 0.001 '**' 0.01 '*' 0.05 '.' 0.1 ' ' 1
## 
## (Dispersion parameter for quasipoisson family taken to be 1.837376)
## 
##     Null deviance: 1817.4  on 914  degrees of freedom
## Residual deviance: 1641.1  on 910  degrees of freedom
## AIC: NA
## 
## Number of Fisher Scoring iterations: 5
\end{verbatim}

Näeme, et punkthinnangud ei muutunud, küll aga läksid standardvead suuremaks. See tähendab, et quaipoissoniga meie mudeli täpsusaste kahaneb (või õigemini esialgse, tavalise Poissoni mudeli puhul hindasime me mudeli täpsust üle). Samuti näeme, et tänu sellele ei ole abielustaatuse koefitsient enam \(95\%\) usaldusnivoo korral statistiliselt oluline ja peaksime selle tunnuse välja jätma.

\begin{Shaded}
\begin{Highlighting}[]
\NormalTok{m6 }\OtherTok{\textless{}{-}} \FunctionTok{glm}\NormalTok{(articles}\SpecialCharTok{\textasciitilde{}}\NormalTok{gender}\SpecialCharTok{+}\NormalTok{kids2}\SpecialCharTok{+}\NormalTok{mentor, }\AttributeTok{family =}\NormalTok{ quasipoisson, }\AttributeTok{data =}\NormalTok{ phd)}
\FunctionTok{summary}\NormalTok{(m6)}
\end{Highlighting}
\end{Shaded}

\begin{verbatim}
## 
## Call:
## glm(formula = articles ~ gender + kids2 + mentor, family = quasipoisson, 
##     data = phd)
## 
## Deviance Residuals: 
##     Min       1Q   Median       3Q      Max  
## -3.4818  -1.5758  -0.3663   0.5443   5.5757  
## 
## Coefficients:
##               Estimate Std. Error t value Pr(>|t|)    
## (Intercept)   0.423844   0.064597   6.561 8.93e-11 ***
## genderfemale -0.233279   0.073885  -3.157  0.00164 ** 
## kids2Jah     -0.179615   0.076785  -2.339  0.01954 *  
## mentor        0.025776   0.002659   9.695  < 2e-16 ***
## ---
## Signif. codes:  0 '***' 0.001 '**' 0.01 '*' 0.05 '.' 0.1 ' ' 1
## 
## (Dispersion parameter for quasipoisson family taken to be 1.843047)
## 
##     Null deviance: 1817.4  on 914  degrees of freedom
## Residual deviance: 1646.8  on 911  degrees of freedom
## AIC: NA
## 
## Number of Fisher Scoring iterations: 5
\end{verbatim}

\hypertarget{mudeli-sobivus-model-fit}{%
\section{\texorpdfstring{Mudeli sobivus (\emph{model fit})}{Mudeli sobivus (model fit)}}\label{mudeli-sobivus-model-fit}}

Mudeli sobivust andmetega (\emph{goodness of fit}) saame analoogselt logistilise regressiooniga hinnata jääkhälbimuse (\emph{Residual deviance}) abil:

\begin{Shaded}
\begin{Highlighting}[]
\NormalTok{res\_dev }\OtherTok{\textless{}{-}} \FunctionTok{deviance}\NormalTok{(m6)}
\NormalTok{res\_df }\OtherTok{\textless{}{-}} \FunctionTok{df.residual}\NormalTok{(m6)}
\FunctionTok{pchisq}\NormalTok{(res\_dev, res\_df, }\AttributeTok{lower.tail=}\ConstantTok{FALSE}\NormalTok{)}
\end{Highlighting}
\end{Shaded}

\begin{verbatim}
## [1] 5.095965e-45
\end{verbatim}

Näeme, et \emph{p}-väärtus on väiksem kui \(0.05\), mis tähendab, et meie mudel ei sobitu andmetega väga hästi (siin tahame, et \emph{p}-väärtus oleks võimalikult suur). Reaaleluliste andmetega ongi tegelikult väga keeruline hästi sobituvat mudelit leida. Seega üldjuhul me lihtsalt lepime, et meie mudel ei ole täiuslik ja jätame selle testi tähelepanuta.

\hypertarget{mudeli-statistiline-olulisus-1}{%
\subsection{Mudeli statistiline olulisus}\label{mudeli-statistiline-olulisus-1}}

Mudeli statistilist olulisust saame jällegi hinnata analoogselt logistilisele regressioonil hälbimuse näitajate abil. Võrdleme nullmudeli hälbimust (\emph{Null deviance}) ja jääkhälbimust (\emph{Residual deviance}). Nullmudeli hälbimus hinnatud \(Y\) hälbimust tegelikest \(Y\) väärtustest juhul kui ühtegi prediktorit mudelis ei ole. Seega kahe hälbimuse vahe näitab kui palju meie mudel tänu sõltumatutele tunnustele paremaks on läinud. See hälbumuste vahe on jälle jaotunud hii-ruut jaotsue alusel, mille vabadusastemeteks on nullmudeli ja hinnatava mudeli vabadusastmete vahe.

\begin{Shaded}
\begin{Highlighting}[]
\NormalTok{dev\_vahe }\OtherTok{\textless{}{-}}\NormalTok{ m6}\SpecialCharTok{$}\NormalTok{null.deviance }\SpecialCharTok{{-}}\NormalTok{ m6}\SpecialCharTok{$}\NormalTok{deviance}
\NormalTok{df\_vahe }\OtherTok{\textless{}{-}}\NormalTok{ m6}\SpecialCharTok{$}\NormalTok{df.null}\SpecialCharTok{{-}}\NormalTok{m6}\SpecialCharTok{$}\NormalTok{df.residual}
\FunctionTok{pchisq}\NormalTok{(dev\_vahe, df\_vahe, }\AttributeTok{lower.tail =}\NormalTok{ F)}
\end{Highlighting}
\end{Shaded}

\begin{verbatim}
## [1] 9.269198e-37
\end{verbatim}

\emph{p}-väärtus on väiksem kui \(0.05\), seega nullmudel ja meie mudel erinevad olulisel määral ning võime järeldada, et meie mudel on statistiliselt oluline.

\hypertarget{juxe4uxe4kide-jaotus-1}{%
\subsection{Jääkide jaotus}\label{juxe4uxe4kide-jaotus-1}}

Poissoni mudeli üheks eelduseks olid üksteisest sõltumatud ja normaaljaotuse järgi jaotunud jäägid.

Paneme joonisele mudeli jäägid (hälbimused) ja (log) prognoositud väärtused. Jäägid peaksid üle prognoositud väärtuste suhteliselt ühtlaselt jaotuma ning mingit selgelt eristuvat mustrit ei tohiks täheldada. Antud juhul ei ole olukord just ideaalne, aga ka mitte kõige hullem. Näeme, et prognoositud väärtuste paremas otsas koonduvad jäägid pigem allapoole, samas jääkide variatiivsus on keskjoonest kõrgemal mõnevõrra suurem. Võib eeldada, et nad ei ole päris normaaljaotuse järgi jaotunud.

\begin{Shaded}
\begin{Highlighting}[]
\NormalTok{res }\OtherTok{\textless{}{-}} \FunctionTok{residuals}\NormalTok{(m6, }\AttributeTok{type=}\StringTok{"deviance"}\NormalTok{)}
\FunctionTok{plot}\NormalTok{(}\FunctionTok{log}\NormalTok{(}\FunctionTok{predict}\NormalTok{(m6)), res)}
\end{Highlighting}
\end{Shaded}

\includegraphics{03-poisson_files/figure-latex/unnamed-chunk-23-1.pdf}

Kontrollime seda ka histogrammiga:

\begin{Shaded}
\begin{Highlighting}[]
\FunctionTok{hist}\NormalTok{(res)}
\end{Highlighting}
\end{Shaded}

\includegraphics{03-poisson_files/figure-latex/unnamed-chunk-24-1.pdf}

Ka siit nähtub, et jäägid ei ole tõesti päris normaalselt jaotunud.

Lõppkokkuvõtteks tuleb tõdeda, et meie mudel ei ole päris ideaalne - ei sobitu kõige paremini andmetega ning ka jääkide struktuur jätab soovida. Põhjus võib olla näiteks selles, et mõni oluline sõltumatu tunnus on mudelist puudu või ka asjaolu, et sõltuva tunnuse jaotus ei vastanud väga hästi Poissoni jaotusele.

\hypertarget{part-mitmetasandiline-regressioon}{%
\part{Mitmetasandiline regressioon}\label{part-mitmetasandiline-regressioon}}

\hypertarget{mitmetasandiline-regressioon}{%
\chapter{Mitmetasandiline regressioon}\label{mitmetasandiline-regressioon}}

\hypertarget{andmete-hierarhiline-struktuur}{%
\section{Andmete hierarhiline struktuur}\label{andmete-hierarhiline-struktuur}}

Valdav osa meie uuritavatest nähtustest ning uurimisobjektidest on mingil määral hierarhiliselt struktureeritud või klasterdunud:

\begin{itemize}
\tightlist
\item
  õpilased klasside kaupa
\item
  patsiendid eri arstide juures
\item
  elanikud linnaosades, linnades, regioonides jne
\item
  töötajad ettevõetetes
\item
  paneeluuringutes (longituuduuringutes) vaatluskorrad inimeste kaupa
\item
  jne
\end{itemize}

Vaatlused klastrite sees kipuvad olema sarnasemad kui klastrite vahel. Samas klassis õppivaid õpilasi ühendavad samad õpetajad ja nende kasutatavad õppemeetodid ning õppematerjalid, üldine klassi mentaliteet jms. Taolised ühised kogemused ühendavad sama klassi õpilasi ja samas eristavad neid teiste klasside õpilastest. Kui näiteks tahame mõõta õpilaste lugemust, ning hinnata, kuivõrd seda mõjutab lapse vanemate kultuuriline kapital, siis peame korrektsete järelduste tegemiseks seda klassidesisest sarnasust ja klassidevahelist erinevust arvesse võtma. Klassis, kus on hea kirjanduse õpetaja või lugemist väärtustav keskkond, võivad ka madala kultuurilise kapitaliga vanemate lapsed lugeda märksa rohkem, kui mõnes teises klassis vähem lugemist soodustavas keskkonnas õppivad kõrge kultuurilise kapitaliga vanemate lapsed.

Võib-olla kõige selgemini on vaatluste hierarhiline struktuur tajutav longituudandmete puhul, kus me mõõdame või vaatleme mingeid omadusi või nähtusi samade respondentide puhul mitmeid kordi mingi aja jooksul. Ehk siis kui tahame näiteks teada kuidas lugemisharjumused muutuvad üle aja, siis võime võime võtta juhuvalimina 100 inimest ning küsida kümne aasta jooksul igal aastal nende selle aasta lugemuse kohta. Nii toimides oleks meil lõpuks koos 1000 erinevat vaatlust või vastust. Kuid on ju ilmne, et me ei saa kõiki neid tuhandet vaatlust sõltumatutena käsitleda. Inimesed, kellele meeldib lugeda, teevad seda tõenäoliselt igal aastal ning inimesed, kes eelistavad muul viisil aega veeta, loevad igal aastal vähem.

Seega andmete hierarhilist struktuuri ja loomulikku klasterdumist arvesse võtmata ei ole meil võimalik korrektselt mingeid mõjusid või tendentse hinnata, sest need mõjud \textbf{sõltuvad} klastreid ühendavatest karakteristikutest. See aga tähendab, et tavaline lineaarne regressioon taoliste mõju hindamiseks ei sobi. Nimelt oli lineaarse regressiooni üheks eelduseks vaatluste (ja jääkide) sõltumatus. Kui me selle eeldustega ei arvesta, siis võivad effektide standardvead olla liiga väikesed, mistõttu võime näha statistiliselt olulisi seoseid seal kus neid tegelikult ei ole.

Mõningatel eriti hulludel juhtudel võivad lisaks standardvigadele olla valed ka regressiooniseosed ise. Sellisel puhul oleksid meie järelduse mitte ainult ebatäpsed vaid totaalselt valed. Vaatame järgmist näidet, kus on matemaatikatestiks õppimisele kulunud aja kaudu üritatakse seletada testi tulemust\footnote{Tegemist on reaalsete andmetega, kuid olen nende tähendust diaktilistel kaalutlustel muutnud. Reaalselt on need \emph{Iris} andmestiku andmed, \(x\)-teljel on iiriste kroonlehtede pikkus ja \(y\)-teljel kroonlehtede laius (vt \texttt{help(iris)})}. Kasutame hinnanguks kõigepealt tavalist lineaarset regressiooni:

\includegraphics{04-multilevel_files/figure-latex/unnamed-chunk-2-1.pdf}

Tulemused tundub kuidagi väga imelikud. Mida kauem õpitakse, seda halvem on testi skoor. Kuida seda seletada? Asi on selles, et tegemist on kolme erineva klassi õpilastega, kelle matemaatilised oskused on suhteliselt erinevad:

\includegraphics{04-multilevel_files/figure-latex/unnamed-chunk-3-1.pdf}

Kui me seda klasside erinevust arvesse võtame ja igas klassis eraldi regressioonivõrrandi hindame, on tulemused juba märgatavalt loogilisemad:

\includegraphics{04-multilevel_files/figure-latex/unnamed-chunk-4-1.pdf}

Kõikides klassides eraldivõetuna on seos õppimisele kulunud aja ja testitulemuse vahel ikkagi positiivne. Lihtsalt korrelatsioonikoefitsiendi suurus (sirge tõus) ja vabaliige (sirge lõikumine \(y\)-teljega kui \(x\) on 0) on kõikides klassides erinev. Seega antud näite puhul oleks meie järeldused, kui me andmete hierarhilist struktuuri arvesse ei võtaks, vastupidised tegelikule olukorrale. Taolist olukorda nimetatakse Simpsoni paradoksiks. Näide on loomulikult mõnevõrra utreeritud, kuid illustreerib hästi andmestiku hierarhilise struktuuriga arvestamise vajalikkust.

\hypertarget{mida-me-uurida-tahame}{%
\subsection{Mida me uurida tahame?}\label{mida-me-uurida-tahame}}

Mida siis teha, kui me teame või eeldame, et meie uurimisobjektid on hierarhiliselt struktureeritud või moodustavad mingite karakteristikute alustel gruppe? Võimalusi on tegelikult mitmeid ning valik nende võimaluste seast sõltub eelkõige nenede hierarhiate või gruppide iseloomust, aga ka sellest, mis meid konkreetselt huvitab.

Klasterdavatest tunnustest lähtuvalt saame analüüsiühikud jagada tasanditeks. \textbf{Esimese tasandi analüüsiühikud} on need mida või keda me otseselt vaatleme või küsitleme. \textbf{Teise tasandi analüüsiühikud} grupid, millesse esimese tasandi ühikud kuuluvad. Näiteks kui õpilased on esimese tasandi analüüsiühikud, siis klassid on teise tasandi analüüsiühikud. Või kui patsiendid on esimene tasand, siis neid ühendavad arstid on teine tasand. Tasandeid võib loomulikult olla ka rohkem. Õpilaste puhul saavad kolmandaks tasandiks olla näiteks koolid, neljandaks tasandiks riigid jne.

\begin{enumerate}
\def\labelenumi{\arabic{enumi}.}
\item
  Meie uurimishuvi võib seisneda ainult esimese tasandi mõjude hindamises. Sellisel juhul käitleme teisi tasandeid lihtsalt takistustena, mida peame oma esimese tasandi analüüsis arvesse võtma, kuid mis meid eraldiseisvalt ei huvita (kas õppimisele kulunud aeg üldiselt mõjutab testitulemust). Võime lihtsalt olla huvitatud ka sellest, kas esimese tasandi mõju on võimalik eristada struktuursetest teguritest (teise tasandi mõjust).
\item
  Meie uurimishuvi võib seisneda ka teise tasandi mõjude hindamisel esimese tasandi väljunditele. Ehk siis kas makrotasandi tegurid mõjutavad indiviiditasandi väljundeid (arvestades sealjuures ka indiviidide erinevusi). Näiteks kas klassi tüüp mõjutab testitulemusi.
\item
  Lõuks võib meie uurimishuvi seisneda ka teise tasandi tegurite mõjus esimese tasandi mõjudele. Kas klassi tüüp mõjutab seost õppimisele kulunud aja ja testitulemuse vahel? Või kas klassis kasutusel olev õppekava maht mõjutab testitulemusi? Ehk siis kas mingid struktuursed tegurid mõjutavad indiviiditasandi mõjusid.
\end{enumerate}

\hypertarget{kuidas-vaatluste-hierahilisust-analuxfcuxfcsis-arvesse-vuxf5tta}{%
\subsection{Kuidas vaatluste hierahilisust analüüsis arvesse võtta?}\label{kuidas-vaatluste-hierahilisust-analuxfcuxfcsis-arvesse-vuxf5tta}}

\begin{enumerate}
\def\labelenumi{\arabic{enumi}.}
\item
  Viimase, õppimisele kulunud aja ja testitulemuste seoste näite puhul, ehk siis olukorras, kus grupitunnuse kategooriaid on suhteliselt vähe (näites oli neid ainult kolm), oleks kõige mõistlikum kasutada tavalist lineaarset regressiooni, kuid kaasta grupeeriv tunnus sõltumatu tunnusena mudelisse. Sellisel juhul on meil tegemist tavalise mitmese regressiooniga (taolist mudelit, kus meil on nii pidev kui kategoriaalne sõltumatu tunnus, nimetatakse ka ANCOVA (\emph{Analysis of Covariane}) mudeliks). Kui me kaasame mudelisse ka sõltumatute tunnuste vahelised interaktsioonid, saame hinnata ka seda kas erinevate gruppide regressioonikoefitsiendid erinevad (ehk siis kas gruppide regressioonisirgete tõusud erinevad). Kõnealuses näites see nii ka oli.
\item
  Kui gruppe on natukene rohkem (näiteks 15 gruppi) ja meid huvitvad eelkõige vaid esimese tasandi mõjud (ehk siis meid ei huvita kuidas grupeeriv tunnus meie uuritavat mõju mõjutab või me eeldame, et ta seda ei mõjuta), siis saame kasutada \textbf{fikseeritud mõjudega mudelit} (\emph{fixed effects model}). Näiteks tahame küsitlusandmetega uurida kuidas sissetulekute suurused mõjutavad eluga rahulolu, kuid kuna meie valim on üle-eestiline, siis peaksime arvestama piirkondlike (näiteks maakondlike) erinevustega sissetulekute suurustes. Meid ei huvita niiväga see sissetulekute erinevus, kuivõrd üldine sissetulekute mõju rahulolule. Seega tahame lihtsalt oma tulemust piirkondlike erinevuste suhtes kontrollida. Fikseeritud mõjudega mudel on tegelikult jällegi tavaline regressioonimudel, millesse on kaasatud grupeeriv kategoriaalne tunnus. Kategoriaalne sõltumatu tunnus on regressioonimudelis alati \emph{dummy}-tunnusena, (see tähendab et kui meil on näiteks 15 kategooriat, siis peame sellest tegema (või no R teeb) 14 binaarset tunnust, mille kaudu meil on kõik 15 kategooriat defineeritud). Seega, kui tahame hinnata mingi sõltumatu pidevtunnuse mõju sõltuvale tunnusele ning andmestiku hierarhilise struktuuri tõttu lisame sinna ka 15 kategooriaga kategoriaalse tunnuse, siis saame mudeli tulemina 15 regressioonikoefitsienti (1 pidevtunnuse koefitsient ja 14 grupeeriva tunnuse koefitsienti). Muidugi võime ka neid neid kategoriaalsete tunnuste koefitsiente tõlgendada (need näitavad kuidas gruppide keskmised referentsgategooriast erinevad), kuid üldjuhul tahame sellise mudeliga lihtsalt oma uuritavat mõju grupeeriva tunnuse suhtes kontrollida (leida grupeerivast tunnusest sõltumatut mõju). Taolise mudeli puhul eeldame vaikimisi, et meie uuritav mõju on kõikide grupeerivate tunnuste lõikes sama (regressioonisirge tõus on kõikides gruppides sama) ja erinevad vaid gruppide keskmised (mudeli kontekstis vabaliikmed). Kui see nii ei ole, siis peaksime mudelisse kaasma ka koosmõjud. Kuid sellisel juhul palju õnne meile nende interaktsioonide tõlgendamiel. Kui lisaks grupeerivale tunnusele on mudelis vaid üks sõltumatu tunnus, siis oleks see tegelikult tehtav (eriti kui referentskategooria on hästi valitud), kuid üldiselt on meil ju mudelis rohkem kui üks sõltumatu tunnus.
\item
  Kui meid huvitavad eelkõige teise tasandi mõjud, siis saame kasutada nn \textbf{kaheastmeliste mudelit} (\emph{two-stage model}). Näiteks kui tahame teada kuidas omavalitsuste investeeringud infrastruktuuri mõjutavad nende elanike eluga rahulolu ja nende elanike sissetulekute mõju eluga rahulolule. Kaheastmeliste mudelite puhul moodustatakse igas klastris või grupis omaette regresioonimudel, seejärel võetakse nende mudelite koefitsiendid ning kasutatake neid omakorda sisenitena teise astme regressioonimudelis. Meie näite puhul peaksime igas KOVis moodustama mudeli, millega hindame sissetulekute mõju eluga rahulolule. Eesti puhul oleks siis vaja moodustada 79 mudelit. Seejärel võtame kõikide mudelite koefitsiendid ning kasutame neid teise tasandi mudelis sõltuva tunnusena. Teise tasandi sõltumatuks tunnuseks oleks KOVide investeeringute suurused. Taolise lähenemisega saame juhul, kui esimeses tasandis on piisavalt vaatlusi, küllaltki adekvaatse ning keerulisemate mudelitega võrreldava tulemuse. Mudeli kasuks räägib eelkõige see, et tegemist on väga lihtsa ja arusaadava moodusega teise tasandi effektide tuletamiseks. Kuid vägagi ilmsed on ka taolise lähenemise kitsaskohad. Gruppe kokku agregeerides kaotame väga palju informatsiooni. Ja mida vähem on meil informatsiooni, seda vähem kindlad me oma järeldustes olla saame. See reegel kehtib ka laiemalt - mida rohkem me oma algseid andmeid agregeerime, seda ebatäpsemad meie järeldused saavad olema.
\item
  Kui meid huvitavad jällegi ainult esimese tasandi mõjud, kuid samas me teame, et meie andmestik on hierahiliselt struktureeritud (ja võib-olla meil ei ole selle hierarhilise struktuuri kohta väga palju infot), siis saame regressioonimudeli hindamisel kasutada nn robustseid standardvigu (\emph{robust standard errors}, \emph{clustered standard errors}). See on puhtalt analüütiline lahendus, mille kaudu me saame oma mudeli tulemuse võimaliku klasterdumise eest kindlustada.
\item
  Viimaks on meil võimalik kasutada ka \textbf{mitmetasandilist analüüsi/mudelit} (\emph{multilevel analysis/models}, nimetatakse ka \emph{hierarchical linear models}, \emph{linear mixed-effect models}, \emph{mixed models}, \emph{nested data models}, \emph{random coefficient models} või \emph{random-effects models}). Järgenvalt vaatamegi seda meetodit lähemalt.
\end{enumerate}

\hypertarget{mitmetasandiline-analuxfcuxfcs}{%
\section{Mitmetasandiline analüüs}\label{mitmetasandiline-analuxfcuxfcs}}

Mitmetasandiline analüüs võimaldab:

\begin{enumerate}
\def\labelenumi{\arabic{enumi}.}
\tightlist
\item
  ühiskonna või muu uurimisobjekti struktureeritust, heterogeensust ja kontekstuaalsust mõjude hindamisel arvesse võtta\\
\item
  ja hinnata sotsiaalse, kultuurilise, institutsionaalse või muu konteksti mõju uuritavatele nähtustele ja uurimisobjektidele
\end{enumerate}

Eelkõige huvitab meid mitmetasandilise analüüsi juures just see teine punkt - gruppide erinevuste (nii grupisiseste mõjude kui gruppide keskmiste erinevuste) selgitamine kõrgema tasandi sõltumatute tunnuste abil. Mitmetasandiline analüüs on oma olemuselt sarnande eelnevalt kirjeldatud kaheastmelisele mudelile, kus saime teise tasandi tunnuste abil (näiteks KOVide investeeringute suurused) seletada gruppidevahelist regressioonikoefitsientide varieeruvust. Kuid erinevalt kaheastmelisest mudelist ei agregeerita siin gruppide koefitsiente enne teise tasandi juurde asumist, vaid käsitletakse esimese ja teise (või ka kolmana või neljana vne) tasandite variatiivsust samas mudelis. Seega ei kaota me mudeli hindamisel meile vajalikku informatsiooni ning järeldused saavad olla täpsemad.

\hypertarget{fikseeritud-ja-juhuslikud-efektid}{%
\subsection{Fikseeritud ja juhuslikud efektid}\label{fikseeritud-ja-juhuslikud-efektid}}

Mitmetasandilises analüüsis käsitleme regressioonikoefitsiente ja vabaliiget mitte konkreetsete ja fikseeritutena, vaid vaid mingi \textbf{teise tasandi grupeeriva tunnuse lõikes varieeruvatena}. Regressiooni koefitsientides (vabaliige ja regressioonikordajad) eristatakse fikseeritud (\emph{fixed}) ja juhuslikku (\emph{random}) osa:

\begin{itemize}
\tightlist
\item
  \textbf{fikseeritud osa} on koefitsientide gruppideülesed üldkeskmised;
\item
  \textbf{juhuslik osa} näitab gruppide koefitsientide varieeruvust ümber üldkeskmiste.
\end{itemize}

Mida see sisuliselt tähendab? Oletame, et meil on regressioonivõrrand iga grupi kohta:

\[Y_{ij} = \beta_{0j} + \beta_{1j}X_{ij}+\epsilon_{ij}\]
kus:

\(Y_{ij}\) on esimese tasandi sõltuva tunnuse väärtus indiviidi \(i\) ja grupi \(j\) jaoks;\\
\(\beta_{0j}\) on grupi \(j\) vabaliige;\\
\(\beta_{1j}\) on regressioonikordaja grupis \(j\);\\
\(X_{ij}\) on esimese tasandi sõltumatu tunnuse väärtus grupi \(j\) kuuluva indiviidi \(i\) jaoks;\\
\(\epsilon_{ij}\) on regressiooni jääk (viga) indiviidi \(i\) jaoks, kes kuulub gruppi \(j\).

Me saame gruppide vabvaliikmetes eristada gruppideülest keskmist vabaliiget \(\gamma_{00}\) ja iga grupi hälbimust sellest keskmisest \(u_{0j}\) (ehk siis iga grupi erinevust üldkeskmisest vabaliikmest):

\[Y_{ij} = \underbrace{\beta_{0j}}_{\beta_{0j}= \gamma_{00}+ u_{0j}} + \beta_{1j}X_{ij}+\epsilon_{ij}\]
kus:

\(\gamma_{00}\) on esimese tasandi vabaliikmete keskmine ehk vabaliikme fikseeritud osa;\\
\(u_{0j}\) on gruppide erinevus keskmisest vabaliikmest ehk vabaliikme juhuslik osa.

Samamoodi saame gruppide regressioonikoefitsientide puhul eristada gruppideülest keskmist koefitsienti \(\gamma_{10}\) ja iga grupi hälbimust sellest keskmisest \(u_{1j}\):

\[Y_{ij} = \beta_{0j} + \underbrace{\beta_{1j}}_{\beta_{1j}= \gamma_{10}+ u_{1j}}X_{ij}+\epsilon_{ij}\]
kus:

\(\gamma_{10}\) on esimese tasandi regressioonikoefitsientide gruppideülene keskmine ehk regressioonikordaja fikseeritud osa;\\
\(u_{1j}\) on gruppide erinevus keskmisest regressioonikordajast ehk regrssioonikordaja juhuslik osa.

Kuna nii vabaliikme kui regressionikoefitseindi puhul on meil nüüd varieeruvad juhuslikud osad (võime neist mõelda kui tavalise regressiooni sõltvatest tunnustest), siis saame nende variatiivsust teise tasandi sõltumatute tunnuste abil regressioonga selgitada. Lisame teise tasandi sõltumatu tunnuse \(W_j\), millega üritame seletada esimese tasandi vabaliikme varieeruvust:

\[Y_{ij} = \underbrace{\beta_{0j}}_{\beta_{0j}= \gamma_{00}+ \gamma_{01}W_j + u_{0j}} + \beta_{1j}X_{ij}+\epsilon_{ij}\]
kus:

\(\gamma_{01}\) on regressioonikordaja, mille läbi teise tasandi sõltumatu tunnus \(W_j\) selgitab esimese tasandi vabaliikmete varieeruvust;\\
\(u_{0j}\) on teise tasandi regressioonivõrrandi viga ehk siis see osa vabaliikmete varieeruvusest, mida me teise tasandi sõltumatu tunnusega selgitada ei suuda.

Täpselt samamoodi saame toimida ka regressioonikoefitseindi juhusliku osaga, mille varieeruvust on jällegi võimalik teise tasandi sõltumatu tunnuse \(W_j\) abil selgitada:

\[Y_{ij} = \beta_{0j} + \underbrace{\beta_{1j}}_{\beta_{1j}= \gamma_{10}+ \gamma_{11}W_j + u_{1j}}X_{ij}+\epsilon_{ij}\]
kus:

\(\gamma_{11}\) on regressioonikordaja, mille läbi teise tasandi sõltumatu tunnus \(W_j\) selgitab esimese tasandi regressioonikoefitseintide varieeruvust;\\
\(u_{1j}\) on teise tasandi regressioonivõrrandi viga ehk siis see osa regressioonikoefitseintide varieeruvusest, mida me teise tasandi sõltumatu tunnusega selgitada ei suuda.

Kui me nüüd nii vabaliikmete kui ka regressioonikoefitsientide fikseeritud ja juhuslikud osad kokku paneme:

\[Y_{ij} = \underbrace{\beta_{0j}}_{\gamma_{00}+ \gamma_{01}W_j + u_{0j}} + \underbrace{\beta_{1j}}_{\gamma_{10}+ \gamma_{11}W_j + u_{1j}}X_{ij}+\epsilon_{ij}\]

siis saamegi mitmetasandilise regressiooni võrrandi:

\[Y_{ij} = \gamma_{00}+ \gamma_{01}W_j  + \gamma_{10}X_{ij}+ \gamma_{11}W_jX_{ij} + u_{0j} + u_{1j}X_{ij} + \epsilon_{ij}\]

kus võime eristada fikseeritud osa ja juhuslikku osa:

\[Y_{ij} = \underbrace{\gamma_{00}+ \gamma_{01}W_j  + \gamma_{10}X_{ij}+ \gamma_{11}W_jX_{ij}}_{\text{fikseeritud osa}} +  \underbrace{u_{0j} + u_{1j}X_{ij} + \epsilon_{ij}}_{\text{juhuslik osa}}\]

Võrrand võib esmapilgul väga kirju ja segane tunduda, kuid kui eelnev tuletuskäik rahulikult läbi mõelda ning kreeka tähtedest end mitte häirida lasta, siis on kõik tegelikult väga loogiline ning võiks neile, kes tavalise lineaarse regressiooniga tuttavad on, mõistetav olla.

\hypertarget{mitmetasandilise-analuxfcuxfcsi-etapid}{%
\subsection{Mitmetasandilise analüüsi etapid}\label{mitmetasandilise-analuxfcuxfcsi-etapid}}

Mitmetasandilisi mudeleid koostatakse üldjuhul etapiviisiliselt, alustades lihtsamast mudelist ning liikudes edasi keerulisemate mudelite poole:

\begin{enumerate}
\def\labelenumi{\arabic{enumi}.}
\tightlist
\item
  Esmalt hinnatakse \textbf{nullmudel} (\emph{null model} või \emph{empty model}), millega hinnatakse vaid gruppidest tulenevat esimese tasandi sõltuva tunnuse varieeruvust. Kui esimese tasandi sõltuv tunnus gruppide vahel statistiliselt oluliselt ei varieeru, siis ei ole mõtet mitmetasandilise analüüsiga edasi minna (kuna siis me ju ei saa tasandeid eristada).
\end{enumerate}

\[Y_{i,j} = \gamma_{00} + u_{0j} + \epsilon_{ij}\]

\begin{enumerate}
\def\labelenumi{\arabic{enumi}.}
\setcounter{enumi}{1}
\tightlist
\item
  Kui nullmudel indikeerib gruppidevahelist erinevust, siis hinnatakse \textbf{juhusliku vabaliikmega mudel} (\emph{random intercept model}), kus lisatakse esimese tasandi sõltumatu tunnus ning hinntakse vabaliikme varieeruvust gruppide vahel.
\end{enumerate}

\[Y_{i,j} = \gamma_{00}+ \gamma_{11}X_{ij} + u_{0j} + \epsilon_{ij}\]

\begin{enumerate}
\def\labelenumi{\arabic{enumi}.}
\setcounter{enumi}{2}
\tightlist
\item
  Kui ka juhusliku vabaliikmega mudel näitab, et vabaliikmed varieeruvad gruppide vahel, siis saab moodustada \textbf{juhusliku regressioonikordajaga mudeli} (\emph{random slope model}), kus lastakse ka esimese tasandi regressioonikordajad gruppide vahel varieeruma.
\end{enumerate}

\[Y_{i,j} = \gamma_{00}+ \gamma_{11}X_{ij} + u_{0j} + u_{1j}X_{ij} + \epsilon_{ij}\]

\begin{enumerate}
\def\labelenumi{\arabic{enumi}.}
\setcounter{enumi}{3}
\tightlist
\item
  Kui vabaliikmed või regressioonikordajad varieeruvad olulisel määral, siis saame vajadusel ja võimalusel nende variatsiooni mingite teise tasandi sõltumatute tunnustega selgitada, ehk siis lülitada mudelisse ka teise tsandi prediktorid (kui näeme variatiivsust vabaliikmete hulgas, kuid mitte regressioonikordajate hulgas, siis saame teise tasandi prediktoreid kasutada loomulikult vaid vabaliikmete selgitamiseks).
\end{enumerate}

Teise tasandi prediktor selgitamas vabaliikme varieeruvust:

\[Y_{i,j} = \gamma_{00}+ \gamma_{01}W_j + \gamma_{10}X_{ij} + u_{0j} + u_{1j}X_{ij} + \epsilon_{ij}\]

Teise tasandi prediktor selgitamas nii vabaliikme kui regressioonikordaja varieeruvust (esimese ja teise tsandi prediktorite koosmõju):

\[Y_{i,j} = \gamma_{00}+ \gamma_{01}W_j  + \gamma_{10}X_{ij}+ \gamma_{11}W_jX_{ij}+  u_{0j} + u_{1j}X_{ij} + \epsilon_{ij}\]

Vaatame selgitava näitena ühe uneuuringu raames tehtud eksperimendi andmestikku\footnote{Gregory Belenky, Nancy J. Wesensten, David R. Thorne, Maria L. Thomas, Helen C. Sing, Daniel P. Redmond, Michael B. Russo and Thomas J. Balkin (2003) Patterns of performance degradation and restoration during sleep restriction and subsequent recovery: a sleep dose-response study. \emph{Journal of Sleep Research} \textbf{12}, 1--12. Andmestik on kättesaadav paketis \emph{lme4} nimega \texttt{sleepstudy}}. Eksperimendiga uuriti kuidas unepuudus reaktsioonikiirust mõjutab. Respondentide reaktsioonikiirust mõõdeti kümnel järjestikusel päeval. Enne esimest mõõtmist (0 päev) lasti respondetidel magada nii kaua nagu nad tavaliselt magavad. Kõikidel järgevatel öödel lasti neil magada kolm tundi. Tulemustena esitatav reaktsioonikiirus on keskmine reaktsiooniaeg erinevate testide alusel, mis päeva jooksul sooritati.

Andtud juhul on siis tegemist longituudandmetikuga (kutsutakse ka paneelandmestikuks), kus teise tasandi ühikuks on respondent ning esimese tasandi vaatlused on grupeeritud indiviidide kaupa.

Vaatame kõigepealt kuidas näeks välja tavaline lineaarne regressioon:

\includegraphics{04-multilevel_files/figure-latex/unnamed-chunk-6-1.pdf}

Pilt näeb välja küllaltki ettearvatav - mida pikemalt und piiratakse, seda suuremaks läheb reaktsiooniaeg. Kuid kuna me teame, et tegemist on paneelandmetega, siis tagamaks oma järelduste korrektsust, peame vähemalt kontrollima teise tasandi efektide olemasolu ja potentsiaalset mõju. Defineerime \textbf{nullmudeli}:

\includegraphics{04-multilevel_files/figure-latex/unnamed-chunk-7-1.pdf}

Näeme, et inimeste keskmised reaktsiooniajad (võtmata arvesse undefitsiidi pikkust) erinevad päris olulisel määral (reaalses analüüsis peame muidugi sellele järeldusele jõudma teststatistiku abil). Seega on õigustatud \textbf{juhusliku vabaliikmega mudeli} defineerimine, kus me jätame regressioonikordajad kõikidel respondentidel samaks, kuid laseme vabaliikmed varieeruma:

\includegraphics{04-multilevel_files/figure-latex/unnamed-chunk-8-1.pdf}

Näeme, et ka vabaliikmed varieeruvad indiviidide vahel päris suurel määral ning liigume edasi \textbf{juhuliku regressioonikordajaga mudeli} juurde:

\includegraphics{04-multilevel_files/figure-latex/unnamed-chunk-9-1.pdf}

Ka regressioonikordajad varieeruvad indiviidide vahel päris olulisel määral. Ühe inimese puhul unedefitsiit isegi vähendab reaktsiooniaega ja päris mitme puhul on muutused üle aja marginaalsed, eristudes seeläbi keskmisest reaktsioonikiirusest.

Edasi saaksime vaadata millised individuaalsed karakteristikud seda mõju erinevust mõjutavad. Kuna tegemist on paneeluuringuga, kus teise tasandi ühikuks on indiviid, siis on ka võimalikud teise tasandi sõltumatud tunnused indiviiditunnused. Näiteks sugu, vanus vms.

\hypertarget{mitmetasandilise-analuxfcuxfcsi-eeldused}{%
\section{Mitmetasandilise analüüsi eeldused}\label{mitmetasandilise-analuxfcuxfcsi-eeldused}}

\begin{itemize}
\tightlist
\item
  Esimese tasandi ühikud peavad moodustama teise tasandi ühikute suhtes (pseudo)populatsiooni\\
\item
  Teise tasandi ühikud on valim teise tasandi ühikute populatsioonist

  \begin{itemize}
  \tightlist
  \item
    Kas teise tasandi ühikuid saab mõista juhuvalimina?
  \item
    Kas teise tasandi ühikuid on piisavalt palju (vähemalt 20, parem kui vähemalt 50), et nende põhjal järeldusi teha?
  \end{itemize}
\item
  Esimese tasandi jäägid peaksid olema jaotunud normaaljaotuse alusel (keskmisega 0)
\item
  Jääkide dispersioon peaks gruppide lõikes võrdne olema
\item
  Teise tasandi jäägid peaksid esimese tasandi jääkidest sõltumatud olema
\item
  Tavalised lineaarse regressiooni eeldused

  \begin{itemize}
  \tightlist
  \item
    Sõltuva ja sõltumatute tunnuste seosed peaksid olema lineaarsed
  \item
    Sõltumatud tunnused peaksid olema mõõdetud ilma mõõtmisveata
  \end{itemize}
\end{itemize}

\hypertarget{mitemtasandiline-analuxfcuxfcs-ris}{%
\section{Mitemtasandiline analüüs Ris}\label{mitemtasandiline-analuxfcuxfcs-ris}}

Ris on mitmetasandilise analüüsi teostamiseks mitmeid pakette. Kasutame siinkohal \texttt{lme4} paketti ja sellest \texttt{lmer()} funktsiooni. \texttt{lme4} pakett ei anna meile koefitsientide \emph{p}-väärtusi. Et neid saada, tuleb laadida ka pakett \texttt{lmerTest}\footnote{Tegelikult piisab ka lihtsalt \texttt{lmerTest} paketi laadimisest, kuna see laeb automaatselt ka \texttt{lme4} paketi.}.

Kõigepealt laeme sessiooniks sisse vajalikud paketid (kui need ei ole installitud, siis tuleb seda teha käsuga \texttt{install.packages()})

\begin{Shaded}
\begin{Highlighting}[]
\FunctionTok{library}\NormalTok{(nlme) }\CommentTok{\# siit saame andmed}
\FunctionTok{library}\NormalTok{(ggplot2) }\CommentTok{\# joonised}
\FunctionTok{library}\NormalTok{(dplyr) }\CommentTok{\# andmete töötlemine}
\FunctionTok{library}\NormalTok{(lme4) }\CommentTok{\# mitmetasanilise analüüsi pakett}
\FunctionTok{library}\NormalTok{(lmerTest) }\CommentTok{\# võimaldab mudelile ka p väärtused külge saada}
\end{Highlighting}
\end{Shaded}

Kasutame andmestikuna \texttt{MathAchieve} näidisandmestikku paketist \texttt{nlme}. Andmestik sisaldab USAs 1982. aastal läbi viidud küsitluse ``High School and Beyond'' tulemusi hõlmates 7185 keskkoliõpilast 160-st koolist. Ehk siis esimene tasand on õpilased ja teine tasand koolid. Andmestik sisaldab muu hulgas õpilasete perede sotsiaalmajanduslikku seisu indeksit ning matemaatikatesti tulemust.
Üritame analüüsida kas pere sotsiaalmajanduslik seis mõjutab matemaatikatesti tulemust ja kas see seos varieerub koolide vahel ning kui jah, siis leida seda variatiivsust selgitavad tegurid.

\hypertarget{andmete-ettevalmistus}{%
\subsection{Andmete ettevalmistus}\label{andmete-ettevalmistus}}

Anname andmetele natukene lihtsama nime ja vaatame millega tegu:

\begin{Shaded}
\begin{Highlighting}[]
\NormalTok{andmed }\OtherTok{\textless{}{-}}\NormalTok{ nlme}\SpecialCharTok{::}\NormalTok{MathAchieve }\SpecialCharTok{\%\textgreater{}\%} 
  \FunctionTok{as.data.frame}\NormalTok{()}
\FunctionTok{str}\NormalTok{(andmed)}
\end{Highlighting}
\end{Shaded}

\begin{verbatim}
## 'data.frame':    7185 obs. of  6 variables:
##  $ School  : Ord.factor w/ 160 levels "8367"<"8854"<..: 59 59 59 59 59 59 59 59 59 59 ...
##  $ Minority: Factor w/ 2 levels "No","Yes": 1 1 1 1 1 1 1 1 1 1 ...
##  $ Sex     : Factor w/ 2 levels "Male","Female": 2 2 1 1 1 1 2 1 2 1 ...
##  $ SES     : num  -1.528 -0.588 -0.528 -0.668 -0.158 ...
##  $ MathAch : num  5.88 19.71 20.35 8.78 17.9 ...
##  $ MEANSES : num  -0.428 -0.428 -0.428 -0.428 -0.428 -0.428 -0.428 -0.428 -0.428 -0.428 ...
\end{verbatim}

Tunnused on järgmised:

\begin{itemize}
\tightlist
\item
  \emph{School} - kooli id
\item
  \emph{Minority} - kas õpilane kuulub vähemuse hulka
\item
  \emph{Sex} - sugu
\item
  \emph{SES} - õpilase pere sotsialmajanduslik staatus (standardiseeritud)
\item
  \emph{MathAch} - matemaatikatesti tulemus
\item
  \emph{MEANSES} - kooli keskmine SES skoor (teise ehk kooli tasandi tunnus)
\end{itemize}

Lisaks on \texttt{nlme} paketis ka andmestik \texttt{MathAchSchool}, mis sisaldab veel erinevaid kooli tasandi tunnuseid. Võtame sellest andmestikus tunnused:

\begin{itemize}
\tightlist
\item
  \emph{School} - koolid id
\item
  \emph{Sector} - kas \emph{Public} või \emph{Chatholic} kool
\item
  \emph{PRACAD} - protsent õpilastest, kes õpivad nn ``akadeemilisel suunal''
\item
  \emph{MEANSES} - kooli keskmine SES skoor
\end{itemize}

\begin{Shaded}
\begin{Highlighting}[]
\NormalTok{koolid }\OtherTok{\textless{}{-}}\NormalTok{ nlme}\SpecialCharTok{::}\NormalTok{MathAchSchool }\SpecialCharTok{\%\textgreater{}\%} 
  \FunctionTok{as.data.frame}\NormalTok{() }\SpecialCharTok{\%\textgreater{}\%} 
  \FunctionTok{select}\NormalTok{(School, Sector, PRACAD, MEANSES)}
\FunctionTok{head}\NormalTok{(koolid)}
\end{Highlighting}
\end{Shaded}

\begin{verbatim}
##      School   Sector PRACAD MEANSES
## 1224   1224   Public   0.35  -0.428
## 1288   1288   Public   0.27   0.128
## 1296   1296   Public   0.32  -0.420
## 1308   1308 Catholic   0.96   0.534
## 1317   1317 Catholic   0.95   0.351
## 1358   1358   Public   0.25  -0.014
\end{verbatim}

Taoline mitmetasandiliste andmete organiseerimise viis, kus erinevate tasandite tunnused on erinevates failides, on küllaltki tavapärane. Seega esimeseks sammuks on nii siin, kui ka mitmetasanilise analüüsi puhul üldiselt need kaks andmestikku kokku liita. Kasutame selleks \emph{dplyr}-i \texttt{left\_join()} funktsiooni.

Mõlemat andmestikku ühendavaks id tunnuseks on \emph{School}. Kuid kuna need tunnused on erinevatest klassidest (üks on \emph{ordered factor}, teine tavaline \emph{factor}), siis teeme nad mõlemad kõigepealt \emph{character} tunnusteks

\begin{Shaded}
\begin{Highlighting}[]
\NormalTok{andmed}\SpecialCharTok{$}\NormalTok{School }\OtherTok{\textless{}{-}} \FunctionTok{as.character}\NormalTok{(andmed}\SpecialCharTok{$}\NormalTok{School)}
\NormalTok{koolid}\SpecialCharTok{$}\NormalTok{School }\OtherTok{\textless{}{-}} \FunctionTok{as.character}\NormalTok{(koolid}\SpecialCharTok{$}\NormalTok{School)}
\CommentTok{\# Nüüd saame andmestikud kokku panna}
\NormalTok{andmed }\OtherTok{\textless{}{-}} \FunctionTok{left\_join}\NormalTok{(andmed, koolid, }\AttributeTok{by =} \StringTok{\textquotesingle{}School\textquotesingle{}}\NormalTok{)}
\CommentTok{\# Vaatame ka tulemust}
\FunctionTok{head}\NormalTok{(andmed)}
\end{Highlighting}
\end{Shaded}

\begin{verbatim}
##   School Minority    Sex    SES MathAch MEANSES.x Sector PRACAD MEANSES.y
## 1   1224       No Female -1.528   5.876    -0.428 Public   0.35    -0.428
## 2   1224       No Female -0.588  19.708    -0.428 Public   0.35    -0.428
## 3   1224       No   Male -0.528  20.349    -0.428 Public   0.35    -0.428
## 4   1224       No   Male -0.668   8.781    -0.428 Public   0.35    -0.428
## 5   1224       No   Male -0.158  17.898    -0.428 Public   0.35    -0.428
## 6   1224       No   Male  0.022   4.583    -0.428 Public   0.35    -0.428
\end{verbatim}

Kuna \emph{MEANSES} tunnus oli mõlemas andmestikus, siis on see uues andmestikus kaks korda (vastavalt suffixitega .x ja .y). Kustutame neist esimese ja muudame \emph{MEANSES.y} nime tagasi \emph{MEANSES}-iks.

\begin{Shaded}
\begin{Highlighting}[]
\NormalTok{andmed}\SpecialCharTok{$}\NormalTok{MEANSES.x }\OtherTok{\textless{}{-}} \ConstantTok{NULL}
\NormalTok{andmed }\OtherTok{\textless{}{-}}\NormalTok{ andmed }\SpecialCharTok{\%\textgreater{}\%} 
  \FunctionTok{rename}\NormalTok{(}\AttributeTok{MEANSES =}\NormalTok{ MEANSES.y)}
\end{Highlighting}
\end{Shaded}

Paneme kõik tunnuste nimed väikestesse tähtedesse\footnote{Lihtsalt, et oleks kergem neid trükkida}:

\begin{Shaded}
\begin{Highlighting}[]
\FunctionTok{names}\NormalTok{(andmed) }\OtherTok{\textless{}{-}} \FunctionTok{tolower}\NormalTok{(}\FunctionTok{names}\NormalTok{(andmed))}
\end{Highlighting}
\end{Shaded}

\hypertarget{uxfclevaade-andmetest}{%
\subsection{Ülevaade andmetest}\label{uxfclevaade-andmetest}}

Nagu enne igat analüüsi, uurime kõigepealt andmeid. Vaatame graafiliselt kuidas sotsiaalmajanduslik indeks matemaatika testi tulemusega seostub:

\begin{Shaded}
\begin{Highlighting}[]
\CommentTok{\# punktide jaoks geom\_point()}
\CommentTok{\# mittelineaarse regressioonijoone jaoks geom\_smooth()}
\NormalTok{andmed }\SpecialCharTok{\%\textgreater{}\%} 
  \FunctionTok{ggplot}\NormalTok{(}\FunctionTok{aes}\NormalTok{(}\AttributeTok{x =}\NormalTok{ ses, }\AttributeTok{y =}\NormalTok{ mathach))}\SpecialCharTok{+}
  \FunctionTok{geom\_point}\NormalTok{(}\AttributeTok{alpha =} \FloatTok{0.2}\NormalTok{)}\SpecialCharTok{+}
  \FunctionTok{geom\_smooth}\NormalTok{()}\SpecialCharTok{+}
  \FunctionTok{theme\_bw}\NormalTok{()}
\end{Highlighting}
\end{Shaded}

\includegraphics{04-multilevel_files/figure-latex/unnamed-chunk-16-1.pdf}

Tundub et üldine seos on täitsa olemas ja see on ka suhteliselt lineaarne.

Järgmiseks vaatame kas koolide keskmised matemaatikasjoorid erinevad. Neid oleks hea vaadata koos usalduspiiridega, seega peame need enne välja arvutame. Andmete töötlemiseks kasutame \texttt{dplyr}-i funktsioone:

\begin{Shaded}
\begin{Highlighting}[]
\NormalTok{andmed }\SpecialCharTok{\%\textgreater{}\%} 
  \FunctionTok{group\_by}\NormalTok{(school) }\SpecialCharTok{\%\textgreater{}\%}
  \FunctionTok{summarise}\NormalTok{(}\AttributeTok{keskmine =} \FunctionTok{mean}\NormalTok{(mathach), }
            \AttributeTok{sd =} \FunctionTok{sd}\NormalTok{(mathach), }
            \AttributeTok{n =} \FunctionTok{n}\NormalTok{()) }\SpecialCharTok{\%\textgreater{}\%} 
  \FunctionTok{mutate}\NormalTok{(}\AttributeTok{l\_ci =}\NormalTok{ keskmine }\SpecialCharTok{{-}} \FloatTok{1.96} \SpecialCharTok{*}\NormalTok{ sd}\SpecialCharTok{/}\FunctionTok{sqrt}\NormalTok{(n),}
         \AttributeTok{u\_ci =}\NormalTok{ keskmine }\SpecialCharTok{+} \FloatTok{1.96} \SpecialCharTok{*}\NormalTok{ sd}\SpecialCharTok{/}\FunctionTok{sqrt}\NormalTok{(n)) }\SpecialCharTok{\%\textgreater{}\%}
  \FunctionTok{mutate}\NormalTok{(}\AttributeTok{school =} \FunctionTok{as.factor}\NormalTok{(school),}
         \AttributeTok{school =} \FunctionTok{reorder}\NormalTok{(school, keskmine)) }\SpecialCharTok{\%\textgreater{}\%} 
  \FunctionTok{ggplot}\NormalTok{(}\FunctionTok{aes}\NormalTok{(}\AttributeTok{x =}\NormalTok{ keskmine, }
             \AttributeTok{xmin =}\NormalTok{ l\_ci, }
             \AttributeTok{xmax =}\NormalTok{ u\_ci, }
             \AttributeTok{y =}\NormalTok{ school))}\SpecialCharTok{+}
  \FunctionTok{geom\_pointrange}\NormalTok{()}\SpecialCharTok{+}
  \FunctionTok{theme\_minimal}\NormalTok{()}\SpecialCharTok{+}
  \FunctionTok{theme}\NormalTok{(}\AttributeTok{panel.grid.minor =} \FunctionTok{element\_blank}\NormalTok{(),}
        \AttributeTok{panel.grid.major.y =} \FunctionTok{element\_blank}\NormalTok{(),}
        \AttributeTok{axis.text.y =} \FunctionTok{element\_blank}\NormalTok{())}
\end{Highlighting}
\end{Shaded}

\includegraphics{04-multilevel_files/figure-latex/unnamed-chunk-17-1.pdf}

Jah, keskmised tunduvad kooliti erinevat. On palju koole mille usalduspiirid ei kattu. Järgmiseks vaatame, kas ka sotsiaalmajandusliku seisundi ja testi tulemuste seosed kooliti erinevad:

\begin{Shaded}
\begin{Highlighting}[]
\NormalTok{andmed }\SpecialCharTok{\%\textgreater{}\%} 
  \FunctionTok{ggplot}\NormalTok{(}\FunctionTok{aes}\NormalTok{(}\AttributeTok{x =}\NormalTok{ ses, }\AttributeTok{y =}\NormalTok{ mathach))}\SpecialCharTok{+}
  \FunctionTok{geom\_point}\NormalTok{(}\AttributeTok{alpha =} \FloatTok{0.2}\NormalTok{)}\SpecialCharTok{+}
  \FunctionTok{geom\_smooth}\NormalTok{(}\FunctionTok{aes}\NormalTok{(}\AttributeTok{color =}\NormalTok{ school), }\AttributeTok{se =}\NormalTok{ F, }\AttributeTok{show.legend =}\NormalTok{ F, }\AttributeTok{method =} \StringTok{\textquotesingle{}lm\textquotesingle{}}\NormalTok{)}\SpecialCharTok{+}
  \FunctionTok{theme\_bw}\NormalTok{()}\SpecialCharTok{+}
  \FunctionTok{scale\_colour\_viridis\_d}\NormalTok{(}\AttributeTok{option =} \StringTok{\textquotesingle{}A\textquotesingle{}}\NormalTok{)}
\end{Highlighting}
\end{Shaded}

\includegraphics{04-multilevel_files/figure-latex/unnamed-chunk-18-1.pdf}

Tundub, et ka regressioonikoefitsiendid erinevad.

Vaatame põgusalt üle ka teiste sõltumatute tunnuste jaotused. Eelkõige seetõttu, et kontrollida neid võimalike vigade suhtes.

\begin{Shaded}
\begin{Highlighting}[]
\FunctionTok{table}\NormalTok{(andmed}\SpecialCharTok{$}\NormalTok{sex, }\AttributeTok{useNA =} \StringTok{\textquotesingle{}always\textquotesingle{}}\NormalTok{)}
\end{Highlighting}
\end{Shaded}

\begin{verbatim}
## 
##   Male Female   <NA> 
##   3390   3795      0
\end{verbatim}

\begin{Shaded}
\begin{Highlighting}[]
\FunctionTok{hist}\NormalTok{(andmed}\SpecialCharTok{$}\NormalTok{ses)}
\end{Highlighting}
\end{Shaded}

\includegraphics{04-multilevel_files/figure-latex/unnamed-chunk-20-1.pdf}

\begin{Shaded}
\begin{Highlighting}[]
\FunctionTok{table}\NormalTok{(andmed}\SpecialCharTok{$}\NormalTok{sector, }\AttributeTok{useNA =} \StringTok{\textquotesingle{}ifany\textquotesingle{}}\NormalTok{)}
\end{Highlighting}
\end{Shaded}

\begin{verbatim}
## 
##   Public Catholic 
##     3642     3543
\end{verbatim}

\begin{Shaded}
\begin{Highlighting}[]
\FunctionTok{hist}\NormalTok{(andmed}\SpecialCharTok{$}\NormalTok{pracad)}
\end{Highlighting}
\end{Shaded}

\includegraphics{04-multilevel_files/figure-latex/unnamed-chunk-22-1.pdf}

\begin{Shaded}
\begin{Highlighting}[]
\FunctionTok{hist}\NormalTok{(andmed}\SpecialCharTok{$}\NormalTok{meanses)}
\end{Highlighting}
\end{Shaded}

\includegraphics{04-multilevel_files/figure-latex/unnamed-chunk-23-1.pdf}

Interpretatsiooni huvides oleks regressioonimudelis mõistlik kasutada sotsiaalmajanduslikku indeksi tunnust, mis on koolide keskmiste alusel tsentreeritud. Kuna vabaliige on sõltuva tunnuse (\(y\)) väärtus juhul, kui sõltumatu tunnus (\(x\)) on 0, siis ilma tsentreerimata on vabaliikme väärtus suhteliselt sisutühi. Kui me selle koolide keskmiste lõikes tsentreerime, näitab vabaliige kooli matemaatikatesti tulemust kooli keskmise sotsiaalmajandusliku indeksi väärtuse korral.

\begin{Shaded}
\begin{Highlighting}[]
\NormalTok{andmed }\OtherTok{\textless{}{-}}\NormalTok{ andmed }\SpecialCharTok{\%\textgreater{}\%} 
  \FunctionTok{mutate}\NormalTok{(}\AttributeTok{tses =}\NormalTok{ ses }\SpecialCharTok{{-}}\NormalTok{ meanses)}
\end{Highlighting}
\end{Shaded}

\hypertarget{nullmudel}{%
\subsection{Nullmudel}\label{nullmudel}}

Esmalt defineerime nullmudeli:

\begin{Shaded}
\begin{Highlighting}[]
\NormalTok{m0 }\OtherTok{\textless{}{-}} \FunctionTok{lmer}\NormalTok{(mathach }\SpecialCharTok{\textasciitilde{}} \DecValTok{1} \SpecialCharTok{+}\NormalTok{ (}\DecValTok{1}\SpecialCharTok{|}\NormalTok{school), }\AttributeTok{data =}\NormalTok{ andmed)}
\end{Highlighting}
\end{Shaded}

\texttt{lmer} mudeli defineerimine on natukene erinev tavapärasest \texttt{ml} või \texttt{glm} mudeli defineerimisest.

\begin{itemize}
\tightlist
\item
  \texttt{athach\ \textasciitilde{}\ 1} - sõltuv tunnus koos tildega ja 1 tähistamas vabaliiget. See on identne \texttt{ml} mudeliga. Tavaliselt me lihtsalt \texttt{ml} mudelis vabaliiget eraldi välja ei too, kuna ainult vabaliikmega \texttt{ml} mudeli tulem oleks lihtsalt sõltuva tunnuse keskmine. Aga kuna siin mudelis meil ühtegi sõltumatut tunnust veel ei ole, siis peame vabaliikme eksplitsiitselt ära märkima.
\item
  \texttt{(1\textbar{}school)} - sulgude sees kirjeldatud mudeli juhuslik osa, ehk siis need tunnused, mida me tahame varieeruvatena näha. 1 tähistab jällegi vabaliiget, mis tähendab, et me laseme gruppide vabaliikmetel varieeruda. \textbar{} märgi järel tuleb mudeli grupeeriv tunnus. Antud juhul on selleks \emph{school}.
\end{itemize}

Vaatame mudeli tulemusi:

\begin{Shaded}
\begin{Highlighting}[]
\FunctionTok{summary}\NormalTok{(m0)}
\end{Highlighting}
\end{Shaded}

\begin{verbatim}
## Linear mixed model fit by REML. t-tests use Satterthwaite's method [
## lmerModLmerTest]
## Formula: mathach ~ 1 + (1 | school)
##    Data: andmed
## 
## REML criterion at convergence: 47116.8
## 
## Scaled residuals: 
##     Min      1Q  Median      3Q     Max 
## -3.0631 -0.7539  0.0267  0.7606  2.7426 
## 
## Random effects:
##  Groups   Name        Variance Std.Dev.
##  school   (Intercept)  8.614   2.935   
##  Residual             39.148   6.257   
## Number of obs: 7185, groups:  school, 160
## 
## Fixed effects:
##             Estimate Std. Error       df t value Pr(>|t|)    
## (Intercept)  12.6370     0.2444 156.6473   51.71   <2e-16 ***
## ---
## Signif. codes:  0 '***' 0.001 '**' 0.01 '*' 0.05 '.' 0.1 ' ' 1
\end{verbatim}

Peamised tulemused on toodud \emph{Random effects:} ja \emph{Fixed effects:} kirjete all.

\begin{itemize}
\tightlist
\item
  Koolide keskmine matemaatikatesti tulemus on 12.6
\item
  Koolide keskmiste dispersioon üldkeskmisest testitulemusest on 8.6 (ja keskmine hälbimus 2.9).\\
\item
  Õpilaste keskmine dispersioon koolide keskmistest testitulemustest on 39.1 (ja keskmine hälbimus 6.3).
\end{itemize}

Nende tulemuste alusel saame arvutda intraklassi korrelatsiooni (\emph{interclass correlatsion} ehk ICC), mis näitab mitu protsenti sõltuva tunnuse varieeruvusest on selgitatav grupitunnuse poole

\begin{Shaded}
\begin{Highlighting}[]
\FloatTok{8.614}\SpecialCharTok{/}\NormalTok{(}\FloatTok{8.614+39.148}\NormalTok{)}
\end{Highlighting}
\end{Shaded}

\begin{verbatim}
## [1] 0.1803526
\end{verbatim}

\emph{ca} 18\% testitulemuste variatiivusest on selgitatav koolie erinevustega.

\hypertarget{juhusliku-vabaliikmega-mudel}{%
\subsection{Juhusliku vabaliikmega mudel}\label{juhusliku-vabaliikmega-mudel}}

Lisame mudelile ka sotsiaalmajandusliku indeksi (tsentreeritud variandi), mis läbi saame tulemuseks mudeli kus vabaliikmed gruppide vahel varieeruvad:

\begin{Shaded}
\begin{Highlighting}[]
\NormalTok{m1 }\OtherTok{\textless{}{-}} \FunctionTok{lmer}\NormalTok{(mathach }\SpecialCharTok{\textasciitilde{}} \DecValTok{1} \SpecialCharTok{+}\NormalTok{ tses }\SpecialCharTok{+}\NormalTok{ (}\DecValTok{1}\SpecialCharTok{|}\NormalTok{school), }\AttributeTok{data =}\NormalTok{ andmed)}
\end{Highlighting}
\end{Shaded}

\begin{Shaded}
\begin{Highlighting}[]
\FunctionTok{summary}\NormalTok{(m1)}
\end{Highlighting}
\end{Shaded}

\begin{verbatim}
## Linear mixed model fit by REML. t-tests use Satterthwaite's method [
## lmerModLmerTest]
## Formula: mathach ~ 1 + tses + (1 | school)
##    Data: andmed
## 
## REML criterion at convergence: 46724
## 
## Scaled residuals: 
##      Min       1Q   Median       3Q      Max 
## -3.09686 -0.73224  0.01941  0.75720  2.91473 
## 
## Random effects:
##  Groups   Name        Variance Std.Dev.
##  school   (Intercept)  8.672   2.945   
##  Residual             37.010   6.084   
## Number of obs: 7185, groups:  school, 160
## 
## Fixed effects:
##              Estimate Std. Error        df t value Pr(>|t|)    
## (Intercept)   12.6493     0.2445  156.7433   51.74   <2e-16 ***
## tses           2.1912     0.1087 7022.0245   20.17   <2e-16 ***
## ---
## Signif. codes:  0 '***' 0.001 '**' 0.01 '*' 0.05 '.' 0.1 ' ' 1
## 
## Correlation of Fixed Effects:
##      (Intr)
## tses 0.003
\end{verbatim}

\begin{itemize}
\tightlist
\item
  Sotsiaalmajandusliku indeksi efekti suurus on 2,2 ja mõju on statistiliselt oluline (\emph{p} \textless{} 0.05). Ehk siis kui sotsiaalmajanduslik indeks kasvab ühe punkti võrra, kasvab matemaatikatesti tulemus 2,2 punkti võrra (võttes arvesse ka koolide erinevust).
\item
  Vabaliikmete dispersioon on 8.672 (ja standardhälve 2,9).
\item
  Esimese tasandi vead läksid natuke väiksemaks (39,1 \textgreater{} 37,0), mis on ka loogiline, kuna iga lisanduv sõltumatu tunnus peaks regressioonijääke vähendama.
\end{itemize}

Saame mudeleid võrrelda \texttt{anova()} funktsiooni ja LRT-testiga:

\begin{Shaded}
\begin{Highlighting}[]
\FunctionTok{anova}\NormalTok{(m0, m1)}
\end{Highlighting}
\end{Shaded}

\begin{verbatim}
## Data: andmed
## Models:
## m0: mathach ~ 1 + (1 | school)
## m1: mathach ~ 1 + tses + (1 | school)
##    npar   AIC   BIC logLik deviance Chisq Df Pr(>Chisq)    
## m0    3 47122 47142 -23558    47116                        
## m1    4 46728 46756 -23360    46720 395.4  1  < 2.2e-16 ***
## ---
## Signif. codes:  0 '***' 0.001 '**' 0.01 '*' 0.05 '.' 0.1 ' ' 1
\end{verbatim}

m1 mudel on statistiliselt oluliselt parem kui m0 mudel.

\hypertarget{juhusliku-regressioonikordajaga-mudel}{%
\subsection{Juhusliku regressioonikordajaga mudel}\label{juhusliku-regressioonikordajaga-mudel}}

Laseme sotsiaalmajandusliku indeksi regressioonikorda samuti vabalt varieeruma. Selleks lisame vastava tunnuse, mille koefitsiente tahame vabaks lasta juhuslike efektide sulgudesse:

\begin{Shaded}
\begin{Highlighting}[]
\NormalTok{m2 }\OtherTok{\textless{}{-}} \FunctionTok{lmer}\NormalTok{(mathach }\SpecialCharTok{\textasciitilde{}} \DecValTok{1} \SpecialCharTok{+}\NormalTok{ tses }\SpecialCharTok{+}\NormalTok{ (}\DecValTok{1}\SpecialCharTok{+}\NormalTok{tses}\SpecialCharTok{|}\NormalTok{school), }\AttributeTok{data =}\NormalTok{ andmed)}
\end{Highlighting}
\end{Shaded}

\begin{Shaded}
\begin{Highlighting}[]
\FunctionTok{summary}\NormalTok{(m2)}
\end{Highlighting}
\end{Shaded}

\begin{verbatim}
## Linear mixed model fit by REML. t-tests use Satterthwaite's method [
## lmerModLmerTest]
## Formula: mathach ~ 1 + tses + (1 + tses | school)
##    Data: andmed
## 
## REML criterion at convergence: 46714.2
## 
## Scaled residuals: 
##      Min       1Q   Median       3Q      Max 
## -3.09680 -0.73194  0.01858  0.75388  2.89928 
## 
## Random effects:
##  Groups   Name        Variance Std.Dev. Corr
##  school   (Intercept)  8.682   2.9465       
##           tses         0.694   0.8331   0.02
##  Residual             36.700   6.0581       
## Number of obs: 7185, groups:  school, 160
## 
## Fixed effects:
##             Estimate Std. Error       df t value Pr(>|t|)    
## (Intercept)  12.6493     0.2445 156.7391   51.73   <2e-16 ***
## tses          2.1932     0.1283 155.2180   17.10   <2e-16 ***
## ---
## Signif. codes:  0 '***' 0.001 '**' 0.01 '*' 0.05 '.' 0.1 ' ' 1
## 
## Correlation of Fixed Effects:
##      (Intr)
## tses 0.012
\end{verbatim}

\begin{itemize}
\tightlist
\item
  Sotsiaalmajandusliku indeksi regresioonikordaja dispersioon on 0.69 (ja standardhälve 0.83)
\end{itemize}

Juhuslike effektide jaoks meil väljundis mingit olulisuse testi ei ole. Aga saame kasutada \texttt{lmerTest} paketi \texttt{ranova()} funktsiooni, mis testib erinevate juhuslike efektide panust mudelisse:

\begin{Shaded}
\begin{Highlighting}[]
\FunctionTok{ranova}\NormalTok{(m2)}
\end{Highlighting}
\end{Shaded}

\begin{verbatim}
## ANOVA-like table for random-effects: Single term deletions
## 
## Model:
## mathach ~ tses + (1 + tses | school)
##                             npar logLik   AIC    LRT Df Pr(>Chisq)   
## <none>                         6 -23357 46726                        
## tses in (1 + tses | school)    4 -23362 46732 9.7617  2    0.00759 **
## ---
## Signif. codes:  0 '***' 0.001 '**' 0.01 '*' 0.05 '.' 0.1 ' ' 1
\end{verbatim}

Sotsiaalmajandusliku indeksi effekt on oluline (\emph{p} \textless{} 0.05).

Saame seda testida ka LRT-testiga:

\begin{Shaded}
\begin{Highlighting}[]
\FunctionTok{anova}\NormalTok{(m1,m2)}
\end{Highlighting}
\end{Shaded}

\begin{verbatim}
## Data: andmed
## Models:
## m1: mathach ~ 1 + tses + (1 | school)
## m2: mathach ~ 1 + tses + (1 + tses | school)
##    npar   AIC   BIC logLik deviance  Chisq Df Pr(>Chisq)   
## m1    4 46728 46756 -23360    46720                        
## m2    6 46723 46764 -23356    46711 9.4331  2   0.008946 **
## ---
## Signif. codes:  0 '***' 0.001 '**' 0.01 '*' 0.05 '.' 0.1 ' ' 1
\end{verbatim}

\hypertarget{teise-tasandi-suxf5ltumatud-muutujad}{%
\subsection{Teise tasandi sõltumatud muutujad}\label{teise-tasandi-suxf5ltumatud-muutujad}}

Lisame teise tasandi sõltumatu muutujana kooli keskmise sotsiaalmajandusliku indeksi ja kooli tüübi:

\begin{Shaded}
\begin{Highlighting}[]
\NormalTok{m3 }\OtherTok{\textless{}{-}} \FunctionTok{lmer}\NormalTok{(mathach }\SpecialCharTok{\textasciitilde{}} \DecValTok{1} \SpecialCharTok{+}\NormalTok{ tses }\SpecialCharTok{+}\NormalTok{ meanses }\SpecialCharTok{+}\NormalTok{ sector }\SpecialCharTok{+}\NormalTok{ (}\DecValTok{1}\SpecialCharTok{+}\NormalTok{tses}\SpecialCharTok{|}\NormalTok{school), }\AttributeTok{data =}\NormalTok{ andmed)}
\end{Highlighting}
\end{Shaded}

\begin{Shaded}
\begin{Highlighting}[]
\FunctionTok{summary}\NormalTok{(m3)}
\end{Highlighting}
\end{Shaded}

\begin{verbatim}
## Linear mixed model fit by REML. t-tests use Satterthwaite's method [
## lmerModLmerTest]
## Formula: mathach ~ 1 + tses + meanses + sector + (1 + tses | school)
##    Data: andmed
## 
## REML criterion at convergence: 46543.3
## 
## Scaled residuals: 
##     Min      1Q  Median      3Q     Max 
## -3.1719 -0.7279  0.0123  0.7562  2.9307 
## 
## Random effects:
##  Groups   Name        Variance Std.Dev. Corr
##  school   (Intercept)  2.3878  1.5453       
##           tses         0.7007  0.8371   0.18
##  Residual             36.7097  6.0589       
## Number of obs: 7185, groups:  school, 160
## 
## Fixed effects:
##                Estimate Std. Error       df t value Pr(>|t|)    
## (Intercept)     12.0451     0.1986 160.3808  60.639  < 2e-16 ***
## tses             2.1950     0.1284 154.8896  17.092  < 2e-16 ***
## meanses          5.2463     0.3683 151.3246  14.244  < 2e-16 ***
## sectorCatholic   1.3722     0.3055 149.8780   4.491  1.4e-05 ***
## ---
## Signif. codes:  0 '***' 0.001 '**' 0.01 '*' 0.05 '.' 0.1 ' ' 1
## 
## Correlation of Fixed Effects:
##             (Intr) tses   meanss
## tses         0.060              
## meanses      0.245 -0.004       
## sectorCthlc -0.696  0.002 -0.356
\end{verbatim}

\begin{itemize}
\tightlist
\item
  Kui kooli keskmine sotsiaalmajanduslik indeks tõuseb 1 punkti võrra, siis tõuseb kooli keskmine matemaatikatesti tulemus 5.9 punkti võrra.\\
\item
  Katoliiklikus koolis on keskmine matemaatika testitulemus ca 1.4 punkti parem kui tavakoolis.
\end{itemize}

\begin{Shaded}
\begin{Highlighting}[]
\FunctionTok{anova}\NormalTok{(m2, m3)}
\end{Highlighting}
\end{Shaded}

\begin{verbatim}
## Data: andmed
## Models:
## m2: mathach ~ 1 + tses + (1 + tses | school)
## m3: mathach ~ 1 + tses + meanses + sector + (1 + tses | school)
##    npar   AIC   BIC logLik deviance  Chisq Df Pr(>Chisq)    
## m2    6 46723 46764 -23356    46711                         
## m3    8 46554 46609 -23269    46538 172.84  2  < 2.2e-16 ***
## ---
## Signif. codes:  0 '***' 0.001 '**' 0.01 '*' 0.05 '.' 0.1 ' ' 1
\end{verbatim}

\hypertarget{esimese-ja-teise-tasandi-suxf5ltumatute-tunnuste-koosmuxf5jud}{%
\subsection{Esimese ja teise tasandi sõltumatute tunnuste koosmõjud}\label{esimese-ja-teise-tasandi-suxf5ltumatute-tunnuste-koosmuxf5jud}}

Lõpuks lisame mudelisse ka esimese tasandi sõltumatu tunnuse (\emph{tses}) ja teise tasandi sõltumatute tunnuste (\emph{meanses} ja \emph{sector}) koosmõjud, mille abil saame hinnata kas teise tasandi tunnused mõjutavbad esimese tasandi mõjusid:

\begin{Shaded}
\begin{Highlighting}[]
\NormalTok{m4 }\OtherTok{\textless{}{-}} \FunctionTok{lmer}\NormalTok{(mathach }\SpecialCharTok{\textasciitilde{}} \DecValTok{1} \SpecialCharTok{+}\NormalTok{ tses }\SpecialCharTok{*}\NormalTok{ meanses }\SpecialCharTok{+}\NormalTok{ tses }\SpecialCharTok{*}\NormalTok{ sector }\SpecialCharTok{+}\NormalTok{ (}\DecValTok{1}\SpecialCharTok{+}\NormalTok{tses }\SpecialCharTok{|}\NormalTok{school), }\AttributeTok{data =}\NormalTok{ andmed)}
\end{Highlighting}
\end{Shaded}

\begin{Shaded}
\begin{Highlighting}[]
\FunctionTok{summary}\NormalTok{(m4)}
\end{Highlighting}
\end{Shaded}

\begin{verbatim}
## Linear mixed model fit by REML. t-tests use Satterthwaite's method [
## lmerModLmerTest]
## Formula: mathach ~ 1 + tses * meanses + tses * sector + (1 + tses | school)
##    Data: andmed
## 
## REML criterion at convergence: 46503.7
## 
## Scaled residuals: 
##      Min       1Q   Median       3Q      Max 
## -3.15921 -0.72319  0.01706  0.75439  2.95822 
## 
## Random effects:
##  Groups   Name        Variance Std.Dev. Corr
##  school   (Intercept)  2.3819  1.5433       
##           tses         0.1014  0.3184   0.39
##  Residual             36.7211  6.0598       
## Number of obs: 7185, groups:  school, 160
## 
## Fixed effects:
##                     Estimate Std. Error       df t value Pr(>|t|)    
## (Intercept)          12.1136     0.1988 159.8921  60.931  < 2e-16 ***
## tses                  2.9388     0.1551 139.3043  18.948  < 2e-16 ***
## meanses               5.3391     0.3693 150.9689  14.457  < 2e-16 ***
## sectorCatholic        1.2167     0.3064 149.5994   3.971 0.000111 ***
## tses:meanses          1.0389     0.2989 160.5528   3.476 0.000656 ***
## tses:sectorCatholic  -1.6426     0.2398 143.3450  -6.850 2.01e-10 ***
## ---
## Signif. codes:  0 '***' 0.001 '**' 0.01 '*' 0.05 '.' 0.1 ' ' 1
## 
## Correlation of Fixed Effects:
##             (Intr) tses   meanss sctrCt tss:mn
## tses         0.080                            
## meanses      0.245  0.020                     
## sectorCthlc -0.697 -0.056 -0.356              
## tses:meanss  0.019  0.282  0.079 -0.028       
## tss:sctrCth -0.056 -0.694 -0.029  0.082 -0.351
\end{verbatim}

\begin{itemize}
\tightlist
\item
  \emph{tses} koefitsient 2.94 on nüüd sotsiaalmajandusliku indeksi keskmine effekt testitulemusele tavakoolis. Katoliiklikus koolis on see efekt 1.64 võrra väiksem. Ehk siis katoliiklikus koolis ei mõjuta pere sotsiaalmajanduslik taust õpitulemusi nii palju kui tavakoolis.\\
\item
  Mida kõrgem on kooli keskmine sotsiaalmajanduslik indeks suurem mõju on pere majanduslikul taustal õpitulemustele. Ühe punktine kasv kooli keskmises tähendab 1.049 punktist mõju kasvu.
\end{itemize}

\hypertarget{part-valimiuuringud}{%
\part{Valimiuuringud}\label{part-valimiuuringud}}

\hypertarget{valimiuuringud}{%
\chapter{Valimiuuringud}\label{valimiuuringud}}

\hypertarget{juhuvalim}{%
\section{Juhuvalim}\label{juhuvalim}}

Järeldava statistika kontseptsioonid eeldavad üldjuhul alati nn juhuvalimit. Juhuslikkus tähedab siikohal, et kõikidel üldkogumi liikmetel on valimisse sattumiseks võrdne tõenäosus. \href{'https://en.wikipedia.org/wiki/Law_of_large_numbers}{Suurte numbrite seadus} (\emph{the law of large numbers}) viitab, et ükskõik millise üldkogumi kohta vjärelduste tegemiseks oleks meil sellest üldkogumist vaja umbes 1000 vaatluselist valimit. Kui me küsiksime Eestis 1000 inimese käest nende sissetuleku suurust, siis 1000 inimese keskmine sissetulek peaks olema küllaltki lähedane terve Eesti keskmise sissetulekuga. Seda aga ainult juhul, kui need 1000 inimest on valitud juhuslikult (kõikidel eestlastel peaks olema võrdne võimalus sattuda nende 1000 sekka). Mis juhtuks, kui me juhusliku valimi asemel võtaksime hoopis igast maakonnast 67 inimest (Eestis on 15 maakonda, seega \(1000 \div 15 \approx 67\)). Tõenäoliselt me alahindaksime keskmist sissetulekut oluliselt. Kõige suuremate sissetulekutega ja samas kõige suurema rahvaarvuga maakond on Harjumaa (598059 inimest 2019 aastal). Kõige väiksemate sissetulekutega ja ka rahvaarvult kõige väiksem maakond on Hiiu maakond (9387 inimest 2019 aastal). Kui me käsitleme mõlemat maakonda võrdselt, siis keskmise arvutamisel võtaksime Hiiu maakonna väikesed sissetulekud arvesse ebaproportsionaalselt suurel määral ja Harjumaa kõrgemad sissetulekud ebaproportsionaalselt v'ikesel määral (Hiiumaa on üleesindatud ja Harjumaa on alaesindatud). Meie maakondade põhine valim ei oleks enam üldkogumi suhtes \textbf{representatiivne}.

\providecommand{\docline}[3]{\noalign{\global\setlength{\arrayrulewidth}{#1}}\arrayrulecolor[HTML]{#2}\cline{#3}}

\setlength{\tabcolsep}{2pt}

\renewcommand*{\arraystretch}{1.5}

\begin{longtable}[c]{|p{1.70in}|p{1.87in}|p{0.94in}}

\caption{Eesti maakondade keskmised brutopalgad 2019 (Statistikaamet)
}\label{tab:unnamed-chunk-1}\\

\hhline{>{\arrayrulecolor[HTML]{666666}\global\arrayrulewidth=2pt}->{\arrayrulecolor[HTML]{666666}\global\arrayrulewidth=2pt}->{\arrayrulecolor[HTML]{666666}\global\arrayrulewidth=2pt}-}

\multicolumn{1}{!{\color[HTML]{000000}\vrule width 0pt}>{\raggedright}p{\dimexpr 1.7in+0\tabcolsep+0\arrayrulewidth}}{\fontsize{11}{11}\selectfont{\textcolor[HTML]{000000}{\global\setmainfont{Arial}{Maakond}}}} & \multicolumn{1}{!{\color[HTML]{000000}\vrule width 0pt}>{\raggedleft}p{\dimexpr 1.87in+0\tabcolsep+0\arrayrulewidth}}{\fontsize{11}{11}\selectfont{\textcolor[HTML]{000000}{\global\setmainfont{Arial}{Keskmine\ brutokuupalk}}}} & \multicolumn{1}{!{\color[HTML]{000000}\vrule width 0pt}>{\raggedleft}p{\dimexpr 0.94in+0\tabcolsep+0\arrayrulewidth}!{\color[HTML]{000000}\vrule width 0pt}}{\fontsize{11}{11}\selectfont{\textcolor[HTML]{000000}{\global\setmainfont{Arial}{Rahvaarv}}}} \\

\hhline{>{\arrayrulecolor[HTML]{666666}\global\arrayrulewidth=2pt}->{\arrayrulecolor[HTML]{666666}\global\arrayrulewidth=2pt}->{\arrayrulecolor[HTML]{666666}\global\arrayrulewidth=2pt}-}

\endfirsthead

\hhline{>{\arrayrulecolor[HTML]{666666}\global\arrayrulewidth=2pt}->{\arrayrulecolor[HTML]{666666}\global\arrayrulewidth=2pt}->{\arrayrulecolor[HTML]{666666}\global\arrayrulewidth=2pt}-}

\multicolumn{1}{!{\color[HTML]{000000}\vrule width 0pt}>{\raggedright}p{\dimexpr 1.7in+0\tabcolsep+0\arrayrulewidth}}{\fontsize{11}{11}\selectfont{\textcolor[HTML]{000000}{\global\setmainfont{Arial}{Maakond}}}} & \multicolumn{1}{!{\color[HTML]{000000}\vrule width 0pt}>{\raggedleft}p{\dimexpr 1.87in+0\tabcolsep+0\arrayrulewidth}}{\fontsize{11}{11}\selectfont{\textcolor[HTML]{000000}{\global\setmainfont{Arial}{Keskmine\ brutokuupalk}}}} & \multicolumn{1}{!{\color[HTML]{000000}\vrule width 0pt}>{\raggedleft}p{\dimexpr 0.94in+0\tabcolsep+0\arrayrulewidth}!{\color[HTML]{000000}\vrule width 0pt}}{\fontsize{11}{11}\selectfont{\textcolor[HTML]{000000}{\global\setmainfont{Arial}{Rahvaarv}}}} \\

\hhline{>{\arrayrulecolor[HTML]{666666}\global\arrayrulewidth=2pt}->{\arrayrulecolor[HTML]{666666}\global\arrayrulewidth=2pt}->{\arrayrulecolor[HTML]{666666}\global\arrayrulewidth=2pt}-}\endhead



\multicolumn{1}{!{\color[HTML]{000000}\vrule width 0pt}>{\raggedright}p{\dimexpr 1.7in+0\tabcolsep+0\arrayrulewidth}}{\fontsize{11}{11}\selectfont{\textcolor[HTML]{000000}{\global\setmainfont{Arial}{Harju\ maakond}}}} & \multicolumn{1}{!{\color[HTML]{000000}\vrule width 0pt}>{\raggedleft}p{\dimexpr 1.87in+0\tabcolsep+0\arrayrulewidth}}{\fontsize{11}{11}\selectfont{\textcolor[HTML]{000000}{\global\setmainfont{Arial}{1\ 531.0}}}} & \multicolumn{1}{!{\color[HTML]{000000}\vrule width 0pt}>{\raggedleft}p{\dimexpr 0.94in+0\tabcolsep+0\arrayrulewidth}!{\color[HTML]{000000}\vrule width 0pt}}{\fontsize{11}{11}\selectfont{\textcolor[HTML]{000000}{\global\setmainfont{Arial}{598\ 059.0}}}} \\





\multicolumn{1}{!{\color[HTML]{000000}\vrule width 0pt}>{\raggedright}p{\dimexpr 1.7in+0\tabcolsep+0\arrayrulewidth}}{\fontsize{11}{11}\selectfont{\textcolor[HTML]{000000}{\global\setmainfont{Arial}{Hiiu\ maakond}}}} & \multicolumn{1}{!{\color[HTML]{000000}\vrule width 0pt}>{\raggedleft}p{\dimexpr 1.87in+0\tabcolsep+0\arrayrulewidth}}{\fontsize{11}{11}\selectfont{\textcolor[HTML]{000000}{\global\setmainfont{Arial}{993.0}}}} & \multicolumn{1}{!{\color[HTML]{000000}\vrule width 0pt}>{\raggedleft}p{\dimexpr 0.94in+0\tabcolsep+0\arrayrulewidth}!{\color[HTML]{000000}\vrule width 0pt}}{\fontsize{11}{11}\selectfont{\textcolor[HTML]{000000}{\global\setmainfont{Arial}{9\ 387.0}}}} \\





\multicolumn{1}{!{\color[HTML]{000000}\vrule width 0pt}>{\raggedright}p{\dimexpr 1.7in+0\tabcolsep+0\arrayrulewidth}}{\fontsize{11}{11}\selectfont{\textcolor[HTML]{000000}{\global\setmainfont{Arial}{Ida-Viru\ maakond}}}} & \multicolumn{1}{!{\color[HTML]{000000}\vrule width 0pt}>{\raggedleft}p{\dimexpr 1.87in+0\tabcolsep+0\arrayrulewidth}}{\fontsize{11}{11}\selectfont{\textcolor[HTML]{000000}{\global\setmainfont{Arial}{1\ 147.0}}}} & \multicolumn{1}{!{\color[HTML]{000000}\vrule width 0pt}>{\raggedleft}p{\dimexpr 0.94in+0\tabcolsep+0\arrayrulewidth}!{\color[HTML]{000000}\vrule width 0pt}}{\fontsize{11}{11}\selectfont{\textcolor[HTML]{000000}{\global\setmainfont{Arial}{136\ 240.0}}}} \\





\multicolumn{1}{!{\color[HTML]{000000}\vrule width 0pt}>{\raggedright}p{\dimexpr 1.7in+0\tabcolsep+0\arrayrulewidth}}{\fontsize{11}{11}\selectfont{\textcolor[HTML]{000000}{\global\setmainfont{Arial}{Jõgeva\ maakond}}}} & \multicolumn{1}{!{\color[HTML]{000000}\vrule width 0pt}>{\raggedleft}p{\dimexpr 1.87in+0\tabcolsep+0\arrayrulewidth}}{\fontsize{11}{11}\selectfont{\textcolor[HTML]{000000}{\global\setmainfont{Arial}{1\ 066.0}}}} & \multicolumn{1}{!{\color[HTML]{000000}\vrule width 0pt}>{\raggedleft}p{\dimexpr 0.94in+0\tabcolsep+0\arrayrulewidth}!{\color[HTML]{000000}\vrule width 0pt}}{\fontsize{11}{11}\selectfont{\textcolor[HTML]{000000}{\global\setmainfont{Arial}{28\ 734.0}}}} \\





\multicolumn{1}{!{\color[HTML]{000000}\vrule width 0pt}>{\raggedright}p{\dimexpr 1.7in+0\tabcolsep+0\arrayrulewidth}}{\fontsize{11}{11}\selectfont{\textcolor[HTML]{000000}{\global\setmainfont{Arial}{Järva\ maakond}}}} & \multicolumn{1}{!{\color[HTML]{000000}\vrule width 0pt}>{\raggedleft}p{\dimexpr 1.87in+0\tabcolsep+0\arrayrulewidth}}{\fontsize{11}{11}\selectfont{\textcolor[HTML]{000000}{\global\setmainfont{Arial}{1\ 192.0}}}} & \multicolumn{1}{!{\color[HTML]{000000}\vrule width 0pt}>{\raggedleft}p{\dimexpr 0.94in+0\tabcolsep+0\arrayrulewidth}!{\color[HTML]{000000}\vrule width 0pt}}{\fontsize{11}{11}\selectfont{\textcolor[HTML]{000000}{\global\setmainfont{Arial}{30\ 286.0}}}} \\





\multicolumn{1}{!{\color[HTML]{000000}\vrule width 0pt}>{\raggedright}p{\dimexpr 1.7in+0\tabcolsep+0\arrayrulewidth}}{\fontsize{11}{11}\selectfont{\textcolor[HTML]{000000}{\global\setmainfont{Arial}{Lääne\ maakond}}}} & \multicolumn{1}{!{\color[HTML]{000000}\vrule width 0pt}>{\raggedleft}p{\dimexpr 1.87in+0\tabcolsep+0\arrayrulewidth}}{\fontsize{11}{11}\selectfont{\textcolor[HTML]{000000}{\global\setmainfont{Arial}{1\ 274.0}}}} & \multicolumn{1}{!{\color[HTML]{000000}\vrule width 0pt}>{\raggedleft}p{\dimexpr 0.94in+0\tabcolsep+0\arrayrulewidth}!{\color[HTML]{000000}\vrule width 0pt}}{\fontsize{11}{11}\selectfont{\textcolor[HTML]{000000}{\global\setmainfont{Arial}{20\ 507.0}}}} \\





\multicolumn{1}{!{\color[HTML]{000000}\vrule width 0pt}>{\raggedright}p{\dimexpr 1.7in+0\tabcolsep+0\arrayrulewidth}}{\fontsize{11}{11}\selectfont{\textcolor[HTML]{000000}{\global\setmainfont{Arial}{Lääne-Viru\ maakond}}}} & \multicolumn{1}{!{\color[HTML]{000000}\vrule width 0pt}>{\raggedleft}p{\dimexpr 1.87in+0\tabcolsep+0\arrayrulewidth}}{\fontsize{11}{11}\selectfont{\textcolor[HTML]{000000}{\global\setmainfont{Arial}{1\ 095.0}}}} & \multicolumn{1}{!{\color[HTML]{000000}\vrule width 0pt}>{\raggedleft}p{\dimexpr 0.94in+0\tabcolsep+0\arrayrulewidth}!{\color[HTML]{000000}\vrule width 0pt}}{\fontsize{11}{11}\selectfont{\textcolor[HTML]{000000}{\global\setmainfont{Arial}{59\ 325.0}}}} \\





\multicolumn{1}{!{\color[HTML]{000000}\vrule width 0pt}>{\raggedright}p{\dimexpr 1.7in+0\tabcolsep+0\arrayrulewidth}}{\fontsize{11}{11}\selectfont{\textcolor[HTML]{000000}{\global\setmainfont{Arial}{Põlva\ maakond}}}} & \multicolumn{1}{!{\color[HTML]{000000}\vrule width 0pt}>{\raggedleft}p{\dimexpr 1.87in+0\tabcolsep+0\arrayrulewidth}}{\fontsize{11}{11}\selectfont{\textcolor[HTML]{000000}{\global\setmainfont{Arial}{1\ 140.0}}}} & \multicolumn{1}{!{\color[HTML]{000000}\vrule width 0pt}>{\raggedleft}p{\dimexpr 0.94in+0\tabcolsep+0\arrayrulewidth}!{\color[HTML]{000000}\vrule width 0pt}}{\fontsize{11}{11}\selectfont{\textcolor[HTML]{000000}{\global\setmainfont{Arial}{25\ 006.0}}}} \\





\multicolumn{1}{!{\color[HTML]{000000}\vrule width 0pt}>{\raggedright}p{\dimexpr 1.7in+0\tabcolsep+0\arrayrulewidth}}{\fontsize{11}{11}\selectfont{\textcolor[HTML]{000000}{\global\setmainfont{Arial}{Pärnu\ maakond}}}} & \multicolumn{1}{!{\color[HTML]{000000}\vrule width 0pt}>{\raggedleft}p{\dimexpr 1.87in+0\tabcolsep+0\arrayrulewidth}}{\fontsize{11}{11}\selectfont{\textcolor[HTML]{000000}{\global\setmainfont{Arial}{1\ 172.0}}}} & \multicolumn{1}{!{\color[HTML]{000000}\vrule width 0pt}>{\raggedleft}p{\dimexpr 0.94in+0\tabcolsep+0\arrayrulewidth}!{\color[HTML]{000000}\vrule width 0pt}}{\fontsize{11}{11}\selectfont{\textcolor[HTML]{000000}{\global\setmainfont{Arial}{85\ 938.0}}}} \\





\multicolumn{1}{!{\color[HTML]{000000}\vrule width 0pt}>{\raggedright}p{\dimexpr 1.7in+0\tabcolsep+0\arrayrulewidth}}{\fontsize{11}{11}\selectfont{\textcolor[HTML]{000000}{\global\setmainfont{Arial}{Rapla\ maakond}}}} & \multicolumn{1}{!{\color[HTML]{000000}\vrule width 0pt}>{\raggedleft}p{\dimexpr 1.87in+0\tabcolsep+0\arrayrulewidth}}{\fontsize{11}{11}\selectfont{\textcolor[HTML]{000000}{\global\setmainfont{Arial}{1\ 200.0}}}} & \multicolumn{1}{!{\color[HTML]{000000}\vrule width 0pt}>{\raggedleft}p{\dimexpr 0.94in+0\tabcolsep+0\arrayrulewidth}!{\color[HTML]{000000}\vrule width 0pt}}{\fontsize{11}{11}\selectfont{\textcolor[HTML]{000000}{\global\setmainfont{Arial}{33\ 311.0}}}} \\





\multicolumn{1}{!{\color[HTML]{000000}\vrule width 0pt}>{\raggedright}p{\dimexpr 1.7in+0\tabcolsep+0\arrayrulewidth}}{\fontsize{11}{11}\selectfont{\textcolor[HTML]{000000}{\global\setmainfont{Arial}{Saare\ maakond}}}} & \multicolumn{1}{!{\color[HTML]{000000}\vrule width 0pt}>{\raggedleft}p{\dimexpr 1.87in+0\tabcolsep+0\arrayrulewidth}}{\fontsize{11}{11}\selectfont{\textcolor[HTML]{000000}{\global\setmainfont{Arial}{1\ 082.0}}}} & \multicolumn{1}{!{\color[HTML]{000000}\vrule width 0pt}>{\raggedleft}p{\dimexpr 0.94in+0\tabcolsep+0\arrayrulewidth}!{\color[HTML]{000000}\vrule width 0pt}}{\fontsize{11}{11}\selectfont{\textcolor[HTML]{000000}{\global\setmainfont{Arial}{33\ 108.0}}}} \\





\multicolumn{1}{!{\color[HTML]{000000}\vrule width 0pt}>{\raggedright}p{\dimexpr 1.7in+0\tabcolsep+0\arrayrulewidth}}{\fontsize{11}{11}\selectfont{\textcolor[HTML]{000000}{\global\setmainfont{Arial}{Tartu\ maakond}}}} & \multicolumn{1}{!{\color[HTML]{000000}\vrule width 0pt}>{\raggedleft}p{\dimexpr 1.87in+0\tabcolsep+0\arrayrulewidth}}{\fontsize{11}{11}\selectfont{\textcolor[HTML]{000000}{\global\setmainfont{Arial}{1\ 426.0}}}} & \multicolumn{1}{!{\color[HTML]{000000}\vrule width 0pt}>{\raggedleft}p{\dimexpr 0.94in+0\tabcolsep+0\arrayrulewidth}!{\color[HTML]{000000}\vrule width 0pt}}{\fontsize{11}{11}\selectfont{\textcolor[HTML]{000000}{\global\setmainfont{Arial}{152\ 976.0}}}} \\





\multicolumn{1}{!{\color[HTML]{000000}\vrule width 0pt}>{\raggedright}p{\dimexpr 1.7in+0\tabcolsep+0\arrayrulewidth}}{\fontsize{11}{11}\selectfont{\textcolor[HTML]{000000}{\global\setmainfont{Arial}{Valga\ maakond}}}} & \multicolumn{1}{!{\color[HTML]{000000}\vrule width 0pt}>{\raggedleft}p{\dimexpr 1.87in+0\tabcolsep+0\arrayrulewidth}}{\fontsize{11}{11}\selectfont{\textcolor[HTML]{000000}{\global\setmainfont{Arial}{1\ 058.0}}}} & \multicolumn{1}{!{\color[HTML]{000000}\vrule width 0pt}>{\raggedleft}p{\dimexpr 0.94in+0\tabcolsep+0\arrayrulewidth}!{\color[HTML]{000000}\vrule width 0pt}}{\fontsize{11}{11}\selectfont{\textcolor[HTML]{000000}{\global\setmainfont{Arial}{28\ 370.0}}}} \\





\multicolumn{1}{!{\color[HTML]{000000}\vrule width 0pt}>{\raggedright}p{\dimexpr 1.7in+0\tabcolsep+0\arrayrulewidth}}{\fontsize{11}{11}\selectfont{\textcolor[HTML]{000000}{\global\setmainfont{Arial}{Viljandi\ maakond}}}} & \multicolumn{1}{!{\color[HTML]{000000}\vrule width 0pt}>{\raggedleft}p{\dimexpr 1.87in+0\tabcolsep+0\arrayrulewidth}}{\fontsize{11}{11}\selectfont{\textcolor[HTML]{000000}{\global\setmainfont{Arial}{1\ 201.0}}}} & \multicolumn{1}{!{\color[HTML]{000000}\vrule width 0pt}>{\raggedleft}p{\dimexpr 0.94in+0\tabcolsep+0\arrayrulewidth}!{\color[HTML]{000000}\vrule width 0pt}}{\fontsize{11}{11}\selectfont{\textcolor[HTML]{000000}{\global\setmainfont{Arial}{46\ 371.0}}}} \\





\multicolumn{1}{!{\color[HTML]{000000}\vrule width 0pt}>{\raggedright}p{\dimexpr 1.7in+0\tabcolsep+0\arrayrulewidth}}{\fontsize{11}{11}\selectfont{\textcolor[HTML]{000000}{\global\setmainfont{Arial}{Võru\ maakond}}}} & \multicolumn{1}{!{\color[HTML]{000000}\vrule width 0pt}>{\raggedleft}p{\dimexpr 1.87in+0\tabcolsep+0\arrayrulewidth}}{\fontsize{11}{11}\selectfont{\textcolor[HTML]{000000}{\global\setmainfont{Arial}{1\ 113.0}}}} & \multicolumn{1}{!{\color[HTML]{000000}\vrule width 0pt}>{\raggedleft}p{\dimexpr 0.94in+0\tabcolsep+0\arrayrulewidth}!{\color[HTML]{000000}\vrule width 0pt}}{\fontsize{11}{11}\selectfont{\textcolor[HTML]{000000}{\global\setmainfont{Arial}{35\ 782.0}}}} \\

\hhline{>{\arrayrulecolor[HTML]{666666}\global\arrayrulewidth=2pt}->{\arrayrulecolor[HTML]{666666}\global\arrayrulewidth=2pt}->{\arrayrulecolor[HTML]{666666}\global\arrayrulewidth=2pt}-}



\end{longtable}

Kui me arvutaksime Eesti keskmise sissetuleku lähtuvalt maakondade keskmistest sissetulekutest, siis saaksime vastuseks:

\[\frac{1531 + 993 + ... + 1113}{15} = 1179\]

Statistikaameti järgi oli Eesti keskmine sissetulek 2019 aastal 1407 eurot. Seega tõepoolest, hinnang alahindab oluliselt tegelikku keskmist sissetulekut.

\hypertarget{tuxf5enuxe4osuslik-valim-probability-sample}{%
\section{Tõenäosuslik valim (probability sample)}\label{tuxf5enuxe4osuslik-valim-probability-sample}}

Võttes oma valimisse igast maakonnast 67 inimest, ei ole meie valim küll enam juhuvalim, kuid niikaua kuni me teame kõikide maakondade rahvaarvu, on see siiski tõenäosuslik valim (antud juhul stratifitseeritud valim). Seda seetõttu, et kõikidel Eesti inimestel on ikkagi võimalus sellesse valimisse sattuda. Harjumaa inimestel küll väiksem ja Hiiumaa inimestel suurem, kuid mingi võimalus on kõigil. Kui me nüüd seda erinevat valimisse sattmise tõenäosust teame, siis on meil võimalik mitte-representatiivne valim representatiivseks muuta. Selleks tuleb meil Hiiumaa elanikud väiksemaks ja Harjumaa elanikud suuremaks kaaluda. Seejärel saame (mõningate mööndustega) jälle kasutada järeldava statistika meetodeid ning teha nende alusel korrektseid järeldusi üldkogumi kohta.

\hypertarget{kaalud}{%
\section{Kaalud}\label{kaalud}}

\hypertarget{valimi-kaalud}{%
\subsection{Valimi kaalud}\label{valimi-kaalud}}

Kõikide Harjumaa elanike jaoks oleks tõenäosus meie valimisse sattuda \(67 \div 598059 \approx 0.0001120\). Kõigi Hiiumaa elanike jaoks oleks see tõenäosus \(67 \div 9387 \approx 0.0071375\). Harjumaa valimis esindaks iga valimisse sattunu \(1 \div 0.00011 \approx 8926\) inimest ja Hiiumaal \(1 \div 0.0071 \approx 140\) inimest. Seega kui me teame iga inimese valimisse sattumise tõenäosust \(\pi\), siis saame välja arvutada ka selle inimese \textbf{valimi kaalud} \(\frac{1}{\pi}\) (nimetatakse ka disainikaaludeks ehk \emph{design weights} või ka \emph{base weights}). Valimikaalude suurus on pöördvõrdeline valimisse sattumise tõenäosusega:

\[w_i = \frac{1}{\pi_i}\]

kusjuures üldkogumi suurs on võrdne kaalude summaga:

\[P = \sum^{n}_{i = 1} w_i\]
ja mingi tunnuse \(y\) üldkogumi kogusumma on:

\[T_y = \sum^{n}_{i = 1}w_i y_i\]
Kui me rakendame neid kaale erinevates maakodades elavatele inimestele, siis saame alaesindatud maakonna (Harjumaa) valimi üles kaaluda \(67 \times 8926 = 598042\) ja üleesindatud maakona (Hiiumaa) valimi alla kaaluda \(67 \times 140 = 9380\) (erinevused algsetest maakondade suurustest on tingitud ümardamisest, kui ei oleks seda teinud, oleksid tulemused identsed).

\providecommand{\docline}[3]{\noalign{\global\setlength{\arrayrulewidth}{#1}}\arrayrulecolor[HTML]{#2}\cline{#3}}

\setlength{\tabcolsep}{2pt}

\renewcommand*{\arraystretch}{1.5}

\begin{longtable}[c]{|p{1.70in}|p{0.84in}|p{0.46in}|p{1.26in}|p{1.05in}}

\caption{Eesti maakondade kaalud rahvaarvu alusel (Statistikaamet)
}\label{tab:unnamed-chunk-2}\\

\hhline{>{\arrayrulecolor[HTML]{666666}\global\arrayrulewidth=2pt}->{\arrayrulecolor[HTML]{666666}\global\arrayrulewidth=2pt}->{\arrayrulecolor[HTML]{666666}\global\arrayrulewidth=2pt}->{\arrayrulecolor[HTML]{666666}\global\arrayrulewidth=2pt}->{\arrayrulecolor[HTML]{666666}\global\arrayrulewidth=2pt}-}

\multicolumn{1}{!{\color[HTML]{000000}\vrule width 0pt}>{\raggedright}p{\dimexpr 1.7in+0\tabcolsep+0\arrayrulewidth}}{\fontsize{11}{11}\selectfont{\textcolor[HTML]{000000}{\global\setmainfont{Arial}{Maakond}}}} & \multicolumn{1}{!{\color[HTML]{000000}\vrule width 0pt}>{\raggedleft}p{\dimexpr 0.84in+0\tabcolsep+0\arrayrulewidth}}{\fontsize{11}{11}\selectfont{\textcolor[HTML]{000000}{\global\setmainfont{Arial}{N}}}} & \multicolumn{1}{!{\color[HTML]{000000}\vrule width 0pt}>{\raggedleft}p{\dimexpr 0.46in+0\tabcolsep+0\arrayrulewidth}}{\fontsize{11}{11}\selectfont{\textcolor[HTML]{000000}{\global\setmainfont{Arial}{n}}}} & \multicolumn{1}{!{\color[HTML]{000000}\vrule width 0pt}>{\raggedleft}p{\dimexpr 1.26in+0\tabcolsep+0\arrayrulewidth}}{\fontsize{11}{11}\selectfont{\textcolor[HTML]{000000}{\global\setmainfont{Arial}{Tõenäosus}}}} & \multicolumn{1}{!{\color[HTML]{000000}\vrule width 0pt}>{\raggedleft}p{\dimexpr 1.05in+0\tabcolsep+0\arrayrulewidth}!{\color[HTML]{000000}\vrule width 0pt}}{\fontsize{11}{11}\selectfont{\textcolor[HTML]{000000}{\global\setmainfont{Arial}{Kaalud}}}} \\

\hhline{>{\arrayrulecolor[HTML]{666666}\global\arrayrulewidth=2pt}->{\arrayrulecolor[HTML]{666666}\global\arrayrulewidth=2pt}->{\arrayrulecolor[HTML]{666666}\global\arrayrulewidth=2pt}->{\arrayrulecolor[HTML]{666666}\global\arrayrulewidth=2pt}->{\arrayrulecolor[HTML]{666666}\global\arrayrulewidth=2pt}-}

\endfirsthead

\hhline{>{\arrayrulecolor[HTML]{666666}\global\arrayrulewidth=2pt}->{\arrayrulecolor[HTML]{666666}\global\arrayrulewidth=2pt}->{\arrayrulecolor[HTML]{666666}\global\arrayrulewidth=2pt}->{\arrayrulecolor[HTML]{666666}\global\arrayrulewidth=2pt}->{\arrayrulecolor[HTML]{666666}\global\arrayrulewidth=2pt}-}

\multicolumn{1}{!{\color[HTML]{000000}\vrule width 0pt}>{\raggedright}p{\dimexpr 1.7in+0\tabcolsep+0\arrayrulewidth}}{\fontsize{11}{11}\selectfont{\textcolor[HTML]{000000}{\global\setmainfont{Arial}{Maakond}}}} & \multicolumn{1}{!{\color[HTML]{000000}\vrule width 0pt}>{\raggedleft}p{\dimexpr 0.84in+0\tabcolsep+0\arrayrulewidth}}{\fontsize{11}{11}\selectfont{\textcolor[HTML]{000000}{\global\setmainfont{Arial}{N}}}} & \multicolumn{1}{!{\color[HTML]{000000}\vrule width 0pt}>{\raggedleft}p{\dimexpr 0.46in+0\tabcolsep+0\arrayrulewidth}}{\fontsize{11}{11}\selectfont{\textcolor[HTML]{000000}{\global\setmainfont{Arial}{n}}}} & \multicolumn{1}{!{\color[HTML]{000000}\vrule width 0pt}>{\raggedleft}p{\dimexpr 1.26in+0\tabcolsep+0\arrayrulewidth}}{\fontsize{11}{11}\selectfont{\textcolor[HTML]{000000}{\global\setmainfont{Arial}{Tõenäosus}}}} & \multicolumn{1}{!{\color[HTML]{000000}\vrule width 0pt}>{\raggedleft}p{\dimexpr 1.05in+0\tabcolsep+0\arrayrulewidth}!{\color[HTML]{000000}\vrule width 0pt}}{\fontsize{11}{11}\selectfont{\textcolor[HTML]{000000}{\global\setmainfont{Arial}{Kaalud}}}} \\

\hhline{>{\arrayrulecolor[HTML]{666666}\global\arrayrulewidth=2pt}->{\arrayrulecolor[HTML]{666666}\global\arrayrulewidth=2pt}->{\arrayrulecolor[HTML]{666666}\global\arrayrulewidth=2pt}->{\arrayrulecolor[HTML]{666666}\global\arrayrulewidth=2pt}->{\arrayrulecolor[HTML]{666666}\global\arrayrulewidth=2pt}-}\endhead



\multicolumn{1}{!{\color[HTML]{000000}\vrule width 0pt}>{\raggedright}p{\dimexpr 1.7in+0\tabcolsep+0\arrayrulewidth}}{\fontsize{11}{11}\selectfont{\textcolor[HTML]{000000}{\global\setmainfont{Arial}{Harju\ maakond}}}} & \multicolumn{1}{!{\color[HTML]{000000}\vrule width 0pt}>{\raggedleft}p{\dimexpr 0.84in+0\tabcolsep+0\arrayrulewidth}}{\fontsize{11}{11}\selectfont{\textcolor[HTML]{000000}{\global\setmainfont{Arial}{598\ 059.0}}}} & \multicolumn{1}{!{\color[HTML]{000000}\vrule width 0pt}>{\raggedleft}p{\dimexpr 0.46in+0\tabcolsep+0\arrayrulewidth}}{\fontsize{11}{11}\selectfont{\textcolor[HTML]{000000}{\global\setmainfont{Arial}{67.0}}}} & \multicolumn{1}{!{\color[HTML]{000000}\vrule width 0pt}>{\raggedleft}p{\dimexpr 1.26in+0\tabcolsep+0\arrayrulewidth}}{\fontsize{11}{11}\selectfont{\textcolor[HTML]{000000}{\global\setmainfont{Arial}{0.0}}}} & \multicolumn{1}{!{\color[HTML]{000000}\vrule width 0pt}>{\raggedleft}p{\dimexpr 1.05in+0\tabcolsep+0\arrayrulewidth}!{\color[HTML]{000000}\vrule width 0pt}}{\fontsize{11}{11}\selectfont{\textcolor[HTML]{000000}{\global\setmainfont{Arial}{8\ 926.3}}}} \\





\multicolumn{1}{!{\color[HTML]{000000}\vrule width 0pt}>{\raggedright}p{\dimexpr 1.7in+0\tabcolsep+0\arrayrulewidth}}{\fontsize{11}{11}\selectfont{\textcolor[HTML]{000000}{\global\setmainfont{Arial}{Hiiu\ maakond}}}} & \multicolumn{1}{!{\color[HTML]{000000}\vrule width 0pt}>{\raggedleft}p{\dimexpr 0.84in+0\tabcolsep+0\arrayrulewidth}}{\fontsize{11}{11}\selectfont{\textcolor[HTML]{000000}{\global\setmainfont{Arial}{9\ 387.0}}}} & \multicolumn{1}{!{\color[HTML]{000000}\vrule width 0pt}>{\raggedleft}p{\dimexpr 0.46in+0\tabcolsep+0\arrayrulewidth}}{\fontsize{11}{11}\selectfont{\textcolor[HTML]{000000}{\global\setmainfont{Arial}{67.0}}}} & \multicolumn{1}{!{\color[HTML]{000000}\vrule width 0pt}>{\raggedleft}p{\dimexpr 1.26in+0\tabcolsep+0\arrayrulewidth}}{\fontsize{11}{11}\selectfont{\textcolor[HTML]{000000}{\global\setmainfont{Arial}{0.0}}}} & \multicolumn{1}{!{\color[HTML]{000000}\vrule width 0pt}>{\raggedleft}p{\dimexpr 1.05in+0\tabcolsep+0\arrayrulewidth}!{\color[HTML]{000000}\vrule width 0pt}}{\fontsize{11}{11}\selectfont{\textcolor[HTML]{000000}{\global\setmainfont{Arial}{140.1}}}} \\





\multicolumn{1}{!{\color[HTML]{000000}\vrule width 0pt}>{\raggedright}p{\dimexpr 1.7in+0\tabcolsep+0\arrayrulewidth}}{\fontsize{11}{11}\selectfont{\textcolor[HTML]{000000}{\global\setmainfont{Arial}{Ida-Viru\ maakond}}}} & \multicolumn{1}{!{\color[HTML]{000000}\vrule width 0pt}>{\raggedleft}p{\dimexpr 0.84in+0\tabcolsep+0\arrayrulewidth}}{\fontsize{11}{11}\selectfont{\textcolor[HTML]{000000}{\global\setmainfont{Arial}{136\ 240.0}}}} & \multicolumn{1}{!{\color[HTML]{000000}\vrule width 0pt}>{\raggedleft}p{\dimexpr 0.46in+0\tabcolsep+0\arrayrulewidth}}{\fontsize{11}{11}\selectfont{\textcolor[HTML]{000000}{\global\setmainfont{Arial}{67.0}}}} & \multicolumn{1}{!{\color[HTML]{000000}\vrule width 0pt}>{\raggedleft}p{\dimexpr 1.26in+0\tabcolsep+0\arrayrulewidth}}{\fontsize{11}{11}\selectfont{\textcolor[HTML]{000000}{\global\setmainfont{Arial}{0.0}}}} & \multicolumn{1}{!{\color[HTML]{000000}\vrule width 0pt}>{\raggedleft}p{\dimexpr 1.05in+0\tabcolsep+0\arrayrulewidth}!{\color[HTML]{000000}\vrule width 0pt}}{\fontsize{11}{11}\selectfont{\textcolor[HTML]{000000}{\global\setmainfont{Arial}{2\ 033.4}}}} \\





\multicolumn{1}{!{\color[HTML]{000000}\vrule width 0pt}>{\raggedright}p{\dimexpr 1.7in+0\tabcolsep+0\arrayrulewidth}}{\fontsize{11}{11}\selectfont{\textcolor[HTML]{000000}{\global\setmainfont{Arial}{Jõgeva\ maakond}}}} & \multicolumn{1}{!{\color[HTML]{000000}\vrule width 0pt}>{\raggedleft}p{\dimexpr 0.84in+0\tabcolsep+0\arrayrulewidth}}{\fontsize{11}{11}\selectfont{\textcolor[HTML]{000000}{\global\setmainfont{Arial}{28\ 734.0}}}} & \multicolumn{1}{!{\color[HTML]{000000}\vrule width 0pt}>{\raggedleft}p{\dimexpr 0.46in+0\tabcolsep+0\arrayrulewidth}}{\fontsize{11}{11}\selectfont{\textcolor[HTML]{000000}{\global\setmainfont{Arial}{67.0}}}} & \multicolumn{1}{!{\color[HTML]{000000}\vrule width 0pt}>{\raggedleft}p{\dimexpr 1.26in+0\tabcolsep+0\arrayrulewidth}}{\fontsize{11}{11}\selectfont{\textcolor[HTML]{000000}{\global\setmainfont{Arial}{0.0}}}} & \multicolumn{1}{!{\color[HTML]{000000}\vrule width 0pt}>{\raggedleft}p{\dimexpr 1.05in+0\tabcolsep+0\arrayrulewidth}!{\color[HTML]{000000}\vrule width 0pt}}{\fontsize{11}{11}\selectfont{\textcolor[HTML]{000000}{\global\setmainfont{Arial}{428.9}}}} \\





\multicolumn{1}{!{\color[HTML]{000000}\vrule width 0pt}>{\raggedright}p{\dimexpr 1.7in+0\tabcolsep+0\arrayrulewidth}}{\fontsize{11}{11}\selectfont{\textcolor[HTML]{000000}{\global\setmainfont{Arial}{Järva\ maakond}}}} & \multicolumn{1}{!{\color[HTML]{000000}\vrule width 0pt}>{\raggedleft}p{\dimexpr 0.84in+0\tabcolsep+0\arrayrulewidth}}{\fontsize{11}{11}\selectfont{\textcolor[HTML]{000000}{\global\setmainfont{Arial}{30\ 286.0}}}} & \multicolumn{1}{!{\color[HTML]{000000}\vrule width 0pt}>{\raggedleft}p{\dimexpr 0.46in+0\tabcolsep+0\arrayrulewidth}}{\fontsize{11}{11}\selectfont{\textcolor[HTML]{000000}{\global\setmainfont{Arial}{67.0}}}} & \multicolumn{1}{!{\color[HTML]{000000}\vrule width 0pt}>{\raggedleft}p{\dimexpr 1.26in+0\tabcolsep+0\arrayrulewidth}}{\fontsize{11}{11}\selectfont{\textcolor[HTML]{000000}{\global\setmainfont{Arial}{0.0}}}} & \multicolumn{1}{!{\color[HTML]{000000}\vrule width 0pt}>{\raggedleft}p{\dimexpr 1.05in+0\tabcolsep+0\arrayrulewidth}!{\color[HTML]{000000}\vrule width 0pt}}{\fontsize{11}{11}\selectfont{\textcolor[HTML]{000000}{\global\setmainfont{Arial}{452.0}}}} \\





\multicolumn{1}{!{\color[HTML]{000000}\vrule width 0pt}>{\raggedright}p{\dimexpr 1.7in+0\tabcolsep+0\arrayrulewidth}}{\fontsize{11}{11}\selectfont{\textcolor[HTML]{000000}{\global\setmainfont{Arial}{Lääne\ maakond}}}} & \multicolumn{1}{!{\color[HTML]{000000}\vrule width 0pt}>{\raggedleft}p{\dimexpr 0.84in+0\tabcolsep+0\arrayrulewidth}}{\fontsize{11}{11}\selectfont{\textcolor[HTML]{000000}{\global\setmainfont{Arial}{20\ 507.0}}}} & \multicolumn{1}{!{\color[HTML]{000000}\vrule width 0pt}>{\raggedleft}p{\dimexpr 0.46in+0\tabcolsep+0\arrayrulewidth}}{\fontsize{11}{11}\selectfont{\textcolor[HTML]{000000}{\global\setmainfont{Arial}{67.0}}}} & \multicolumn{1}{!{\color[HTML]{000000}\vrule width 0pt}>{\raggedleft}p{\dimexpr 1.26in+0\tabcolsep+0\arrayrulewidth}}{\fontsize{11}{11}\selectfont{\textcolor[HTML]{000000}{\global\setmainfont{Arial}{0.0}}}} & \multicolumn{1}{!{\color[HTML]{000000}\vrule width 0pt}>{\raggedleft}p{\dimexpr 1.05in+0\tabcolsep+0\arrayrulewidth}!{\color[HTML]{000000}\vrule width 0pt}}{\fontsize{11}{11}\selectfont{\textcolor[HTML]{000000}{\global\setmainfont{Arial}{306.1}}}} \\





\multicolumn{1}{!{\color[HTML]{000000}\vrule width 0pt}>{\raggedright}p{\dimexpr 1.7in+0\tabcolsep+0\arrayrulewidth}}{\fontsize{11}{11}\selectfont{\textcolor[HTML]{000000}{\global\setmainfont{Arial}{Lääne-Viru\ maakond}}}} & \multicolumn{1}{!{\color[HTML]{000000}\vrule width 0pt}>{\raggedleft}p{\dimexpr 0.84in+0\tabcolsep+0\arrayrulewidth}}{\fontsize{11}{11}\selectfont{\textcolor[HTML]{000000}{\global\setmainfont{Arial}{59\ 325.0}}}} & \multicolumn{1}{!{\color[HTML]{000000}\vrule width 0pt}>{\raggedleft}p{\dimexpr 0.46in+0\tabcolsep+0\arrayrulewidth}}{\fontsize{11}{11}\selectfont{\textcolor[HTML]{000000}{\global\setmainfont{Arial}{67.0}}}} & \multicolumn{1}{!{\color[HTML]{000000}\vrule width 0pt}>{\raggedleft}p{\dimexpr 1.26in+0\tabcolsep+0\arrayrulewidth}}{\fontsize{11}{11}\selectfont{\textcolor[HTML]{000000}{\global\setmainfont{Arial}{0.0}}}} & \multicolumn{1}{!{\color[HTML]{000000}\vrule width 0pt}>{\raggedleft}p{\dimexpr 1.05in+0\tabcolsep+0\arrayrulewidth}!{\color[HTML]{000000}\vrule width 0pt}}{\fontsize{11}{11}\selectfont{\textcolor[HTML]{000000}{\global\setmainfont{Arial}{885.4}}}} \\





\multicolumn{1}{!{\color[HTML]{000000}\vrule width 0pt}>{\raggedright}p{\dimexpr 1.7in+0\tabcolsep+0\arrayrulewidth}}{\fontsize{11}{11}\selectfont{\textcolor[HTML]{000000}{\global\setmainfont{Arial}{Põlva\ maakond}}}} & \multicolumn{1}{!{\color[HTML]{000000}\vrule width 0pt}>{\raggedleft}p{\dimexpr 0.84in+0\tabcolsep+0\arrayrulewidth}}{\fontsize{11}{11}\selectfont{\textcolor[HTML]{000000}{\global\setmainfont{Arial}{25\ 006.0}}}} & \multicolumn{1}{!{\color[HTML]{000000}\vrule width 0pt}>{\raggedleft}p{\dimexpr 0.46in+0\tabcolsep+0\arrayrulewidth}}{\fontsize{11}{11}\selectfont{\textcolor[HTML]{000000}{\global\setmainfont{Arial}{67.0}}}} & \multicolumn{1}{!{\color[HTML]{000000}\vrule width 0pt}>{\raggedleft}p{\dimexpr 1.26in+0\tabcolsep+0\arrayrulewidth}}{\fontsize{11}{11}\selectfont{\textcolor[HTML]{000000}{\global\setmainfont{Arial}{0.0}}}} & \multicolumn{1}{!{\color[HTML]{000000}\vrule width 0pt}>{\raggedleft}p{\dimexpr 1.05in+0\tabcolsep+0\arrayrulewidth}!{\color[HTML]{000000}\vrule width 0pt}}{\fontsize{11}{11}\selectfont{\textcolor[HTML]{000000}{\global\setmainfont{Arial}{373.2}}}} \\





\multicolumn{1}{!{\color[HTML]{000000}\vrule width 0pt}>{\raggedright}p{\dimexpr 1.7in+0\tabcolsep+0\arrayrulewidth}}{\fontsize{11}{11}\selectfont{\textcolor[HTML]{000000}{\global\setmainfont{Arial}{Pärnu\ maakond}}}} & \multicolumn{1}{!{\color[HTML]{000000}\vrule width 0pt}>{\raggedleft}p{\dimexpr 0.84in+0\tabcolsep+0\arrayrulewidth}}{\fontsize{11}{11}\selectfont{\textcolor[HTML]{000000}{\global\setmainfont{Arial}{85\ 938.0}}}} & \multicolumn{1}{!{\color[HTML]{000000}\vrule width 0pt}>{\raggedleft}p{\dimexpr 0.46in+0\tabcolsep+0\arrayrulewidth}}{\fontsize{11}{11}\selectfont{\textcolor[HTML]{000000}{\global\setmainfont{Arial}{67.0}}}} & \multicolumn{1}{!{\color[HTML]{000000}\vrule width 0pt}>{\raggedleft}p{\dimexpr 1.26in+0\tabcolsep+0\arrayrulewidth}}{\fontsize{11}{11}\selectfont{\textcolor[HTML]{000000}{\global\setmainfont{Arial}{0.0}}}} & \multicolumn{1}{!{\color[HTML]{000000}\vrule width 0pt}>{\raggedleft}p{\dimexpr 1.05in+0\tabcolsep+0\arrayrulewidth}!{\color[HTML]{000000}\vrule width 0pt}}{\fontsize{11}{11}\selectfont{\textcolor[HTML]{000000}{\global\setmainfont{Arial}{1\ 282.7}}}} \\





\multicolumn{1}{!{\color[HTML]{000000}\vrule width 0pt}>{\raggedright}p{\dimexpr 1.7in+0\tabcolsep+0\arrayrulewidth}}{\fontsize{11}{11}\selectfont{\textcolor[HTML]{000000}{\global\setmainfont{Arial}{Rapla\ maakond}}}} & \multicolumn{1}{!{\color[HTML]{000000}\vrule width 0pt}>{\raggedleft}p{\dimexpr 0.84in+0\tabcolsep+0\arrayrulewidth}}{\fontsize{11}{11}\selectfont{\textcolor[HTML]{000000}{\global\setmainfont{Arial}{33\ 311.0}}}} & \multicolumn{1}{!{\color[HTML]{000000}\vrule width 0pt}>{\raggedleft}p{\dimexpr 0.46in+0\tabcolsep+0\arrayrulewidth}}{\fontsize{11}{11}\selectfont{\textcolor[HTML]{000000}{\global\setmainfont{Arial}{67.0}}}} & \multicolumn{1}{!{\color[HTML]{000000}\vrule width 0pt}>{\raggedleft}p{\dimexpr 1.26in+0\tabcolsep+0\arrayrulewidth}}{\fontsize{11}{11}\selectfont{\textcolor[HTML]{000000}{\global\setmainfont{Arial}{0.0}}}} & \multicolumn{1}{!{\color[HTML]{000000}\vrule width 0pt}>{\raggedleft}p{\dimexpr 1.05in+0\tabcolsep+0\arrayrulewidth}!{\color[HTML]{000000}\vrule width 0pt}}{\fontsize{11}{11}\selectfont{\textcolor[HTML]{000000}{\global\setmainfont{Arial}{497.2}}}} \\





\multicolumn{1}{!{\color[HTML]{000000}\vrule width 0pt}>{\raggedright}p{\dimexpr 1.7in+0\tabcolsep+0\arrayrulewidth}}{\fontsize{11}{11}\selectfont{\textcolor[HTML]{000000}{\global\setmainfont{Arial}{Saare\ maakond}}}} & \multicolumn{1}{!{\color[HTML]{000000}\vrule width 0pt}>{\raggedleft}p{\dimexpr 0.84in+0\tabcolsep+0\arrayrulewidth}}{\fontsize{11}{11}\selectfont{\textcolor[HTML]{000000}{\global\setmainfont{Arial}{33\ 108.0}}}} & \multicolumn{1}{!{\color[HTML]{000000}\vrule width 0pt}>{\raggedleft}p{\dimexpr 0.46in+0\tabcolsep+0\arrayrulewidth}}{\fontsize{11}{11}\selectfont{\textcolor[HTML]{000000}{\global\setmainfont{Arial}{67.0}}}} & \multicolumn{1}{!{\color[HTML]{000000}\vrule width 0pt}>{\raggedleft}p{\dimexpr 1.26in+0\tabcolsep+0\arrayrulewidth}}{\fontsize{11}{11}\selectfont{\textcolor[HTML]{000000}{\global\setmainfont{Arial}{0.0}}}} & \multicolumn{1}{!{\color[HTML]{000000}\vrule width 0pt}>{\raggedleft}p{\dimexpr 1.05in+0\tabcolsep+0\arrayrulewidth}!{\color[HTML]{000000}\vrule width 0pt}}{\fontsize{11}{11}\selectfont{\textcolor[HTML]{000000}{\global\setmainfont{Arial}{494.1}}}} \\





\multicolumn{1}{!{\color[HTML]{000000}\vrule width 0pt}>{\raggedright}p{\dimexpr 1.7in+0\tabcolsep+0\arrayrulewidth}}{\fontsize{11}{11}\selectfont{\textcolor[HTML]{000000}{\global\setmainfont{Arial}{Tartu\ maakond}}}} & \multicolumn{1}{!{\color[HTML]{000000}\vrule width 0pt}>{\raggedleft}p{\dimexpr 0.84in+0\tabcolsep+0\arrayrulewidth}}{\fontsize{11}{11}\selectfont{\textcolor[HTML]{000000}{\global\setmainfont{Arial}{152\ 976.0}}}} & \multicolumn{1}{!{\color[HTML]{000000}\vrule width 0pt}>{\raggedleft}p{\dimexpr 0.46in+0\tabcolsep+0\arrayrulewidth}}{\fontsize{11}{11}\selectfont{\textcolor[HTML]{000000}{\global\setmainfont{Arial}{67.0}}}} & \multicolumn{1}{!{\color[HTML]{000000}\vrule width 0pt}>{\raggedleft}p{\dimexpr 1.26in+0\tabcolsep+0\arrayrulewidth}}{\fontsize{11}{11}\selectfont{\textcolor[HTML]{000000}{\global\setmainfont{Arial}{0.0}}}} & \multicolumn{1}{!{\color[HTML]{000000}\vrule width 0pt}>{\raggedleft}p{\dimexpr 1.05in+0\tabcolsep+0\arrayrulewidth}!{\color[HTML]{000000}\vrule width 0pt}}{\fontsize{11}{11}\selectfont{\textcolor[HTML]{000000}{\global\setmainfont{Arial}{2\ 283.2}}}} \\





\multicolumn{1}{!{\color[HTML]{000000}\vrule width 0pt}>{\raggedright}p{\dimexpr 1.7in+0\tabcolsep+0\arrayrulewidth}}{\fontsize{11}{11}\selectfont{\textcolor[HTML]{000000}{\global\setmainfont{Arial}{Valga\ maakond}}}} & \multicolumn{1}{!{\color[HTML]{000000}\vrule width 0pt}>{\raggedleft}p{\dimexpr 0.84in+0\tabcolsep+0\arrayrulewidth}}{\fontsize{11}{11}\selectfont{\textcolor[HTML]{000000}{\global\setmainfont{Arial}{28\ 370.0}}}} & \multicolumn{1}{!{\color[HTML]{000000}\vrule width 0pt}>{\raggedleft}p{\dimexpr 0.46in+0\tabcolsep+0\arrayrulewidth}}{\fontsize{11}{11}\selectfont{\textcolor[HTML]{000000}{\global\setmainfont{Arial}{67.0}}}} & \multicolumn{1}{!{\color[HTML]{000000}\vrule width 0pt}>{\raggedleft}p{\dimexpr 1.26in+0\tabcolsep+0\arrayrulewidth}}{\fontsize{11}{11}\selectfont{\textcolor[HTML]{000000}{\global\setmainfont{Arial}{0.0}}}} & \multicolumn{1}{!{\color[HTML]{000000}\vrule width 0pt}>{\raggedleft}p{\dimexpr 1.05in+0\tabcolsep+0\arrayrulewidth}!{\color[HTML]{000000}\vrule width 0pt}}{\fontsize{11}{11}\selectfont{\textcolor[HTML]{000000}{\global\setmainfont{Arial}{423.4}}}} \\





\multicolumn{1}{!{\color[HTML]{000000}\vrule width 0pt}>{\raggedright}p{\dimexpr 1.7in+0\tabcolsep+0\arrayrulewidth}}{\fontsize{11}{11}\selectfont{\textcolor[HTML]{000000}{\global\setmainfont{Arial}{Viljandi\ maakond}}}} & \multicolumn{1}{!{\color[HTML]{000000}\vrule width 0pt}>{\raggedleft}p{\dimexpr 0.84in+0\tabcolsep+0\arrayrulewidth}}{\fontsize{11}{11}\selectfont{\textcolor[HTML]{000000}{\global\setmainfont{Arial}{46\ 371.0}}}} & \multicolumn{1}{!{\color[HTML]{000000}\vrule width 0pt}>{\raggedleft}p{\dimexpr 0.46in+0\tabcolsep+0\arrayrulewidth}}{\fontsize{11}{11}\selectfont{\textcolor[HTML]{000000}{\global\setmainfont{Arial}{67.0}}}} & \multicolumn{1}{!{\color[HTML]{000000}\vrule width 0pt}>{\raggedleft}p{\dimexpr 1.26in+0\tabcolsep+0\arrayrulewidth}}{\fontsize{11}{11}\selectfont{\textcolor[HTML]{000000}{\global\setmainfont{Arial}{0.0}}}} & \multicolumn{1}{!{\color[HTML]{000000}\vrule width 0pt}>{\raggedleft}p{\dimexpr 1.05in+0\tabcolsep+0\arrayrulewidth}!{\color[HTML]{000000}\vrule width 0pt}}{\fontsize{11}{11}\selectfont{\textcolor[HTML]{000000}{\global\setmainfont{Arial}{692.1}}}} \\





\multicolumn{1}{!{\color[HTML]{000000}\vrule width 0pt}>{\raggedright}p{\dimexpr 1.7in+0\tabcolsep+0\arrayrulewidth}}{\fontsize{11}{11}\selectfont{\textcolor[HTML]{000000}{\global\setmainfont{Arial}{Võru\ maakond}}}} & \multicolumn{1}{!{\color[HTML]{000000}\vrule width 0pt}>{\raggedleft}p{\dimexpr 0.84in+0\tabcolsep+0\arrayrulewidth}}{\fontsize{11}{11}\selectfont{\textcolor[HTML]{000000}{\global\setmainfont{Arial}{35\ 782.0}}}} & \multicolumn{1}{!{\color[HTML]{000000}\vrule width 0pt}>{\raggedleft}p{\dimexpr 0.46in+0\tabcolsep+0\arrayrulewidth}}{\fontsize{11}{11}\selectfont{\textcolor[HTML]{000000}{\global\setmainfont{Arial}{67.0}}}} & \multicolumn{1}{!{\color[HTML]{000000}\vrule width 0pt}>{\raggedleft}p{\dimexpr 1.26in+0\tabcolsep+0\arrayrulewidth}}{\fontsize{11}{11}\selectfont{\textcolor[HTML]{000000}{\global\setmainfont{Arial}{0.0}}}} & \multicolumn{1}{!{\color[HTML]{000000}\vrule width 0pt}>{\raggedleft}p{\dimexpr 1.05in+0\tabcolsep+0\arrayrulewidth}!{\color[HTML]{000000}\vrule width 0pt}}{\fontsize{11}{11}\selectfont{\textcolor[HTML]{000000}{\global\setmainfont{Arial}{534.1}}}} \\

\hhline{>{\arrayrulecolor[HTML]{666666}\global\arrayrulewidth=2pt}->{\arrayrulecolor[HTML]{666666}\global\arrayrulewidth=2pt}->{\arrayrulecolor[HTML]{666666}\global\arrayrulewidth=2pt}->{\arrayrulecolor[HTML]{666666}\global\arrayrulewidth=2pt}->{\arrayrulecolor[HTML]{666666}\global\arrayrulewidth=2pt}-}



\end{longtable}

Kui me nüüd tahame arvutada Eesti keskmise sissetuleku lähtuvalt maakondade kaalutud keskmisest, siis peame esmalt maakondade keskmised kaaludega läbi korrutama, kokku liitma ning seejärel kaalude summaga läbi jagama.

\[\frac{(1531 \times 8926) + (993 \times 140) + ... + (1113 \times 534)}{8926 + 140 + ... + 534} = 1351\]

Märksa lähemal tõelisele Eesti keskmisele sissetulekule. Päris sama summat ei saanud me peamiselt seetõttu, et võtsime kaalumise aluseks rahvaarvu, mitte töötavate inimese arvu. Lisaks on tõenäoline, et sissetulekute jaotused erinevad maakondade lõikes, mis tähendab, et tegelikult ei oleks siinkohal väga õige aritmeetilist keskmist kasutada.

\hypertarget{post-stratifikatsiooni-kaalud}{%
\subsection{Post-stratifikatsiooni kaalud}\label{post-stratifikatsiooni-kaalud}}

Post-stratifikatsiooni abil saame parandada valimi representatiivsust peale valimi moodustamist. Näiteks juhul kui meie valim on moodustatud aadresside alusel, siis ei saa me planeerida oma valimit vastajate vanuselisest ja soolisest jaotusest lähtuvalt, kuigi üldkogumi kohta on meil need proportsioonid teada. Teine ja märksa levinum põhjus post-stratifikatsiooni jaoks on mitte-vastamine (\emph{non-response}). Mõnede gruppide puhul kipub vastamisaktiivsus olema mõnevõrra madalam (näiteks noored mehed). Ka mitte-vastamise puhul ei ole meil valimi disainimise juures palju teha (võime muidugi suurendada noorte meeste valimisse sattumise tõenäosust, kuid me ei saa kindlustada kogu valimi vastavust üldkogumi proportsioonidele) ja selle ulatust näeme me vaid peale küsitluse läbiviimist. Kui me siis hiljem avastame, et kuigi mehi peaks olema üldkogumis 50\%, on neid valimisse sattunud 40\%, siis tähendab see seda, et meie valim ei ole soo lõikes representatiivne. Naised on üleesindatud ja mehed alaesindatud. Sellisel juhul on võimalik välja arvutada post-stratifikatsioonikaalud. Kaalude moodustamisel peame naisi vähemaks kaaluma \(0.5 \div 0.6 \approx 0.8333\) ja mehi rohkemaks kaaluma \(0.5 \div 0.4 = 1.25\). Kui valimi kaalud on eelnevalt olemas, siis modifitseerime valimi kaale korrutades iga meessoost respondendi kaalud läbi \(1.25\)-ga ja iga naissoost respondendi kaalud läbi \(0.83\)-ga.

Post-stratifikatsiooni kasutatakse tavaliselt valimi representatiivsuse tõestmiseks erinevate sotsiaaldemograafiliste tunnuste lõikes (sugu, vanus, haridus jne). Kuid kaalude moodustamiseks peame teadma kõikide post-stratifikatsioonitunnuste ristlõigete osakaale. Ehk siis kui tahame post-stratifitseerida hariduse ja soo lõikes, siis peame teadma kõrgharitud meeste osakaalu, kõrgharitud naiste osakaalu, keskharitud meeste osakaalu jne. Tihti taolist ristõikelist üldkogumi jaotust me ei tea. Lisaks võivad nii moodustatud grupid minna väga väikeseks ja tekib oht, et mõnda ristlõiget meil valimis ei olegi. Näiteks kui lisaksime haridusele ja soole veel ka vanuse (näiteks vanused 15-75), siis oleks meil juba \(2\times3\times60 = 360\) gruppi. Arvestades tavapärast valimimahtu (\emph{ca} 1000 vaatlust) on ülimalt tõenäoline, et osasid post-stratifikatsiooni ristlõikesid meie valimisse lihtsalt ei sattunud.

\hypertarget{raking-ja-calibration}{%
\subsubsection{Raking ja calibration}\label{raking-ja-calibration}}

Miks me ei saa post-stratifikatsioonikaale iga sotsiaaldemograafilise tunnuse lõikes eraldi välja arvutada ja seejärel need erinevad kaalud läbi korrutada? Kui me arvutaksime kõigepealt välja soo kaalud, siis kaalude rakendamisel oleks meie valim soo lõikes representatiivne. Kui me seejärel arvutaks välja hariduse kaalud ja rakendaks ka need, siis oleks meie valim hariduse lõikes representatiivne, kuid soo proportsioonid oleksid jälle paigast ära. Ja kui me siis kasutaks veel ka vanuse kaale, oleks meie valim representatiivne vanuse lõikes, kuid mitte enam soo ega hariduse lõikes. Peaksime nüüd uuesti arvutama soo kaalud, siis uuesti hariduse kaalud ja siis uuesti vanuse kaalud. Ja seejärel jälle otsast peale. Niikaua, kuni oleme leidnud mingisuguse ekviliibriumi, kus ühegi tunnuse kaal enam ei muutu. Taolist iteratiivset protsessi nimetatakse rakinguks (ka \emph{iterative proportional fitting}).

Alternatiivne viis erinevate tunnuste lõikes valimi proportsioonide korrigeerimiseks on kalibreerimine. Kalibreerimise puhul kasutakse valimi üldkogumiga proportsiooni viimiseks regressioonimudelit, mille abil arvutatakse kalibratsioonikaalud, millega siis korrigeeritakse valimi kaale.

\hypertarget{mitte-vastamise-kaalud-non-response-weights}{%
\subsection{Mitte-vastamise kaalud (non-response weights)}\label{mitte-vastamise-kaalud-non-response-weights}}

Kuigi post-stratifitseerimisega on võimalik neid gruppe, kus vastamismäär oli madalam kui valimiraam ette nägi, üles kaaluda, on tihti otstarbekas pöörata tähelepanu ka otseselt mitte-vastanutele. Kui post-stratifikatsiooni või rakingu/kalibreerimisega üritatakse tasanda eelkõige valimiraami ja üldkogumi erinevusi (juhuvalim, tänu sellele, et see on juhuslik, ei taga alati, et valim vastaks üldkogumi proportsioonidele) terves valimis, siis mitte-vastamise kaalude abil üritatakse üles kaaluda ainult neid gruppe, kus vastamismäär oli madalam ja gruppides, kus vastamismäär oli 100\%, kaale ei muudeta. Tulles tagasi eelneva soolise erinevuse näite juurde: oletame, et meie valimi suuruseks oli 1000 inimest, kellest 50\% olid valimi järgi naised ja 50\% mehed (valimis vastavalt 500 ja 500), kuid peale küsitlust avastame, et algsest 1000-st respondendist õnnestus küsitleda 500-t naist ja 400-t meest. Ehk siis naiste vastamismäär oli 100\% ja meeste vastamismäär 80\%. Kuna naiste valimi osaga on kõik korras, ei taha me nende puhul midagi muuta. Küll aga tahame suurendada meeste kaale nii, et meeste valimi osa vastaks algsele valimiraamile. See tähendab, et me peaksime suurendama meeste kaale \(500\div400 = 1.25\) võrra. Iga valimisse sattunud meest arvstame seega 1.25 kordselt.

Teine võimalus mitte-vastamise kaalude arvutamiseks on kasutada logistilist regressiooni, kus valimiraamist lähtuvad sotsiaaldemograafilised tunnused on aluseks hinnangule vastamise ja mitte-vastamise kohta.

\hypertarget{kaaludest-uxfcldiselt}{%
\subsection{Kaaludest üldiselt}\label{kaaludest-uxfcldiselt}}

Ise kaale arvutades peaksime jälgima, et meie arvutatud kaalud liiga suureks ei läheks. See juhtub siis, kui mõni grupp mille lõikes me kaale arvutame, on valimis väga alaesindatud. Näiteks kui tahame tagada, et meie valim vastaks üldkogumile töötuse tunnuse lõikes, kuid valimisse on sattunud töötuid vaid 1\%, samas kui üldkogumis on töötuid näiteks 10\%. Me peaksime kõik olemasolevad töötud kümnekordselt üles kaaluma. Seda on aga ilmselgelt liiga palju. Me ei taha mõne töötu pealt tehtud järeldusi, mis võivad olla suhteliselt juhuslikud, laiendada küllaltki suurele osale populatsioonist. Seetõttu tuleks pärast kaalude moodustamist alati kontrollida, et need ikka mõistlikkuse piiresse jääks (see mõistlikkuse piir oleneb kontekstist) ja vajadusel suuremaid kaalusid väiksemaks teha (\emph{weight trimming}).

Kui andmestikus on mitu kaalude tunnust, näiteks valimi kaalud ja mitte-vastamise kaalud, siis saame need ühendada üheks kaalu tunnuseks. Selleks peame iga vaatluse jaoks tema kaalud läbi korrutama. Ehk siis kui konkreetse vaatluse jaoks on valimi kaalud väärtusega 0.8 ja mitte-vastamise kaalud 1.2, siis vaatluse lõplikeks kaaludeks kujuneb \(0.8 \times1.2 = 0.96\).

\hypertarget{valimidisain}{%
\section{Valimidisain}\label{valimidisain}}

Olukorras, kus kõigil üldkogumi liikmetel ei ole võrdset võimalust valimisse sattuda (kuid kõigil on see võimalus mingi tõenäosusega siiski olemas) või valimi valikuühik ei ole sama mis analüüsiühik (näiteks kui valim on moodustatud leibkondade põhjal aga analüüsime lebkondades olevaid isikuid), on meil tegemist mingi \textbf{valimidisaini} alusel moodustatud valimiga. Valimidisainiga defineeritakse iga analüüsiühiku erinevad valimisse sattumise tõenäosused. Kui me valimi aluseks olevat disaini andmete anlüüsimisel arvesse ei võta, võime analüüsi tulemusel teha väga valesid järeldusi. Seda nii punkthinnangute (nagu eelnevas näites) kui ka standardvigade osas. Punkthinnanguid saame üldjuhul korrigeerida valimi kaaludega. Näiteks kaalutud keskmisi saame ka käsitsi välja arvutada, nagu me eelnevalt tegime, või kasutada selleks vastavaid funktsioone (Ri baasfunktsioon \texttt{weighted.mean()} võtab sisendiks kaalumata keskmiste vektori ja kaalude vektori).

Standardvigade korrigeerimine asi mõnevõrra keerulisem ja nende puhul tuleks kasutada spetsiifilisemaid lähenemisi. Eelnevalt vaatasime tavalise juhuvalimi standardvigade arvutamise loogikat (standardviga on valimijaotuse standardhälve, näidates kui suur on meie hinnangu keskmine viga kui me võtaksime samast popultaioonist lõputult valimeid). Keerulisemate valimidisainide puhul juhuvalimi loogika enam ei toimi, kuna vaatlused ei ole omavahel sõltumatud ja/või valiku ühikuks ei ole analüüsiühik. Sellises olukorras on meil tegelikult päris mitmeid võimalusi, kuidas standardvigasid, ehk siis valimijaotuse varieeruvust, hinnata. Peamisteks kasutatavateks meetoditeks on:

\begin{itemize}
\tightlist
\item
  Taylori seeriate meetod (\emph{Taylor series linearisation})
\item
  Bootstrapi replikatsiooni meetod
\item
  Jackknife replikatsiooni meetod
\item
  BRR (\emph{balanced repeated replication})
\end{itemize}

Kui Taylori seeriate meetodi puhul arvutatakse standardvead analüütiliselt, siis ülejäänud replikatsioonipõhiste meetodite puhul empiiriliselt. Üldiselt ei pea nende meetodite hingeelu nende kasutamiseks väga põhjalikult tundma (hiljem siiski vaatame replikatsioonimeetodeid, kuna nende loogika on küllaltki lihtne).

Laias laastus võib erinevad valimidisainid jagada kolme suuremasse gruppi: stratifitseeritud valimid ja klastervalimid. Üldjuhul kasutatakse reaalselt nende kahe tüübi kombinatsioone või variatsioone. Konkreetse uuringu juures kasutatav valimidisain on tavaliselt kirjeldatud uuringu dokumentatsioonis (kui see nii ei ole, siis tasub alati uuringu läbiviijalt seda küsida). Tihti on selle seletuse juures ära toodud ka juhised edasiseks analüüsiks. Seega esimeseks sammuks mingi uuringu kasutamisel peaks alati olema uuringu dokumentatsiooniga tutvumine.

Kuid miks üldse kasutatakse mingeid keerulisi valimidisaine ja ei piirduta tavaliste juhuvalimitega, mida oleks standardmeetoditega lihtne analüüsida? Esimene ja tihti määravaim põhjus on küsitlusega kaasnev kulu. Keerulisemad valimidisainid võimaldavad kontsentreerida küsitluste läbiviimist, lihtsustades seeläbi küsitlusega kaasnevat logistikat. Teine, ja tegelikult mõnevõrra olulisem põhjus on teatud keerulisemate valimidisainidega kaasnev tulemuste kvaliteedi tõus. Me saame valimidisainiga näiteks sihtida konkreetseid gruppe või tõsta uuringu üldist täpsusastet.

\hypertarget{stratifitseeritud-juhuvalim}{%
\subsection{Stratifitseeritud juhuvalim}\label{stratifitseeritud-juhuvalim}}

Üldkogum jagatakse teineteist välistavatesse gruppidesse (stratad) ja igas grupis viiakse läbi juhuvalik. Stratifitseeritud valimiga on võimalik tagada, et stratifitseeriva tunnuse lõikes on valim ühtlaselt jaotunud ning vähem varieeruv. Seetõttu on ka selle põhjal tehtavad hinnangu täpsemad (standardvead on väiksemad).

Tihti tahame teha järeldusi mõne küllaltki marginaalse grupi kohta. Et need järeldused oleksid statistiliselt täpsed, on meil vaja tagada, et see grupp oleks esindatud suuremalt kui see juhuvalikuga võimalik on (mida suurem on uuritava grupi valim, seda täpsemaid hinnanguid me selle kohta teha saame). Näiteks kui tahaksime üle-Eestilises uuringus käsitleda mõnda spetsiifilist küsimust Hiiumaa kohta, siis 1000 inimesega juhuvalimiga satuks meie valimisse 7 hiidldast (Hiiumaa rahvaarv moodustab 0.7\% Eesti rahvastikust). Seda on aga liiga vähe et me saaksime Hiiumaa kohta midagi olulist järeldada. Seega peaksime hiidlaste valimisse sattumise tõenäosust mõnevõrra suurendama.

Lisaks võimaldab stratifitseeritud valim mõningal määral lihtsustada andmekogumise administreerimist (Näiteks igas maakonnas eraldi uuringu läbiviija) või isegi kasutada stratades erinevaid valimi moodustamise meetodeid.

Stratifitseeritud valim eeldab, et strtifitseerimise aluseks olev tunnus oleks valimi moodustamisel iga üldkogumi liikme jaoks teada. See võib teatud juhtudel aga küllaltki problemaatiline olla. Näiteks kui me tahame stratifitseerida maakonna alusel ja üldkogumi moodustamisel lähtuda rahvastikuregistrist, siis on küllaltki tõenäoline, et rahvastikuregistris märgitud elukoha maakond (kui see seal üldse märgitud on) erineb reaalsest elukoha maakonnast.

\hypertarget{klastervalim}{%
\subsection{Klastervalim}\label{klastervalim}}

Klastervalimi puhul käsitletakse analüüsiühikuid mingite klastrite liikmetena ning valimi moodustamisel ei valita mitte analüüsiühikuid otse vaid neist moodustunud klstereid. Juhuvaliku alusel valitakse klastrid ning kõikide valimisse sattunud klastrite liikmeid küsitletakse. Kõige lihtsama näitena võib siinkohal tuua koolivõrgu. Eestis on ca 530 üldhariduskooli. Kui me tahame teha küsitlust, mis oleks representatiivne kõigi Eesti õpetajate suhtes, siis juhuvalimi korral peaksime olema valmis, et kõikidest koolidest satub meie valimisse mõni õpetaja (kuna kõigil õpetajatel on võrdne tõenäosus valimisse sattuda). Kõik koolid läbi käia ja seal paari õpetajat küsitleda oleks logistilisel küllaltki tülikas ning mitte eriti aja- ja kuluefektiivne. Lihtsam, kiirem ja odavam oleks valimi alusena käsitleda koole, teha koolide juhuvalim ning valimisse sattunud koolides küsitleda kõiki õpetajaid. Sellisel juhul oleks meie üldkogum ikkagi kogu Eesti õpetajaskond (kuna kõikidel koolidel ja seeläbi ka kõikidel õpetajatel õpetajatel oli võimalus valimisse sattuda) ja kui me taoliselt moodustatud valimidisaini hiljem analüüsi käigus arvestame, siis saaksime tehtud järeldused üldistada ka kõikidele Eesti õpetajatele.

Võib-olla ei ole isegi mõtet kõiki valimisse sattunud koolide õpetajaid intervjueerida. Me saaksime tegelikult ka igas koolis teha eraldi teise tasandi juhuvalimi ja küsitleda ainult valimisse sattunud õpetajaid. Seega esmalt valime juhuvalimi alusel koolid ja seejärel juhuvalimi alusel õpetajad neis koolides. Taolist valimidisani nimetatakse mitmetasandiliseks klastervalimiks. Neid tasandeid võib olla ka rohkem kui kaks. Näiteks esimene juhuvalik on koolide tasandil, teine klasside tasandil ja kolmas õpilaste tasandil. Esimese tasandi valimiühikuid nimetatakse \emph{primary sampling units} või \emph{PSU} (koolid), teise tasandi ühikuid \emph{secondary sampling units } või \emph{SSU} (klassid) jne.

Võrreldes juhu- või stratifitseeritud valimiga klastervalim sama valimimahu juures üldiselt vähendab hinnangute täpsust, kuna inimesed klastrite sees kipuvad olema sarnasemad kui klastrite vahel. Kuid samas võimaldab klastervalimi kuluefektiivsus koostada suuremaid valimeid, mis omakorda suurendavad täpsust. Seega kokkuvõttes võib mõnevõrra suurema mahuga klastervalimiga saada väiksema mahuga juhuvalimiga võrreldava täpsusastmega hinnagud väiksema raha eest.

\hypertarget{replikatsioonikaalud-replicate-weights}{%
\subsection{Replikatsioonikaalud (replicate weights)}\label{replikatsioonikaalud-replicate-weights}}

Keerulisemate valimidisainide puhul ei ole võimalik standardvigasid analüütiliste meetoditega arvutada. Sel juhul on standardvigade leidmisks võimalik kasutada erinevad replikatsioonimeetodid, nagu \emph{bootstrap} või \emph{jackknife} (reaalsuses küll tavaliselt mõnda nende variatsioonidest). Paljude suuremate uuringute puhul on andmestikuga kaasas replikatsioonikaalud, mis võimaldavad replikatsioonimeetodeid kasutada. Näiteks OECD uuringutes nagu PISA või PIIAC. Replikatsioonimeetodeid võib ja saab kasutada ka tavalise juhuvalimi korral. Ajalooliselt on nende laiem kasutus jäänud arvutusvõimsuste taha, kuid tänapäeval, kui see enam probleem ei ole, on need järjest rohkem hakanud tavapäraseid analüütilisi standardvigade arvutamise meetodeid asendama.

Kõikide replikatsioonimeetodite üldprintsiip ja loogika on lihtne. Olemasolevast valimist võetakse palju alamvalimeid (replikatsioone), ehk siis valimit käsitletakse üldpopultaioonina, millest võetkse omakorda valimid. Kõikide alamvalimite puhul arvutatakse huvipakkuv statistik (näiteks mingi tunnuse keskmine). Alamvalimite statistikutest moodustub (pseudo)valimijaotus, mille standardhälve ongi standardviga. Seega kui tavaline standardvea arvutamise metoodika lähtub küll potentsiaalsest eeldusest, et valimijaotuse aluseks olevaid valimeid on lõputult, kuid tuletab standardvea analüütiliselt (\(se = \frac{sd}{\sqrt{n}}\)), siis replikatsioonimeetodid tuletavad olemasolevast valimist suurel hulgal alamvalimeid, mille alusel moodustavad valimijaotuse ja tuletavad standardvea empiiriliselt.

Repikatsioonimeetodid erinevad üksteisest selle poolest kuidas nad neid alavalimeid moodustavad. \emph{Bootstrap} meetodiga võetakse üldvalimist juhuslikkuse alusel sama palju vaatlusi kui seal algselt oli, kuid kasutatakse nn tagasipaneku meetodit. See tähendab, et mõni algse valimi vaatlus võib alamvalimisse sattuda mitu korda ja mõni üldse mitte. Kui palju alamvalimeid võetakse, on analüütiku otsustada (mida rohkem, seda parem, kuid võiks olla vähemalt 100).

\emph{Jackknife} meetodiga võetakse alamvalimeid nii palju kui algses valimis vaatlusi oli. Kuid iga alamvalimi puhul jäetakse mingi reegli alusel üks vaatlustest välja. Ülejäänud vaatlused korrutatakse läbi koefitsiendiga, mis korrigeerib alamvalimi suuruse võrdseks algse valimiga. Näiteks kui algses valimis oli 10 inimest, siis võetakse 10 alamvalimit, milles kõigis on 9 inimest ja Kõik alavalimid korrutatakse läbi 1.1-ga (\(\frac{10}{9}\))

Replikatsioonikaalud annavad meile info kuidas valimite replikeerimine peaks toimuma.

\hypertarget{kuxfcsitlusandmed-ris}{%
\section{Küsitlusandmed Ris}\label{kuxfcsitlusandmed-ris}}

Ris on küsitlusandmete analüüsiks spetsiaalne pakett \emph{survey}, mis võimaldab kasutada erinevaid analüüsimeetodeid arvestades samal ajal ka valimidisainist tulenevate eripäradega. Kuna praktiliselt kõik suuremad küsitlused kasutavad mingeid keerukamaid valimidisaine, siis peame nende andmete analüüsimisel alati ka vastavate disainidega arvestama ega saa kasutada tavapäraseid meetodeid.

Kõigepealt installige (kui te seda juba teinud ei ole) \emph{survey} pakett ja lugege see sisse. Nagu alati, siis installima peab paketi ainult ühe korra, kuid igaks sessiooniks tuleb see uuesti sisse lugeda. \emph{survey} paketiga on kaasas ka näidisanmdestikud. Lugege ka need sisse (´data(api)´). \emph{api} andmestikuga on mõõdetud akadeemilise võimekuse indeksit kõikides Kalifornia koolides. Andmestiku analüüsiühikuks on kool ehk siis andmestik koondab koolitsandi infot.

\emph{survey} paketis peab kõigepealt defineerima valimidisaini, milles kirjeldatakse kõiki valimi moodustamise eripärasid, kaale jne. Disaini defineerimise läbi koondatakse kogu valimidisainist lähtuv info ühte andmeojekti. Disaini deineerimiseks on funktsioon \texttt{svydesign()}:

\hypertarget{kaaludega-tavalise-juhuvalimi-defineerimine}{%
\subsection{Kaaludega tavalise juhuvalimi defineerimine}\label{kaaludega-tavalise-juhuvalimi-defineerimine}}

Valimidisain tavalise juhuvalimi puhul kui meile on ainult valimi kaalud ja/või mingid mingid kaalud. Isegi tavalise juhuvalimi puhul peaksime ikkagi kasutama vähemalt valimi kaale (need lubavad meil hinnata näiteks populatsiooni suurust). Lisaks valimi kaaludele või valimi kaalude asemel võivad andmestikus olla ka mitte-vastamise kaalud või poststratifikatsiooni kaalud (või oleme need ise välja arvutanud), mille olemasolu või vajadus ei sõltu sellest, kas tegemist on tavalise juhuvalimiga või keerulisema disainiga. Kui andmestikus on mitu kaalude tunnust, mida meil on vaja kasutada, siis peaksime need omavahel läbi korrutades üheks tunnuseks koondama.

\texttt{svydesign()} argumentidena peame defineerima:

\begin{itemize}
\tightlist
\item
  \texttt{ids} argumendiga defineeritakse valimiühikud (\emph{PSU}). Need on klastri id'd klastervalimi puhul, ehk siis tunnus, mille lõikes klastervalimi valik toimus. Kui valik toimus analüüsitasandil ilma klastriteta, siis tuleks märkida \texttt{ids\ =\ \textasciitilde{}1}
\item
  \texttt{weights} tähistab kaalu tunnust. Tunnuse nime ette tuleb kindlasti panna tilde märk (või viidata tunnusele otse apisrs\$pw)
\item
  \texttt{data} on meie algne küsitlusandmestik
\end{itemize}

Väiksemate üldkogumite korral, kui me teame üldkogumi suurust, saab kasutada lõpliku populatsiooni korrektsiooni (\emph{finite population correction} ehk \emph{fpc}), mis võimaldab kasutada mõnevõrra väiksemaid standardvigasid. Kui üldkogum on küllaltki väike, siis on ka valim üldkogumile tõenäoliselt sarnasem, võrreldes juhuga kui üldkogum on väga suur. Seega on valimis ka vähem määramatust ning standardvead väiksemad.

\emph{fpc} defineeritakse parameetriga:

\begin{itemize}
\tightlist
\item
  \texttt{fpc} mille kaudu peab deineerima tunnuse, mis sisaldab iga vaatluse kohta üldkogumi suurust. Tavalise juhuvalimi puhul on see suurus kõikide vaatuste jaoks sama.
\end{itemize}

Küsitluse disainist ülevaate saamiseks saame kasutada \texttt{summary()} funktsiooni:

\begin{verbatim}
## Independent Sampling design
## svydesign(ids = ~1, weights = ~pw, fpc = ~fpc, data = apisrs)
## Probabilities:
##    Min. 1st Qu.  Median    Mean 3rd Qu.    Max. 
## 0.03229 0.03229 0.03229 0.03229 0.03229 0.03229 
## Population size (PSUs): 6194 
## Data variables:
##  [1] "cds"      "stype"    "name"     "sname"    "snum"     "dname"   
##  [7] "dnum"     "cname"    "cnum"     "flag"     "pcttest"  "api00"   
## [13] "api99"    "target"   "growth"   "sch.wide" "comp.imp" "both"    
## [19] "awards"   "meals"    "ell"      "yr.rnd"   "mobility" "acs.k3"  
## [25] "acs.46"   "acs.core" "pct.resp" "not.hsg"  "hsg"      "some.col"
## [31] "col.grad" "grad.sch" "avg.ed"   "full"     "emer"     "enroll"  
## [37] "api.stu"  "pw"       "fpc"
\end{verbatim}

\hypertarget{stratifitseeritud-valimi-defineerimine}{%
\subsection{Stratifitseeritud valimi defineerimine}\label{stratifitseeritud-valimi-defineerimine}}

Stratifitseeritud valimi puhul tuleb meil defineerida stratifitseeriv tunnus. Andmestikus \emph{apistrat} on selleks \emph{stype} (koolitüüp) tunnus, mille lõikes koolid on stratifitseeritud.

\begin{itemize}
\tightlist
\item
  \texttt{strata} argumendiga defineeritakse valimit stratifitseeriv tunnus
\item
  \texttt{fpc} argumendiga saab määrata iga strata suuruse (kui see on olemas, siis ole vaja valimi kaale eraldi märkida, survey arvutab vaatluste valimisse sattumise tõenäosused ja kaalud ise välja)
\end{itemize}

\hypertarget{uxfchetasandilise-klastervalimi-defineerimine}{%
\subsection{Ühetasandilise klastervalimi defineerimine}\label{uxfchetasandilise-klastervalimi-defineerimine}}

Kasutame \emph{apiclus1} andmestikku, milles on koolid klasterdatud piirkonna (\emph{dnum}) järgi. Ehk siis piirkonnad on esimese tasandi valikuks. Valimisse sattunud piirkondade kõik koolid on valimisse kaasatud.

\begin{itemize}
\tightlist
\item
  \texttt{ids} argumeniga defineeritakse \emph{PSU} ehk klastri id (\emph{dnum})
\item
  \texttt{weights} argumendiga määratakse valimi kaalud. Jällegi, kui tegemist on ainult valimi kaaludega ja \texttt{fpc} on defineeritud, siis ei pea kaale määrama (\emph{survey} arvutab need ise välja)
\item
  \texttt{fpc} valimi suurus (klastrite koguarv)
\end{itemize}

\hypertarget{mitmetasandilise-klastervalimi-defineerimine}{%
\subsection{Mitmetasandilise klastervalimi defineerimine}\label{mitmetasandilise-klastervalimi-defineerimine}}

Andmestikuks on \emph{apiclus2}, kus esimese tasandina (\emph{PSU}) on valitud kooli piirkonnad ja teise tasandina (\emph{SSU}) koolid. Ehk kõigepealt on tehtud piirkondade valim ja seejärel igas valimisse sattunud piirkonnas omakorda koolide valim.

\begin{itemize}
\tightlist
\item
  \texttt{ids} argumeniga on defineeritud PSU ja SSU id'd
\item
  \texttt{fpc} argumendiga on defineeritud PSU ja SSU üldkogumite suurused (vastavalt piirkondade koguarv ja koolide koguarv piirkonnas)
\end{itemize}

\hypertarget{replikatsioonikaaludega-valimi-defineerimine}{%
\subsection{Replikatsioonikaaludega valimi defineerimine}\label{replikatsioonikaaludega-valimi-defineerimine}}

Replikatsioonikaalude tegemiseks on \emph{survey} funktsioon \texttt{as.svrepdesign()}, mis võtab sisendiks olemasoleva ilma replikatsioonikaaludeta disainiobjekti ja annab väljundiks kaaludega disainiobjekti.

\begin{itemize}
\tightlist
\item
  \texttt{design} argumendiga defineeritakse survey disain, millele tahetakse replikatsioonikaale arvutada
\item
  \texttt{type} argumendiga defineeritakse replikatsioonimeetodi tüüp. Variandid on ``auto'', ``JK1'', ``JKn'', ``BRR'', ``bootstrap'', ``subbootstrap'',``mrbbootstrap'',``Fay''. Täpsemalt on ndende kohta võimalik lugeda funktsiooni abifailist: \texttt{?as.svrepdesign}
\end{itemize}

Replikatsioonikaalude kasutamiseks, juhul kui me ise neid kaale ei arvuta ja need on meie andmestikkus juba olemas, peame defineerima jällegi vastava survey disainiobjekti. Näidisandmestikes ühtegi replikatsioonikaaludega andmestikku ei ole. Kuid saame selle vähese vaevaga eelneva näite replikatsioonikaalude objekti abil teha:

\begin{verbatim}
##  [1] "cds"      "stype"    "name"     "sname"    "snum"     "dname"   
##  [7] "dnum"     "cname"    "cnum"     "flag"     "pcttest"  "api00"   
## [13] "api99"    "target"   "growth"   "sch.wide" "comp.imp" "both"    
## [19] "awards"   "meals"    "ell"      "yr.rnd"   "mobility" "acs.k3"  
## [25] "acs.46"   "acs.core" "pct.resp" "not.hsg"  "hsg"      "some.col"
## [31] "col.grad" "grad.sch" "avg.ed"   "full"     "emer"     "enroll"  
## [37] "api.stu"  "fpc"      "pw"       "indeks"   "rep_w_1"  "rep_w_2" 
## [43] "rep_w_3"  "rep_w_4"  "rep_w_5"  "rep_w_6"  "rep_w_7"  "rep_w_8" 
## [49] "rep_w_9"  "rep_w_10" "rep_w_11" "rep_w_12" "rep_w_13" "rep_w_14"
## [55] "rep_w_15"
\end{verbatim}

Kui kaaludega andmestik on olemas, siis saame replikatsioonikaalude disaini defineerida funktsiooniga \texttt{svrepdesign()}. Funktsioonis on kindlasti vaja defineerida järgmised argumendid:

\begin{itemize}
\tightlist
\item
  \texttt{repweights} argumendiga defineeritakse replikatsioonikaalud. Neid kaale on tavaliselt küllaltki palju. Seega on otstarbekas need enne andmestikust välja võtta. Näiteks \emph{dplyr} paketi \texttt{select()} funktsiooniga. Kuna tavaliselt on kaalud mingi kindla reegli alusel nimetatud, siis on mugav kasutada \texttt{starts\_with()} funktsiooni, mis võimaldab valida kõik ühtse nimeosaga tunnused.
\item
  \texttt{type} argumendiga määratakse replikatsioonikaalude tüüp. Võimalikud tüübid on ``BRR'',``Fay'',``JK1'', ``JKn'',``bootstrap'',``ACS'',``successive-difference'',``JK2'' ja ``other''. See, millist tüüpi on vaja kasutada, peaks olema kirjas küsitluse dokumentatsioonis. Näiteks PISA uuringus on kasutatud ``Fay'' meetodit.
\item
  \texttt{data} argumendiga defineeritakse küsitluse andmestik.
\item
  \texttt{combined.weights} argument ütleb, kas valimi kaalud on juba liidetud replikatsioonikaaludele. Tvaliselt see nii ongi (ja \texttt{combined.weights} vaikimisi väärtus on T), kuid meie isetehtud andmestikus ei ole.
\end{itemize}

\hypertarget{valimiandmete-analuxfcuxfcs}{%
\section{Valimiandmete analüüs}\label{valimiandmete-analuxfcuxfcs}}

Kui valimidisain on defineeritud, siis saab seda kasutada edasistes analüüsides. Selleks on terve hulk \emph{survey} funktsioone, mis kõik algavd \emph{svy} eeliitega. Tasub meeles pidada, et \emph{survey} diainiobjektid sobivad sisendina ainult \emph{survey} funktsioonidele (mõne erandiga). Näiteks tavalise lineaarse regressiooni \texttt{lm()} funktsiooni kasutada ei saa (küll aga \emph{survey} \texttt{glm()} funktsiooni, mis võimaldab meil samuti lineaarset regressiooni jooksutada). Silmas tasub pidada ka seda, et enamke funktsioonide puhul on vaja kasutada tilde märki.

\textbf{Üldkogumi kogusumma:}

\begin{verbatim}
##          total     SE
## enroll 3404940 941611
\end{verbatim}

\textbf{Tunnuse keskmine:}

\begin{verbatim}
##         mean     SE
## api00 644.17 23.779
\end{verbatim}

Kui tahame keskmisele usalduspiire, siis saame kasutada funktsiooni \texttt{confint()}:

\begin{verbatim}
##          2.5 %   97.5 %
## api00 597.5634 690.7754
\end{verbatim}

Võime ka korraga mitme tunnuse keskmist hinnata:

\begin{verbatim}
##         mean     SE
## api00 644.17 23.779
## api99 606.98 24.469
\end{verbatim}

Saame ka kategoriaalsete tunnuste proportsioone hinnata:

\begin{verbatim}
##            mean     SE
## stypeE 0.786885 0.0468
## stypeH 0.076503 0.0271
## stypeM 0.136612 0.0299
\end{verbatim}

\textbf{Kvantiilid:}

\begin{verbatim}
## $api00
##      quantile ci.2.5 ci.97.5       se
## 0.25      552    491     628 31.93791
## 0.5       652    559     715 36.36725
## 0.75      719    696     777 18.88300
## 
## attr(,"hasci")
## [1] TRUE
## attr(,"class")
## [1] "newsvyquantile"
\end{verbatim}

\textbf{Gruppide kaupa hinnagud:}

\begin{verbatim}
##   stype    api00       se
## E     E 648.8681 22.58731
## H     H 618.5714 38.40263
## M     M 631.4400 31.92737
\end{verbatim}

Saame hinnata ka mitut tunnust mitme grupi lõikes:

\begin{verbatim}
##       stype sch.wide    api00    api99 se.api00 se.api99
## E.No      E       No 596.3333 601.6667 43.92749 47.75128
## H.No      H       No 659.3333 662.0000 27.27433 29.52400
## M.No      M       No 606.3750 611.3750 41.53039 41.53240
## E.Yes     E      Yes 653.6439 608.3485 20.52153 21.74141
## H.Yes     H      Yes 607.4545 577.6364 44.14423 46.96892
## M.Yes     M      Yes 643.2353 607.2941 42.55219 42.95820
\end{verbatim}

Kui on vaja usalduspiire:

\begin{verbatim}
##       stype sch.wide    api00    api99 ci_l.api00 ci_l.api99 ci_u.api00
## E.No      E       No 596.3333 601.6667   510.2370   508.0759   682.4296
## H.No      H       No 659.3333 662.0000   605.8766   604.1340   712.7900
## M.No      M       No 606.3750 611.3750   524.9769   529.9730   687.7731
## E.Yes     E      Yes 653.6439 608.3485   613.4225   565.7361   693.8654
## H.Yes     H      Yes 607.4545 577.6364   520.9334   485.5790   693.9756
## M.Yes     M      Yes 643.2353 607.2941   559.8345   523.0976   726.6361
##       ci_u.api99
## E.No    695.2575
## H.No    719.8660
## M.No    692.7770
## E.Yes   650.9609
## H.Yes   669.6938
## M.Yes   691.4906
\end{verbatim}

Regressioonimudelite jaoks on funktsioon \texttt{svyglm()}, mis on väga sarnane \texttt{glm()} funktsioonile.

\textbf{Tavaline lineaarne regressioon:}

\begin{verbatim}
## 
## Call:
## svyglm(formula = api00 ~ ell + meals + mobility, design = des_clus, 
##     family = gaussian())
## 
## Survey design:
## svydesign(ids = ~dnum, weights = ~pw, data = apiclus1)
## 
## Coefficients:
##             Estimate Std. Error t value Pr(>|t|)    
## (Intercept) 819.2791    21.6051  37.921 5.18e-13 ***
## ell          -0.5167     0.3273  -1.579    0.143    
## meals        -3.1232     0.2809 -11.119 2.54e-07 ***
## mobility     -0.1689     0.4494  -0.376    0.714    
## ---
## Signif. codes:  0 '***' 0.001 '**' 0.01 '*' 0.05 '.' 0.1 ' ' 1
## 
## (Dispersion parameter for gaussian family taken to be 3157.85)
## 
## Number of Fisher Scoring iterations: 2
\end{verbatim}

\textbf{Logistiline regressioon:}

\begin{verbatim}
## 
## Call:
## svyglm(formula = I(awards == "Yes") ~ ell + meals + mobility, 
##     design = des_clus, family = binomial())
## 
## Survey design:
## svydesign(ids = ~dnum, weights = ~pw, data = apiclus1)
## 
## Coefficients:
##              Estimate Std. Error t value Pr(>|t|)  
## (Intercept)  0.425949   0.351961   1.210   0.2516  
## ell          0.041120   0.015248   2.697   0.0208 *
## meals       -0.014643   0.007044  -2.079   0.0618 .
## mobility     0.007534   0.013571   0.555   0.5899  
## ---
## Signif. codes:  0 '***' 0.001 '**' 0.01 '*' 0.05 '.' 0.1 ' ' 1
## 
## (Dispersion parameter for binomial family taken to be 0.9907103)
## 
## Number of Fisher Scoring iterations: 4
\end{verbatim}

\hypertarget{plausible-values}{%
\section{Plausible values}\label{plausible-values}}

\emph{Plausible values} või eesti keeles siis ehk usutavad väärtused ei ole tegelikult otseselt küsitlusuuringute teema selles mõttes, et need ei ole seotud valimidisainiga. Küll aga kasutatakse neid küsitlusuuringutes (näiteks PIAACi ja PISA uuringutes), seega on asjakohane neid siin ka põgusalt käsitleda. PIAACi uuringu eesmärgiks on mõõta täiskasvanute oskusi (lugemis-, arvutamis- ja probleemilahendusoskused). Kui me aga vaatame PIAACi andmefaili, siis näeme seal ei ole iga mõõdetava oskuse kohta ühte tunnust, vaid tervelt 10 erinevat tunnust. Need 10 tunnust iga oskuse kohta on nende oskuste usutavad väärtused. Need väärtused on tuletatud iga inimese jaoks lähtuvalt selle inimese vastustest oskusi mõõtvatele küsimustele. Iga inimese jaoks kontrueeritakse tema oskuse tõenäosusjaotus (selleks kasutatakse IRT (\emph{Item Response Theory}) meetodeid). Seejärel võetakse sellest jaotusest 10 juhuslikku väärtus, mis ongi selle inimese usutavad väärtused.
Kui me tahame nüüd teada, et kas meeste või naiste lugemisoskus on keskmiselt parem, siis mida me tegema peaksime? Me ei tohiks kindlasti arvutada iga inimese jaoks tema usutavate väärtuste keskmist. Samuti ei tohiks me kasutada ainult ühte usutavat väärtust. Mõlemal juhul oleks meie järeldused valed.\\
Et saada korrektseid tulemusi, peaksime võrdlema esimese usutava väärtuse tulemusi meeste ja naiste vahel, seejärel võrdlema teise usutava väärtuse tulemusi meeste ja naiste vahel ja nii kümme korda. Seejärel peaksime arvutama kõikide kümne meeste ja naiste erinevuse keskmise, mis olekski korrektne meeste ja naiste erinevus.\\
Tundub päris aeganõudev ja tüütu. Õnneks on ka lihtsam viis - kasutada jällegi Ri \emph{survey} paketti (koos \emph{mitools} paketiga):

  \bibliography{book.bib,packages.bib}

\end{document}
